\section{데이터 및 빈도 불일치 자료의 구성}

\subsection{분석 개요 및 데이터 출처}

\begin{itemize}
  \item 목적:
    \begin{itemize}
      \item 월별 산업활동지표(전산업생산지수)를 종속변수로 사용.
      \item 주별 전력거래량을 고빈도 설명변수로 사용.
      \item MIDAS-AR 및 AR(1) 벤치마크, XGBoost 확장을 통해 nowcasting 성능 비교.
    \end{itemize}
  \item 데이터 출처:
    \begin{itemize}
      \item 월별 전산업생산지수 (계절조정): 국가 통계 자료.
      \item 전력거래량: 공공데이터포털에서 시간대별·지역별 거래량 자료를 수집 후 일·주 단위로 집계.
      \item 기업경기실사지수(BSI): 월별 심리지표.
    \end{itemize}
  \item 소프트웨어:
    \begin{itemize}
      \item R 사용.
      \item 주요 패키지: \texttt{readxl}, \texttt{dplyr}, \texttt{lubridate}, \texttt{forecast}, \texttt{xgboost} 등.
    \end{itemize}
\end{itemize}

\subsection{종속변수: 월별 산업생산 지표}

\begin{itemize}
  \item 표기:
    \begin{itemize}
      \item $I^{\text{tot}}_t$: 전산업생산지수(계절조정, 월 $t$).
      \item $y^{\text{tot}}_t$: 월간 로그 성장률.
      \item $y^{\text{tot,YoY}}_t$: 전년동월비.
      \item $t$: 월(month) 인덱스.
      \item 사용 가능 기간: 2000년 1월–2025년 9월 (실제 추정에선 샘플 제약 존재).
    \end{itemize}
  \item 정의:
    \begin{align}
      y^{\text{tot}}_t 
        &= 100\left( \log I^{\text{tot}}_t - \log I^{\text{tot}}_{t-1} \right), \\
      y^{\text{tot,YoY}}_t
        &= 100\left( \frac{I^{\text{tot}}_t}{I^{\text{tot}}_{t-12}} - 1 \right)
          \approx 100\left( \log I^{\text{tot}}_t - \log I^{\text{tot}}_{t-12} \right).
    \end{align}
  \item 실제 추정:
    \begin{itemize}
      \item $y^{\text{tot}}_t$ 또는 $y^{\text{tot,YoY}}_t$ 중 하나를 선택하여 $y_t$로 표기.
      \item 두 경우 모두 계절조정된 전산업생산지수를 기반으로 계산.
    \end{itemize}
\end{itemize}

\subsection{설명변수: 주별/일별 전력거래량 성장률}

\begin{itemize}
  \item 표기:
    \begin{itemize}
      \item $P_s$: 주 $s$에서의 전력거래량.
      \item $\ell_s = \log P_s$: 로그 전력거래량.
      \item $\ell^{\text{SA}}_s$: 계절조정된 로그 전력거래량.
      \item $x_s$: 계절조정된 로그 성장률.
    \end{itemize}
  \item 계절조정 및 성장률:
    \begin{align}
      \ell_s &= \log P_s, \\
      \ell^{\text{SA}}_s &= \text{seasadj}\big( \text{STL}(\ell_s) \big), \\
      x_s &= 100\left( \ell^{\text{SA}}_s - \ell^{\text{SA}}_{s-1} \right).
    \end{align}
  \item $x_s$를 MIDAS 회귀의 고빈도 설명변수로 사용.
\end{itemize}

\section{MIDAS-AR 모형 설정과 추정 방법}

\subsection{MIDAS-AR(1) 모형}

\begin{itemize}
  \item 기본 구조:
    \begin{align}
      y_t
        &= \lambda y_{t-1}
           + \beta_0 + \beta_1 Z_t(K,\theta)
           + \varepsilon_t.
    \end{align}
  \item 구성 요소:
    \begin{itemize}
      \item $\lambda$: AR(1) 계수.
      \item $\beta_0, \beta_1$: 상수항 및 MIDAS 회귀자 계수.
      \item $\theta = (\theta_1,\theta_2)$: exp–Almon 가중치 모수.
      \item $K$: 사용하는 고빈도 래그 개수.
      \item $Z_t(K,\theta)$: 고빈도 성장률의 가중 합.
    \end{itemize}
  \item 안정성 제약:
    \begin{itemize}
      \item 재파라미터화 예: $\lambda = \tanh(\lambda^{\text{raw}})$ 등.
      \item $\lvert \lambda \rvert < 1$ 자동 충족.
    \end{itemize}
\end{itemize}

\subsection{exp–Almon 가중치와 MIDAS regressor}

\begin{itemize}
  \item 월별 cutoff:
    \begin{itemize}
      \item $T_t$: 월 $t$의 말일.
      \item cutoff까지의 고빈도 성장률 $\{x_s : s \le T_t\}$ 중 최근 $K$개를 선택:
      \[
        x_{t,1},\dots,x_{t,K},
      \]
      \item $x_{t,1}$: 가장 최근 주, $x_{t,K}$: 가장 오래된 래그.
    \end{itemize}
  \item exp–Almon 가중치:
    \begin{align}
      Z_t(K,\theta) 
        &= \sum_{k=1}^K w_k(\theta)\, x_{t,k}, \\
      w_k(\theta_1,\theta_2)
        &= \frac{\exp(\theta_1 k + \theta_2 k^2)}
                {\sum_{j=1}^K \exp(\theta_1 j + \theta_2 j^2)}, \quad k=1,\dots,K.
    \end{align}
  \item 제약 및 재파라미터화:
    \begin{itemize}
      \item $w_k(\theta)\ge 0$, $\sum_k w_k(\theta)=1$.
      \item $\theta_2 < 0$ 제약을 통해 오래된 래그 가중치 감소 유도.
      \item 구현:
      \begin{equation*}
        \theta_1 = 300 \tanh(\theta_{1,\text{raw}}), \quad
        \theta_2 = -10 \frac{\exp(\theta_{2,\text{raw}})}{1 + \exp(\theta_{2,\text{raw}})}.
      \end{equation*}
      \item 결과: $|\theta_1|<300$, $-10<\theta_2<0$ 보장.
      \item 목적: 수치적 안정성 확보, corner solution 방지.
    \end{itemize}
  \item $K$ 선택:
    \begin{itemize}
      \item 후보: $K \in \{8,13,26,52\}$.
      \item Validation RMSE 최소 기준으로 최적 $K^*$ 선택.
    \end{itemize}
\end{itemize}

\subsection{\citet{clements2008macroeconomic} 추정 절차}

\begin{itemize}
  \item 샘플 분할:
    \begin{itemize}
      \item Train: 2002–2020년.
      \item Validation: 2021–2022년.
      \item Test: 2023–2024년.
      \item COVID 구간 포함 (제외 여부에 따른 성능 변화 미미).
    \end{itemize}
  \item 1단계 (Standard MIDAS, AR 없음):
    \begin{align}
      y_t = \beta_0 + \beta_1 Z_t(K,\theta) + \varepsilon_t.
    \end{align}
    \begin{itemize}
      \item NLS로 $(\beta_0,\beta_1,\theta)$ 추정 → 초기값 획득.
    \end{itemize}
  \item 2단계 (잔차의 AR 추정):
    \begin{align}
      \hat{\varepsilon}_t
        &= \lambda^{(0)} \hat{\varepsilon}_{t-1} + u_t.
    \end{align}
    \begin{itemize}
      \item OLS로 $\lambda^{(0)}$ 추정.
    \end{itemize}
  \item 3단계 ($\lambda$ 고정 MIDAS-AR):
    \begin{align}
      y_t
        &= \lambda^{(0)} y_{t-1}
           + \beta_0 + \beta_1 \widetilde{Z}_t(K,\theta;\lambda^{(0)})
           + \varepsilon_t.
    \end{align}
    \begin{itemize}
      \item $y_t$의 자기회귀 구조를 반영하여 고빈도 성장률을 한 번 조정한 뒤 가중치 부여.
      \item $y_{t-1}$이 설명하는 저빈도 움직임과 고빈도 정보의 분리를 목표.
      \item NLS로 $(\beta_0,\beta_1,\theta)$ 재추정.
    \end{itemize}
  \item 4단계 (Full MIDAS-AR 공동 추정):
    \begin{itemize}
      \item $(\lambda,\beta_0,\beta_1,\theta)$를 동시에 NLS로 추정.
      \item 초기값: 3단계 추정치.
      \item 최적화 알고리즘: BFGS.
    \end{itemize}
\end{itemize}

\subsection{K 선택과 예측 설계}

\begin{itemize}
  \item 튜닝 단계:
    \begin{itemize}
      \item 후보 $K$: $\{8,13,26,52\}$.
      \item 각 $K$에 대해 위 절차(1–4단계)를 수행.
      \item Validation 구간에서 one-step-ahead 예측치 $\hat{y}_t$ 계산.
      \item Validation RMSE:
      \begin{align}
        \text{RMSE}_{\text{val}}(K,1)
          = \sqrt{\frac{1}{T_{\text{val}}}
                  \sum_{t\in\text{val}} (y_t - \hat{y}_t)^2 }.
      \end{align}
      \item RMSE 최소가 되는 $K^*$ 선택.
    \end{itemize}
  \item 최종 예측:
    \begin{itemize}
      \item Train+Val(2002–2022) 전체로 재추정 → $(\hat{\lambda},\hat{\beta}_0,\hat{\beta}_1,\hat{\theta})$.
      \item one-step-ahead 예측:
      \begin{align}
        \hat{y}_t
          &= \hat{\lambda} y_{t-1} 
             + \hat{\beta}_0 + \hat{\beta}_1 Z_t(K^*,\hat{\theta}).
      \end{align}
      \item Test RMSE:
      \begin{align}
        \text{RMSE}_{\text{test}}
          &= \sqrt{\frac{1}{T_{\text{test}}}
                  \sum_{t\in\text{test}} (y_t - \hat{y}_t)^2 }.
      \end{align}
    \end{itemize}
\end{itemize}

\section{AR(1) 벤치마크와 Vintage별 MIDAS–AR(1)}

\subsection{공통 샘플 및 AR(1) 벤치마크}

\begin{itemize}
  \item 공통 샘플:
    \begin{itemize}
      \item $y_t$ 사용 가능 시점: 2002-03–2024-12 (가정).
      \item 주별 전력 성장률 $x_s$: 2001-04 이후.
      \item MIDAS 래그가 정의 가능한 월만 남긴 공통 집합:
      \[
        \mathcal{T}_\text{common} \subset \{2002\text{-03},\dots,2024\text{-12}\}.
      \]
    \end{itemize}
  \item AR(1) 모형:
    \begin{align}
      y_t &= \alpha + \phi y_{t-1} + e_t.
    \end{align}
  \item 추정 및 예측:
    \begin{itemize}
      \item 2002–2022년 (공통 샘플 내)에서 OLS로 $(\hat{\alpha},\hat{\phi})$ 추정.
      \item one-step-ahead 예측:
      \begin{align}
        \hat{y}_t^{\text{AR(1)}} &= \hat{\alpha} + \hat{\phi} y_{t-1}.
      \end{align}
      \item Test RMSE:
      \begin{align}
        \text{RMSE}_\text{test}^{\text{AR(1)}}
          &= \sqrt{\frac{1}{T_\text{test}} \sum_{t \in \mathcal{T}_\text{test}}
                    ( y_t - \hat{y}_t^{\text{AR(1)}} )^2}.
      \end{align}
      \item $\mathcal{T}_\text{test} = \{ t \in \mathcal{T}_\text{common} : 2023\text{-01} \le t \le 2024\text{-12}\}$.
      \item AR(1)은 고빈도 정보를 사용하지 않으므로 vintage와 무관하게 동일한 RMSE를 가짐.
    \end{itemize}
\end{itemize}

\subsection{Vintage 정보세트 정의}

\begin{itemize}
  \item 정의:
    \begin{itemize}
      \item $m_t$: 월 $t$의 첫날.
      \item $T_t$: 월 $t$의 말일.
      \item 각 주 $s$에 대해 월(month) 및 주차(week-in-month) 계산.
    \end{itemize}
  \item cutoff date $C_t^{(h)}$:
    \begin{itemize}
      \item h0: $C_t^{(\text{h0})} = m_t - 1$ (전월 말까지).
      \item h1: $t$월 week $\le 1$인 주 중 마지막 주 날짜.
      \item h2: $t$월 week $\le 2$인 주 중 마지막 주 날짜.
      \item h3: $t$월 week $\le 3$인 주 중 마지막 주 날짜.
      \item h4: $C_t^{(\text{h4})} = T_t$ (당월 전체).
    \end{itemize}
  \item 각 vintage별 MIDAS 래그:
    \begin{itemize}
      \item $\{x_s : s \le C_t^{(h)}\}$ 중 최근 $K$개 선택:
      \[
        x^{(h)}_{t,1},\dots,x^{(h)}_{t,K}.
      \]
      \item exp–Almon 가중치 적용:
      \[
        Z_t^{(h)}(K,\theta) = \sum_{k=1}^K w_k(\theta)\, x^{(h)}_{t,k}.
      \]
    \end{itemize}
\end{itemize}

\subsection{Vintage별 MIDAS–AR(1) 추정 및 요약}

\begin{itemize}
  \item 모형:
    \begin{align}
      y_t = \lambda^{(h)} y_{t-1}
           + \beta_0^{(h)} + \beta_1^{(h)} Z_t^{(h)}(K,\theta^{(h)})
           + \varepsilon_t^{(h)}.
    \end{align}
  \item 모수:
    \begin{itemize}
      \item $\lambda^{(h)}$: AR(1) 계수.
      \item $\beta_0^{(h)},\beta_1^{(h)}$: 상수 및 MIDAS 계수.
      \item $\theta^{(h)}$: exp–Almon 모수.
      \item $K$: 래그 개수 (vintage별 최적 $K^*(h)$ 선택).
    \end{itemize}
  \item 제약 및 최적화:
    \begin{itemize}
      \item 재파라미터화로 $|\lambda^{(h)}|<1$, $|\theta_1^{(h)}|<300$, $-10<\theta_2^{(h)}<0$ 보장.
      \item BFGS로 NLS 추정.
    \end{itemize}
  \item 튜닝 및 최종 예측:
    \begin{itemize}
      \item 각 $h$에 대해 $K \in \{8,13,26,52\}$ 중 Validation RMSE 최소인 $K^*(h)$ 선택.
      \item 2002–2022년 전체로 재추정 → $(\hat{\lambda}^{(h)},\hat{\beta}_0^{(h)},\hat{\beta}_1^{(h)},\hat{\theta}^{(h)})$.
      \item 테스트 구간 예측:
      \[
        \hat{y}_t^{(h)}
          = \hat{\lambda}^{(h)} y_{t-1}
           + \hat{\beta}_0^{(h)} 
           + \hat{\beta}_1^{(h)} Z_t^{(h)}(K^*(h),\hat{\theta}^{(h)}).
      \]
      \item Test RMSE:
      \[
        \text{RMSE}_\text{test}^{(h)}
          = \sqrt{\frac{1}{T_\text{test}} \sum_{t \in \mathcal{T}_\text{test}}
             ( y_t - \hat{y}_t^{(h)} )^2 }.
      \]
    \end{itemize}
  \item exp–Almon 가중치 해석:
    \begin{itemize}
      \item 각 vintage에서 $w_k^{(h)} = w_k(\hat{\theta}_1^{(h)},\hat{\theta}_2^{(h)})$ 계산.
      \item $k=1$: 가장 최근 주, $k=K^*(h)$: 가장 오래된 래그.
      \item $w_k^{(h)}$ 패턴으로 어떤 시점의 전력 정보가 중요하게 사용되는지 확인.
    \end{itemize}
\end{itemize}

\section{실증 결과 요약: AR(1) vs MIDAS–AR(1)}

\subsection{종속변수: 전산업생산지수 성장률}

\paragraph{정태성 검정}

\begin{itemize}
  \item ADF 검정:
    \begin{itemize}
      \item R 함수: \texttt{adf.test()}.
      \item lag 선택: $k = \lfloor (T-1)^{1/3} \rfloor = 6$.
      \item 검정통계량: 약 $-7.59$.
      \item p-value: $< 0.01$.
      \item 결론: 1\% 유의수준에서 단위근 귀무가설 기각 → 수준에서 정상성(stationarity) 가정 가능.
    \end{itemize}
\end{itemize}

\paragraph{Vintage별 MIDAS–AR(1) 인샘플 적합}

\begin{figure}[H]
    \centering
    \includegraphics[width=\textwidth]{midas/images/midasar_sample_fit.png}
    \caption{MIDAS–AR(1) 인샘플 적합: 실제값 vs 예측값}
    \label{fig:midasar_insample}
\end{figure}

\begin{itemize}
  \item Figure~\ref{fig:midasar_insample}:
    \begin{itemize}
      \item 2002–2022년 구간 인샘플 적합.
      \item 모든 vintage에서 인샘플 RMSE: 약 1.24–1.26.
      \item AR(1)이 설명하는 저빈도 움직임 위에 MIDAS 회귀자가 세부 변동을 추가 설명.
      \item $K^*(h)$:
        \begin{itemize}
          \item $h0$, $h1$: 상대적으로 짧은 래그($K=8$).
          \item $h2$, $h4$: $K=52$ 등 긴 래그 선택.
        \end{itemize}
    \end{itemize}
\end{itemize}

\paragraph{Vintage별 exp–Almon 가중치}

\begin{figure}[H]
    \centering
    \includegraphics[width=\textwidth]{midas/images/midasar_weight.png}
    \caption{exp–Almon 가중치}
    \label{fig:midasar_weights}
\end{figure}

\begin{itemize}
  \item Figure~\ref{fig:midasar_weights}:
    \begin{itemize}
      \item 전반적으로 최근 몇 주(1–3주)에 가중치 집중.
      \item $h0$: (t-1)월 정보만 사용 → 최근 주 중심 우하향 패턴.
      \item $h2$, $h4$: $K^*(h)=52$이나 실질 가중치는 최근 1–4주에 집중, 나머지는 0에 근접.
      \item $h3$: 가장 최근 1주에 거의 전 가중치 집중.
    \end{itemize}
\end{itemize}

\paragraph{Vintage별 Test RMSE: AR(1) vs MIDAS–AR(1)}

\begin{table}[H]
\centering
\begin{tabular}{lcc}
\toprule
Vintage & AR(1) & MIDAS-AR(1) \\
\midrule
h0 & 0.950 (0.0) & 0.952 (-0.2) \\
h1 & 0.950 (0.0) & 0.951 (-0.1) \\
h2 & 0.950 (0.0) & 0.952 (-0.2) \\
h3 & 0.950 (0.0) & 0.951 (-0.1) \\
h4 & 0.950 (0.0) & 0.945 (0.5)  \\
\bottomrule
\end{tabular}
\caption{테스트 RMSE: AR(1) vs MIDAS-AR(1)}
\label{tab:midasar_rmse_table}
\end{table}

\begin{itemize}
  \item 표 해석:
    \begin{itemize}
      \item 괄호 안 숫자: AR(1) 대비 RMSE 감소율(\%).
      \item $h0$–$h3$: MIDAS–AR(1)의 감소율 $\approx -0.2\%$~$-0.1\%$ (AR(1)보다 약간 열악).
      \item $h4$: MIDAS–AR(1)의 RMSE가 약 $0.5\%$ 감소 (소폭 개선).
    \end{itemize}
\end{itemize}

\paragraph{테스트 기간 예측 경로}

\begin{figure}[H]
    \centering
    \includegraphics[width=\textwidth]{midas/images/midasar_test_fit.png}
    \caption{테스트 기간 예측 경로: AR(1) vs MIDAS-AR(1)}
    \label{fig:midasar_test}
\end{figure}

\begin{itemize}
  \item Figure~\ref{fig:midasar_test}:
    \begin{itemize}
      \item 2023–2024년 테스트 구간에서 AR(1) vs MIDAS–AR(1) 예측 경로 비교.
      \item 전 vintage에서 두 모형의 궤적이 거의 동일.
      \item $h0$–$h3$: 일부 달에서 미세한 차이 있으나 RMSE 개선으로 연결되지 않음.
      \item $h4$: 몇몇 국면에서 MIDAS–AR(1)이 실제값에 조금 더 근접.
    \end{itemize}
  \item 결론(월간 성장률):
    \begin{itemize}
      \item AR(1)만으로도 단기 예측력이 높음.
      \item 주별 전력거래량을 추가한 MIDAS–AR(1)은 full month 정보(h4)에서만 약간의 개선.
      \item 월 중(h0–h2) 정보만으로는 추가 예측력 거의 없음.
    \end{itemize}
\end{itemize}

\subsection{종속변수: 전년동월비}

\paragraph{정태성 검정}

\begin{itemize}
  \item ADF 검정:
    \begin{itemize}
      \item R \texttt{adf.test()} 사용.
      \item lag: 자동 선택 ($k \approx 6$).
      \item 검정통계량: $-5.56$.
      \item p-value: $<0.01$.
      \item 결론: 수준에서 정상(stationary)으로 판단.
    \end{itemize}
\end{itemize}

\paragraph{인샘플 적합 및 exp–Almon 가중치}

\begin{figure}[H]
    \centering
    \includegraphics[width=\textwidth]{midas/images/midasar_mom_sample_fit.png}
    \caption{MIDAS–AR(1) 인샘플 적합: 전년동월비}
    \label{fig:midasar_yoy_insample}
\end{figure}

\begin{figure}[H]
    \centering
    \includegraphics[width=\textwidth]{midas/images/midasar_mom_weight.png}
    \caption{exp–Almon 가중치: 전년동월비 모형}
    \label{fig:midasar_yoy_weights}
\end{figure}

\begin{itemize}
  \item Figure~\ref{fig:midasar_yoy_insample}:
    \begin{itemize}
      \item 전년동월비의 변동성이 크고 스파이크가 존재.
      \item MIDAS–AR(1) 적합치는 각 vintage에서 실제값을 비교적 잘 따라감.
      \item 인샘플 RMSE: 대략 1.78–1.88 (월간 성장률 모형보다 크다).
    \end{itemize}
  \item Figure~\ref{fig:midasar_yoy_weights}:
    \begin{itemize}
      \item $h0$: $K^*(h0)=52$, 실제 가중치는 최근 5–7주에 집중.
      \item $h1$, $h2$: 종 모양(hump-shaped) 패턴; cutoff 이전 4–6주 전 래그에 큰 가중치.
      \item $h3$, $h4$: 짧은 래그($K^*=8$) 선택, 단조 감소형 가중치.
      \item 전반적으로 최근 1–2개월 내 전력 사용 정보가 중요, 일부 vintage에서 약간 더 이전 주에 비중.
    \end{itemize}
\end{itemize}

\paragraph{Vintage별 Test RMSE (전년동월비)}

\begin{table}[H]
\centering
\begin{tabular}{lcc}
\toprule
Vintage & AR(1) & MIDAS-AR(1) \\
\midrule
h0 & 1.49 (0.0) & 1.50 (-0.7) \\
h1 & 1.49 (0.0) & 1.60 (-7.4) \\
h2 & 1.49 (0.0) & 1.47 (1.3)  \\
h3 & 1.49 (0.0) & 1.50 (-0.7) \\
h4 & 1.49 (0.0) & 1.49 (0.0)  \\
\bottomrule
\end{tabular}
\caption{테스트 RMSE: 전년동월비 (MIDAS-AR)}
\label{tab:midasar_rmse_yoy}
\end{table}

\begin{itemize}
  \item 해석:
    \begin{itemize}
      \item $h0$, $h3$: MIDAS–AR(1)의 RMSE가 AR(1)보다 약 $0.7\%$ 악화.
      \item $h1$: RMSE 약 $7.4\%$ 악화.
      \item $h2$: RMSE 약 $1.3\%$ 개선(크기 작음).
      \item $h4$: 두 모형 RMSE 동일.
    \end{itemize}
\end{itemize}

\paragraph{테스트 기간 예측 경로(전년동월비)}

\begin{figure}[H]
    \centering
    \includegraphics[width=\textwidth]{midas/images/midasar_mom_test_fit.png}
    \caption{테스트 기간 예측 경로: 전년동월비}
    \label{fig:midasar_yoy_test}
\end{figure}

\begin{itemize}
    \item Figure~\ref{fig:midasar_yoy_test}:
    \begin{itemize}
      \item 전반적으로 AR(1)과 MIDAS–AR(1)의 궤적 유사.
      \item $h1$: MIDAS–AR(1)이 일부 구간에서 노이즈를 과도하게 따라가며 RMSE 악화.
      \item $h2$: 일부 구간(예: 2023년·2024년 초 상승 국면)에서 MIDAS–AR(1)이 실제값에 더 근접.
      \item $h0$, $h3$, $h4$: 두 모형 간 차이가 미미.
    \end{itemize}
  \item 결론(전년동월비):
    \begin{itemize}
      \item AR(1)만으로도 상당한 단기 예측력 확보.
      \item 주별 전력거래량을 활용한 MIDAS–AR(1)은 대부분의 vintage에서 AR(1) 대비 개선 없음.
      \item $h2$에서만 소폭의 RMSE 감소, 그 외 vintage에서는 개선 불충분 또는 악화.
    \end{itemize}
\end{itemize}

\section{XGBoost를 활용한 비선형 확장}\label{sec:xgb}

\subsection{기본 아이디어 및 표본 분할}

\begin{itemize}
  \item 기본 가정:
    \begin{itemize}
      \item $y_{t-1}$만으로도 강한 자기회귀 구조 존재 → AR(1) 벤치마크.
      \item 전력거래량, BSI 등 고빈도·심리지표는 비선형 추가정보 가능.
    \end{itemize}
  \item 비교 대상:
    \begin{itemize}
      \item (1) 선형 ARX: AR(1) + 고빈도 feature의 선형 효과.
      \item (2) AR(1) 잔차에 대한 XGBoost 보정: $e_t$를 비선형 함수로 설명.
      \item (3) XGBoost 직접 예측: $(y_{t-1}, x_{t,h})$를 입력으로 $y_t$ 직접 예측.
    \end{itemize}
  \item 표본 분할:
    \begin{itemize}
      \item 데이터: $y_t$, 주별 전력, BSI가 모두 존재하는 2003년 이후.
      \item Train: 2003–2020년.
      \item Validation: 2021–2022년.
      \item Test: 2023–2024년.
      \item vintage $h0$–$h4$: 모두 동일한 분할, feature만 다름.
    \end{itemize}
\end{itemize}

\subsection{feature 구성 및 vintage별 정보세트(요약)}

\begin{table}[H]
  \centering
  \label{tab:vintage_features}
  \begin{tabular}{lccccc}
    \toprule
    Feature & $h0$ & $h1$ & $h2$ & $h3$ & $h4$ \\
    \midrule
    \multicolumn{6}{l}{\textbf{(t-1)월 주별 전력거래량}} \\
    \quad pw\_tm1\_w1 & \checkmark & \checkmark & \checkmark & \checkmark & \checkmark \\
    \quad pw\_tm1\_w2 & \checkmark & \checkmark & \checkmark & \checkmark & \checkmark \\
    \quad pw\_tm1\_w3 & \checkmark & \checkmark & \checkmark & \checkmark & \checkmark \\
    \quad pw\_tm1\_w4 & \checkmark & \checkmark & \checkmark & \checkmark & \checkmark \\
    \addlinespace[0.5ex]
    \multicolumn{6}{l}{\textbf{(t-1)월 BSI 지수}} \\
    \quad bsi\_tm1      & \checkmark & \checkmark & \checkmark & \checkmark & \checkmark \\
    \quad bsi\_tm1\_yoy & \checkmark & \checkmark & \checkmark & \checkmark & \checkmark \\
    \addlinespace[0.5ex]
    \multicolumn{6}{l}{\textbf{t월 주별 전력거래량}} \\
    \quad pw\_t\_w1 &           & \checkmark & \checkmark & \checkmark & \checkmark \\
    \quad pw\_t\_w2 &           &            & \checkmark & \checkmark & \checkmark \\
    \quad pw\_t\_w3 &           &            &            & \checkmark & \checkmark \\
    \quad pw\_t\_w4 &           &            &            &            & \checkmark \\
    \addlinespace[0.5ex]
    \multicolumn{6}{l}{\textbf{t월 주별 전력거래량 전년동월대비}} \\
    \quad pwyoy\_t\_w1 &        & \checkmark & \checkmark & \checkmark & \checkmark \\
    \quad pwyoy\_t\_w2 &        &            & \checkmark & \checkmark & \checkmark \\
    \quad pwyoy\_t\_w3 &        &            &            & \checkmark & \checkmark \\
    \quad pwyoy\_t\_w4 &        &            &            &            & \checkmark \\
    \addlinespace[0.5ex]
    \multicolumn{6}{l}{\textbf{t월 BSI 지수 (동행)}} \\
    \quad bsi\_t      &         &            &            &            & \checkmark \\
    \quad bsi\_t\_yoy &         &            &            &            & \checkmark \\
    \bottomrule
  \end{tabular}
\caption{Vintage별 설명변수}
\end{table}

\begin{itemize}
  \item 공통 전월 정보:
    \begin{itemize}
      \item $(t-1)$월 주별 전력: \texttt{pw\_tm1\_w1}–\texttt{pw\_tm1\_w4}.
      \item $(t-1)$월 BSI: \texttt{bsi\_tm1}, \texttt{bsi\_tm1\_yoy}.
    \end{itemize}
  \item 현재 월 주별 전력 및 YoY:
    \begin{itemize}
      \item vintage별로 사용 가능한 주차까지 \texttt{pw\_t\_w$j$}, \texttt{pwyoy\_t\_w$j$} 추가.
      \item h1: 1주, h2: 1–2주, h3: 1–3주, h4: 1–4주.
    \end{itemize}
  \item 동월 BSI:
    \begin{itemize}
      \item h4: \texttt{bsi\_t}, \texttt{bsi\_t\_yoy} 포함.
    \end{itemize}
  \item feature 벡터:
    \[
      x_{t,h} = (\text{pw\_tm1\_w1-w4},\; \text{bsi\_tm1}, \text{bsi\_tm1\_yoy},
      \text{pw\_t\_w1-w4}, \text{pwyoy\_t\_w1-w4}, \text{bsi\_t},\text{bsi\_t\_yoy})'.
    \]
\end{itemize}


\subsection{모형별 정의}

\paragraph{(1) 선형 ARX}

\begin{itemize}
  \item 모형:
    \begin{equation}
      y_t = \alpha + \phi y_{t-1} + \beta_h' x_{t,h} + \varepsilon_{t,h}.
    \end{equation}
  \item 추정:
    \begin{itemize}
      \item 2003–2022년 전체로 OLS.
      \item 테스트 구간에서 out-of-sample 예측.
    \end{itemize}
\end{itemize}

\paragraph{(2) AR(1)+XGB\_residual}

\begin{itemize}
  \item 1단계: AR(1) 적합.
    \begin{itemize}
      \item 모형: $y_t = \alpha + \phi y_{t-1} + u_t$.
      \item 잔차: $e_t = y_t - \hat{y}_t^{AR(1)}$.
    \end{itemize}
  \item 2단계: 잔차에 대한 XGBoost:
    \begin{equation}
      e_t = f_h(x_{t,h}) + \eta_{t,h}.
    \end{equation}
  \item 테스트 예측:
    \[
      \hat y_{t,h}^{AR(1)+XGB\_res}
      = \hat y_t^{AR(1)} + \hat f_h(x_{t,h}).
    \]
\end{itemize}

\paragraph{(3) XGB\_direct}

\begin{itemize}
  \item 입력:
    \[
      z_{t,h} = (y_{t-1}, x_{t,h})'.
    \]
  \item 모형:
    \[
      y_t = g_h(z_{t,h}) + \xi_{t,h}.
    \]
  \item 테스트 예측:
    \[
      \hat y_{t,h}^{XGB\_direct} = \hat g_h(z_{t,h}).
    \]
\end{itemize}

\subsection{하이퍼파라미터 및 롤링 검증}

\begin{itemize}
  \item XGBoost 설정:
    \begin{itemize}
      \item \texttt{objective} = "reg:squarederror".
      \item \texttt{eval\_metric} = "rmse".
      \item $\eta = 0.05$, \texttt{max\_depth} = 3.
      \item \texttt{subsample} = 0.8, \texttt{colsample\_bytree} = 0.8.
    \end{itemize}
  \item 선택 파라미터:
    \begin{itemize}
      \item 부스팅 반복 횟수 $n_{\text{round}} \in \{50,100,150,200\}$.
    \end{itemize}
  \item 롤링 교차검증:
    \begin{itemize}
      \item 2003–2022년 중 뒤쪽 24개월을 네 개의 6개월 validation 윈도우로 분할.
      \item 각 윈도우 $j$: $t < v_j$는 train, $v_j \le t \le v_j+5$는 validation.
      \item 각 $n_{\text{round}}$에 대해 네 윈도우의 평균 RMSE 계산.
      \item 평균 RMSE 최소의 $n_{\text{round}}$ 선택.
    \end{itemize}
  \item 최종 학습 및 평가:
    \begin{itemize}
      \item 선택된 $n_{\text{round}}$로 2003–2022년 전체로 재학습.
      \item 2023–2024년 테스트 구간에 대한 예측 수행.
      \item 성능지표:
      \[
        \text{RMSE}_{h}^{(m)} =
          \sqrt{\frac{1}{T_{\text{test}}}
              \sum_{t \in \text{test}}
              (y_t - \hat y_{t,h}^{(m)})^2}.
      \]
    \end{itemize}
\end{itemize}


\section{실증 결과 요약}

\subsection{종속변수: 전산업생산지수 성장률}

\subsubsection{예측 성능: RMSE 및 상대 개선율}

\begin{table}[H]
\centering
\label{tab:rmse-xgb}
\begin{tabular}{lcccc}
\toprule
Vintage & AR(1) & ARX (linear) & AR(1)+XGB\_residual & XGB-direct \\
\midrule
h0 & 0.952 (0.0)         & \textbf{0.950} (0.2) & 1.110 (-10.3) & 1.030 (-4.5) \\
h1 & \textbf{0.953} (0.0)& 0.964 (-1.2)        & 1.040 (-11.2) & 0.979 (-2.6) \\
h2 & \textbf{0.953} (0.0)& 0.964 (-1.2)        & 1.040 (-10.2) & 1.000 (-4.4) \\
h3 & \textbf{0.953} (0.0)& 0.964 (-1.2)        & 1.000 (-7.0)  & 1.000 (-7.0)  \\
h4 & 0.953 (0.0)         & \textbf{0.940} (1.4)& 1.040 (-7.0)  & 0.951 (0.2)  \\
\bottomrule
\end{tabular}
\begin{flushleft}
\caption{테스트 RMSE: XGBoost 모형}
\footnotesize
\textit{주}: 각 셀은 2023–2024년 테스트 구간에서의 RMSE와,
괄호 안의 AR(1) 대비 RMSE 감소율(\%)을 함께 보고한다.
감소율은 $100 \times (1 - \text{RMSE}_{m,h} / \text{RMSE}_{\text{AR(1)},h})$로 정의되며,
양수 값은 동일한 vintage에서 AR(1) 모형보다 예측 오차가 작다는 것을 의미한다.
\end{flushleft}
\end{table}

\begin{itemize}
  \item ARX:
    \begin{itemize}
      \item 대부분의 vintage에서 AR(1) 대비 감소율 $\approx -1.2\%$ 수준 (비슷하거나 약간 열악).
      \item $h4$: 약 1.4\% RMSE 감소 (소폭 개선).
    \end{itemize}
  \item AR(1)+XGB\_residual:
    \begin{itemize}
      \item 모든 vintage에서 RMSE가 AR(1)보다 7–11\% 증가.
      \item 잔차에 대한 부스팅이 노이즈를 과적합하는 경향.
    \end{itemize}
  \item XGB-direct:
    \begin{itemize}
      \item 대부분의 vintage에서 AR(1) 대비 2.6–4.5\% 수준 성능 저하.
      \item $h3$, $h4$: 각각 약 0.3\%, 0.2\% 개선(크기는 매우 작음).
    \end{itemize}
\end{itemize}

\subsubsection{시각적 비교와 feature importance}

\paragraph{테스트 예측 경로}

\begin{figure}[H]
    \centering
    \includegraphics[width=\textwidth]{midas/images/xgboost_test_plots.png}
    \caption{테스트 기간 예측 경로: XGBoost 모형}
    \label{fig:xgboost_test_plots}
\end{figure}

\begin{itemize}
  \item Figure~\ref{fig:xgboost_test_plots}:
    \begin{itemize}
      \item 각 vintage별 테스트 구간에서 AR(1), ARX, AR(1)+XGB\_res, XGB\_direct 예측 경로 비교.
      \item 네 모형 모두 AR(1) 궤적과 매우 유사한 패턴.
      \item AR(1)+XGB\_residual: 일부 시점에서 진폭 과대 → RMSE 악화와 일치.
    \end{itemize}
\end{itemize}

\paragraph{Feature importance (Gain 기준)}

\begin{figure}[H]
    \centering
    \includegraphics[width=\textwidth]{midas/images/heatmap_iip.png}
    \caption{변수 중요도 히트맵}
    \label{fig:xgb-imp-heatmap_iip}
\end{figure}

\begin{itemize}
  \item 정의:
    \begin{itemize}
      \item Gain: 해당 변수로 분할했을 때의 손실 감소량 $\Delta L$을 모든 노드에서 합산한 값.
      \item vintage별로 합이 1이 되도록 정규화해 feature importance로 사용.
    \end{itemize}
  \item Figure~\ref{fig:xgb-imp-heatmap_iip}:
    \begin{itemize}
      \item XGB-direct와 AR(1)+XGB\_residual의 normalized Gain을 히트맵으로 비교.
    \end{itemize}
  \item XGB-direct:
    \begin{itemize}
      \item 모든 vintage에서 $y_{t-1}$의 중요도가 가장 높음.
      \item XGBoost가 AR(1) 구조를 재현하는 방향으로 작동.
      \item 보조적으로 $(t-1)$월 BSI 및 $h4$의 동월 BSI가 사용됨.
      \item 현재 달 1주 전력거래량 및 YoY는 일부 vintage에서만 제한적으로 중요.
    \end{itemize}
  \item AR(1)+XGB\_residual:
    \begin{itemize}
      \item \texttt{bsi\_tm1\_yoy}, \texttt{pw\_tm1\_w1}의 중요도가 상대적으로 큼.
      \item AR(1)이 설명하지 못한 잔차 패턴이 전월 BSI YoY와 전월 초 전력거래량과 관련됨을 시사.
      \item 잔차의 신호대잡음비가 낮아 RMSE 개선으로는 이어지지 않음.
    \end{itemize}
\end{itemize}

\subsubsection{ARX 모형: BSI와 전력거래량의 역할}

\begin{table}[H]
\centering
\label{tab:arx_bsi}
\begin{tabular}{lrrr}
\toprule
변수 & 계수 추정치 & 표준오차 & t값 \\
\midrule
상수항          & -1.699        & 1.906 & -0.89 \\
$y_{t-1}$       & -0.403$^{***}$& 0.071 & -5.67 \\
$\text{pw}_{t-1,w1}$     &  0.000        & 0.000 &  0.21 \\
$\text{BSI}_{t-1}$       & -0.076$^{*}$  & 0.035 & -2.20 \\
$\text{BSI}_{t-1}^{\text{YoY}}$ & -0.051$^{*}$  & 0.022 & -2.33 \\
$\text{pw}_{t,w1}$       &  0.000        & 0.000 & -0.21 \\
$\text{pw}_{t,w1}^{\text{YoY}}$ & -0.000       & 0.005 & -0.09 \\
$\text{BSI}_{t}$         &  0.096$^{**}$ & 0.035 &  2.76 \\
$\text{BSI}_{t}^{\text{YoY}}$   &  0.053$^{*}$  & 0.022 &  2.47 \\
\midrule
$R^{2}$       & \multicolumn{3}{r}{0.239} \\
조정 $R^{2}$  & \multicolumn{3}{r}{0.205} \\
관측치 수     & \multicolumn{3}{r}{188} \\
\bottomrule
\end{tabular}
\begin{flushleft}
\caption{ARX 모형 추정 결과}
\footnotesize
\textit{주}: 종속변수는 전산업생산지수 월별 성장률($y_t$)이며,
$y_{t-1}$은 1기 시차, $\text{pw}$는 월별(또는 주별) 전력거래량 관련 변수,
$\text{BSI}$는 기업경기실사지수, ``YoY''는 전년동월 대비 변화를 의미한다.
각 열은 순서대로 계수 추정치, 표준오차, t값을 포함한다.
유의수준: \textit{***} $p<0.01$, \textit{**} $p<0.05$, \textit{*} $p<0.10$.
\end{flushleft}
\end{table}

\begin{itemize}
  \item $y_{t-1}$:
    \begin{itemize}
      \item 계수 약 $-0.40$, 1\% 유의수준에서 유의.
      \item 기본적인 평균회귀 패턴 반영.
    \end{itemize}
  \item 전력거래량 변수:
    \begin{itemize}
      \item 계수 크기 매우 작고 모두 비유의.
      \item 본 사양에서는 추가 설명력 제한적.
    \end{itemize}
  \item BSI 관련 변수:
    \begin{itemize}
      \item $t-1$, $t$ 시점의 BSI 수준 및 전년동월비 모두 5\% 수준 내에서 유의.
      \item 경기국면 및 실물활동에 대한 선행·동행 정보 제공.
      \item AR(1) 대비 추가적인 예측력의 상당 부분을 BSI가 제공하는 것으로 해석 가능.
    \end{itemize}
\end{itemize}

\subsection{종속변수: 전년동월비}

\subsubsection{예측 성능: RMSE 및 상대 개선율}

\begin{table}[H]
\centering
\begin{tabular}{lcccc}
\toprule
Vintage & AR(1) & ARX (linear) & AR(1)+XGB\_residual & XGB-direct \\
\midrule
h0 & 1.49 (0.0)         & 1.51 (-1.6)        & 1.52 (-2.3)        & \textbf{1.42} (4.4) \\
h1 & \textbf{1.48} (0.0)& 1.58 (-6.4)        & 1.55 (-4.3)        & 1.61 (-8.6)         \\
h2 & \textbf{1.48} (0.0)& 1.58 (-6.4)        & 1.58 (-6.7)        & 1.55 (-4.3)         \\
h3 & \textbf{1.48} (0.0)& 1.58 (-6.4)        & 1.55 (-4.3)        & 1.58 (-6.7)         \\
h4 & \textbf{1.48} (0.0)& 1.53 (-2.9)        & 1.53 (-2.9)        & 1.52 (-2.3)         \\
\bottomrule
\end{tabular}
\begin{flushleft}
\caption{테스트 RMSE: 전년동월비 (XGBoost)}
\footnotesize\textit{주}: 각 셀은 2023–2024년 테스트 구간에서의 RMSE와, 괄호 안의 AR(1) 대비 RMSE 감소율(\%)을 함께 보고한다. 감소율은 $100 \times (1 - \mathrm{RMSE}_{m,h}/\mathrm{RMSE}_{\mathrm{AR(1)},h})$로 정의되며, 양수 값은 동일한 vintage에서 AR(1) 모형보다 예측 오차가 작다는 것을 의미한다.
\end{flushleft}
\label{tab:xgb_rmse_yoy}
\end{table}

\begin{itemize}
  \item \textbf{벤치마크(AR(1)) 성능}
  \begin{itemize}
    \item 모든 vintage에서 AR(1)의 테스트 RMSE는 약 1.48--1.49 수준으로 거의 동일함.
    \item 전년동월비 기준에서도 단순 AR(1)이 안정적인 기준 예측력을 제공함.
  \end{itemize}

  \item \textbf{ARX (linear)}
  \begin{itemize}
    \item 모든 vintage에서 AR(1) 대비 RMSE 감소율이 음수(약 $-1.6\%\sim -6.4\%$)로, 선형으로 고빈도 변수를 추가하면 예측력이 일관되게 악화됨.
    \item 특히 $h1$--$h3$에서는 약 $6\%$ 수준의 성능 저하가 반복적으로 관찰됨.
  \end{itemize}

  \item \textbf{AR(1)+XGB\_residual}
  \begin{itemize}
    \item 모든 vintage에서 감소율이 약 $-2.3\%\sim -6.7\%$로, AR(1)보다 RMSE가 항상 더 큼.
    \item AR(1)이 이미 설명한 구조 위에 잔차에 대해 부스팅을 적용하면, 전년동월비 기준에서는 잔차의 노이즈를 과적합하는 경향이 강함.
  \end{itemize}

  \item \textbf{XGB-direct}
  \begin{itemize}
    \item $h0$에서만 AR(1) 대비 약 $4.4\%$의 RMSE 감소가 관찰되며, 전월까지의 정보만 사용 가능한 시점에서는 비선형 모형이 일부 개선 효과를 보임.
    \item $h1$--$h4$에서는 감소율이 모두 음수(약 $-2.3\%\sim -8.6\%$)로 나타나, 고빈도 정보를 비선형으로 활용하더라도 AR(1)보다 예측오차가 더 큼.
  \end{itemize}

  \item \textbf{종합}
  \begin{itemize}
    \item 전년동월비를 종속변수로 사용한 경우, 대부분의 vintage에서 AR(1) 벤치마크가 가장 안정적인 선택으로 나타남.
    \item $h0$에서의 XGB-direct만 제한적인 개선을 보이며, 그 외의 경우에는 고빈도 변수의 선형·비선형 확장이 예측력을 유의미하게 개선하지 못함.
  \end{itemize}
\end{itemize}

\subsubsection{시각적 비교와 feature importance}

\paragraph{테스트 예측 경로}

\begin{figure}
    \centering
    \includegraphics[width=\textwidth]{midas/images/xgboost_test_mom_plot.png}
    \caption{테스트 기간 예측 경로: 전년동월비 (XGBoost)}
    \label{fig:xgboost_test_mom_plots}
\end{figure}

\begin{itemize}
    \item AR(1)+XGB\_residual 모형은 일부 시점에서 진폭을 과도하게 확대하여 실제값 주변의 노이즈까지 추종하는 현상이 관찰되며, 이는 앞서 확인된 RMSE 결과의 성능 악화와 일치하는 패턴임.
    \item XGB-direct 모형은 일부 vintage에서 국지적으로 실제값에 더 근접하는 구간이 있으나, 전반적인 궤적은 여전히 AR(1)과 유사하며 전체 RMSE 관점에서는 뚜렷한 우위를 보이지 않음.
    \item 종합적으로, 전년동월비 기준 테스트 구간에서도 고빈도 변수를 활용한 비선형 확장이 AR(1) 벤치마크 대비 예측 경로를 체계적으로 개선하지 못함.
\end{itemize}

\paragraph{Feature importance (Gain 기준)}

\begin{figure}[H]
    \centering
    \includegraphics[width=\textwidth]{midas/images/heatmap_mom.png}
    \caption{변수 중요도 히트맵: 전년동월비}
    \label{fig:xgb-imp-heatmap_iip}
\end{figure}
\begin{itemize}
  \item XGB-direct:
    \begin{itemize}
      \item 모든 vintage에서 1기 시차 종속변수 $y_{t-1}$의 중요도가
            가장 높음.
      \item XGBoost가 기본적으로 강한 자기회귀 구조를 먼저 학습하고 있음을 시사.
      \item 다음으로는 전월 BSI 수준(\texttt{bsi\_tm1})과 전월 BSI 전년동월비(\texttt{bsi\_tm1\_yoy})의 중요도가 상대적으로 크게 나타나, 경기심리지표가 보조적인 예측 정보를 제공하고 있음을 보여줌
      \item 현재 월 주별 전력거래량 및 YoY 변수(\texttt{pw\_t\_w1},
            \texttt{pwyoy\_t\_w1} 등)는 일부 vintage에서만 제한적인
            중요도를 가지며, 전체적으로는 기여도가 크지 않음.
    \end{itemize}
  \item AR(1)+XGB\_residual:
    \begin{itemize}
      \item 잔차를 설명하는 단계에서는 전월 BSI 전년동월비 (\texttt{bsi\_tm1\_yoy})와 전월 1주 전력거래량 (\texttt{pw\_tm1\_w1})의 중요도가 상대적으로 크게 나타남.
      \item 이는 AR(1)이 설명하지 못한 부분이 전월 경기심리의 변화와 전월 초 전력 사용 변화와 연관되어 있음을 시사하지만, 잔차의 신호대잡음비가 낮아 전체 예측 RMSE 개선으로 이어지지는 않음.
      \item 동월 BSI(\texttt{bsi\_t}, \texttt{bsi\_t\_yoy})는 주로 모든 정보가 이용 가능한 $h4$에서만 의미 있는 중요도를 보임.
    \end{itemize}
  \item 전반적으로, XGBoost 모형에서 가장 핵심적인 설명변수는 종속변수의 래그값과 BSI이며, 주별 전력거래량 변수의 한계적 기여는 제한적인 것으로 해석됨.
\end{itemize}


\subsubsection{ARX 모형: BSI와 전력거래량의 역할}

\begin{table}[H]
\centering
\begin{tabular}{lrrr}
\toprule
변수 & 계수 추정치 & 표준오차 & t값 \\
\midrule
상수항                    & -11.587$^{***}$ &  2.754 & -4.21 \\
$y_{t-1}$                 &   0.398$^{***}$ &  0.066 &  6.07 \\
$\text{pw}_{t-1,w1}$      &  -0.000         &  0.000 & -0.45 \\
$\text{BSI}_{t-1}$        &   0.113$^{*}$   &  0.046 &  2.48 \\
$\text{BSI}_{t-1}^{\text{YoY}}$
                          &  -0.076$^{**}$  &  0.029 & -2.64 \\
$\text{pw}_{t,w1}$        &   0.000         &  0.000 &  1.58 \\
$\text{pw}_{t,w1}^{\text{YoY}}$
                          &  -0.008         &  0.006 & -1.28 \\
$\text{BSI}_{t}$          &   0.004         &  0.045 &  0.08 \\
$\text{BSI}_{t}^{\text{YoY}}$
                          &   0.123$^{***}$ &  0.028 &  4.40 \\
\midrule
$R^{2}$       & \multicolumn{3}{r}{0.788} \\
조정 $R^{2}$  & \multicolumn{3}{r}{0.779} \\
관측치 수     & \multicolumn{3}{r}{188}   \\
\bottomrule
\end{tabular}
\caption{ARX 모형 추정 결과}
\begin{flushleft}
\footnotesize
\textit{주}: 종속변수는 전산업생산지수 월별 성장률($y_t$)이며,
$y_{t-1}$은 1기 시차, $\text{pw}$는 전력거래량 관련 변수,
$\text{BSI}$는 기업경기실사지수, ``YoY''는 전년동월 대비 변화를 의미한다.
각 열은 순서대로 계수 추정치, 표준오차, t값을 포함한다.
유의수준: \textit{***} $p<0.01$, \textit{**} $p<0.05$, \textit{*} $p<0.10$.
\end{flushleft}
\label{tab:arx_mom}
\end{table}

\begin{itemize}
  \item $y_{t-1}$:
    \begin{itemize}
      \item 계수 약 $0.40$, 1\% 유의수준에서 매우 유의.
      \item 전년동월비가 상당한 자기상관(지속성)을 갖고 있음을 보여줌.
    \end{itemize}

  \item 전력거래량 변수:
    \begin{itemize}
      \item 전월·당월 1주 전력 및 그 전년동월비 계수는 모두 0에 가깝고 비유의.
      \item 이 사양에서 전력거래량만으로는 추가적인 설명력이 거의 없는 것으로 나타남.
    \end{itemize}

  \item BSI 관련 변수:
    \begin{itemize}
      \item 전월 BSI 수준($\text{BSI}_{t-1}$)과 전월 BSI 전년동월비($\text{BSI}^\text{YoY}_{t-1}$)는 각각 10\%, 5\% 수준에서 유의하며, 과거 경기심리가 향후 전년동월비 변화에 정보를 제공함을 시사.
      \item 동월 BSI 수준($\text{BSI}_{t}$)은 비유의이지만, 동월 BSI 전년동월비($\text{BSI}^\text{YoY}_{t}$)는 1\% 수준에서 유의한 양(+)의 계수:
      \item 전반적으로, AR(1)에서 포착하지 못한 부분 중 상당 부분은 BSI 수준·증감에 의해 설명되고, 전력 변수의 추가 기여는 제한적인 것으로 해석 가능.
    \end{itemize}

  \item 적합도:
    \begin{itemize}
      \item $R^{2}\approx 0.79$, 조정 $R^{2}\approx 0.78$로 비교적 높은 설명력.
      \item 다만 앞선 테스트 RMSE 결과와 함께 볼 때, 높은 인샘플 적합에 비해 out-of-sample 예측력 개선은 제한적이라는 점을 시사.
    \end{itemize}
\end{itemize}

\section{요약: 변수 선택 관점에서 본 결과}

\begin{itemize}
    \item \textbf{전월대비 성장률 기준}
        \begin{itemize}
            \item 전력거래량
            \begin{itemize}
                \item 시간대별 자료를 로그–STL 계절조정 후 주간 로그 성장률로 변환.
                \item (t-1)월, t월 주별 성장률 및 전년동월비를 MIDAS 가중합·ARX·XGB feature로 사용.
                \item 대부분의 모형에서 계수·Gain 모두 작고 비유의, AR(1) 대비 RMSE 개선 거의 없음.
            \end{itemize}
            \item BSI
            \begin{itemize}
                \item 수준(\texttt{bsi\_tm1}, \texttt{bsi\_t})과 전년동월비(\texttt{bsi\_tm1\_yoy}, \texttt{bsi\_t\_yoy})로 분해해 사용.
                \item 선형 ARX와 XGBoost에서 전월·동월 BSI 변수들의 계수 및 Gain이 상대적으로 큼.
                \item 다만 전월대비 기준 out-of-sample RMSE는 AR(1) 대비 소폭·불안정한 개선에 그침.
            \end{itemize}
        \end{itemize}

    \item \textbf{전년동월비 기준}
    \begin{itemize}
        \item 전력거래량
        \begin{itemize}
            \item 동일한 변환(로그–STL 계절조정 후 주간 성장률·전년동월비)을 적용해 feature 구성.
            \item ARX, AR(1)+XGB\_residual, XGB-direct에서 중요도 낮고, 대부분 vintage에서 AR(1)보다 RMSE 악화.
            \item 예외적으로 $h0$–XGB-direct만 약 4\% 수준의 RMSE 개선, 다른 vintage로 일반화되지 않음.
        \end{itemize}
        \item BSI
        \begin{itemize}
            \item 전월 수준/전년동월비, 동월 전년동월비가 유의한 계수 및 높은 Gain을 보임.
            \item 과거·현재 경기심리 변화가 전년동월 기준 실물 변동에 일정 정보 제공.
            \item 그럼에도 대부분 vintage에서 AR(1)이 여전히 가장 안정적인 예측 성능 유지.
        \end{itemize}
    \end{itemize}

    \item \textbf{종합}
    \begin{itemize}
        \item 두 종속변수 모두에서 가장 일관된 설명력은 1기 시차 종속변수 $y_{t-1}$에서 나옴.
        \item 전력거래량 고빈도 변수는 다양한 변환·모형에도 불구하고 한계적 기여(보조 변수 수준)에 머묾.
        \item BSI 수준·전년동월비는 인샘플 적합과 feature importance 관점에서 의미 있는 정보 제공.
        \item 그러나 테스트 RMSE 기준으로는 AR(1) 벤치마크 대비 뚜렷한·일관된 예측력 개선까지는 이어지지 않음.
        \item 현재 변환 설정(주별 전력 성장률, BSI 수준·전년동월비)을 기준으로 볼 때, nowcasting에서 핵심 변수는 $y_{t-1}$과 BSI 계열 변수, 전력 변수는 부차적 설명 변수로 정리됨.
    \end{itemize}
    \end{itemize}

