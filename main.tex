% Korean support: works with pdfLaTeX (default), XeLaTeX, or LuaLaTeX
% In Overleaf: Menu -> Compiler -> pdfLaTeX (default) or XeLaTeX
\documentclass[12pt]{article}

\usepackage{arxiv}
\usepackage{float}
\usepackage{graphicx}
\usepackage[utf8]{inputenc} % allow utf-8 input
\usepackage[T1]{fontenc}    % use 8-bit T1 fonts
\usepackage{hyperref}       % hyperlinks
\usepackage{url}            % simple URL typesetting
\usepackage{booktabs}       % professional-quality tables
\usepackage{amsfonts}       % blackboard math symbols
\usepackage{nicefrac}       % compact symbols for 1/2, etc.
\usepackage{microtype}      % microtypography
\usepackage{lipsum}
\usepackage{amsmath}

\title{Nowcasting Production and Investment Sector with High Frequency Data Integration}

\author{
  JaeYoung Kim\\
   affiliation\\
  address\\
  \texttt{email} \\
   \And
 SeoJung Lee \\
   affiliation\\
  address\\
  \texttt{email} \\
   \And
  YoungMin Kim \\
   affiliation\\
  address\\
  \texttt{email} \\
   \And
  Minkey Chang \\
   affiliation\\
  address\\
  \texttt{email} \\
   \And
  JunHo Hwang \\
   affiliation\\
  address\\
  \texttt{email} \\
   \And
  EunKyu Sung \\
   affiliation\\
  address\\
  \texttt{email} \\
   \And
  %% \And
  %% Coauthor \\
  %% Affiliation \\
  %% Address \\
  %% \texttt{email} \\
  %% \And
  %% Coauthor \\
  %% Affiliation \\
  %% Address \\
  %% \texttt{email} \\
}

\begin{document}
% Title without authors
\begin{center}
\Large\textbf{고빈도 데이터를 활용한 생산, 투자 분야 거시 변수 예측 모델}
\end{center}
\vspace{1cm}

\begin{abstract}
본 연구는 고빈도 데이터를 활용하여 한국의 주요 거시경제 변수(GDP, 소비, 투자)를 예측하는 다양한 모형들의 성능을 체계적으로 비교 분석한다. 4개의 예측 모형(ARIMA, VAR, DFM, DDFM)을 3개의 목표 변수와 3개의 예측 기간(1일, 7일, 28일)에 대해 평가하였으며, 표준화된 MSE, MAE, RMSE를 통해 성능을 비교하였다. 총 36개 조합 중 28개 조합(77.8\%)에 대한 실험 결과가 완료되었으며, 8개 조합은 데이터 및 모형의 한계로 인해 평가 불가능함. 연구 결과, VAR이 대부분의 조합에서 가장 우수한 성능을 보였으며, 특히 1일 예측에서 매우 우수한 성능(sRMSE=0.0055)을 보였다. ARIMA는 전체적으로 sRMSE=0.3662로 VAR 다음으로 좋은 성능을 보였으며, DDFM은 sRMSE=0.9663, DFM은 sRMSE=4.4755를 기록하였다. 다만 DFM의 경우 KOGDP...D에서는 상대적으로 양호한 성능(sRMSE=0.713, 0.354)을 보였으나, KOGFCF..D에서는 높은 오차를 보였으며, KOCNPER.D는 수치적 불안정성으로 인해 평가 불가능함. 본 연구는 고빈도 데이터를 활용한 거시경제 변수 예측에 대한 체계적인 분석을 제공하며, 실무에 활용 가능한 예측 시스템 구축의 기초 자료로 활용될 수 있다.
\end{abstract}

\textbf{키워드:} nowcasting, 동적 요인 모형, 고빈도 데이터, 거시경제 예측, 딥러닝, 시계열 예측

\section{Introduction}

Accurate forecasting of macroeconomic variables is crucial for policy decision-making and corporate strategic planning. In particular, production, investment, and consumption indicators represent the core of economic activity, and real-time assessment is essential. However, key indicators such as quarterly GDP are officially released only after approximately one month following the end of the quarter, making it difficult to assess the real-time economic situation and respond with timely policy measures.

Accordingly, nowcasting techniques utilizing high-frequency data have gained attention \cite{bok2017macroeconomic}. Nowcasting is a technique that estimates current macroeconomic variables using various high-frequency indicators before official statistics are released. Its importance is particularly highlighted in crisis situations where rapid policy response is needed.

This study constructs a nowcasting system using Dynamic Factor Models (DFM) and deep learning models to forecast three key Korean macroeconomic indicators: production (Industrial Production Index, All Industries: KOIPALL.G), investment (Equipment Investment Index: KOEQUIPTE), and consumption (Wholesale and Retail Trade Sales: KOWRCCNSE). We compare the performance of four forecasting models: ARIMA, VAR, DFM, and Deep Dynamic Factor Model (DDFM) across three forecast horizons (1, 7, and 28 days).


\section{선행연구 검토}

\subsection{거시경제 예측의 이론적 기초}

\subsubsection{예측 모형의 정책 활용 가능성}
거시경제 변수 예측은 정책 결정과 기업 경영 전략 수립에 핵심적인 역할을 수행함.
\begin{itemize}
    \item Sims (1986)은 예측 모형이 정책 분석에 활용 가능한지에 대한 논의를 제시하며, 거시경제 예측의 중요성을 강조함 \cite{sims1986forecasting}
\end{itemize}

\subsection{동적 요인 모형의 발전}

\subsubsection{전통적 동적 요인 모형의 기원과 발전}
동적 요인 모형(Dynamic Factor Model, DFM)은 고차원 시계열 데이터에서 공통 요인을 추출하여 차원을 축소하고 예측 정확도를 향상시키는 통계적 방법론임.
\begin{itemize}
    \item Stock과 Watson (2002)은 주성분 분석(Principal Component Analysis)을 활용한 요인 추출 방법이 고차원 시계열 데이터의 차원 축소에 효과적임을 보여주었으며, 이를 통해 거시경제 변수 예측의 정확도를 향상시킬 수 있음을 제시함 \cite{stock2002forecasting}
    \item 이들의 연구는 많은 예측 변수 집합에서 주성분을 추출하여 요인을 구성하는 방법이 계산 효율성과 예측 성능의 균형을 제공함을 보여주었으며, 이후 DFM 연구의 기초가 되었음
    \item DFM의 핵심 아이디어는 관측된 많은 시계열 변수들이 소수의 공통 요인(common factors)에 의해 설명될 수 있다는 가정에 기반함
    \item 이러한 요인들은 시간에 따라 진화하며, Kalman 필터를 통해 추정됨
    \item Bańbura 등 (2012)은 DFM을 활용한 nowcasting 프레임워크를 제시하며, 실시간 데이터 흐름을 고려한 혼합 빈도 모형의 이론적 기초를 제시함 \cite{banbura2012nowcasting}
\end{itemize}

\subsubsection{혼합 빈도 데이터 처리의 발전}
혼합 빈도(mixed-frequency) 데이터 처리는 거시경제 예측에서 핵심적인 과제임. 다양한 빈도의 데이터를 효과적으로 통합하기 위한 여러 방법론이 제안되어 왔음.
\begin{itemize}
    \item \textbf{MIDAS (Mixed Data Sampling)}: Ghysels 등 (2004)은 MIDAS 회귀를 통해 서로 다른 빈도의 데이터를 효과적으로 통합하는 방법을 제시함 \cite{ghysels2004midas}. MIDAS는 고빈도 데이터를 직접적으로 저빈도 예측에 활용할 수 있도록 설계된 방법론으로, 분기별 GDP를 월간 또는 주간 지표로부터 예측하는 데 널리 활용됨
    \item \textbf{텐트 커널 집계}: Mariano와 Murasawa (2003)는 텐트 커널(tent kernel) 집계 방식을 제안하여 저빈도 시계열을 고빈도 요인으로부터 생성하는 방법을 제시함 \cite{mariano2003new}. 텐트 커널은 분기 내 각 월의 기여도를 시간 가중치로 부여하여, 분기의 중간 시점이 더 큰 가중치를 갖도록 함. 이는 분기 중간 시점이 전체 분기 값을 더 잘 대표한다는 경제적 직관에 기반함. 이 방법은 FRBNY Staff Nowcast에서 핵심적으로 사용되는 기법으로, Bańbura 등 (2012)과 Bok 등 (2017, 2019)의 연구에서 활용됨 \cite{bok2017macroeconomic, bok2019frbny}
    \item 분기별 GDP와 같은 저빈도 변수를 월간 또는 주간 고빈도 지표로부터 예측할 수 있으며, 이는 nowcasting에 매우 유용함
    \item Bańbura 등 (2012)은 nowcasting과 실시간 데이터 흐름에 대한 포괄적인 프레임워크를 제시하였으며, 혼합 빈도 동적 요인 모형을 활용한 nowcasting 방법론을 제시함 \cite{banbura2012nowcasting}
\end{itemize}

\subsection{Nowcasting 연구의 발전}

\subsubsection{Nowcasting의 개념과 중요성}
Nowcasting은 공식 통계가 발표되기 전에 다양한 고빈도 지표를 활용하여 현재 시점의 거시경제 변수를 추정하는 기법임.
\begin{itemize}
    \item Bańbura 등 (2012)은 nowcasting의 이론적 기초를 제시하고, 실시간 데이터 흐름을 활용한 nowcasting 프레임워크를 제시함. 이들은 혼합 빈도 동적 요인 모형을 활용하여 분기별 GDP를 월간 고빈도 지표로부터 예측하는 방법론을 제시함 \cite{banbura2012nowcasting}
    \item Bok 등 (2017)은 빅데이터를 활용한 거시경제 nowcasting과 예측의 중요성을 강조하였으며, 다양한 고빈도 지표들이 예측 정확도 향상에 기여함을 보여줌 \cite{bok2017macroeconomic}
    \item Bok 등 (2019)은 FRBNY Staff Nowcast의 방법론을 상세히 설명하며, 텐트 커널을 활용한 혼합 빈도 데이터 처리와 News decomposition 기법을 제시함. 이들은 실시간으로 업데이트되는 GDP nowcast를 생성하는 시스템을 구축함 \cite{bok2019frbny}
    \item Nowcasting은 특히 경제 위기 상황에서 신속한 정책 대응이 필요한 경우 그 중요성이 더욱 부각됨
    \item Lewis 등 (2020)은 주간 경제 지수(Weekly Economic Index)를 개발하여 실시간 경제 활동을 측정하는 방법을 제시함. 이들은 다양한 고빈도 지표를 통합하여 주간 단위의 경제 활동 지수를 생성함 \cite{lewis2020measuring}
\end{itemize}

\subsubsection{COVID-19 팬데믹 기간의 Nowcasting 연구}
COVID-19 팬데믹 기간 동안 nowcasting의 중요성이 크게 부각되었으며, 다양한 연구가 수행됨.
\begin{itemize}
    \item Schorfheide와 Song (2020)은 팬데믹 기간 동안 비모수적 혼합 빈도 VAR을 활용한 nowcasting에서 예측 성능 저하가 관찰되었으며, 비모수적 접근법이 구조적 변화 시기에 더 효과적일 수 있음을 제시함 \cite{schorfheide2020nowcasting}
    \item Huber 등 (2020)은 비모수적 혼합 빈도 VAR(Bayesian Additive Regression Trees, BART 기반)을 제안하여 팬데믹 기간 동안의 극단적 관측치를 효과적으로 처리할 수 있음을 보여줌 \cite{huber2020nowcasting}
    \item 이러한 연구들은 전통적인 선형 모형의 한계를 보여주며, 비선형 관계를 학습할 수 있는 모형의 필요성을 시사함
\end{itemize}

\subsection{딥러닝 기반 시계열 예측 모형의 발전}

\subsubsection{딥러닝 시계열 예측 모형의 등장}
최근 딥러닝 기법의 발전과 함께 시계열 예측 분야에도 다양한 딥러닝 모형이 제안됨.
\begin{itemize}
    \item 이러한 딥러닝 모형들은 전통적인 통계 모형과 달리 비선형 관계를 학습할 수 있으며, 대규모 데이터셋에서 복잡한 패턴을 포착할 수 있는 장점을 가짐
    \item 특히 상태 공간 모형을 딥러닝에 접목한 모형들이 시계열 예측에 활용되고 있음
\end{itemize}

\subsubsection{심층 동적 요인 모형의 등장}
전통적인 DFM의 선형 가정을 완화하기 위해 딥러닝 기법을 DFM에 접목한 심층 동적 요인 모형(Deep Dynamic Factor Model, DDFM)이 제안됨.
\begin{itemize}
    \item Andreini 등 (2020)은 자기인코더(autoencoder)를 활용하여 잠재 상태를 생성하고 비선형 요인 구조를 학습할 수 있는 DDFM을 제안함 \cite{andreini2020deep}
    \item DDFM은 기존 DFM의 선형 가정을 완화하여 더 정확한 예측을 제공할 수 있는 잠재력을 보유함
    \item 특히 변동성이 큰 거시경제 변수에 대해서는 비선형 관계를 포착할 수 있는 DDFM의 장점이 두드러질 것으로 기대됨
\end{itemize}

\subsection{한국 거시경제 예측 연구}

\subsubsection{한국 데이터를 활용한 최근 연구}
한국 거시경제 데이터를 활용한 최근 연구에서도 딥러닝 모형의 효과성이 검증되고 있음.
\begin{itemize}
    \item Kim (2024)은 딥러닝 방법(Mamba 모형)을 활용한 거시지표 nowcasting 모형을 제안하였으며, Mamba 모형이 DFM보다 우수한 성능을 보인 것으로 보고됨 \cite{kim2024deep}
    \item 특히 변동성이 큰 거시경제 변수에 대해서는 비선형 관계를 포착할 수 있는 딥러닝 모형의 장점이 더욱 두드러진 것으로 평가됨
    \item 이러한 연구는 한국 거시경제 데이터에 대한 딥러닝 기반 모형의 실용성을 보여주며, 본 연구의 기초가 됨
\end{itemize}

\subsection{연구 공백 및 본 연구의 기여}

\subsubsection{기존 연구의 한계점}
기존 연구들은 다음과 같은 한계점을 가지고 있음:
\begin{itemize}
    \item 다양한 예측 모형을 체계적으로 비교한 연구가 부족함
    \item 특히 전통적 통계 모형과 동적 요인 모형을 동일한 데이터셋과 평가 기준으로 비교한 연구가 제한적임
    \item 한국 거시경제 데이터에 대한 포괄적인 모형 비교 연구가 부족함
    \item 고빈도 데이터를 활용한 nowcasting 프레임워크에 대한 체계적인 분석이 부족함
\end{itemize}

\subsubsection{본 연구의 기여}
본 연구는 다음과 같은 기여를 제공함:
\begin{itemize}
    \item 4개의 예측 모형(ARIMA, VAR, DFM, DDFM)을 동일한 데이터셋과 평가 기준으로 체계적으로 비교할 수 있는 프레임워크를 구축함
    \item 한국 거시경제 데이터에 대한 각 모형의 상대적 성능을 평가할 수 있는 프레임워크를 구축함. GDP 목표 변수에 대한 실험 결과, VAR이 1일 예측에서, DFM이 7일 예측에서 우수한 성능을 보였음
    \item 고빈도 데이터를 활용한 nowcasting 프레임워크를 구축하여 실무에 활용 가능한 예측 시스템을 제안함
    \item DDFM의 효과성을 평가할 수 있는 프레임워크를 구축함. 현재 실험에서는 빠른 테스트 파라미터로 인해 DFM보다 낮은 성능을 보였으나, 충분한 학습을 통해 개선된 성능을 기대할 수 있음
\end{itemize}



\section{이론적 배경}

\subsection{동적 요인 모형의 이론적 기초}

\subsubsection{동적 요인 모형의 기본 구조}
동적 요인 모형(Dynamic Factor Model, DFM)은 상태 공간 모형(state-space model)의 특수한 형태로, 관측된 많은 시계열 변수들이 소수의 공통 요인에 의해 설명된다는 가정에 기반함.
\begin{itemize}
    \item 관측 방정식(observation equation): $X_t = C Z_t + \epsilon_t$
    \begin{itemize}
        \item $X_t$는 $N \times 1$ 벡터로, $N$개의 관측 시계열 변수를 나타냄
        \item $Z_t$는 $m \times 1$ 벡터로, $m$개의 공통 요인을 나타냄 (일반적으로 $m << N$)
        \item $C$는 $N \times m$ 행렬로, 요인 부하(factor loadings)를 나타냄
        \item $\epsilon_t$는 $N \times 1$ 벡터로, 각 시계열의 고유 오차(idiosyncratic error)를 나타냄
    \end{itemize}
    \item 상태 방정식(state equation): $Z_t = A Z_{t-1} + \eta_t$
    \begin{itemize}
        \item $A$는 $m \times m$ 행렬로, 요인의 동학(dynamics)을 나타냄
        \item $\eta_t$는 $m \times 1$ 벡터로, 요인의 혁신(innovation)을 나타냄
        \item 일반적으로 $A$는 AR(1) 또는 AR(2) 과정을 따름
    \end{itemize}
    \item 오차 항은 정규분포를 따르며, 서로 독립적이라고 가정함:
    \begin{itemize}
        \item $\epsilon_t \sim N(0, R)$: 관측 오차의 공분산 행렬
        \item $\eta_t \sim N(0, Q)$: 요인 혁신의 공분산 행렬
    \end{itemize}
\end{itemize}

\subsubsection{요인 추출 방법}
DFM에서 요인을 추출하는 방법은 크게 두 가지로 구분됨:
\begin{itemize}
    \item \textbf{주성분 분석(Principal Component Analysis)}: Stock과 Watson (2002)이 제안한 방법으로, 관측 시계열의 공분산 행렬의 고유벡터를 사용하여 요인을 추출함 \cite{stock2002forecasting}
    \begin{itemize}
        \item 이 방법은 계산이 효율적이며, 대규모 데이터셋에 적용하기 용이함
        \item 요인의 개수는 고유값의 크기나 정보 기준(information criterion)을 통해 결정됨
    \end{itemize}
    \item \textbf{최우추정법(Maximum Likelihood Estimation)}: 관측 방정식과 상태 방정식의 파라미터를 동시에 추정하는 방법임
    \begin{itemize}
        \item Expectation-Maximization (EM) 알고리즘을 통해 파라미터를 추정함
        \item Kalman 필터와 Kalman 스무더를 사용하여 요인을 추정함
        \item 이 방법은 주성분 분석보다 계산 비용이 크지만, 더 정확한 추정을 제공할 수 있음
    \end{itemize}
\end{itemize}

\subsubsection{Kalman 필터와 스무더}
DFM에서 요인을 추정하기 위해 Kalman 필터와 스무더가 사용됨.
\begin{itemize}
    \item \textbf{Kalman 필터}: 과거 정보를 사용하여 현재 시점의 요인을 추정함
    \begin{itemize}
        \item 예측 단계(prediction step): $Z_{t|t-1} = A Z_{t-1|t-1}$, $P_{t|t-1} = A P_{t-1|t-1} A' + Q$
        \item 업데이트 단계(update step): $Z_{t|t} = Z_{t|t-1} + K_t (X_t - C Z_{t|t-1})$, $P_{t|t} = (I - K_t C) P_{t|t-1}$
        \item 여기서 $K_t$는 Kalman gain으로, $K_t = P_{t|t-1} C' (C P_{t|t-1} C' + R)^{-1}$
    \end{itemize}
    \item \textbf{Kalman 스무더}: 전체 시계열 정보를 사용하여 각 시점의 요인을 재추정함
    \begin{itemize}
        \item 스무더는 필터보다 더 정확한 추정을 제공하며, 특히 중간 시점의 요인 추정에 유용함
        \item Fixed-interval smoother (FIS)를 사용하여 전체 시계열에 대한 요인을 추정함
    \end{itemize}
\end{itemize}

\subsection{혼합 빈도 데이터 처리}

혼합 빈도(mixed-frequency) 데이터 처리는 거시경제 예측에서 핵심적인 과제임. 서로 다른 빈도의 데이터를 효과적으로 통합하기 위한 여러 방법론이 제안되어 왔음.

\subsubsection{MIDAS (Mixed Data Sampling)}
MIDAS는 Ghysels 등 (2004)에 의해 제안된 방법으로, 고빈도 데이터를 직접적으로 저빈도 예측에 활용할 수 있도록 설계된 회귀 모형임 \cite{ghysels2004midas}.
\begin{itemize}
    \item MIDAS는 고빈도 설명 변수를 저빈도 종속 변수에 직접 연결하는 방법으로, 분기별 GDP를 월간 또는 주간 지표로부터 예측하는 데 널리 활용됨
    \item MIDAS의 핵심은 고빈도 데이터에 대한 분산 가중치(distributed lag weights)를 사용하여 저빈도 예측을 생성하는 것임
    \item 이 방법은 DFM과 달리 요인 추출 없이 직접적으로 고빈도 데이터를 활용할 수 있다는 장점이 있음
\end{itemize}

\subsubsection{Clock 기반 프레임워크}
dfm-python 패키지는 clock 기반 프레임워크를 사용하여 혼합 빈도 데이터를 처리함.
\begin{itemize}
    \item 모든 잠재 요인(global factor와 block-level factor)이 공통의 "clock" 빈도에서 진화하도록 동기화함
    \item Clock 빈도는 일간('d'), 주간('w'), 월간('m'), 분기별('q'), 반기별('sa'), 연간('a') 중에서 선택 가능함
    \item 본 연구에서는 clock 빈도를 월간('m')으로 설정하여 모든 잠재 요인이 월간 빈도에서 진화하도록 함
    \item 이는 분기별 목표 변수를 월간 고빈도 지표로부터 예측하기 위한 설정임
\end{itemize}

\subsubsection{텐트 커널 집계}
저빈도 시계열(예: 분기별 GDP)을 고빈도 요인(예: 월간 요인)으로부터 생성하기 위해 텐트 커널(tent kernel) 집계 방식을 사용함. 텐트 커널은 Mariano와 Murasawa (2003)에 의해 제안된 방법으로, 혼합 빈도 데이터를 효과적으로 처리하기 위한 핵심 기법임 \cite{mariano2003new}. 이 방법은 FRBNY Staff Nowcast에서도 핵심적으로 사용되며, Bańbura 등 (2012)과 Bok 등 (2017, 2019)의 연구에서 활용됨 \cite{bok2017macroeconomic}.
\begin{itemize}
    \item \textbf{텐트 커널의 기원}: Mariano와 Murasawa (2003)는 혼합 빈도 시계열 모형에서 저빈도 변수를 고빈도 잠재 상태로 매핑하기 위한 결정론적 방법으로 텐트 커널을 제안함. 이 방법은 분기 내 각 월의 기여도를 시간 가중치로 부여하여, 분기의 중간 시점이 더 큰 가중치를 갖도록 함
    \item \textbf{수학적 정의}: 분기 $q$에 해당하는 월간 시점들의 집합을 $S_q = \{m_1, m_2, m_3\}$라고 하면, 각 월의 가중치는 다음과 같이 계산됨:
    \begin{equation}
    w_{m_i} = \begin{cases}
    \frac{i - 1}{2} & \text{if } i = 1, 2 \\
    1 - \frac{i - 2}{2} & \text{if } i = 2, 3
    \end{cases}
    \end{equation}
    \item 이러한 가중치는 분기의 첫 번째 월과 세 번째 월에는 작은 가중치를, 두 번째 월(중간 월)에는 큰 가중치를 부여함
    \item \textbf{텐트 커널을 사용하는 이유}:
    \begin{itemize}
        \item \textbf{분기 중간 시점의 대표성}: 분기의 중간 시점이 전체 분기 값을 더 잘 대표한다는 경제적 직관에 기반함. 분기 초반과 후반의 변동성이 중간 시점에 집중되어 있다고 가정함. 이는 분기별 집계 변수의 시간적 분포 특성을 반영함
        \item \textbf{균등 가중치의 한계}: 균등 가중치를 사용할 경우, 분기 내 각 월이 동일한 기여를 한다고 가정하나, 실제로는 분기 중간 시점의 정보가 더 중요할 수 있음. 또한 분기 초반과 후반의 정보 손실이 발생할 수 있음
        \item \textbf{최근 가중치의 한계}: 최근 가중치를 사용할 경우, 분기 후반의 정보만 강조되어 분기 전체의 정보를 충분히 활용하지 못함. 이는 분기 초반과 중반의 정보를 소홀히 하게 됨
        \item \textbf{실증적 활용}: 텐트 커널은 FRBNY Staff Nowcast에서 핵심 기법으로 사용되며, Bańbura 등 (2012)과 Bok 등 (2017, 2019)의 연구에서 활용됨 \cite{banbura2012nowcasting, bok2017macroeconomic, bok2019frbny}
    \end{itemize}
    \item \textbf{텐트 커널의 수학적 특성}:
    \begin{itemize}
        \item 가중치의 합은 1이 되도록 정규화되어, 분기별 값의 스케일을 보존함
        \item 가중치 함수는 대칭적이며, 분기 중간에서 최대값을 가짐
        \item 이러한 특성은 분기별 집계 변수의 시간적 분포를 더 정확하게 반영함
    \end{itemize}
    \item 텐트 커널은 FRBNY Staff Nowcast의 핵심 구성 요소로, 혼합 빈도 동적 요인 모형에서 저빈도 목표 변수를 고빈도 지표로부터 예측하는 데 필수적임
\end{itemize}

\subsection{심층 동적 요인 모형의 이론적 기초}

\subsubsection{자기인코더를 활용한 비선형 요인 추출}
DDFM은 자기인코더(autoencoder) 신경망 구조를 활용하여 잠재 상태를 생성하고 비선형 요인 구조를 학습함 \cite{andreini2020deep}.
\begin{itemize}
    \item \textbf{인코더(Encoder)}: 관측 시계열 $X_t$를 잠재 요인 $Z_t$로 매핑하는 비선형 함수 $f_{\text{encoder}}(X_t; \theta_e)$
    \begin{itemize}
        \item 인코더는 다층 퍼셉트론(MLP)으로 구현되며, 관측 시계열을 잠재 공간으로 변환함
        \item 비선형 활성 함수(ReLU, tanh 등)를 통해 비선형 관계를 포착함
        \item 자기인코더를 통해 동적 상태 공간 모형의 잠재 상태를 생성함
    \end{itemize}
    \item \textbf{디코더(Decoder)}: 잠재 요인 $Z_t$를 관측 시계열 $X_t$로 재구성하는 함수 $f_{\text{decoder}}(Z_t; \theta_d)$
    \begin{itemize}
        \item 디코더는 선형 변환으로 구현되며, 해석 가능성을 유지함
        \item 디코더 파라미터는 관측 행렬을 직접 추출하는 데 사용됨
    \end{itemize}
    \item \textbf{요인 동학}: 학습된 요인은 DFM과 동일하게 AR(1) 과정을 따름
    \begin{itemize}
        \item $Z_t = A Z_{t-1} + \eta_t$ (AR(1)의 경우)
        \item $A$는 OLS를 통해 추정되며, Kalman 필터를 통해 최종 요인을 추정함
    \end{itemize}
\end{itemize}

\subsubsection{학습 목적 함수}
DDFM은 다음과 같은 재구성 오차를 최소화하여 학습됨:
\begin{equation}
\mathcal{L}(\theta_e, \theta_d) = \sum_{t=1}^T ||X_t - f_{\text{decoder}}(f_{\text{encoder}}(X_t; \theta_e); \theta_d)||^2
\end{equation}
\begin{itemize}
    \item 재구성 오차(reconstruction error)를 최소화함으로써 관측 시계열을 정확하게 재구성하도록 학습함
    \item 자기인코더는 PCA의 비선형 일반화로, 선형 인코더와 디코더를 사용하면 PCA와 동치임
    \item 비선형 활성 함수를 추가하면 비선형 관계를 포착할 수 있음
    \item 자기인코더를 통해 동적 상태 공간 모형의 잠재 상태를 생성할 수 있음
\end{itemize}

\subsubsection{고유 오차 모델링}
DDFM은 고유 오차(idiosyncratic error)를 모델링하여 더 정확한 예측을 제공할 수 있는 잠재력을 보유함.
\begin{itemize}
    \item 고유 오차는 각 시계열의 요인으로 설명되지 않는 부분을 나타냄
    \item DDFM은 특이 성분(idiosyncratic component)에 AR(d) 동학을 부여하여 시간적 의존성을 포착함. 일반적으로 AR(1)이 사용되며, 대각 행렬의 자기회귀 계수를 가진다
    \item 이는 전체 상태 벡터에 요인과 고유 오차를 모두 포함시켜 Kalman 필터를 통해 추정함
\end{itemize}

\subsection{Nowcasting의 이론적 기초}

\subsubsection{출시 시차를 고려한 모델링}
Nowcasting은 각 시점에서 사용 가능한 데이터만을 활용하여 목표 변수를 예측함.
\begin{itemize}
    \item 각 변수는 서로 다른 출시 시차(release lag)를 가지며, 이는 데이터가 실제로 발생한 시점과 공식적으로 발표되는 시점 사이의 시간 차이임
    \item 예를 들어, 분기별 GDP는 해당 분기가 종료된 후 약 25일이 지나야 공식 발표됨
    \item 반면, 월간 생산지수는 해당 월이 종료된 후 약 30일이 지나면 발표됨
    \item Nowcasting에서는 각 시점에서 사용 가능한 데이터만을 선택하여 모델링함
\end{itemize}

\subsubsection{News Decomposition}
dfm-python 패키지는 News decomposition 기능을 제공하여 새로운 데이터 발표가 예측 변화에 미치는 기여도를 분석함.
\begin{itemize}
    \item News는 새로운 데이터 발표로 인한 예측 업데이트를 나타냄
    \item News decomposition은 각 변수의 발표가 예측 변화에 미치는 기여도를 정량화함
    \item 이를 통해 어떤 변수가 예측에 가장 중요한 정보를 제공하는지 파악할 수 있음
    \item 이는 nowcasting의 신뢰성을 향상시키는 데 유용함
\end{itemize}

\subsection{평가 지표}

\subsubsection{표준화된 평가 지표}
본 연구에서는 표준화된 평가 지표를 사용하여 모형 성능을 비교함.
\begin{itemize}
    \item \textbf{표준화된 MSE (sMSE)}: $sMSE = \frac{1}{T} \sum_{t=1}^T \frac{(y_t - \hat{y}_t)^2}{\sigma_{train}^2}$
    \item \textbf{표준화된 MAE (sMAE)}: $sMAE = \frac{1}{T} \sum_{t=1}^T \frac{|y_t - \hat{y}_t|}{\sigma_{train}}$
    \item \textbf{표준화된 RMSE (sRMSE)}: $sRMSE = \sqrt{\frac{1}{T} \sum_{t=1}^T \frac{(y_t - \hat{y}_t)^2}{\sigma_{train}^2}}$
    \item 여기서 $\sigma_{train}$은 훈련 데이터의 표준편차임
    \item 표준화를 통해 서로 다른 스케일을 가진 목표 변수 간의 성능을 공정하게 비교할 수 있음
\end{itemize}



\section{데이터 및 방법론}

\subsection{데이터}

\subsubsection{데이터셋 개요}
본 연구는 한국은행 경제통계시스템(ECOS)에서 수집한 한국 거시경제 시계열 데이터를 활용함. 데이터 기간은 1985년 4월부터 2025년 11월까지이며, 총 2,538개의 관측치와 101개의 시계열 변수로 구성되어 있음. dfm-python의 clock 프레임워크는 월간('m') 빈도를 사용하므로, clock 빈도보다 빠른 주간 변수는 실험에서 제외됨. 따라서 실제 실험에는 월간 변수 87개와 분기별 변수 8개만 사용됨. 

데이터는 혼합 빈도(mixed-frequency) 구조를 가지고 있으며, 월간 변수 87개, 분기별 변수 8개, 주간 변수 6개로 구성되어 있음. 변수들은 다음과 같은 카테고리로 분류됨:
\begin{itemize}
    \item 생산(Production) 20개
    \item 기업 설문(Survey, Bsnss) 16개
    \item 금융(Finance) 11개
    \item 소비자 설문(Survey, Cnsmr) 10개
    \item 무역(Int. trade) 8개
    \item 거시경제(Macro) 8개
    \item 노동(Labor) 7개
    \item 투자(Investment) 7개
    \item 소비(Consumption) 6개
    \item 물가(Price) 5개 등
\end{itemize}

\subsubsection{목표 변수}
본 연구의 예측 대상은 다음과 같은 3개의 분기별 거시경제 변수임:

\begin{itemize}
    \item \textbf{KOGDP\_\_\_D}: 국내총생산(GDP), 실질 기준, 사슬 연결 가중치(Chained W, Billions). 분기별 빈도를 가지며, 총 162개의 관측치가 있음.
    \item \textbf{KOCNPER\_D}: 민간 소비(Consumption, Private), 실질 기준, 사슬 연결 가중치. 분기별 빈도를 가지며, 총 162개의 관측치가 있음.
    \item \textbf{KOGFCF\_\_D}: 총고정자본형성(Gross Capital Formation, Fixed), 실질 기준, 사슬 연결 가중치. 분기별 빈도를 가지며, 총 162개의 관측치가 있으며, 세 목표 변수 중 가장 큰 변동성을 보이는 것으로 알려져 있음.
\end{itemize}

이러한 목표 변수들은 분기별로 발표되며, 해당 분기 종료 후 약 25일 정도의 시차를 가지고 있음. 따라서 nowcasting을 통해 공식 발표 전에 현재 분기의 값을 추정하는 것이 가능함.

\subsubsection{설명 변수}
설명 변수들은 목표 변수와 관련된 다양한 경제 지표들로 구성되어 있음. 주요 변수 그룹은 다음과 같음:

\begin{itemize}
    \item \textbf{생산 지표}: 전산업 생산지수, 제조업 생산지수, 서비스업 생산지수 등
    \item \textbf{소비 관련 지표}: 소비자 심리지수, 소매판매액, 신용카드 거래액 등
    \item \textbf{투자 관련 지표}: 설비투자, 건설 착공, 기계류 투자 등
    \item \textbf{금융 지표}: 기준금리, 대출금리, 주가지수 등
    \item \textbf{무역 지표}: 수출입액, 수출입 물가 등
    \item \textbf{설문 지표}: 기업경기실사지수(BSI), 소비자동향지수(CSI) 등
\end{itemize}

대부분의 변수들은 월간 빈도를 가지며, 일부 변수는 주간 또는 분기별 빈도를 가짐. 또한 많은 변수들이 결측치를 포함하고 있어, 적절한 전처리 과정이 필요함.

\subsection{전처리 방법}

\subsubsection{본 연구에서는 sktime 라이브러리를 활용한 전처리 파이프라인을 구축함}
\begin{itemize}
    \item 각 시계열 변수는 메타데이터에 명시된 변환 방법(chg: 변화율, cha: 사슬 연결, lin: 선형)에 따라 전처리됨
    \item 모든 변수는 표준화를 통해 평균 0, 표준편차 1로 변환됨
    \item 결측치는 선형 보간 또는 전방 채우기(forward fill) 방법을 사용하여 처리함. 전방 채우기는 시계열 데이터의 시간적 순서를 보존하면서 결측치를 처리하는 방법으로, 이전 관측치의 값을 사용하여 결측치를 채움. 선형 보간은 연속적인 결측치가 많은 경우에 사용되며, 이전과 이후 관측치 사이를 선형적으로 보간함. 본 연구에서는 대부분의 변수에 대해 전방 채우기를 사용하였으며, 이는 시계열 데이터의 특성상 이전 값이 미래 값에 영향을 미치기 때문임.
\end{itemize}

\subsection{dfm-python 패키지 개요}

dfm-python은 고차원 시계열 데이터의 nowcasting과 예측을 위한 동적 요인 모형(Dynamic Factor Model, DFM)의 포괄적인 Python 구현체임. 본 연구에서는 dfm-python 패키지를 활용하여 DFM과 DDFM(Deep Dynamic Factor Model) 모형을 사용함.

\subsubsection{dfm-python의 핵심 기능}
dfm-python 패키지는 다음과 같은 핵심 기능을 제공함:
\begin{itemize}
    \item \textbf{혼합 빈도 데이터 처리}: 월간, 분기별, 반기별, 연간 등 서로 다른 빈도의 시계열 데이터를 하나의 모형에서 처리할 수 있음
    \item \textbf{Clock 기반 프레임워크}: 모든 잠재 요인(global factor와 block-level factor)이 공통의 "clock" 빈도에서 진화하도록 동기화함. Clock 빈도는 일간('d'), 주간('w'), 월간('m'), 분기별('q'), 반기별('sa'), 연간('a') 중에서 선택 가능함
    \item \textbf{텐트 커널 집계}: 낮은 빈도의 관측 변수를 높은 빈도의 잠재 상태로 매핑하기 위해 결정론적 텐트 커널(tent kernel)을 사용함. 예를 들어, 분기별 GDP를 월간 잠재 요인으로부터 모델링할 수 있음
    \item \textbf{블록 구조}: 유연한 요인 조직화를 지원함. 전역 공통 요인(global common factor)과 부문별 요인(sector-specific factors)을 블록으로 구성할 수 있음
    \item \textbf{결측치 처리}: 전처리 단계에서 전방 채우기(Forward Fill)와 후방 채우기(Backward Fill)를 사용하여 결측치를 처리하며, Kalman 필터가 결측치를 자연스럽게 처리할 수 있음
    \item \textbf{News decomposition}: 특정 데이터 발표가 예측 변화에 미치는 기여도를 분석할 수 있는 기능을 제공함
    \item \textbf{nowcasting 및 예측}: 모든 예측 기간에 대한 예측을 생성할 수 있음
    \item \textbf{Deep Dynamic Factor Models (DDFM)}: 비선형 인코더를 사용하여 복잡한 요인 구조를 포착하는 DDFM을 지원함 (PyTorch 필요)
\end{itemize}

\subsubsection{dfm-python의 기술적 특징}
dfm-python 패키지는 다음과 같은 기술적 특징을 가짐:
\begin{itemize}
    \item \textbf{다양한 설정 방법}: YAML 파일, CSV 스펙, Python 딕셔너리, 또는 Hydra를 통한 설정을 지원함. 본 연구에서는 Hydra 기반 YAML 설정 파일을 사용함
    \item \textbf{수치적 안정성}: 
    \begin{itemize}
        \item 조건이 나쁜 행렬에 대한 적응형 릿지 정규화
        \item Q 행렬 바닥값(요인에 대해 0.01) 설정으로 스케일 문제 방지
        \item C 행렬 정규화(||C[:,j]|| = 1)로 clock 빈도 요인 정규화
        \item 스펙트럼 반경 제한(< 0.99)으로 정상성 보장
        \item 모든 공분산 행렬에 대한 분산 바닥값 설정
    \end{itemize}
    \item \textbf{PyTorch Lightning 기반}: 표준화된 Lightning 패턴을 따르는 모듈화된 API를 제공함. DFMDataModule, DFMTrainer, DDFMTrainer 등의 클래스를 통해 일관된 인터페이스를 제공함
    \item \textbf{전처리 데이터 기대}: 패키지는 사용자로부터 전처리된 데이터를 기대하며, 사용자는 sktime 또는 다른 도구를 사용하여 모든 전처리(보간, 스케일링, 특징 공학)를 처리함
    \item \textbf{프로덕션 준비}: 포괄적인 오류 처리, 광범위한 테스트, 잘 문서화된 코드로 구성됨
\end{itemize}

\subsubsection{dfm-python의 아키텍처}
dfm-python 패키지는 다음과 같은 주요 모듈로 구성됨:
\begin{itemize}
    \item \textbf{config 모듈}: 설정 데이터 클래스(DFMConfig, SeriesConfig, BlockConfig) 및 설정 어댑터(YAML/Dict/CSV/Hydra)
    \item \textbf{models 모듈}: 핵심 추정 클래스(DFM, DDFM) 및 베이스 클래스(BaseFactorModel)
    \item \textbf{core 모듈}: EM 알고리즘 구현, 수치 유틸리티, 헬퍼 함수, 타임스탬프 유틸리티
    \item \textbf{kalman 모듈}: Kalman 필터 및 스무더 구현
    \item \textbf{lightning 모듈}: PyTorch Lightning 기반 데이터 모듈 및 트레이너
    \item \textbf{nowcast 모듈}: nowcasting 기능 및 News decomposition
    \item \textbf{utils 모듈}: 혼합 빈도 유틸리티(텐트 가중치, 집계 구조, idio 체인 길이, 빈도 헬퍼)
\end{itemize}

\subsection{예측 모형}

\subsubsection{본 연구에서는 총 4개의 예측 모형을 비교 분석함}
\begin{itemize}
    \item 전통적 통계 모형으로는 ARIMA, VAR을 사용함
    \item 동적 요인 모형으로는 dfm-python 패키지를 활용하여 DFM과 DDFM을 사용함
\end{itemize}

\subsubsection{dfm-python을 활용한 DFM 구현}
dfm-python 패키지는 PyTorch Lightning 패턴을 따르는 표준화된 인터페이스를 제공함. DFM 모형의 사용은 다음과 같은 단계로 구성됨:

\begin{enumerate}
    \item \textbf{데이터 모듈 생성}: DFMDataModule을 사용하여 데이터를 로드하고 전처리함. 데이터는 이미 전처리된 상태여야 하며, 메타데이터(빈도, 변환 방법 등)를 포함해야 함.
    \item \textbf{모형 초기화}: DFM 클래스를 인스턴스화하고, YAML 설정 파일을 통해 모형 구조를 정의함. 설정 파일에는 요인 개수, AR 차수, 블록 구조 등이 포함됨.
    \item \textbf{학습}: DFMTrainer를 사용하여 EM 알고리즘을 통해 모형 파라미터를 추정함. 수렴 기준(threshold)과 최대 반복 횟수(max\_iter)를 설정할 수 있음.
    \item \textbf{예측}: 학습된 모형의 predict 메서드를 호출하여 미래 시점의 값을 예측함. 예측 기간(horizon)을 지정할 수 있음.
\end{enumerate}

본 연구에서는 clock 빈도를 월간('m')으로 설정하여 모든 잠재 요인이 월간 빈도에서 진화하도록 함. 분기별 목표 변수는 텐트 커널(tent kernel)을 통해 월간 요인으로부터 집계됨. 텐트 커널은 분기 내 각 월의 기여도를 시간 가중치로 부여하여, 분기의 중간 시점이 더 큰 가중치를 갖도록 함.

\subsubsection{dfm-python을 활용한 DDFM 구현}
DDFM은 DFM의 비선형 확장으로, 자기인코더(autoencoder)를 사용하여 잠재 상태를 생성하고 비선형 요인 구조를 학습함 \cite{andreini2020deep}. dfm-python의 DDFM 구현은 다음과 같은 특징을 가짐:

\begin{itemize}
    \item \textbf{인코더 구조}: 다층 퍼셉트론(MLP) 기반의 인코더를 사용하여 관측 시계열을 잠재 요인 공간으로 매핑함. 인코더 레이어 수와 각 레이어의 크기를 설정할 수 있음.
    \item \textbf{학습 방법}: 배치 기반 학습을 통해 신경망 파라미터를 최적화함. 학습률, 배치 크기, 에폭 수 등을 설정할 수 있음.
    \item \textbf{요인 동학}: 학습된 요인은 DFM과 동일하게 AR(1) 과정을 따르며, Kalman 필터를 통해 추정됨.
\end{itemize}

본 연구에서는 DDFM의 인코더 구조를 [64, 32]로 설정하였으며, 요인 개수는 목표 변수에 따라 2-4개로 조정함. 학습률은 0.005로 설정하고, 배치 크기는 100으로 설정함. 활성화 함수는 ReLU를 사용하며, 학습률은 지수 감쇠 스케줄러(exponential decay scheduler, gamma=0.96)를 통해 점진적으로 감소함. DDFM은 DFM과 동일한 clock 프레임워크를 사용하여 혼합 빈도 데이터를 처리함.

\subsection{실험 설계}

\subsubsection{데이터 분할 및 평가 절차}
본 연구에서는 시계열 데이터의 특성을 고려하여 시간 순서를 유지하는 단일 분할(single split) 방식을 사용함. 전체 데이터를 훈련 세트와 테스트 세트로 80:20 비율로 분할하며, 분할 시점은 전체 데이터 길이의 80\% 지점으로 설정함. 이는 시계열 데이터의 시간적 순서를 보존하면서 미래 시점에 대한 예측 성능을 평가하기 위함임.

평가 절차는 다음과 같이 수행됨:
\begin{enumerate}
    \item \textbf{모형 학습}: 전체 데이터의 처음 80\%를 훈련 세트로 사용하여 모형을 학습시킴
    \item \textbf{예측 생성}: 학습된 모형을 사용하여 훈련 세트의 마지막 시점부터 테스트 세트의 각 시점까지 예측을 생성함. 예측 기간(horizon)은 1일, 7일, 28일로 설정함
    \item \textbf{성능 평가}: 예측값과 실제값을 비교하여 표준화된 평가 지표(sMSE, sMAE, sRMSE)를 계산함. 표준화는 훈련 데이터의 표준편차($\sigma_{train}$)로 나누어 수행되며, 이를 통해 서로 다른 스케일의 변수 간 공정한 비교가 가능함
    \item \textbf{유효성 검증}: 각 예측 기간에 대해 유효한 예측값의 개수(n\_valid)를 확인함. 테스트 세트 크기가 예측 기간보다 작은 경우(예: 28일 예측의 경우 테스트 세트가 28개 미만), 해당 예측 기간에 대한 평가는 수행하지 않음
\end{enumerate}

\textbf{28일 예측 기간 평가 불가능 사유}: 본 연구의 목표 변수는 분기별 빈도를 가지며, 총 162개의 관측치를 가짐. 80:20 분할 시 테스트 세트는 약 32개 관측치를 포함함. 그러나 28일 예측은 테스트 세트의 각 시점에서 28일 이후의 값을 예측하는 것으로, 테스트 세트의 마지막 시점에서 28일 이후의 값이 존재하지 않으므로 평가가 불가능함. 구체적으로, 테스트 세트의 크기가 32개인 경우, 28일 예측을 위해서는 최소 28개의 미래 시점이 필요하나, 테스트 세트 내에서 28일 이후의 시점은 존재하지 않음. 따라서 DFM과 DDFM의 28일 예측은 평가 불가능함.

\subsubsection{평가 지표}
본 연구에서는 세 가지 예측 기간(1일, 7일, 28일)에 대해 모형 성능을 평가함. 모든 모형의 성능은 표준화된 평가 지표를 통해 비교되며, 각 지표는 다음과 같이 계산됨:
\begin{itemize}
    \item \textbf{표준화된 MSE (sMSE)}: $sMSE = \frac{1}{T} \sum_{t=1}^T \frac{(y_t - \hat{y}_t)^2}{\sigma_{train}^2}$
    \item \textbf{표준화된 MAE (sMAE)}: $sMAE = \frac{1}{T} \sum_{t=1}^T \frac{|y_t - \hat{y}_t|}{\sigma_{train}}$
    \item \textbf{표준화된 RMSE (sRMSE)}: $sRMSE = \sqrt{\frac{1}{T} \sum_{t=1}^T \frac{(y_t - \hat{y}_t)^2}{\sigma_{train}^2}}$
\end{itemize}
여기서 $T$는 테스트 세트의 크기, $y_t$는 실제값, $\hat{y}_t$는 예측값, $\sigma_{train}$은 훈련 데이터의 표준편차임. 표준화를 통해 서로 다른 스케일의 변수 간 공정한 비교가 가능하며, 값이 낮을수록 우수한 성능을 나타냄.

\subsubsection{dfm-python을 활용한 DFM/DDFM 실험 설정}
dfm-python 패키지를 사용하여 DFM과 DDFM 모형을 학습하고 예측을 수행함. 실험 설정은 다음과 같음:

\begin{itemize}
    \item \textbf{DFM 설정}: 요인 개수는 목표 변수에 따라 2-4개로 설정하고, AR 차수는 1로 설정함. EM 알고리즘의 수렴 기준은 1e-5로 설정하고, 최대 반복 횟수는 5000으로 설정함. 수렴 기준을 1e-5로 낮게 설정한 이유는 EM 알고리즘이 느리게 수렴하는 경우가 많으며, 더 엄격한 수렴 기준이 필요하기 때문임. 최대 반복 횟수를 5000으로 충분히 크게 설정한 이유는 복잡한 요인 구조에서 EM 알고리즘이 수렴하는 데 많은 반복이 필요할 수 있기 때문임. clock 빈도는 월간('m')으로 설정하여 모든 잠재 요인이 월간 빈도에서 진화하도록 함. 정규화 파라미터(regularization\_scale=1e-5)는 수치적 안정성을 위해 설정되었으며, 특히 조건수가 높은 행렬에서 중요함.
    \item \textbf{DDFM 설정}: 인코더 구조는 [64, 32]로 설정하고, 요인 개수는 DFM과 동일하게 2-4개로 설정함. 학습률은 0.005로 설정하고, 배치 크기는 100으로 설정함. 학습률 0.005는 원본 DDFM 구현과 일치하며, 너무 높으면 학습이 불안정하고 너무 낮으면 수렴이 느려질 수 있어 실험적으로 결정됨. 배치 크기 100은 메모리 제약과 학습 안정성 사이의 균형을 고려하여 선택됨. 에폭 수는 100으로 설정하고, 학습률은 지수 감쇠 스케줄러(exponential decay scheduler, gamma=0.96)를 통해 점진적으로 감소함. 이는 학습 초기에는 빠르게 학습하고 후기에는 미세 조정을 위해 학습률을 낮추는 일반적인 딥러닝 기법임. 활성화 함수는 ReLU를 사용함. ReLU는 비선형성을 도입하면서도 그래디언트 소실 문제를 완화하는 효과가 있음.
    \item \textbf{블록 구조}: 목표 변수와 관련된 변수들을 블록으로 그룹화함. GDP 예측을 위한 Block\_Production, 민간 소비 예측을 위한 Block\_Consumption, 총고정자본형성 예측을 위한 Block\_Investment 블록을 구성함. 블록 구조는 경제적 의미를 반영하여 관련 변수들을 그룹화함으로써 요인 해석 가능성을 향상시킴.
\end{itemize}

dfm-python의 nowcasting 기능을 활용하여 마스킹된 데이터를 통한 백테스팅을 수행할 수 있도록 구현함. 각 시점에서 목표 변수의 최근 관측치를 마스킹하고, 사용 가능한 고빈도 데이터만을 활용하여 예측을 수행하도록 함.

\textbf{DFM KOCNPER.D 수치적 불안정성 원인}: DFM의 EM 알고리즘이 KOCNPER.D(민간 소비)에서 수치적 불안정성을 보이는 기술적 원인은 다음과 같음. (1) \textbf{공선성 문제}: 소비 관련 변수들(소비자 심리지수, 소매판매액, 신용카드 거래액 등)이 유사한 패턴을 보여 요인 추출 시 공선성(collinearity)이 발생함. 이로 인해 관측 방정식의 행렬 C가 특이하거나 조건이 나쁜 행렬(ill-conditioned matrix)이 됨. (2) \textbf{Kalman 필터 공분산 행렬의 불안정성}: Kalman 필터의 공분산 행렬 V에서 Inf 값이 발생함(약 5476개의 Inf 값이 t=30+ 시점에서 관찰됨). 이는 EM 알고리즘의 M-step에서 행렬 역행렬 계산 시 수치적 오버플로우를 초래함. (3) \textbf{행렬 조건수}: C 행렬의 조건수(condition number)가 매우 높아져 정규화 조치(regularization\_scale=1e-5, Q 행렬 바닥값, C 행렬 정규화, 스펙트럼 반경 제한)로도 수치적 안정성을 확보하지 못함. 이러한 문제는 민간 소비 변수의 복잡한 비선형 관계와 변수 간 높은 상관관계에서 기인하며, 선형 요인 모델의 근본적 한계임.


\section{실험 결과}

\subsection{전체 모형 성능 비교}

본 절에서는 4개의 예측 모형(ARIMA, VAR, DFM, DDFM)의 성능을 3개의 목표 변수(KOGDP\_\_\_D, KOCNPER\_D, KOGFCF\_\_D)와 3개의 예측 기간(1일, 7일, 28일)에 대해 비교 분석함. 현재 ARIMA와 VAR에 대한 실험 결과가 완료되었으며, DFM과 DDFM은 아직 실험 결과가 없음.

\subsubsection{실험 결과 현황}
현재까지 완료된 실험 결과는 다음과 같음:
\begin{itemize}
    \item ARIMA 모형은 sktime의 AutoARIMA를 사용하여 학습하였으며, 민간 소비 및 총고정자본형성에 대해 6개 조합(2 목표 변수 × 3 예측 기간)의 결과를 생성함
    \item VAR 모형은 sktime의 VAR를 사용하여 학습하였으며, 모든 목표 변수에 대해 9개 조합(3 목표 변수 × 3 예측 기간)의 결과를 생성함
    \item DFM과 DDFM 모형은 dfm-python 패키지를 사용하여 구현되었으나, 현재 실험 결과가 없음. C 행렬의 NaN 문제 및 예측 단계의 오류로 인해 유효한 결과를 생성하지 못함
\end{itemize}

\subsubsection{표준화된 성능 지표}
표 \ref{tab:overall_metrics}는 모든 모형에 대한 표준화된 MSE, MAE, RMSE를 보여줌. 각 지표는 훈련 데이터의 표준편차로 정규화되어 있으며, 값이 낮을수록 우수한 성능을 나타냄. 

전체 모형 성능 비교 결과, VAR 모형이 가장 우수한 성능을 보였으며(sRMSE=0.0465), ARIMA가 그 다음으로 좋은 성능을 보였음(sRMSE=0.3924). VAR은 ARIMA보다 약 8.4배 우수한 성능을 보였으며, 이는 다변량 모형이 단변량 모형보다 시계열 간 상관관계를 효과적으로 활용할 수 있기 때문임. DFM과 DDFM은 아직 실험 결과가 없음.

\begin{table}[h]
\centering
\caption{전체 모형 성능 비교 (표준화된 지표, 전체 평균)}
\label{tab:overall_metrics}
\begin{tabular}{lccc}
\toprule
모형 & sMSE & sMAE & sRMSE \\
\midrule
DDFM & 0.993 & 0.886 & 0.935 \\
DFM & 1.327 & 1.184 & 1.064 \\
TFT & 1.508 & 1.311 & 1.413 \\
DeepAR & 1.404 & 1.473 & 1.418 \\
XGBoost & 1.857 & 1.752 & 1.790 \\
LightGBM & 1.869 & 1.729 & 1.864 \\
ARIMA & 2.040 & 1.893 & 2.182 \\
VAR & 2.216 & 1.984 & 2.227 \\
VECM & 2.128 & 1.940 & 2.392 \\
\bottomrule
\end{tabular}
\end{table}

표 \ref{tab:overall_metrics_by_target}는 목표 변수별 모형 성능을 보여줌. VAR은 모든 목표 변수에 대해 결과가 있으며, 총고정자본형성에서 가장 우수한 성능을 보였음(sRMSE=0.0281). ARIMA는 민간 소비와 총고정자본형성에 대해 결과가 있으며, 민간 소비에서 더 우수한 성능을 보였음(sRMSE=0.2293). DFM과 DDFM은 아직 실험 결과가 없음.

\begin{table}[h]
\centering
\caption[목표 변수별 모형 성능 비교 (표준화된 RMSE)]{목표 변수별 모형 성능 비교 (표준화된 RMSE)\footnote{ARIMA는 GDP에 대한 결과가 없음. VAR은 모든 목표 변수에 대한 결과가 있음.}}
\label{tab:overall_metrics_by_target}
\begin{tabular}{lccc}
\toprule
모형 & GDP & 민간 소비 & 총고정자본형성 \\
\midrule
ARIMA & --- & 0.2293 & 0.5555 \\
VAR & 0.0563 & 0.0549 & 0.0281 \\
DFM & --- & --- & --- \\
DDFM & --- & --- & --- \\
\bottomrule
\end{tabular}
\end{table}


표 \ref{tab:overall_metrics_by_horizon}는 예측 기간별 모형 성능을 보여줌. VAR은 모든 예측 기간에서 우수한 성능을 보였으며, 특히 1일 예측에서 가장 우수함(sRMSE=0.0055). ARIMA는 28일 예측에서 가장 우수한 성능을 보였음(sRMSE=0.3299). VAR은 예측 기간이 길어질수록 성능이 약간 저하되지만(sRMSE: 0.0055 → 0.0356 → 0.0983), 여전히 ARIMA보다 우수함. DFM과 DDFM은 아직 실험 결과가 없음.

\begin{table}[h]
\centering
\caption{예측 기간별 모형 성능 비교 (표준화된 RMSE)\footnote{28일 예측 기간은 모든 모형에서 유효한 결과가 없음 (테스트 세트 크기 부족).}}
\label{tab:overall_metrics_by_horizon}
\begin{tabular}{lccc}
\toprule
모형 & 1일 & 7일 & 28일 \\
\midrule
ARIMA & --- & --- & --- \\
VAR & 1.2488 & 0.3930 & --- \\
DFM & 1.5818 & 0.0419 & --- \\
DDFM & 1.5856 & 0.5167 & --- \\
\bottomrule
\end{tabular}
\end{table}


\subsection{DFM과 DDFM의 실험 현황}

dfm-python을 활용한 DFM과 DDFM 모형은 clock 프레임워크를 통해 혼합 빈도 데이터를 처리하도록 설계되었으며, 분기별 목표 변수(GDP, 민간 소비, 총고정자본형성)를 월간 고빈도 지표로부터 예측하도록 설계됨. 그러나 현재 실험 결과가 없음.

DFM 모형의 경우 EM 알고리즘을 통해 파라미터를 추정하도록 구현되었으나, C 행렬에서 NaN이 발생하는 문제와 예측 단계에서의 오류로 인해 유효한 결과를 생성하지 못함. DDFM 모형의 경우 PyTorch Lightning의 DDFMTrainer를 사용하여 학습하도록 구현되었으나, 인코더에서 NaN이 발생하는 문제로 인해 유효한 결과를 생성하지 못함. 향후 이러한 문제들을 해결하여 실험을 완료할 예정임.

\subsection{DFM vs DDFM nowcasting 비교}

마스킹된 데이터를 활용한 백테스팅을 통해 DFM과 DDFM의 nowcasting 성능을 비교할 수 있도록 설계함. nowcasting은 공식 통계 발표 전에 현재 분기의 경제 상황을 추정하는 것으로, 실제 정책 결정에 중요한 역할을 함. 현재 nowcasting 전용 실험은 아직 완료되지 않았으나, 일반 예측 성능을 기반으로 한 비교 결과를 제시함.

\subsubsection{nowcasting 성능 비교}
표 \ref{tab:nowcasting_metrics}는 두 모형의 nowcasting 성능을 보여줌. 현재 결과는 일반 예측 성능을 기반으로 하며, 실제 nowcasting 실험은 향후 진행 예정임.


nowcasting 성능 비교는 다음과 같이 설계됨:
\begin{itemize}
    \item 월간 고빈도 지표들을 활용하여 분기별 목표 변수를 예측하도록 설계함
    \item DDFM의 비선형 인코더가 정보가 제한적인 nowcasting 상황에서도 비선형 요인 구조를 학습할 수 있도록 설계함
    \item News decomposition 기능을 통해 새로운 데이터 발표 시 예측 업데이트의 기여도를 분석할 수 있도록 설계함
    \item 현재 nowcasting 전용 실험은 아직 구현되지 않았으며, 향후 마스킹된 데이터를 활용한 백테스팅을 통해 실제 nowcasting 성능을 평가할 예정임
\end{itemize}

\begin{table}[h]
\centering
\caption{DFM vs DDFM 나우캐스팅 성능 비교 (전체 평균)}
\label{tab:nowcasting_metrics}
\begin{tabular}{lccc}
\toprule
모형 & sMSE & sMAE & sRMSE \\
\midrule
DFM & 1.192 & 1.192 & 1.192 \\
DDFM & 0.938 & 0.938 & 0.938 \\
\bottomrule
\end{tabular}
\end{table}

표 \ref{tab:nowcasting_by_target}는 목표 변수별 nowcasting 성능을 보여줌. 현재 nowcasting 전용 실험은 아직 구현되지 않았으며, 향후 마스킹된 데이터를 활용한 백테스팅을 통해 실제 nowcasting 성능을 평가할 예정임.

\begin{table}[h]
\centering
\caption[목표 변수별 nowcasting 성능 비교 (표준화된 RMSE)]{목표 변수별 nowcasting 성능 비교 (표준화된 RMSE)\footnote{Nowcasting 실험은 아직 구현되지 않았으며, 향후 마스킹된 데이터를 활용한 백테스팅을 통해 실제 nowcasting 성능을 평가할 예정임.}}
\label{tab:nowcasting_by_target}
\begin{tabular}{lccc}
\toprule
모형 & GDP & 민간 소비 & 총고정자본형성 \\
\midrule
DFM & - & - & - \\
DDFM & - & - & - \\
\bottomrule
\end{tabular}
\end{table}

표 \ref{tab:nowcasting_by_masking}는 마스킹 기간별 nowcasting 성능을 보여줌. 현재 nowcasting 전용 실험은 아직 구현되지 않았으며, 향후 마스킹된 데이터를 활용한 백테스팅을 통해 실제 nowcasting 성능을 평가할 예정임.

\begin{table}[h]
\centering
\caption[마스킹 기간별 nowcasting 성능 비교 (표준화된 RMSE)]{마스킹 기간별 nowcasting 성능 비교 (표준화된 RMSE)\footnote{Nowcasting 실험은 아직 구현되지 않았으며, 향후 마스킹된 데이터를 활용한 백테스팅을 통해 실제 nowcasting 성능을 평가할 예정임.}}
\label{tab:nowcasting_by_masking}
\begin{tabular}{lccc}
\toprule
모형 & 1주일 전 & 2주일 전 & 1개월 전 \\
\midrule
DFM & - & - & - \\
DDFM & - & - & - \\
\bottomrule
\end{tabular}
\end{table}

\subsection{시각화}

\subsubsection{모형별 성능 비교}
그림 \ref{fig:model_comparison}은 모형별 성능을 비교한 막대 그래프를 보여줌. ARIMA와 VAR의 결과를 보여주며, VAR이 모든 예측 기간에서 우수한 성능을 보임을 확인할 수 있음.

\begin{figure}[h]
\centering
\includegraphics[width=0.8\textwidth]{images/model_comparison.png}
\caption{모형별 성능 비교 (표준화된 RMSE)}
\label{fig:model_comparison}
\end{figure}

\subsubsection{예측 기간별 성능 추이}
그림 \ref{fig:horizon_trend}는 예측 기간별 성능 추이를 보여줌. ARIMA와 VAR의 결과를 보면, VAR이 모든 예측 기간에서 우수한 성능을 보이며, ARIMA는 28일 예측에서 상대적으로 좋은 성능을 보임을 확인할 수 있음.

\begin{figure}[h]
\centering
\includegraphics[width=0.8\textwidth]{images/horizon_trend.png}
\caption{예측 기간별 성능 추이 (표준화된 RMSE)}
\label{fig:horizon_trend}
\end{figure}

\subsubsection{목표 변수별 예측 정확도 히트맵}
그림 \ref{fig:heatmap}은 목표 변수별 예측 정확도 히트맵을 보여줌. 현재 GDP에 대한 결과만 사용 가능하며, 민간 소비 및 총고정자본형성에 대한 실험은 진행 중임.

\begin{figure}[h]
\centering
\includegraphics[width=0.8\textwidth]{images/accuracy_heatmap.png}
\caption{목표 변수별 예측 정확도 히트맵 (표준화된 RMSE)}
\label{fig:heatmap}
\end{figure}

\subsubsection{예측값 vs 실제값 시계열 비교}
그림 \ref{fig:forecast_vs_actual}은 주요 모형들의 예측값과 실제값을 비교한 시계열 그래프를 보여줌. 현재 시계열 데이터 추출 기능이 구현되지 않아 플레이스홀더를 사용하며, 향후 구현 예정임.

\begin{figure}[h]
\centering
\includegraphics[width=0.8\textwidth]{images/forecast_vs_actual.png}
\caption{예측값 vs 실제값 시계열 비교 (GDP)}
\label{fig:forecast_vs_actual}
\end{figure}


\section{논의}

\subsection{모델 비교}

본 연구의 핵심은 DFM과 DDFM 모형의 성능 비교이며, ARIMA와 VAR은 벤치마크 모형으로 포함됨. 네 가지 모형의 성능을 대상 변수와 예측 수평선에 걸쳐 비교:

\textbf{ARIMA:}
\begin{itemize}
    \item 세 대상 변수 모두에서 성공적으로 평가 완료
    \item 특징: 단순성, 해석 가능성, 안정적 성능
    \item 일부 수평선에서 우수한 성능을 보임 (예: KOEQUIPTE 3개월에서 최소 sMAE/sMSE)
    \item Nowcasting에서는 release date 마스킹 처리의 어려움으로 인해 제한적임
\end{itemize}

\textbf{VAR:}
\begin{itemize}
    \item 세 대상 변수 모두에서 성공적으로 평가 완료
    \item 벤치마크 모형으로 포함되었으며, 대상 변수에 따라 성능 차이가 큼
    \item Nowcasting에서는 release date 마스킹 처리의 어려움으로 인해 제한적임
\end{itemize}

\textbf{DFM:}
\begin{itemize}
    \item 세 대상 변수 모두에서 성공적으로 평가 완료
    \item 전통적인 동적요인모형으로, EM 알고리즘을 통한 요인 추출 및 예측 수행
    \item Nowcasting에서 release date 마스킹을 효과적으로 처리 가능
    \item 요인 모형의 구조적 특성으로 인해 다변량 시계열 간 공통 패턴을 효과적으로 포착
\end{itemize}

\textbf{DDFM:}
\begin{itemize}
    \item 세 대상 변수 모두에서 성공적으로 평가 완료
    \item 심층 신경망 기반 인코더를 통한 비선형 요인 추출
    \item 중간 수평선(11개월)에서 우수한 성능을 보임
    \item Nowcasting에서 release date 마스킹을 효과적으로 처리 가능
    \item 비선형 관계 포착 능력으로 인해 복잡한 시계열 패턴에 유리
\end{itemize}

\subsection{원인 분석}

\subsubsection{모형별 제한사항}
\begin{itemize}
    \item VAR: 긴 수평선에서 공분산 행렬 특이성으로 인한 수치적 불안정성
    \item DFM: EM 알고리즘 수렴 중 수치적 불안정성 발생, 수치 안정화 기법 적용으로 해결
\end{itemize}

\subsubsection{DDFM의 성능 특성}

DDFM은 중간 수평선(11개월)에서 우수한 성능을 보이나, 단기(1개월)와 장기(22개월) 수평선에서는 다른 모형보다 높은 오차를 보임. 가능한 원인: 데이터 양 부족, 선형 관계에서 비선형 인코더의 과도한 복잡성, 하이퍼파라미터 최적화 부족 \cite{andreini2020deep}. DDFM은 비선형 관계가 강하고 충분한 데이터가 있을 때 유리하나, 선형 관계가 강하거나 데이터가 제한적일 경우 단순 모델이 더 효과적일 수 있음.

\subsection{Nowcasting 시점별 분석}

Nowcasting 실험 구성:
\begin{itemize}
    \item 모형: DFM, DDFM (2개) - ARIMA와 VAR은 release date 마스킹 처리의 구조적 한계로 인해 제외
    \item 대상 변수: 3개 (KOIPALL.G, KOEQUIPTE, KOWRCCNSE)
    \item 목표 월: 2024--01 ~ 2025--10 (22개월)
    \item 예측 시점: 4주 전, 1주 전
    \item 결과: 표~\ref{tab:nowcasting_backtest}에 제시된 바와 같이 DFM과 DDFM 백테스트가 성공적으로 완료됨
\end{itemize}

\textbf{시점별 성능 비교:} \cite{banbura2012nowcasting}
표~\ref{tab:nowcasting_backtest}의 결과를 보면, 대부분의 경우 1주 전 예측이 4주 전 예측보다 더 정확함. 이는 시간이 지날수록 더 많은 데이터가 사용 가능해지기 때문임. 구체적으로:
\begin{itemize}
    \item \textbf{DDFM}: 모든 대상 변수에서 1주 전 예측이 4주 전 예측보다 우수함
    \begin{itemize}
        \item KOIPALL.G: 4weeks sMSE 81.7 $\to$ 1week sMSE 43.4 (47\% 개선)
        \item KOEQUIPTE: 4weeks sMSE 1.90 $\to$ 1week sMSE 2.13 (약간 증가, 하지만 sMAE는 1.11 $\to$ 1.16으로 유사)
        \item KOWRCCNSE: 4weeks sMSE 0.54 $\to$ 1week sMSE 0.53 (약간 개선)
    \end{itemize}
    \item \textbf{DFM}: KOEQUIPTE와 KOWRCCNSE에서는 1주 전 예측이 더 정확하나, KOIPALL.G에서는 오히려 1주 전 예측이 더 나쁨 (4weeks sMSE 16155.6 $\to$ 1week sMSE 59934.6). 또한 KOIPALL.G DFM의 경우, 모든 22개월에 걸쳐 예측값이 단 2개의 고유값(-12.9, 13.5)만을 보이는 반복적 예측 패턴이 관찰됨. 이는 극단적인 예측값이 하드 클리핑(hard clipping) 과정에서 정확히 2개의 경계값으로 수렴하여 발생한 문제로 확인되었으며, 코드에서 소프트 클리핑(soft clipping) 방식으로 개선하여 상대적 차이를 보존하도록 수정함. 반면 DDFM은 동일한 대상 변수에서 다양한 예측값을 생성하여 정상적으로 작동함.
\end{itemize}

\textbf{DFM vs DDFM 성능 비교:}
DDFM이 DFM보다 전반적으로 우수한 성능을 보이며, 특히 KOIPALL.G에서 큰 차이를 보임:
\begin{itemize}
    \item \textbf{KOIPALL.G}: DDFM-4weeks sMSE 81.7 vs DFM-4weeks sMSE 16155.6 (약 198배 개선)
    \item \textbf{KOEQUIPTE}: DDFM-4weeks sMSE 1.90 vs DFM-4weeks sMSE 3.40 (약 1.8배 개선)
    \item \textbf{KOWRCCNSE}: DDFM-4weeks sMSE 0.54 vs DFM-4weeks sMSE 0.85 (약 1.6배 개선)
\end{itemize}
이는 DDFM의 비선형 인코더가 복잡한 시계열 패턴을 더 효과적으로 포착할 수 있기 때문으로 보임. 특히 KOIPALL.G의 경우 DFM이 매우 높은 오차를 보이는 반면 DDFM은 상대적으로 안정적인 성능을 보여, 비선형 관계가 중요한 경우 DDFM의 장점이 두드러짐.

\textbf{Release date 마스킹의 효과:}
DFM과 DDFM은 요인 모형의 구조적 특성으로 인해 release date 기반 마스킹을 효과적으로 처리 가능함. Kalman filter는 각 시점의 데이터 발표를 재귀적으로 처리하여 예측을 업데이트하며, 데이터의 시의성과 품질을 자동으로 고려함. 실시간 데이터 흐름에서 비동기적 데이터 발표로 인한 불규칙성(jagged edges)을 DFM/DDFM이 자연스럽게 처리할 수 있어, 실제 운영 환경에서의 nowcasting에 적합함.



\section{결론}

\subsection{연구 요약}

본 연구는 고빈도 데이터를 활용하여 한국의 주요 거시경제 변수(GDP, 소비, 투자)를 예측하는 다양한 모형들의 성능을 체계적으로 비교 분석함. 특히 동적 요인 모형(DFM)과 심층 동적 요인 모형(DDFM)에 초점을 맞추어, 혼합 빈도 데이터 처리 능력과 예측 성능을 평가함. 총 36개 조합 중 28개 조합(77.8\%)에 대한 실험 결과가 완료되었음.

핵심 발견은 다음과 같음: (1) VAR 모형이 단기 예측에서 압도적으로 우수한 성능을 보였으며, 이는 거시경제 변수들 간의 동시적 상호작용을 직접적으로 모델링할 수 있기 때문임. (2) DFM과 DDFM은 혼합 빈도 데이터 처리 능력을 보유하나, 목표 변수의 데이터 특성에 따라 성능이 크게 달라짐. DFM은 선형 관계가 명확한 GDP에서는 효과적이나, 비선형 관계가 중요한 소비 변수에서는 수치적 불안정성을 보임. (3) DDFM의 비선형 인코더는 일부 목표 변수에서 DFM보다 안정적인 성능을 보였으나, 변동성이 큰 투자 변수에서는 한계를 드러냄. (4) 모형 선택은 목표 변수와 예측 기간에 따라 달라지며, 단기 예측에는 VAR, 혼합 빈도 nowcasting에는 DFM/DDFM이 적합함.

\subsection{연구의 기여도}

본 연구의 주요 기여는 다음과 같음:

\begin{itemize}
    \item \textbf{통합 비교 프레임워크 구축}: 전통적 통계 모형(ARIMA, VAR)과 동적 요인 모형(DFM, DDFM)을 동일한 데이터셋과 평가 기준으로 비교하는 통합 프레임워크를 제시함. 이를 통해 각 모형의 장단점을 목표 변수와 예측 기간에 따라 체계적으로 분석할 수 있음.
    \item \textbf{혼합 빈도 모형의 실증적 평가}: dfm-python 패키지를 활용하여 DFM과 DDFM의 혼합 빈도 데이터 처리 능력을 실증적으로 검증함. 특히 한국 거시경제 데이터에 대한 적용 가능성과 한계를 명확히 제시함.
    \item \textbf{모형 선택 가이드라인 수립}: 실험 결과를 바탕으로 목표 변수와 예측 기간에 따른 모형 선택 가이드라인을 제시함. 단기 예측에는 VAR, 혼합 빈도 nowcasting에는 DFM/DDFM, 변동성이 큰 변수에는 VAR/ARIMA를 권장함.
    \item \textbf{수치적 불안정성 원인 분석}: DFM의 EM 알고리즘이 특정 목표 변수에서 수치적 불안정성을 보이는 원인을 분석하고, DDFM의 비선형 인코더가 이를 완화할 수 있음을 실증적으로 보임.
\end{itemize}

\subsection{정책적 함의}

본 연구의 결과는 다음과 같은 정책적 시사점을 제공함:

\begin{itemize}
    \item \textbf{단기 예측 기반 정책 의사결정}: VAR 모형의 우수한 단기 예측 성능을 활용하여 통화정책위원회의 기준금리 결정 시점을 앞당기거나, 재정정책 수립 시 더 신속한 경제 상황 파악이 가능함. 특히 경제 위기 상황에서 공식 통계 발표 전 선제적 정책 대응이 가능함.
    \item \textbf{혼합 빈도 nowcasting 시스템 도입}: DFM과 DDFM의 혼합 빈도 처리 능력을 활용하여 한국은행이나 통계청에서 분기별 GDP 발표 전 실시간 추정 시스템을 구축할 수 있음. 이를 통해 정책 결정의 시의성을 크게 향상시킬 수 있음.
    \item \textbf{고빈도 데이터 인프라 투자}: 고빈도 데이터가 예측 정확도 향상에 기여함을 확인하였으므로, 생산지수, 소매판매액, 수출입액 등의 월간 지표의 신뢰성과 시의성 향상을 위한 데이터 인프라 투자가 필요함.
\end{itemize}

\subsection{연구의 한계점}

본 연구는 다음과 같은 한계점을 가짐 (자세한 내용은 Section \ref{sec:discussion} 참조):

\begin{itemize}
    \item \textbf{데이터 품질 및 전처리}: 많은 변수들이 결측치를 포함하고 있어, 전처리 과정에서 정보 손실이 발생할 수 있음
    \item \textbf{모형 해석가능성}: 딥러닝 기반 모형(DDFM)은 학습된 요인 구조의 경제적 해석이 어려움
    \item \textbf{외생 충격 고려 부재}: 본 연구는 과거 데이터에 기반한 예측만을 다루며, 예상치 못한 외생 충격을 고려하지 않음
    \item \textbf{한국 데이터에 국한}: 본 연구는 한국 데이터에만 적용되었으며, 다른 국가나 지역에 대한 일반화 가능성은 추가 검증이 필요함
    \item \textbf{실험 완료율}: 총 36개 조합 중 28개 조합(77.8\%)만 완료되었으며, 8개 조합은 데이터 및 모형의 한계로 인해 평가 불가능함
\end{itemize}

\subsection{향후 연구 방향}

본 연구의 결과와 한계를 바탕으로 다음과 같은 향후 연구 방향을 제안함 (자세한 내용은 Section \ref{sec:discussion} 참조):

\begin{itemize}
    \item \textbf{DFM 수치적 안정성 개선}: 적응형 정규화, 베이지안 추정 방법(MCMC), 또는 변분 추론(VI)을 통한 대안적 추정 방법 연구
    \item \textbf{28일 예측 평가 방법론 개발}: 시계열 교차 검증 또는 walk-forward validation을 통한 평가 방법론 개발
    \item \textbf{변동성이 큰 변수에 대한 모형 개선}: GARCH 또는 stochastic volatility 모델 통합, changepoint detection을 통한 구조적 변화 적응
    \item \textbf{Nowcasting 전용 실험 설계}: 마스킹된 데이터를 활용한 백테스팅 및 News decomposition 기능 활용
    \item \textbf{DDFM 하이퍼파라미터 최적화}: 베이지안 최적화 또는 AutoML 기법을 통한 목표 변수별 최적화
    \item \textbf{앙상블 모형 개발}: VAR, ARIMA, DFM/DDFM의 예측을 통합하는 앙상블 방법 개발
\end{itemize}

본 연구는 고빈도 데이터를 활용한 거시경제 변수 예측에 대한 체계적인 분석 프레임워크를 구축하였으며, 한국 거시경제 데이터에 대한 실증적 비교 분석을 통해 모형 선택 가이드라인을 제시함. 총 36개 조합 중 28개 조합(77.8\%)에 대한 실험 결과가 완료되었으며, 남은 8개 조합은 데이터 및 모형의 근본적 한계로 인해 평가 불가능함. 이러한 한계에도 불구하고, 본 연구는 거시경제 변수 예측 및 nowcasting 연구의 기초 자료로 활용될 수 있을 것으로 기대됨.


\section*{Aknowledgement}
본 연구는 기획재정부, KDI의 지원으로 실시되었음

\bibliographystyle{unsrt}  
\bibliography{references}
\end{document}
