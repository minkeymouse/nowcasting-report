% Korean support: works with pdfLaTeX (default), XeLaTeX, or LuaLaTeX
% In Overleaf: Menu -> Compiler -> pdfLaTeX (default) or XeLaTeX
\documentclass[12pt]{article}

\usepackage{arxiv}
\usepackage{float}
\usepackage{graphicx}
\usepackage[utf8]{inputenc} % allow utf-8 input
\usepackage[T1]{fontenc}    % use 8-bit T1 fonts
\usepackage{hyperref}       % hyperlinks
\usepackage{url}            % simple URL typesetting
\usepackage{booktabs}       % professional-quality tables
\usepackage{amsfonts}       % blackboard math symbols
\usepackage{nicefrac}       % compact symbols for 1/2, etc.
\usepackage{microtype}      % microtypography
\usepackage{lipsum}
\usepackage{amsmath}

\title{Nowcasting Production and Investment Sector with High Frequency Data Integration}

\author{
  JaeYoung Kim\\
   affiliation\\
  address\\
  \texttt{email} \\
   \And
 SeoJung Lee \\
   affiliation\\
  address\\
  \texttt{email} \\
   \And
  YoungMin Kim \\
   affiliation\\
  address\\
  \texttt{email} \\
   \And
  Minkey Chang \\
   affiliation\\
  address\\
  \texttt{email} \\
   \And
  JunHo Hwang \\
   affiliation\\
  address\\
  \texttt{email} \\
   \And
  EunKyu Sung \\
   affiliation\\
  address\\
  \texttt{email} \\
   \And
  %% \And
  %% Coauthor \\
  %% Affiliation \\
  %% Address \\
  %% \texttt{email} \\
  %% \And
  %% Coauthor \\
  %% Affiliation \\
  %% Address \\
  %% \texttt{email} \\
}

\begin{document}
% Title without authors
\begin{center}
\Large\textbf{고빈도 데이터를 활용한 생산, 투자, 소비 분야 거시 변수 예측 모델}
\end{center}
\vspace{1cm}

\begin{abstract}
본 연구는 고빈도 데이터를 활용하여 한국의 주요 거시경제 변수(생산: 전산업생산지수, 제조업생산지수, 투자: 설비투자지수, 소비: 도소매판매액)를 예측하는 동적 요인 모형(DFM)과 딥러닝 모형의 성능을 비교 분석한다. 월간 및 분기 데이터를 이용한 DFM 모형을 통해 월간 지수를 산출하고, 고빈도 금융시장 데이터를 추가한 고빈도 DFM 모형과 딥러닝 모형의 nowcasting 성과를 평가한다. 연구 결과, DFM 모형은 분기 데이터로부터 월간 지수를 추정하는 데 양호한 성과를 보였으며, 고빈도 모형은 실시간 경기진단에 효과적임을 확인하였다. 딥러닝 모형은 DFM 모형 대비 예측오차가 개선되었으나 변동폭 과소 추정 경향이 있어, 두 모형의 평균값 사용이 바람직함을 제시한다.
\end{abstract}

\textbf{키워드:} nowcasting, 동적 요인 모형, 고빈도 데이터, 거시경제 예측, 딥러닝

\section{Introduction}

Accurate forecasting of macroeconomic variables is crucial for policy decision-making and corporate strategic planning. In particular, production, investment, and consumption indicators represent the core of economic activity, and real-time assessment is essential. However, key indicators such as quarterly GDP are officially released only after approximately one month following the end of the quarter, making it difficult to assess the real-time economic situation and respond with timely policy measures.

Accordingly, nowcasting techniques utilizing high-frequency data have gained attention \cite{bok2017macroeconomic}. Nowcasting is a technique that estimates current macroeconomic variables using various high-frequency indicators before official statistics are released. Its importance is particularly highlighted in crisis situations where rapid policy response is needed.

This study constructs a nowcasting system using Dynamic Factor Models (DFM) and deep learning models to forecast three key Korean macroeconomic indicators: production (Industrial Production Index, All Industries: KOIPALL.G), investment (Equipment Investment Index: KOEQUIPTE), and consumption (Wholesale and Retail Trade Sales: KOWRCCNSE). We compare the performance of four forecasting models: ARIMA, VAR, DFM, and Deep Dynamic Factor Model (DDFM) across three forecast horizons (1, 7, and 28 days).


\section{결과 비교}
\label{sec:results}

\subsection{실험 설계}
\label{subsec:experimental_design}

\subsubsection{실험 셋업}

\begin{itemize}
    \item \textbf{대상 변수:} KOEQUIPTE, KOWRCCNSE, KOIPALL.G (3개)
    \item \textbf{모형:} ARIMA, VAR, DFM, DDFM (4개)
    \item \textbf{평가:} 22개 예측 수평선(1--22개월), 모형-대상 조합별 평균 계산
\end{itemize}

\begin{table}[h]
\centering
\caption{Dataset and Model Parameters}
\label{tab:dataset_params}
\begin{tabular}{lccc}
\toprule
Target Variable & Series Count & Training Period & Forecast Period \\
\midrule
KOIPALL.G & 43 & 1985-2019 & 2024-2025 \\
KOEQUIPTE & 43 & 1985-2019 & 2024-2025 \\
KOWRCCNSE & 43 & 1985-2019 & 2024-2025 \\
\bottomrule
\end{tabular}
\end{table}


표~\ref{tab:dataset_params} 요약:
\begin{itemize}
    \item 각 대상 변수마다 평균 43개 시계열 사용
    \item 훈련 기간: 1985--2019년
    \item 예측 기간: 2024--2025년
\end{itemize}

\subsubsection{데이터 전처리}

\begin{itemize}
    \item \textbf{변환:} 시계열별 변환 유형('lin', 'log', 'chg' 등) 적용
    \item \textbf{결측치 처리:} forward-fill $\to$ backward-fill $\to$ naive forecaster 순차 적용
    \item \textbf{표준화:}
    \begin{itemize}
        \item ARIMA/VAR: 원본 스케일 유지
        \item DFM/DDFM: StandardScaler 적용 (평균 0, 표준편차 1)
    \end{itemize}
\end{itemize}

\subsubsection{데이터 품질 문제 및 시리즈 제거}

데이터 품질 개선을 위해 다음 시리즈를 제거:
\begin{itemize}
    \item 높은 상관관계(> 0.95) 시리즈
    \item 극단적 결측치(91.3\%) 시리즈 (pmiall, pmiout)
    \item 블록 구조 단일 글로벌 블록으로 단순화, 요인 수 3개 통일
\end{itemize}

\subsubsection{예측 모형}

\textbf{ARIMA:} 자기회귀 및 이동평균 성분 포착, 정상성을 위해 차분 사용, 단변량 시계열 예측. 차수 (1,1,1) 사용.

\textbf{VAR:} ARIMA를 다변량으로 확장, 여러 시계열 간 동적 관계 포착. 시차 1 사용. 긴 수평선에서 수치적 불안정성 발생 가능.

\textbf{DFM:} 많은 시계열에서 공통 요인 추출, 차원 축소, 혼합주기 데이터 처리 \cite{stock2002forecasting}. DFM은 state-space 형태로 표현되며, measurement equation과 transition equation으로 구성됨. EM 알고리즘으로 파라미터 추정, 칼만 필터와 스무더로 요인 추정. 칼만 필터는 실시간 데이터 흐름을 재귀적으로 처리하여 각 시점의 예측을 업데이트하며, 데이터의 품질과 시의성을 기반으로 가중치를 부여함. 이는 nowcasting에 특히 유용한 특성으로, 비동기적 데이터 발표와 결측치를 자연스럽게 처리할 수 있음. 혼합주기 데이터의 경우, 텐트 커널(tent kernel) 집계 방법을 사용하여 서로 다른 주기의 데이터를 통합함.

\textbf{DDFM:} 오토인코더 기반 아키텍처로 비선형 요인 관계 학습 \cite{andreini2020deep}. DDFM은 인코더를 통해 관측 변수에서 잠재 요인을 추출하고, 디코더를 통해 요인에서 관측 변수로 재구성함. 이 과정에서 선형 DFM의 제약을 완화하여 더 복잡한 요인 구조를 학습할 수 있음. 대규모 데이터셋에서도 효과적으로 작동하며, 전통적인 DFM의 계산적 한계를 극복함.

\subsubsection{Forecasting과 Nowcasting}

\textbf{Forecasting:} 과거 데이터로 미래 값 예측. 각 모형 훈련 후 1--22개월 수평선에 대해 예측 생성.

\textbf{Nowcasting:} 공식 통계 발표 전 현재 시점 거시경제 변수 추정 \cite{banbura2012nowcasting}. Nowcasting은 실시간 경제 모니터링의 핵심 기법으로, 중앙은행과 정책기관에서 널리 활용됨. 각 목표 월에 대해 4주 전, 1주 전 시점에서 예측을 수행하며, 시리즈별 발표 시차(publication lag)를 기준으로 미발표 데이터를 마스킹함. 이는 실제 운영 환경에서 특정 시점에 사용 가능한 데이터만을 사용하여 예측하는 상황을 시뮬레이션함. 시간이 지날수록 더 많은 데이터가 사용 가능해지므로, 예측 정확도가 향상될 것으로 기대됨. DFM과 DDFM은 요인 모형의 구조적 특성으로 인해 release date 기반 마스킹을 효과적으로 처리할 수 있으나, ARIMA와 VAR은 이러한 구조적 유연성이 부족하여 nowcasting에 제한적임.


\section{생산 모형: KOIPALL.G}

\subsection{대상 변수}

전산업생산지수(Industrial Production Index, All Industries: KOIPALL.G)는 생산 지표로 작용하며, 한국 경제의 전체 산업 활동을 나타낸다. 이 지수는 모든 산업의 생산을 집계하며 경제 활동 평가를 위한 핵심 지표이다.

\subsection{데이터 구성}

생산 모형은 산업생산과 관련된 월간 및 분기 시계열 데이터를 활용한다. 데이터셋에는 고용, 산업생산, 기업 서베이 및 전체 산업 활동을 예측하는 기타 경제 지표와 관련된 변수들이 포함된다.

\subsection{모형 비교 결과}

KOIPALL.G에 대해 네 가지 모형(ARIMA, VAR, DFM, DDFM)의 예측 성능을 세 가지 예측 수평선(1일, 7일, 28일)에서 비교한다. 성능 지표(표준화된 MSE, MAE, RMSE)는 표~\ref{tab:overall_metrics_by_target}에 제시되며 그림~\ref{fig:forecast_vs_actual_koipallg}에 시각화된다.

\subsection{예측 성능}

예측 대 실제 플롯(그림~\ref{fig:forecast_vs_actual_koipallg})은 평가 기간 동안의 역사적 시계열과 모형 예측을 보여준다. 예측 수평선별 상세 성능 지표는 표~\ref{tab:overall_metrics_by_horizon}에 제시된다. KOIPALL.G에 대한 모든 모형-수평선 조합의 상세 지표는 표~\ref{tab:metrics_36_rows}에서 확인할 수 있다.

\begin{figure}[h]
\centering
\includegraphics[width=0.9\textwidth]{images/forecast_vs_actual_koipall_g.png}
\caption{예측 대 실제: 전산업생산지수 (KOIPALL.G). 30개월의 역사적 데이터와 ARIMA, VAR, DFM, DDFM 모형의 30개월 예측을 보여준다.}
\label{fig:forecast_vs_actual_koipallg}
\end{figure}

\subsection{논의}

KOIPALL.G에 대한 실험 결과는 예측 수평선에 걸쳐 모형 성능의 유의한 차이를 보여준다. ARIMA는 가장 일관된 성능을 보인다: 1일 예측에서 우수(sMSE = 0.0034, sRMSE = 0.0584), 7일 예측에서 보통(sMSE = 2.28, sRMSE = 1.51), 28일 예측에서 합리적(sMSE = 0.39, sRMSE = 0.62). 1일에서 7일 예측으로의 성능 저하는 단기 패턴이 중기 추세보다 예측하기 쉬움을 시사한다.

VAR은 1일 예측에서 탁월한 성능을 보이지만(sMSE $\approx$ 3.5$\times$10$^{-9}$, sRMSE $\approx$ 6.0$\times$10$^{-5}$), 더 긴 수평선에서는 심각한 수치적 불안정성을 겪는다. 7일 및 28일 예측의 경우, VAR은 극도로 큰 오차를 생성한다(h=7일 경우 sRMSE $>$ 10$^{22}$, h=28일 경우 $>$ 10$^{58}$), 이는 이 대상에 대해 다단계 앞 예측에 적합하지 않음을 나타낸다. 이 불안정성은 매우 짧은 수평선을 넘어 예측할 때 VAR 모형의 알려진 제한사항이다.

DFM은 1일 예측에서 보통 성능을 보이지만(sRMSE = 5.92), 7일 예측에서 개선된다(sRMSE = 5.28). 이는 요인 모형이 단기 변동보다 중기 추세를 더 잘 포착함을 나타낸다. 그러나 DFM 성능은 모든 수평선에서 ARIMA보다 현저히 낮다. DDFM은 1일 예측에서 우수한 성능(sRMSE = 0.46)과 7일 예측에서 탁월한 성능(sRMSE = 0.18)을 보이며, 이러한 수평선에서 ARIMA를 능가한다. 28일 수평선은 80/20 훈련-테스트 분할 후 테스트 데이터 부족으로 인해 DFM과 DDFM 모두에서 사용할 수 없다.

전반적으로, ARIMA는 모든 수평선에 걸쳐 산업생산에 대한 가장 신뢰할 수 있는 예측을 제공하며, 예측 수평선이 증가함에 따라 성능이 점진적으로 저하된다.


\section{투자 모형: KOEQUIPTE}

\subsection{대상 변수}

설비투자지수(Equipment Investment Index: KOEQUIPTE)는 투자 지표로 작용하며, 설비 및 기계에 대한 자본 지출을 측정한다. 이 지수는 고정자본형성의 핵심 구성요소이며 기업 투자 활동을 반영한다.

\subsection{데이터 구성}

투자 모형은 설비투자와 관련된 월간 및 분기 시계열 데이터를 활용한다. 데이터셋에는 제조업 및 건설업 고용, 자본재 수입, 생산자물가지수, 설비투자 지표, 건설 활동, 설비자금 대출 및 기업 대출금리와 같은 금융 지표와 관련된 변수들이 포함된다.

\subsection{모형 비교 결과}

KOEQUIPTE에 대해 네 가지 모형(ARIMA, VAR, DFM, DDFM)의 예측 성능을 세 가지 예측 수평선(1일, 7일, 28일)에서 비교한다. 성능 지표(표준화된 MSE, MAE, RMSE)는 표~\ref{tab:overall_metrics_by_target}에 제시되며 그림~\ref{fig:forecast_vs_actual_koequipte}에 시각화된다.

\subsection{예측 성능}

예측 대 실제 플롯(그림~\ref{fig:forecast_vs_actual_koequipte})은 평가 기간 동안의 역사적 시계열과 모형 예측을 보여준다. 예측 수평선별 상세 성능 지표는 표~\ref{tab:overall_metrics_by_horizon}에 제시된다. KOEQUIPTE에 대한 모든 모형-수평선 조합의 상세 지표는 표~\ref{tab:metrics_36_rows}에서 확인할 수 있다.

\begin{figure}[h]
\centering
\includegraphics[width=0.9\textwidth]{images/forecast_vs_actual_koequipte.png}
\caption{예측 대 실제: 설비투자지수 (KOEQUIPTE). 30개월의 역사적 데이터와 ARIMA, VAR, DFM, DDFM 모형의 30개월 예측을 보여준다.}
\label{fig:forecast_vs_actual_koequipte}
\end{figure}

\subsection{논의}

KOEQUIPTE에 대한 결과는 ARIMA가 모든 수평선에 걸쳐 보통 성능을 제공함을 보여준다. 표준화된 RMSE 값은 0.32(1일), 1.59(7일), 1.67(28일)이다. 모형은 예측 수평선이 증가함에 따라 상대적으로 안정적인 성능을 유지하지만, 오차는 다른 대상보다 높다. 이는 설비투자가 생산 또는 소비 지표보다 예측하기 어려울 수 있음을 시사한다.

VAR은 다시 1일 예측 정확도에서 우수한 성능을 보이지만(sMSE $\approx$ 3.7$\times$10$^{-9}$, sRMSE $\approx$ 6.0$\times$10$^{-5}$), 더 긴 수평선에서는 완전히 실패하며, 오차가 비현실적인 크기로 폭발한다(h=7일 경우 sRMSE $>$ 10$^{13}$, h=28일 경우 $>$ 10$^{60}$). 이 수치적 불안정성은 즉시 다음 기간을 넘어 투자 예측에 VAR을 부적합하게 만든다.

KOEQUIPTE에 대한 ARIMA 오차가 다른 대상에 비해 상대적으로 높은 것(평균 sRMSE = 1.19 vs. 다른 대상의 0.71-0.73)은 설비투자가 더 큰 변동성 또는 예측하기 어려운 패턴을 보임을 나타낸다. 이는 경기 순환 효과 및 정책 변화의 영향을 받는 자본 투자 결정의 덩어리 특성 때문일 수 있다.

DFM은 KOEQUIPTE에 대해 낮은 성능을 보이며, sRMSE 값이 4.21(1일) 및 6.11(7일)로 ARIMA보다 현저히 낮다. 이는 요인 모형이 설비투자의 변동성 및 불규칙한 패턴과 어려움을 겪고 있음을 시사한다. 그러나 DDFM은 1일 예측에서 탁월한 성능(sRMSE = 0.0103)을 보이며, ARIMA를 포함한 모든 다른 모형을 능가한다. 7일 예측의 경우, DDFM은 sRMSE = 1.91을 달성하며, 이는 ARIMA의 1.59보다 나쁘지만 여전히 합리적이다. 28일 수평선은 80/20 훈련-테스트 분할 후 테스트 데이터 부족으로 인해 DFM과 DDFM 모두에서 사용할 수 없다. DDFM의 우수한 단기 성능은 딥러닝 인코더가 전통적인 모형이 놓치는 투자 데이터의 복잡한 패턴을 효과적으로 포착함을 시사한다.


\section{소비부문 모형}

\subsection{데이터 구성}

소비부문 데이터는 고용, 소비, 설문 등 주요 월간 지표를 포함하여 구성됨. 주요 변수는 고용/노동(취업자 수: 도·소매업), 수출입(수입: 소비재), 소비/지출(소매판매액, 도·소매업 판매지수: 내구재, 반내구재, 비내구재, 신용카드 거래액, 사이버쇼핑 거래액), 물가(소비자물가지수, 소비자물가: 농산물·유류 제외, 식료품·에너지 제외), 산업생산(제조업 출하/재고: 소비재, 생산: 비내구재, 소비재, 서비스업: 도·소매, 숙박·음식점), 기업경기(BSI, FKI 지수), 소비자동향(CSI: 종합, 생활형편, 가계소득 전망, 소비지출 계획, 경기판단, 고용상황 전망, 가계저축/부채 전망), 금융(주택담보대출, 가계대출금리) 등임.

\subsection{DFM 모형 추정}

4개 공통요인을 가정하고 DFM 모형을 추정하여, 이를 통해 월간 소비 지수를 산출함. 추정된 공통요인과 모수, 잔차항을 이용하여 월간 도소매판매액 지수를 추정함. 추정된 월간 소비 지수는 관측된 소매판매액 지수와 유사성을 보임.

\subsection{고빈도 DFM 모형}

고빈도 데이터를 포함하여 주, 월간 데이터로 구성된 고빈도 DFM 모형을 추정함. 고빈도 데이터는 주가, 금리, 환율 등 금융시장 데이터와 뉴스 심리지수를 활용함. 금융시장 특성을 반영하기 위해 요인 개수를 1개 추가한 5개로 가정함.

고빈도 DFM 모형은 월간 소비 지수에 대해 양호한 nowcasting 성과를 보임. 평균 절대 예측오차는 4주전 및 1주전 수준에서 양호한 성능을 보임.

\subsection{딥러닝 모형 비교}

동일한 데이터를 이용하여 딥러닝 모형으로 추정 시 nowcasting 성과가 개선됨. 생산 및 투자 모형과 마찬가지로 DFM 모형 대비 예측오차가 개선되었으나, 월간 변동폭을 과소 추정하는 경향이 있어 DFM 모형과 DNN 모형의 평균값 사용이 바람직함.



\section{결론}

본 연구는 고빈도 데이터를 활용하여 한국의 주요 거시경제 변수(생산, 투자, 소비)를 예측하는 동적 요인 모형과 딥러닝 모형의 성능을 비교 분석함.

주요 결과는 다음과 같음: (1) DFM 모형은 분기 데이터로부터 월간 지수를 추정하는 데 양호한 성과를 보였으며, 추정된 월간 지수는 관측 지수와 높은 상관관계를 보임. (2) 고빈도 DFM 모형은 실시간 경기진단에 효과적이며, 1주전 예측에서 양호한 성능을 보임. (3) 딥러닝 모형은 DFM 모형 대비 예측오차가 개선되었으나, 월간 변동폭을 과소 추정하는 경향이 있어 두 모형의 평균값 사용이 바람직함.

본 연구의 기여는 다음과 같음: (1) 생산, 투자, 소비 부문별 nowcasting 시스템 구축, (2) 고빈도 데이터를 활용한 실시간 경기진단 프레임워크 제시, (3) DFM과 딥러닝 모형의 앙상블 전략 제안.

향후 연구 방향으로는 더 많은 고빈도 데이터(실시간 소비 데이터, 카드 거래 데이터 등)의 통합, 구조적 변화 시점 감지 및 모형 적응, 외생 충격 고려 등이 제안됨.


\bibliographystyle{unsrt}  
\bibliography{references}
\end{document}
