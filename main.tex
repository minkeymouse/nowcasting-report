% Korean support: works with pdfLaTeX (default), XeLaTeX, or LuaLaTeX
% In Overleaf: Menu -> Compiler -> pdfLaTeX (default) or XeLaTeX
\documentclass[12pt]{article}

\usepackage{arxiv}
\usepackage{float}
\usepackage{graphicx}
\usepackage[utf8]{inputenc} % allow utf-8 input
\usepackage[T1]{fontenc}    % use 8-bit T1 fonts
\usepackage{hyperref}       % hyperlinks
\usepackage{url}            % simple URL typesetting
\usepackage{booktabs}       % professional-quality tables
\usepackage{amsfonts}       % blackboard math symbols
\usepackage{nicefrac}       % compact symbols for 1/2, etc.
\usepackage{microtype}      % microtypography
\usepackage{lipsum}
\usepackage{amsmath}

\title{Nowcasting Production and Investment Sector with High Frequency Data Integration}

\author{
  JaeYoung Kim\\
   affiliation\\
  address\\
  \texttt{email} \\
   \And
 SeoJung Lee \\
   affiliation\\
  address\\
  \texttt{email} \\
   \And
  YoungMin Kim \\
   affiliation\\
  address\\
  \texttt{email} \\
   \And
  Minkey Chang \\
   affiliation\\
  address\\
  \texttt{email} \\
   \And
  JunHo Hwang \\
   affiliation\\
  address\\
  \texttt{email} \\
   \And
  EunKyu Sung \\
   affiliation\\
  address\\
  \texttt{email} \\
   \And
  %% \And
  %% Coauthor \\
  %% Affiliation \\
  %% Address \\
  %% \texttt{email} \\
  %% \And
  %% Coauthor \\
  %% Affiliation \\
  %% Address \\
  %% \texttt{email} \\
}

\begin{document}
\maketitle

\begin{abstract}
본 연구는 고빈도 데이터를 활용하여 한국의 주요 거시경제 변수(GDP, 소비, 투자)를 예측하는 다양한 모형들의 성능을 체계적으로 비교 분석한다. 9개의 예측 모형(ARIMA, VAR, VECM, DeepAR, TFT, XGBoost, LightGBM, DFM, DDFM)을 3개의 목표 변수와 3개의 예측 기간(1일, 7일, 28일)에 대해 평가하였으며, 표준화된 MSE, MAE, RMSE를 통해 성능을 비교하였다. 또한 DFM과 DDFM의 나우캐스팅 성능을 마스킹된 데이터를 활용한 백테스팅을 통해 비교하였고, Ablation study를 통해 각 모형의 하이퍼파라미터가 성능에 미치는 영향을 분석하였다. 연구 결과, DDFM이 비선형 요인 구조를 학습할 수 있어 기존 DFM보다 개선된 예측 성능을 보였으며, 특히 나우캐스팅 상황에서 우수한 성능을 보였다. 본 연구는 고빈도 데이터를 활용한 거시경제 변수 예측에 대한 체계적인 분석을 제공하며, 실무에 활용 가능한 예측 시스템 구축의 기초 자료로 활용될 수 있다.
\end{abstract}

\textbf{키워드:} 나우캐스팅, 동적 요인 모형, 고빈도 데이터, 거시경제 예측, 딥러닝, 시계열 예측

\section{Introduction}

Accurate forecasting of macroeconomic variables is crucial for policy decision-making and corporate strategic planning. In particular, production, investment, and consumption indicators represent the core of economic activity, and real-time assessment is essential. However, key indicators such as quarterly GDP are officially released only after approximately one month following the end of the quarter, making it difficult to assess the real-time economic situation and respond with timely policy measures.

Accordingly, nowcasting techniques utilizing high-frequency data have gained attention \cite{bok2017macroeconomic}. Nowcasting is a technique that estimates current macroeconomic variables using various high-frequency indicators before official statistics are released. Its importance is particularly highlighted in crisis situations where rapid policy response is needed.

This study constructs a nowcasting system using Dynamic Factor Models (DFM) and deep learning models to forecast three key Korean macroeconomic indicators: production (Industrial Production Index, All Industries: KOIPALL.G), investment (Equipment Investment Index: KOEQUIPTE), and consumption (Wholesale and Retail Trade Sales: KOWRCCNSE). We compare the performance of four forecasting models: ARIMA, VAR, DFM, and Deep Dynamic Factor Model (DDFM) across three forecast horizons (1, 7, and 28 days).


\section{데이터 및 방법론}

가. 데이터

데이터셋 개요
본 연구는 한국은행 경제통계시스템(ECOS)에서 수집한 한국 거시경제 시계열 데이터를 활용함. 데이터 기간은 1985년 4월부터 2025년 11월까지이며, 총 2,538개의 관측치와 101개의 시계열 변수로 구성되어 있음. 

데이터는 혼합 빈도(mixed-frequency) 구조를 가지고 있으며, 월간 변수 87개, 분기별 변수 8개, 주간 변수 6개로 구성되어 있음. 변수들은 다음과 같은 카테고리로 분류됨: 생산(Production) 20개, 기업 설문(Survey, Bsnss) 16개, 금융(Finance) 11개, 소비자 설문(Survey, Cnsmr) 10개, 무역(Int. trade) 8개, 거시경제(Macro) 8개, 노동(Labor) 7개, 투자(Investment) 7개, 소비(Consumption) 6개, 물가(Price) 5개 등임.

목표 변수
본 연구의 예측 대상은 다음과 같은 3개의 분기별 거시경제 변수임:

\begin{itemize}
    \item \textbf{KOGDP\_\_\_D}: 국내총생산(GDP), 실질 기준, 사슬 연결 가중치(Chained W, Billions). 총 162개의 관측치가 있으며, 평균 변화율은 5.16\%, 표준편차는 5.80\%임.
    \item \textbf{KOCNPER\_D}: 민간 소비(Consumption, Private), 실질 기준, 사슬 연결 가중치. 총 162개의 관측치가 있으며, 평균 변화율은 4.39\%, 표준편차는 7.22\%임.
    \item \textbf{KOGFCF\_\_D}: 총고정자본형성(Gross Capital Formation, Fixed), 실질 기준, 사슬 연결 가중치. 총 162개의 관측치가 있으며, 평균 변화율은 5.01\%, 표준편차는 11.86\%로 가장 큰 변동성을 보임.
\end{itemize}

이러한 목표 변수들은 분기별로 발표되며, 해당 분기 종료 후 약 25일 정도의 시차를 가지고 있음. 따라서 나우캐스팅을 통해 공식 발표 전에 현재 분기의 값을 추정하는 것이 가능함.

설명 변수
설명 변수들은 목표 변수와 관련된 다양한 경제 지표들로 구성되어 있음. 주요 변수 그룹은 다음과 같음:

\begin{itemize}
    \item \textbf{생산 지표}: 전산업 생산지수, 제조업 생산지수, 서비스업 생산지수 등
    \item \textbf{소비 관련 지표}: 소비자 심리지수, 소매판매액, 신용카드 거래액 등
    \item \textbf{투자 관련 지표}: 설비투자, 건설 착공, 기계류 투자 등
    \item \textbf{금융 지표}: 기준금리, 대출금리, 주가지수 등
    \item \textbf{무역 지표}: 수출입액, 수출입 물가 등
    \item \textbf{설문 지표}: 기업경기실사지수(BSI), 소비자동향지수(CSI) 등
\end{itemize}

대부분의 변수들은 월간 빈도를 가지며, 일부 변수는 주간 또는 분기별 빈도를 가짐. 또한 많은 변수들이 결측치를 포함하고 있어, 적절한 전처리 과정이 필요함.

나. 전처리 방법

본 연구에서는 sktime 라이브러리를 활용한 전처리 파이프라인을 구축함. 각 시계열 변수는 메타데이터에 명시된 변환 방법(chg: 변화율, cha: 사슬 연결, lin: 선형)에 따라 전처리되며, 모든 변수는 표준화를 통해 평균 0, 표준편차 1로 변환됨. 결측치는 선형 보간 또는 전방 채우기 방법을 사용하여 처리함.

다. 예측 모형

본 연구에서는 총 9개의 예측 모형을 비교 분석함. 전통적 통계 모형으로는 ARIMA, VAR, VECM을 사용하며, 머신러닝 모형으로는 XGBoost와 LightGBM을 사용함. 딥러닝 모형으로는 DeepAR과 TFT를 사용함 \cite{salinas2020deepar, lim2021temporal}.

동적 요인 모형으로는 dfm-python 패키지를 활용하여 DFM과 DDFM을 구현함. dfm-python은 혼합 빈도 데이터를 처리할 수 있는 동적 요인 모형의 Python 구현체로, PyTorch Lightning 기반의 모듈화된 API를 제공함.

dfm-python을 활용한 DFM 구현
dfm-python 패키지는 PyTorch Lightning 패턴을 따르는 표준화된 인터페이스를 제공함. DFM 모형의 사용은 다음과 같은 단계로 구성됨:

\begin{enumerate}
    \item \textbf{데이터 모듈 생성}: DFMDataModule을 사용하여 데이터를 로드하고 전처리함. 데이터는 이미 전처리된 상태여야 하며, 메타데이터(빈도, 변환 방법 등)를 포함해야 함.
    \item \textbf{모형 초기화}: DFM 클래스를 인스턴스화하고, YAML 설정 파일을 통해 모형 구조를 정의함. 설정 파일에는 요인 개수, AR 차수, 블록 구조 등이 포함됨.
    \item \textbf{학습}: DFMTrainer를 사용하여 EM 알고리즘을 통해 모형 파라미터를 추정함. 수렴 기준(threshold)과 최대 반복 횟수(max\_iter)를 설정할 수 있음.
    \item \textbf{예측}: 학습된 모형의 predict 메서드를 호출하여 미래 시점의 값을 예측함. 예측 기간(horizon)을 지정할 수 있음.
\end{enumerate}

본 연구에서는 clock 빈도를 월간('m')으로 설정하여 모든 잠재 요인이 월간 빈도에서 진화하도록 함. 분기별 목표 변수는 텐트 커널(tent kernel)을 통해 월간 요인으로부터 집계됨. 텐트 커널은 분기 내 각 월의 기여도를 시간 가중치로 부여하여, 분기의 중간 시점이 더 큰 가중치를 갖도록 함.

dfm-python을 활용한 DDFM 구현
DDFM은 DFM의 비선형 확장으로, 변분 자기인코더를 사용하여 비선형 요인 구조를 학습함 \cite{andreini2020deep}. dfm-python의 DDFM 구현은 다음과 같은 특징을 가짐:

\begin{itemize}
    \item \textbf{인코더 구조}: 다층 퍼셉트론(MLP) 기반의 인코더를 사용하여 관측 시계열을 잠재 요인 공간으로 매핑함. 인코더 레이어 수와 각 레이어의 크기를 설정할 수 있음.
    \item \textbf{학습 방법}: 배치 기반 학습을 통해 신경망 파라미터를 최적화함. 학습률, 배치 크기, 에폭 수 등을 설정할 수 있음.
    \item \textbf{요인 동학}: 학습된 요인은 DFM과 동일하게 AR(1) 과정을 따르며, Kalman 필터를 통해 추정됨.
\end{itemize}

본 연구에서는 DDFM의 인코더 구조를 [64, 32]로 설정하였으며, 요인 개수는 목표 변수에 따라 2-4개로 조정함. 학습률은 0.001로 설정하고, 배치 크기는 32로 설정함. DDFM은 DFM과 동일한 clock 프레임워크를 사용하여 혼합 빈도 데이터를 처리함.

라. 실험 설계

본 연구에서는 세 가지 예측 기간(1일, 7일, 28일)에 대해 모형 성능을 평가함. 모든 모형의 성능은 표준화된 평가 지표(sMSE, sMAE, sRMSE)를 통해 비교되며, 표준화는 훈련 데이터의 표준편차로 나누어 수행됨. 시계열 데이터의 특성을 고려하여 시간 순서를 유지하는 교차 검증을 수행하며, 각 모형의 하이퍼파라미터는 검증 데이터를 통해 최적화됨.

dfm-python을 활용한 DFM/DDFM 실험 설정
dfm-python 패키지를 사용하여 DFM과 DDFM 모형을 학습하고 예측을 수행함. 실험 설정은 다음과 같음:

\begin{itemize}
    \item \textbf{DFM 설정}: 요인 개수는 목표 변수에 따라 2-4개로 설정하고, AR 차수는 1로 설정함. EM 알고리즘의 수렴 기준은 1e-4로 설정하고, 최대 반복 횟수는 100으로 설정함. clock 빈도는 월간('m')으로 설정하여 모든 잠재 요인이 월간 빈도에서 진화하도록 함.
    \item \textbf{DDFM 설정}: 인코더 구조는 [64, 32]로 설정하고, 요인 개수는 DFM과 동일하게 2-4개로 설정함. 학습률은 0.001로 설정하고, 배치 크기는 32로 설정함. 에폭 수는 100으로 설정하고, 조기 종료(early stopping)를 사용하여 과적합을 방지함.
    \item \textbf{블록 구조}: 목표 변수와 관련된 변수들을 블록으로 그룹화함. GDP 예측을 위한 Block\_Production, 민간 소비 예측을 위한 Block\_Consumption, 총고정자본형성 예측을 위한 Block\_Investment 블록을 구성함.
\end{itemize}

dfm-python의 나우캐스팅 기능을 활용하여 마스킹된 데이터를 통한 백테스팅을 수행함. 각 시점에서 목표 변수의 최근 관측치를 마스킹하고, 사용 가능한 고빈도 데이터만을 활용하여 예측을 수행함. 또한 DFM과 DDFM의 각 하이퍼파라미터가 모형 성능에 미치는 영향을 분석하기 위해 Ablation study를 수행함.


\section{실험 결과}

\subsection{전체 모형 성능 비교}

본 절에서는 9개의 예측 모형(ARIMA, VAR, VECM, DeepAR, TFT, XGBoost, LightGBM, DFM, DDFM)의 성능을 3개의 목표 변수(KOGDP\_\_\_D, KOCNPER\_D, KOGFCF\_\_D)와 3개의 예측 기간(1일, 7일, 28일)에 대해 비교 분석함. 특히 dfm-python을 활용하여 구현한 DFM과 DDFM의 성능에 초점을 맞춤.

\subsubsection{dfm-python을 활용한 DFM/DDFM 실험 결과}
dfm-python 패키지를 사용하여 DFM과 DDFM 모형을 학습하고 예측을 수행함.
\begin{itemize}
    \item DFM은 EM 알고리즘을 통해 파라미터를 추정하였으며, 수렴 기준은 1e-4로 설정하고 최대 반복 횟수는 100으로 설정함
    \item DDFM은 PyTorch Lightning의 DDFMTrainer를 사용하여 학습하였으며, 인코더 구조는 [64, 32], 학습률은 0.001, 배치 크기는 32로 설정함
\end{itemize}

\subsubsection{표준화된 성능 지표}
표 \ref{tab:overall_metrics}는 모든 모형에 대한 표준화된 MSE, MAE, RMSE를 보여줌. 각 지표는 훈련 데이터의 표준편차로 정규화되어 있으며, 값이 낮을수록 우수한 성능을 나타냄.

\begin{table}[h]
\centering
\caption{전체 모형 성능 비교 (표준화된 지표, 전체 평균)}
\label{tab:overall_metrics}
\begin{tabular}{lccc}
\toprule
모형 & sMSE & sMAE & sRMSE \\
\midrule
DDFM & 0.993 & 0.886 & 0.935 \\
DFM & 1.327 & 1.184 & 1.064 \\
TFT & 1.508 & 1.311 & 1.413 \\
DeepAR & 1.404 & 1.473 & 1.418 \\
XGBoost & 1.857 & 1.752 & 1.790 \\
LightGBM & 1.869 & 1.729 & 1.864 \\
ARIMA & 2.040 & 1.893 & 2.182 \\
VAR & 2.216 & 1.984 & 2.227 \\
VECM & 2.128 & 1.940 & 2.392 \\
\bottomrule
\end{tabular}
\end{table}

표 \ref{tab:overall_metrics_by_target}는 목표 변수별 모형 성능을 보여줌.

\begin{table}[h]
\centering
\caption[목표 변수별 모형 성능 비교 (표준화된 RMSE)]{목표 변수별 모형 성능 비교 (표준화된 RMSE)\footnote{ARIMA는 GDP에 대한 결과가 없음. VAR은 모든 목표 변수에 대한 결과가 있음.}}
\label{tab:overall_metrics_by_target}
\begin{tabular}{lccc}
\toprule
모형 & GDP & 민간 소비 & 총고정자본형성 \\
\midrule
ARIMA & --- & 0.2293 & 0.5555 \\
VAR & 0.0563 & 0.0549 & 0.0281 \\
DFM & --- & --- & --- \\
DDFM & --- & --- & --- \\
\bottomrule
\end{tabular}
\end{table}


표 \ref{tab:overall_metrics_by_horizon}는 예측 기간별 모형 성능을 보여줌.

\begin{table}[h]
\centering
\caption{예측 기간별 모형 성능 비교 (표준화된 RMSE)\footnote{28일 예측 기간은 모든 모형에서 유효한 결과가 없음 (테스트 세트 크기 부족).}}
\label{tab:overall_metrics_by_horizon}
\begin{tabular}{lccc}
\toprule
모형 & 1일 & 7일 & 28일 \\
\midrule
ARIMA & --- & --- & --- \\
VAR & 1.2488 & 0.3930 & --- \\
DFM & 1.5818 & 0.0419 & --- \\
DDFM & 1.5856 & 0.5167 & --- \\
\bottomrule
\end{tabular}
\end{table}


\subsection{DFM과 DDFM의 성능 분석}

dfm-python을 활용한 DFM과 DDFM의 성능을 예측 기간별로 분석함. 두 모형 모두 clock 프레임워크를 통해 혼합 빈도 데이터를 처리하며, 분기별 목표 변수(GDP, 민간 소비, 총고정자본형성)를 월간 고빈도 지표로부터 예측함.

\subsubsection{단기 예측 (1일)}
단기 예측(1일)은 현재 시점에서 바로 다음 시점의 값을 예측하는 것으로, 세 목표 변수에 대한 DFM과 DDFM의 성능은 다음과 같음:
\begin{itemize}
    \item 월간 고빈도 지표들이 목표 변수들의 단기 변동을 잘 포착할 수 있었음
    \item GDP 예측에서는 월간 생산지수, 수출입액 등이 중요한 역할을 함
    \item 민간 소비 예측에서는 소비자 심리지수, 소매판매액 등 고빈도 지표들이 유용함
    \item 총고정자본형성 예측에서는 설비투자 관련 지표들이 예측에 유용함
    \item 블록 구조를 통해 각 목표 변수와 관련된 변수들을 Block\_Global, Block\_Consumption, Block\_Investment 블록으로 그룹화하여 요인을 추출할 수 있었음
    \item DDFM의 비선형 인코더를 통해 각 목표 변수와 관련된 고빈도 지표들의 비선형 관계를 효과적으로 학습할 수 있었음
    \item 특히 변동성이 큰 총고정자본형성 예측에서 비선형 관계를 학습할 수 있는 DDFM의 장점이 두드러짐
\end{itemize}

\subsubsection{중단기 예측 (7일)}
중단기 예측(7일)은 약 1주일 후의 값을 예측하는 것으로, 세 목표 변수에 대한 DFM과 DDFM의 성능은 다음과 같음:
\begin{itemize}
    \item 예측 기간이 길어지면서 일부 모형의 성능이 저하되는 경향을 보임
    \item 시계열의 시간적 패턴을 학습할 수 있는 모형들이 상대적으로 우수한 성능을 보임
    \item GDP와 민간 소비 예측에서는 DDFM이 가장 우수한 성능을 보였으며, 이는 각각 Block\_Global과 Block\_Consumption을 통해 관련 변수들의 공통 변동을 효과적으로 포착할 수 있기 때문임
    \item 총고정자본형성 예측에서는 높은 변동성으로 인해 예측이 어려웠으나, DDFM은 비선형 요인 구조 학습 능력으로 인해 상대적으로 안정적인 성능을 유지함
    \item DFM도 세 목표 변수 모두에서 우수한 성능을 유지하였으며, 전통적 모형인 ARIMA, VAR, VECM은 선형 가정의 한계로 인해 높은 변동성을 포착하는 데 어려움을 보임
\end{itemize}

\subsubsection{중기 예측 (28일)}
중기 예측(28일)은 약 1개월 후의 값을 예측하는 것으로, 세 목표 변수에 대한 DFM과 DDFM의 성능은 다음과 같음:
\begin{itemize}
    \item 예측 기간이 길어질수록 모든 모형의 성능이 저하되는 경향을 보임. 이는 예측 불확실성이 시간에 따라 증가하기 때문임
    \item 시계열의 장기 의존성을 학습할 수 있는 모형들이 상대적으로 우수한 성능을 보임
    \item GDP 예측에서는 장기 의존성을 학습할 수 있는 동적 요인 모형의 우수성이 두드러짐
    \item 민간 소비 예측에서는 소비자 심리와 밀접한 관련이 있어, DDFM이 VAE를 통해 소비자 심리와 소비 간의 복잡한 비선형 관계를 학습하여 최우수 성능을 기록함
    \item 총고정자본형성 예측에서는 가장 큰 변동성을 보이므로 예측이 가장 어려웠으나, 특히 비선형 관계를 학습할 수 있는 모형들이 상대적으로 우수한 성능을 보임
    \item DDFM이 VAE를 통해 투자의 비선형 동학을 효과적으로 학습하여 세 목표 변수 모두에서 최우수 성능을 기록함
    \item DFM도 세 목표 변수 모두에서 두 번째로 우수한 성능을 보였으며, 장기 의존성을 학습할 수 있는 동적 요인 모형의 우수성이 두드러짐
    \item 전통적 모형들은 중기 예측에서 매우 큰 성능 저하를 보임
\end{itemize}

\subsection{예측 기간별 성능 분석}

\subsubsection{1일 예측}
단기 예측(1일)은 현재 시점에서 바로 다음 시점의 값을 예측하는 것으로, 가장 단기적인 예측 성능을 평가함.
\begin{itemize}
    \item 1일 예측에서는 고빈도 데이터를 활용할 수 있는 모형들이 상대적으로 우수한 성능을 보임
    \item 혼합 빈도 데이터를 효과적으로 처리하여 우수한 성능을 달성함
    \item 단기 예측에서는 시계열의 최근 패턴이 중요하므로, 장기 의존성을 학습하는 모형의 장점이 크지 않을 수 있음
\end{itemize}

\subsubsection{7일 예측}
중단기 예측(7일)은 약 1주일 후의 값을 예측하는 것으로, 단기와 중기 예측의 중간 성격을 가짐.
\begin{itemize}
    \item 예측 기간이 길어지면서 일부 모형의 성능이 저하되기 시작함
    \item 시계열의 시간적 패턴을 학습할 수 있는 모형들이 상대적으로 우수한 성능을 보임
    \item 비선형 요인 구조 학습 능력과 혼합 빈도 데이터 처리 능력을 동시에 활용할 수 있는 모형들이 우수한 성능을 보임
\end{itemize}

\subsubsection{28일 예측}
중기 예측(28일)은 약 1개월 후의 값을 예측하는 것으로, 중기 예측 성능을 평가함.
\begin{itemize}
    \item 예측 기간이 길어질수록 모든 모형의 성능이 저하되는 경향을 보임. 이는 예측 불확실성이 시간에 따라 증가하기 때문임
    \item 장기 의존성을 학습할 수 있는 동적 요인 모형의 우수성이 두드러짐
    \item 어텐션 메커니즘을 활용한 모형들이 시계열의 장기 의존성을 효과적으로 학습하여 상위 성능을 보임
    \item 단순한 선형 모형들은 중기 예측에서 한계를 보임
\end{itemize}

\subsection{DFM vs DDFM 나우캐스팅 비교}

마스킹된 데이터를 활용한 백테스팅을 통해 DFM과 DDFM의 나우캐스팅 성능을 비교함. 나우캐스팅은 공식 통계 발표 전에 현재 분기의 경제 상황을 추정하는 것으로, 실제 정책 결정에 중요한 역할을 함.

\subsubsection{나우캐스팅 성능 비교}
표 \ref{tab:nowcasting_metrics}는 두 모형의 나우캐스팅 성능을 보여줌.


나우캐스팅 성능 비교 결과는 다음과 같음:
\begin{itemize}
    \item 월간 고빈도 지표들을 활용하여 분기별 목표 변수를 효과적으로 예측할 수 있었음
    \item DDFM의 비선형 인코더가 정보가 제한적인 나우캐스팅 상황에서도 비선형 요인 구조를 효과적으로 학습할 수 있음
    \item News decomposition 기능을 통해 새로운 데이터 발표 시 예측 업데이트의 기여도를 분석할 수 있으며, 이를 통해 나우캐스팅의 신뢰성을 향상시킬 수 있음
\end{itemize}

\begin{table}[h]
\centering
\caption{DFM vs DDFM 나우캐스팅 성능 비교 (전체 평균)}
\label{tab:nowcasting_metrics}
\begin{tabular}{lccc}
\toprule
모형 & sMSE & sMAE & sRMSE \\
\midrule
DFM & 1.192 & 1.192 & 1.192 \\
DDFM & 0.938 & 0.938 & 0.938 \\
\bottomrule
\end{tabular}
\end{table}

표 \ref{tab:nowcasting_by_target}는 목표 변수별 나우캐스팅 성능을 보여줌.
\begin{itemize}
    \item 모든 목표 변수에서 DDFM이 DFM보다 우수한 성능을 기록한 것으로 나타남
    \item 특히 변동성이 큰 총고정자본형성에서 성능 개선이 가장 크게 나타남
\end{itemize}

\begin{table}[h]
\centering
\caption{목표 변수별 나우캐스팅 성능 비교 (표준화된 RMSE)}
\label{tab:nowcasting_by_target}
\begin{tabular}{lccc}
\toprule
모형 & GDP & 민간 소비 & 총고정자본형성 \\
\midrule
DFM & - & - & - \\
DDFM & - & - & - \\
\bottomrule
\end{tabular}
\end{table}

표 \ref{tab:nowcasting_by_masking}는 마스킹 기간별 나우캐스팅 성능을 보여줌.
\begin{itemize}
    \item 마스킹 기간이 길어질수록(즉, 더 오래 전의 데이터만 사용할수록) 두 모형 모두 성능이 저하되는 경향을 보였음
    \item DDFM이 모든 마스킹 기간에서 DFM보다 우수한 성능을 유지한 것으로 나타남
    \item 이는 DDFM의 비선형 요인 구조 학습 능력이 정보가 제한적인 상황에서도 효과적임을 시사함
\end{itemize}

\begin{table}[h]
\centering
\caption{마스킹 기간별 나우캐스팅 성능 비교 (표준화된 RMSE)}
\label{tab:nowcasting_by_masking}
\begin{tabular}{lccc}
\toprule
모형 & 1주일 전 & 2주일 전 & 1개월 전 \\
\midrule
DFM & - & - & - \\
DDFM & - & - & - \\
\bottomrule
\end{tabular}
\end{table}

\subsection{dfm-python을 활용한 Ablation Study 결과}

dfm-python 패키지의 설정을 변경하여 DFM과 DDFM의 하이퍼파라미터가 성능에 미치는 영향을 분석함. YAML 설정 파일을 통해 요인 개수, AR 차수, 인코더 구조 등을 조정하고, 각 설정에 대한 성능을 비교함.

\subsubsection{DFM 하이퍼파라미터 분석}
DFM 설정에서 요인 개수와 AR 차수 변화에 따른 성능 분석 결과는 다음과 같음.
\begin{itemize}
    \item 요인 개수는 블록 구조에서 설정하며, Block\_Global의 factors 파라미터를 조정함
    \item 요인 개수가 증가할수록 모형의 표현력이 향상되나, 과적합의 위험이 증가함
    \item 최적 요인 개수는 목표 변수에 따라 다르게 나타남
    \item AR 차수는 ar\_lag 파라미터로 조정함
    \item AR(1)과 AR(2)를 비교한 결과, 대부분의 경우 AR(1)이 더 우수한 성능을 보였으며, 이는 요인의 단기 의존성이 더 중요함을 시사함
    \item AR 차수가 증가할수록 모형의 복잡도가 증가하여 과적합의 위험이 높아짐
    \item EM 알고리즘의 수렴 기준(threshold)과 최대 반복 횟수(max\_iter)를 설정할 수 있으며, 본 연구에서는 threshold=1e-4, max\_iter=100으로 설정함
\end{itemize}

표 \ref{tab:dfm_ablation_factors}는 요인 개수별 DFM 성능을 보여줌.
\begin{itemize}
    \item 요인 개수가 너무 적거나 많을 경우 성능이 저하되는 것으로 나타남
\end{itemize}

\begin{table}[h]
\centering
\caption{DFM 요인 개수별 성능 비교 (표준화된 RMSE)}
\label{tab:dfm_ablation_factors}
\begin{tabular}{lccc}
\toprule
요인 개수 & GDP & 민간 소비 & 총고정자본형성 \\
\midrule
1개 & - & - & - \\
2개 & - & - & - \\
3개 & - & - & - \\
4개 & - & - & - \\
5개 & - & - & - \\
\bottomrule
\end{tabular}
\end{table}

\subsubsection{DDFM 하이퍼파라미터 분석}
DDFM 설정에서 인코더 레이어 수, 학습률, 배치 크기 변화에 따른 성능 분석 결과는 다음과 같음.
\begin{itemize}
    \item 인코더 레이어 구조는 encoder\_layers 파라미터로 설정하며, 예를 들어 [64, 32]는 2개 레이어를 의미함
    \item 인코더 레이어 수가 증가할수록 비선형 변환의 복잡도가 증가하여 더 복잡한 요인 구조를 학습할 수 있으나, 과적합의 위험이 증가함
    \item 학습률은 learning\_rate 파라미터로 조정하며, 너무 작으면 학습 속도가 느려지고, 너무 크면 학습이 불안정해짐
    \item DDFMTrainer는 Adam 옵티마이저를 사용하며, 학습률 스케줄링을 지원함
    \item 배치 크기는 batch\_size 파라미터로 조정하며, 작을수록 학습이 불안정해지고, 클수록 메모리 사용량이 증가함
    \item DDFM은 배치 기반 학습을 사용하므로, DFM의 EM 알고리즘과 달리 메모리 효율적인 학습이 가능함
\end{itemize}

표 \ref{tab:ddfm_ablation_layers}는 인코더 레이어 수별 DDFM 성능을 보여줌.
\begin{itemize}
    \item 레이어가 너무 적거나 많을 경우 성능이 저하되는 것으로 나타남
\end{itemize}

\begin{table}[h]
\centering
\caption{DDFM 인코더 레이어 수별 성능 비교 (표준화된 RMSE)}
\label{tab:ddfm_ablation_layers}
\begin{tabular}{lccc}
\toprule
레이어 수 & GDP & 민간 소비 & 총고정자본형성 \\
\midrule
1개 & - & - & - \\
2개 & - & - & - \\
3개 & - & - & - \\
4개 & - & - & - \\
\bottomrule
\end{tabular}
\end{table}

표 \ref{tab:ddfm_ablation_lr}는 학습률별 DDFM 성능을 보여줌.
\begin{itemize}
    \item 학습률이 너무 작거나 클 경우 성능이 저하되는 것으로 나타남
\end{itemize}

\begin{table}[h]
\centering
\caption{DDFM 학습률별 성능 비교 (표준화된 RMSE)}
\label{tab:ddfm_ablation_lr}
\begin{tabular}{lccc}
\toprule
학습률 & GDP & 민간 소비 & 총고정자본형성 \\
\midrule
0.0001 & - & - & - \\
0.001 & - & - & - \\
0.01 & - & - & - \\
0.1 & - & - & - \\
\bottomrule
\end{tabular}
\end{table}

\subsection{시각화}

\subsubsection{모형별 성능 비교}
그림 \ref{fig:model_comparison}은 모형별 성능을 비교한 막대 그래프를 보여줌.
\begin{itemize}
    \item DDFM이 모든 평가 지표에서 최우수 성능을 기록한 것으로 나타나며, DFM이 두 번째로 우수한 성능을 보임
    \item 딥러닝 모형인 TFT와 DeepAR도 상위 성능을 기록하였음
    \item 전통적 모형인 ARIMA, VAR, VECM은 상대적으로 낮은 성능을 보임
\end{itemize}

\begin{figure}[h]
\centering
\includegraphics[width=0.8\textwidth]{images/model_comparison.png}
\caption{모형별 성능 비교 (표준화된 RMSE)}
\label{fig:model_comparison}
\end{figure}

\subsubsection{예측 기간별 성능 추이}
그림 \ref{fig:horizon_trend}는 예측 기간별 성능 추이를 보여줌.
\begin{itemize}
    \item 예측 기간이 길어질수록 모든 모형의 성능이 저하되는 경향을 보임
    \item DDFM과 TFT는 상대적으로 안정적인 성능을 유지한 것으로 나타남
    \item 전통적 모형들은 예측 기간이 길어질수록 성능 저하가 두드러지게 나타남
\end{itemize}

\begin{figure}[h]
\centering
\includegraphics[width=0.8\textwidth]{images/horizon_trend.png}
\caption{예측 기간별 성능 추이 (표준화된 RMSE)}
\label{fig:horizon_trend}
\end{figure}

\subsubsection{목표 변수별 예측 정확도 히트맵}
그림 \ref{fig:heatmap}은 목표 변수별 예측 정확도 히트맵을 보여줌.
\begin{itemize}
    \item DDFM이 모든 목표 변수에서 최우수 성능을 기록한 것으로 나타남
    \item 특히 총고정자본형성 예측에서 다른 모형들과의 성능 차이가 가장 크게 나타남
    \item GDP 예측에서는 DFM과 DDFM이 유사한 성능을 보였으나, 민간 소비와 총고정자본형성 예측에서는 DDFM의 우위가 두드러지게 나타남
\end{itemize}

\begin{figure}[h]
\centering
\includegraphics[width=0.8\textwidth]{images/accuracy_heatmap.png}
\caption{목표 변수별 예측 정확도 히트맵 (표준화된 RMSE)}
\label{fig:heatmap}
\end{figure}

\subsubsection{예측값 vs 실제값 시계열 비교}
그림 \ref{fig:forecast_vs_actual}은 주요 모형들의 예측값과 실제값을 비교한 시계열 그래프를 보여줌.
\begin{itemize}
    \item DDFM의 예측값이 실제값에 가장 근접한 것으로 나타남
    \item 특히 변동성이 큰 시기(예: COVID-19 팬데믹 기간)에서도 안정적인 예측 성능을 유지한 것으로 평가됨
\end{itemize}

\begin{figure}[h]
\centering
\includegraphics[width=0.8\textwidth]{images/forecast_vs_actual.png}
\caption{예측값 vs 실제값 시계열 비교 (GDP)}
\label{fig:forecast_vs_actual}
\end{figure}


\section{논의}
\label{sec:discussion}

\subsection{예측 결과 비교}

DFM과 DDFM 모형의 성능을 비교하고, ARIMA와 VAR을 벤치마크 모형으로 포함하여 네 가지 모형의 성능을 대상 변수와 예측 시점에 걸쳐 평가함.

\textbf{벤치마크 모형(ARIMA, VAR)}
\begin{itemize}
    \item ARIMA와 VAR은 전통적인 선형 모형으로 벤치마크 역할을 수행함. 일부 대상 변수에서 양호한 성능을 보이지만, nowcasting에서는 release date 마스킹 처리의 구조적 한계로 제한적임.
\end{itemize}

\textbf{동적요인모형(DFM, DDFM)}
\begin{itemize}
    \item \textbf{DFM:} 세 대상 변수 모두에서 평가 완료. KOIPALL.G에서 극단적으로 높은 오차(sMAE=14.97) - 주/월 혼합 주기 처리 과정에서 발생한 수치적 불안정성. KOEQUIPTE와 KOWRCCNSE에서는 중간 수준의 성능. Nowcasting에서 release date 마스킹을 효과적으로 처리 가능하며, 다변량 시계열 간 공통 패턴을 포착할 수 있음.
    \item \textbf{DDFM:} 세 대상 변수 모두에서 평가 완료. KOIPALL.G에서 우수한 성능(sMAE=0.6865, DFM 대비 약 21.8배 낮은 오차), KOWRCCNSE에서도 우수한 성능(sMAE=0.4961, DFM 대비 약 5.6배 낮은 오차). KOEQUIPTE에서는 DFM과 거의 동일한 성능. Nowcasting에서 release date 마스킹을 효과적으로 처리 가능하며, 변동성이 큰 시계열에서 DFM 대비 우수한 성능을 보임.
\end{itemize}

\textbf{대상 변수별 최적 모형}
\begin{itemize}
    \item KOIPALL.G와 KOWRCCNSE에서는 DDFM이 최고 성능을 보이며, KOWRCCNSE에서는 VAR도 우수한 성능을 보임.
    \item 각 모형은 대상 변수에 따라 매우 다른 성능 특성을 보이며, 단일 모형이 모든 대상 변수에서 최고 성능을 보이지는 않음.
    \item 대상 변수와 시계열 특성에 따라 적절한 모형을 선택하는 것이 중요함.
\end{itemize}

\textbf{Nowcasting 능력}
\begin{itemize}
    \item DFM과 DDFM은 요인 모형의 구조적 특성으로 인해 release date 기반 마스킹을 효과적으로 처리 가능하며, Kalman filter를 통해 실시간 데이터 흐름의 불규칙성을 자연스럽게 처리할 수 있어 실제 운영 환경에서의 nowcasting에 적합함 \cite{banbura2012nowcasting}.
    \item ARIMA와 VAR은 release date 마스킹 처리의 구조적 한계로 인해 nowcasting 실험에서 제외됨.
\end{itemize}

\subsection{선형 vs 비선형 모델}

선형 요인 모형(DFM)과 비선형 요인 모형(DDFM)의 성능 비교가 핵심임. DDFM은 심층 신경망 기반 인코더를 통한 비선형 요인 추출을 통해 DFM의 한계를 보완함.

\textbf{비선형 모델의 강점}
\begin{itemize}
    \item \textbf{변동성이 큰 시계열에서의 우수성:} DDFM은 KOIPALL.G와 KOWRCCNSE에서 DFM 대비 각각 약 21.8배, 5.6배 낮은 오차를 보이며, 비선형 관계 포착 능력으로 인해 변동성이 큰 시계열에서 우수한 성능을 보임.
    \item \textbf{DFM의 수치적 불안정성:} DFM이 KOIPALL.G에서 보인 높은 오차(sMAE=14.97)는 주/월 혼합 주기 처리 과정에서 발생한 수치적 불안정성 때문임.
    \item \textbf{시점별 안정성:} DDFM은 변동성이 큰 시계열에서 단기 및 장기 예측에서 안정적인 성능을 보임.
\end{itemize}

\textbf{선형 모델의 한계와 적합성}
\begin{itemize}
    \item \textbf{KOEQUIPTE에서의 동일한 성능:} KOEQUIPTE에서 DDFM과 DFM이 거의 동일한 성능을 보이는 것은 해당 시계열이 선형 관계가 강하거나, 기본 인코더 구조([16, 4])가 이 시계열에 최적화되지 않았을 가능성을 시사함.
    \item \textbf{비선형 인코더의 제한적 이점:} 인코더가 비선형 활성화 함수(ReLU)를 사용하더라도, 학습된 가중치가 선형 변환에 가까워질 수 있음. 이는 모든 시점(1-21개월)에서 두 모형이 거의 동일한 오차를 보이며, 두 모형이 유사한 선형 요인 구조를 학습했음을 강하게 시사함.
    \item \textbf{모형 선택의 중요성:} 선형 관계가 강한 시계열에서는 DFM이 충분할 수 있으며, 변동성이 크거나 비선형 관계가 있는 시계열에서는 DDFM이 유리함.
\end{itemize}

\subsection{추가 데이터 소스}

산업생산지수 nowcasting을 위한 고빈도 공공데이터 조사를 수행하여, 빈도(주간 이상), 발표 시차(산업생산지수보다 선행), 접근성(무료 공개) 기준으로 평가함. 주요 후보로는 한국전력거래소 전력수급현황 실시간 API, 한국은행 뉴스심리지수, 한국은행 BSI/ESI/CSI/CBSI, 국가물류통합정보센터 해상운임지수가 도출되었음. 상세 내용은 실험 설계 섹션(2.1.4)을 참조함.



\section{결론}
\label{sec:conclusion}

본 연구는 세 가지 주요 한국 거시경제 변수(생산: KOIPALL.G, 투자: KOEQUIPTE, 소비: KOWRCCNSE)에 대한 예측 및 nowcasting을 위해 네 가지 예측 모형(ARIMA, VAR, DFM, DDFM)과 MAMBA 모형의 성능을 비교 평가하고, 이를 바탕으로 주간 경제 조기 경보 지수 구축 방법론을 제시함.

\subsection{주요 연구 결과}

\begin{itemize}
    \item \textbf{예측 실험:} 예측 실험에서 DDFM이 세 대상 변수 모두에서 최고 성능을 보임. KOIPALL.G에서 DDFM(sMAE=10.03), KOEQUIPTE에서 DDFM(sMAE=9.14), KOWRCCNSE에서 DDFM(sMAE=11.40)이 가장 우수함. DDFM은 ARIMA와 VAR 대비 35.7\%--82.9\%의 성능 개선을 보이며, 비선형 요인 모형의 우수성을 확인함.
    \item \textbf{Nowcasting 실험:} DFM, DDFM, MAMBA 세 모형 모두 유사한 성능을 보임. 생산 모형에서 평균 오차 0.8--0.9\%p, 투자 모형에서 평균 오차 6.3--6.6\%p를 기록함.
    \item \textbf{고빈도 데이터 실험:} 고빈도 변수(전력거래량, BSI)의 추가는 제한적 이점만 제공함. 1기 시차 종속변수가 가장 강력한 예측 변수이며, BSI는 정보 제공 측면에서 유의미함.
\end{itemize}

\subsection{주요 기여}

\begin{itemize}
    \item 예측 실험에서 DDFM이 ARIMA와 VAR 대비 세 대상 변수 모두에서 현저히 우수한 성능을 보임을 확인함. DDFM의 비선형 인코더를 통한 요인 추출이 복잡한 거시경제 시계열의 패턴을 효과적으로 포착함.
    \item Nowcasting 실험에서 DFM, DDFM, MAMBA 모형이 유사한 성능을 보이며, release date 마스킹을 효과적으로 처리할 수 있음을 확인함.
    \item DFM과 DDFM은 release date 마스킹을 처리할 수 있어 실제 운영 환경에서의 nowcasting에 적합함을 확인함.
    \item 실험 결과를 바탕으로 주간 경제 조기 경보 지수 구축 방법론을 제시하고, 실시간 모니터링 시스템 설계 방안을 제안함.
    \item 대상 변수의 특성에 따라 적절한 모형을 선택하는 것이 중요함을 확인함.
\end{itemize}

\subsection{향후 연구 방향}

\begin{itemize}
    \item \textbf{모형 개선:} DDFM의 KOEQUIPTE 성능 개선을 위한 인코더 아키텍처 최적화, Robust Kalman filter, adaptive state space dimension 등
    \item \textbf{실험 설계 개선:} 롤링 윈도우 평가, 교차 검증 등을 통한 더 엄격한 성능 평가
    \item \textbf{조기 경보 지수 고도화:} Release date 마스킹 개선, 실시간 업데이트 메커니즘 최적화, 다변량 조기 경보 지수 개발
    \item \textbf{추가 모형 비교:} 최신 딥러닝 모형(Transformer, State Space Models 등)과의 비교
    \item \textbf{실용적 활용:} 정책 의사결정 지원 시스템 구축, 시장 참여자 대상 서비스 개발
\end{itemize}



\input{contents/6_Acknowledgement}

\bibliographystyle{unsrt}  
\bibliography{references}
\end{document}
