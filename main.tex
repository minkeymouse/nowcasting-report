\documentclass[12pt]{article}

\usepackage{arxiv}
\usepackage{float}
\usepackage{graphicx}
\usepackage[utf8]{inputenc} % allow utf-8 input
\usepackage[T1]{fontenc}    % use 8-bit T1 fonts
\usepackage{hyperref}       % hyperlinks
\usepackage{url}            % simple URL typesetting
\usepackage{booktabs}       % professional-quality tables
\usepackage{amsfonts}       % blackboard math symbols
\usepackage{nicefrac}       % compact symbols for 1/2, etc.
\usepackage{microtype}      % microtypography
\usepackage{lipsum}
\usepackage{amsmath}

\title{Nowcasting Production and Investment Sector with High Frequency Data Integration}

\author{
  JaeYoung Kim\\
   affiliation\\
  address\\
  \texttt{email} \\
   \And
 SeoJung Lee \\
   affiliation\\
  address\\
  \texttt{email} \\
   \And
  YoungMin Kim \\
   affiliation\\
  address\\
  \texttt{email} \\
   \And
  Minkey Chang \\
   affiliation\\
  address\\
  \texttt{email} \\
   \And
  JunHo Hwang \\
   affiliation\\
  address\\
  \texttt{email} \\
   \And
  EunKyu Sung \\
   affiliation\\
  address\\
  \texttt{email} \\
   \And
  %% \And
  %% Coauthor \\
  %% Affiliation \\
  %% Address \\
  %% \texttt{email} \\
  %% \And
  %% Coauthor \\
  %% Affiliation \\
  %% Address \\
  %% \texttt{email} \\
}

\begin{document}
% Title without authors
\begin{center}
\Large\textbf{대한민국 거시경제 Nowcasting: 고빈도 데이터 통합과 동태요인모형 분석을 중심으로}
\end{center}
\vspace{1cm}

\begin{abstract}
본 연구는 세 가지 주요 한국 거시경제 변수에 대한 나우캐스팅을 위해 네 가지 예측 모형(ARIMA, VAR, 동적요인모형, 심층 동적요인모형)의 성능을 비교한다. 대상 변수는 생산(전산업생산지수: KOIPALL.G), 투자(설비투자지수: KOEQUIPTE), 소비(도소매판매액: KOWRCCNSE)이다. 모형들은 22개 예측 시점(1개월부터 22개월까지)에서 표준화된 지표를 사용하여 평가되며, 각 시점에 대한 지표를 평균하여 최종 성능 지표로 사용한다. 이를 통해 서로 다른 시계열 규모 간 공정한 비교가 가능하다. 실험적 평가를 통해 모형 성능을 대상 변수와 예측 시점에 걸쳐 제시한다.
\end{abstract}

\textbf{키워드:} 나우캐스팅, 동적요인모형, 고빈도 데이터, 거시경제 예측, 딥러닝

\section{Introduction}

Accurate forecasting of macroeconomic variables is crucial for policy decision-making and corporate strategic planning. In particular, production, investment, and consumption indicators represent the core of economic activity, and real-time assessment is essential. However, key indicators such as quarterly GDP are officially released only after approximately one month following the end of the quarter, making it difficult to assess the real-time economic situation and respond with timely policy measures.

Accordingly, nowcasting techniques utilizing high-frequency data have gained attention \cite{bok2017macroeconomic}. Nowcasting is a technique that estimates current macroeconomic variables using various high-frequency indicators before official statistics are released. Its importance is particularly highlighted in crisis situations where rapid policy response is needed.

This study constructs a nowcasting system using Dynamic Factor Models (DFM) and deep learning models to forecast three key Korean macroeconomic indicators: production (Industrial Production Index, All Industries: KOIPALL.G), investment (Equipment Investment Index: KOEQUIPTE), and consumption (Wholesale and Retail Trade Sales: KOWRCCNSE). We compare the performance of four forecasting models: ARIMA, VAR, DFM, and Deep Dynamic Factor Model (DDFM) across three forecast horizons (1, 7, and 28 days).


\section{실험 설계}
\label{sec:experiment_setup}

\subsection{데이터}

\textbf{예측 실험 데이터}
\begin{itemize}
    \item \textbf{대상 변수:} 생산(전산업생산지수: KOIPALL.G), 투자(설비투자지수: KOEQUIPTE), 소비(도소매판매액: KOWRCCNSE) 3개 변수
    \item \textbf{생산 부문 모형:} 총 41개 변수로 구성됨. 고용, 산업생산, 서베이(기업경기, 소비자 동향) 등 주요 월간 지수와 주간 데이터를 포함함. 기업경기동향 조사는 해당월 중 발표되어 속보성이 높으며, 주가지수 등 금융변수, 뉴스심리지수, 미국 경제정책불확실성 지수를 포함함. 상세 변수 구성은 부록의 표~\ref{tab:production_variables}를 참조함.
    \item \textbf{투자 부문 모형:} 총 41개 변수로 구성됨. 고용, 설비투자, 건설 등 주요 지표와 속보성 높은 서베이(기업경기, 소비자 동향) 등 주요 월간 지수와 주간 데이터 9개를 포함함. 고빈도 데이터는 주가지수 등 금융변수, 뉴스심리지수, 미국 경제정책불확실성 지수 및 투자관련 섹터 주가지수를 포함함. 상세 변수 구성은 부록의 표~\ref{tab:investment_variables}를 참조함.
    \item \textbf{모형:} ARIMA, VAR, DFM, DDFM 4개 모형을 비교함
\end{itemize}

\textbf{Nowcasting 실험 데이터}
\begin{itemize}
    \item \textbf{대상 변수:} 생산(전산업생산지수: KOIPALL.G), 투자(설비투자지수: KOEQUIPTE) 2개 변수
    \item \textbf{데이터 구성:} 생산 및 투자 부문 모형의 변수 구성을 활용하며, 주간 및 월간 데이터를 혼합하여 사용함. 혼합 주기 처리 방법은 동적요인모형 섹션(2.2.1)을 참조함.
    \item \textbf{모형:} DFM, DDFM, MAMBA 모형을 활용함
    \item \textbf{평가 시점:} 각 목표 월에 대해 4주 전, 1주 전 시점에서 예측을 수행하며, 시리즈별 발표 시차(publication lag)를 기준으로 미발표 데이터를 마스킹함
\end{itemize}

\textbf{고빈도 데이터 실험 데이터}
\begin{itemize}
    \item \textbf{종속변수:} 월별 전산업생산지수(계절조정)의 전월대비 성장률 및 전년동월비
    \item \textbf{설명변수:} 주별 전력거래량(로그--STL 계절조정 후 주간 성장률), 월별 BSI(수준 및 전년동월비)
    \item \textbf{표본 분할:} Train(2002--2020년), Validation(2021--2022년), Test(2023--2024년)
    \item \textbf{Vintage:} h0(전월 말), h1--h4(당월 1--4주)
    \item \textbf{모형:} MIDAS-AR(1), AR(1) 벤치마크, 선형 ARX, XGBoost 기반 비선형 모형을 비교함
\end{itemize}

\textbf{데이터 전처리}
\begin{itemize}
    \item \textbf{변환:} 각 시계열의 특성에 따라 적절한 변환을 적용함. 변환 유형은 다음과 같음:
    \begin{itemize}
        \item \textbf{lin (linear):} 변환 없이 원본 수준값 사용
        \item \textbf{log:} 로그 변환으로 비율 변화를 선형화하고 분산 안정화
        \item \textbf{chg (change):} 전기 대비 차분으로 정상성 확보 (월간: 1개월 차분, 주간: 1주 차분, 분기: 1분기 차분)
        \item \textbf{ch1:} 전년동기 대비 차분으로 계절성 제거 (월간: 12개월 차분, 주간: 52주 차분, 분기: 4분기 차분)
        \item \textbf{pch (percent change):} 전기 대비 성장률 (백분율, 1기 시차)
        \item \textbf{pc1:} 전년동기 대비 성장률 (백분율, 연간 시차)
        \item \textbf{cha (change annualized):} 연율화 차분 변환
        \item \textbf{pca (percent change annualized):} 연율화 성장률 변환
    \end{itemize}
    각 시계열의 변환 유형은 시계열의 특성(수준값/성장률, 계절성, 추세 등)을 고려하여 설정되며, 시계열별 설정 파일(config/series/\{series\_id\}.yaml)에서 관리됨. 변환은 원본 시계열의 주파수(주간/월간/분기)에서 적용되며, 이후 주파수 변환(리샘플링)이 수행됨. 예를 들어, 월간 시계열에 대해 'ch1' 변환을 적용하면 12개월 차분이 계산되고, 주간 시계열에 대해서는 52주 차분이 계산됨.
    \item \textbf{주파수 변환:} 모형의 요구사항에 따라 주파수 변환이 수행됨. ARIMA와 VAR 모형의 경우 주간 데이터를 월간으로 리샘플링하며, 이는 각 월에 속한 주간 관측값의 평균을 사용함. DFM과 DDFM의 경우, 모형 설정에 따라 주간 클럭(clock='w')을 사용하면 주간 데이터를 그대로 유지하고, 월간 클럭을 사용하면 주간 데이터를 월간으로 리샘플링함. 혼합 주기 모형의 경우 주간 클럭을 사용하며, tent kernel을 통해 자동으로 주간/월간 데이터를 통합 처리함.
    \item \textbf{결측치 처리:} forward-fill $\to$ backward-fill $\to$ naive forecaster 순차 적용. 먼저 전방 채움(forward-fill)을 적용하고, 여전히 결측치가 남아있는 경우 후방 채움(backward-fill)을 적용하며, 마지막으로 naive forecaster(마지막 관측값 사용)를 적용함. 모든 방법을 적용한 후에도 결측치가 남아있는 행은 제거됨. 결측치 처리는 변환 및 주파수 변환 이후에 수행되며, 각 시계열별로 독립적으로 적용됨.
    \item \textbf{인덱스 정규화:} 시계열 데이터의 인덱스를 DatetimeIndex로 정규화하고, 중복된 날짜가 있는 경우 마지막 값을 유지함. 주파수 정보가 없는 경우 자동으로 추론하여 설정함.
    \item \textbf{표준화:}
    \begin{itemize}
        \item ARIMA/VAR: 원본 스케일 유지 (표준화 미적용)
        \item DFM/DDFM: RobustScaler 적용. RobustScaler는 중앙값(median)을 0으로, 사분위수 범위(IQR, Interquartile Range)를 1로 조정하여 이상치에 강건한 표준화를 수행함. 이는 StandardScaler(평균과 표준편차 기반) 대비 이상치의 영향을 덜 받아 안정적인 전처리를 제공함. 표준화는 변환, 주파수 변환, 결측치 처리 이후에 수행되며, 훈련 데이터에 맞춰 학습된 스케일러를 테스트 데이터에 동일하게 적용함.
    \end{itemize}
    \item \textbf{혼합 주기 처리:} 주간 데이터와 월간 데이터를 함께 활용하는 경우, DFM의 기본 주파수를 주간('w')으로 설정하고 tent kernel을 통해 자동으로 혼합 주기 변환이 수행됨. 월간 데이터는 주간 인덱스에 배치되며(월말에 해당하는 주에 배치), tent kernel이 주간 데이터를 월간 수준으로 집계함. 이 과정에서 Mariano \& Murasawa (2003) 방법을 따르며, 월간 전월대비 상승률은 주간 전월비 상승률 4개의 평균값으로 표현됨.
\end{itemize}

\textbf{추가 데이터셋(실험 미활용)}
\begin{itemize}
    \item 산업생산지수 nowcasting을 위한 고빈도 공공데이터 조사를 수행하였으며, 실험에 직접 활용하지 않은 추가 데이터 소스들을 정리함. 상세 내용은 부록 A를 참조함.
    \item 주요 후보: 한국전력거래소 전력수급현황 API, 한국은행 뉴스심리지수, NLIC 주별 해상운임지수, 한국은행 BSI/ESI/CSI/CBSI, 항만 물동량 통계 등
\end{itemize}

\subsection{예측 모형}

\subsubsection{동적요인모형}
\begin{itemize}
    \item 동적요인모형(DFM)은 많은 시계열에서 공통 요인을 추출해 소수의 동태적 요인으로 설명하는 대표적 차원축소 기법으로, 관측식과 상태식을 갖는 state-space 형태를 취함 \cite{stock2002forecasting}. 대규모 이질적 거시 지표 간의 공분산 구조를 소수 요인으로 집약해 수십~수백 개 변수의 동시 예측이 가능하며, Kalman filter를 통해 누락·비동기 데이터(혼합주기, jagged edges)를 자연스럽게 처리할 수 있다는 점에서 나우캐스팅에 핵심적으로 활용됨 \cite{banbura2012nowcasting, bok2019frbny}.
    \item DFM의 기본 구조는 다음과 같음:
    \begin{align}
    y_t &= \lambda_i' f_t + e_t \\
    f_t &= A_1 f_{t-1} + A_2 f_{t-2} + A_3 f_{t-3} + A_4 f_{t-4} + u_t
    \end{align}
    여기서 $y_t$는 관측 데이터, $f_t$는 은닉 요인(latent factors) 벡터임.
    \item DFM은 state-space 형태로 표현되며, measurement equation과 transition equation으로 구성됨. EM 알고리즘으로 파라미터 추정, 칼만 필터와 스무더로 요인 추정 \cite{bok2019frbny}. 칼만 필터는 실시간 데이터 흐름을 재귀적으로 처리하여 각 시점의 예측을 업데이트하며, 데이터의 품질과 시의성을 기반으로 가중치를 부여함. 이는 nowcasting에 특히 유용한 특성으로, 비동기적 데이터 발표와 결측치를 자연스럽게 처리할 수 있음 \cite{banbura2012nowcasting}.
    \item DFM 모형에서 요인 식별을 위한 factor loading 제약 가정이 nowcasting 성과를 저해하는 요소로 추정됨에 따라, 요인식별 가정 없이 DFM 모형을 추정하고 nowcasting 성과를 측정함. 요인 개수는 Ahn \& Horenstein (2013)의 Eigenvalue ratio 테스트 등을 참고하여 설정함 \cite{ahn2013eigenvalue}.
    \item \textbf{주/월 혼합 주기 처리:} 주간 데이터와 월간 데이터를 함께 활용하는 혼합 주기 모형을 사용함. DFM의 기본 frequency를 주간('w')으로 설정하고, dfm-python의 mixed\_freq=True 옵션을 통해 tent kernel이 자동으로 적용됨. 혼합주기 변환은 Mariano \& Murasawa (2003) 방법을 따름 \cite{mariano2003new}.
    \begin{itemize}
        \item 월간 지수의 전월대비 상승률은 주간지수의 전월비 상승률 4개의 평균값으로 표현됨:
        \begin{align}
        y_t^m &= \frac{1}{4} y_t + \frac{1}{4} y_{t-1} + \frac{1}{4} y_{t-2} + \frac{1}{4} y_{t-3} = \frac{1}{4}(I + L + L^2 + L^3)y_t \\
        \Delta y_t^m &= y_t^m - y_{t-4}^m = \frac{1}{4}(I + L + L^2 + L^3)(y_t - L^4 y_t) \\
        &= \frac{1}{4} \Delta^4 y_t + \frac{1}{4} \Delta^4 y_{t-1} + \frac{1}{4} \Delta^4 y_{t-2} + \frac{1}{4} \Delta^4 y_{t-3}
        \end{align}
        여기서 $y_t^m$는 월간 수준값, $\Delta y_t^m$는 월간 전월대비 상승률, $y_t$는 주간 수준값, $\Delta^4 y_t$는 주간 전월비 상승률, $L$은 시차 연산자임.
        \item 혼합주기 모형에서 분기 성장률은 Mariano \& Murasawa (2003)에 따라 시차(lag) 4개 월간 성장률의 가중합으로 표현됨 \cite{mariano2003new}:
        \begin{align}
        y_t^q &= \frac{1}{3} y_t + \frac{1}{3} y_{t-1} + \frac{1}{3} y_{t-2} = \frac{1}{3}(I + L + L^2)y_t \\
        \Delta y_t^q &= y_t^q - y_{t-3}^q = \frac{1}{3}(I + L + L^2)(y_t - L^3 y_t) \\
        &= \frac{1}{3}(I + L + L^2)(\Delta y_t + \Delta y_{t-1} + \Delta y_{t-2}) \\
        &= \frac{1}{3} \Delta y_t + \frac{2}{3} \Delta y_{t-1} + \frac{3}{3} \Delta y_{t-2} + \frac{2}{3} \Delta y_{t-3} + \frac{1}{3} \Delta y_{t-4}
        \end{align}
        여기서 $y_t^q$는 분기 수준값, $\Delta y_t^q$는 분기 성장률, $y_t$는 월간 수준값, $\Delta y_t$는 월간 성장률, $L$은 시차 연산자임.
        \item 분기지수의 전분기 대비 성장률은 주간지수 상승률 20개 시차의 가중 평균값으로 표현됨:
        \begin{align}
        \Delta y_t^q &= \frac{1}{3}(I + L + L^2)(\Delta y_t^m + \Delta y_{t-1}^m + \Delta y_{t-2}^m) \\
        &= \frac{1}{3} \Delta y_t^m + \frac{2}{3} \Delta y_{t-1}^m + \frac{3}{3} \Delta y_{t-2}^m + \frac{2}{3} \Delta y_{t-3}^m + \frac{1}{3} \Delta y_{t-4}^m \\
        &= \frac{1}{12}(\Delta^4 y_t + \Delta^4 y_{t-1} + \Delta^4 y_{t-2} + \Delta^4 y_{t-3}) + \frac{1}{12}(\Delta^4 y_{t-4} + \cdots) \\
        &\quad + \frac{3}{12}(\Delta^4 y_{t-8} + \cdots) + \frac{2}{12}(\Delta^4 y_{t-12} + \cdots) + \frac{1}{12}(\Delta^4 y_{t-16} + \cdots + \Delta^4 y_{t-19})
        \end{align}
        따라서 공동요인 5개, 잔차항의 자기회귀, 5개 주간 지표, 30개 월간 지표, 6개 분기 데이터로 구성된 DFM 모형을 가정하면, $20 \times 5 + (5 + 4 \times 30 + 20 \times 6) = 345$개의 상태변수가 필요하여 모형 추정이 어려워짐. 주간과 월간 데이터만을 활용하여 모형 복잡성을 관리함.
    \end{itemize}
\end{itemize}

\subsubsection{DDFM}
\begin{itemize}
    \item 심층 동적요인모형(DDFM)은 오토인코더 기반 비선형 인코더를 사용해 요인 구조를 학습함으로써 전통적 DFM의 선형 가정을 완화한다 \cite{andreini2020deep}. 비선형 인코더는 고차원 거시 데이터의 복잡한 상호작용을 더 적은 요인으로 포착하면서도, 요인층 뒤에는 여전히 선형 state-space(예: VAR(1))를 두어 필터링·스무딩 안정성을 유지한다.
    \item DDFM은 인코더를 통해 관측 변수에서 잠재 요인을 추출하고, 디코더를 통해 요인에서 관측 변수로 재구성함. 이 과정에서 선형 DFM의 제약을 완화하여 더 복잡한 요인 구조를 학습할 수 있음. 대규모 데이터셋에서도 효과적으로 작동하며, 전통적인 DFM의 계산적 한계를 극복함.
    \item DDFM의 성능 개선을 위해 대상 변수별 인코더 아키텍처 최적화, 활성화 함수 선택(tanh), Huber 손실 함수, 가중치 감쇠, 그래디언트 클리핑, 향상된 가중치 초기화, 증가된 사전 훈련, 배치 크기 최적화 등을 적용함.
\end{itemize}

\subsubsection{MIDAS}
\begin{itemize}
    \item MIDAS(Mixed Data Sampling)는 서로 다른 주기의 데이터를 통합하여 예측하는 모형으로, 고빈도 데이터(주간, 일간)와 저빈도 데이터(월간)를 함께 활용함 \cite{ghysels2004midas, clements2008macroeconomic}.
    \item MIDAS-AR 모형을 수행하여 고빈도 지표의 단일변수 예측에서의 활용 가능성을 탐색함. MIDAS-AR 모형은 exp-Almon 가중치를 사용하여 고빈도 변수를 저빈도 종속변수에 매핑함. 이를 통해 주별 전력거래량과 같은 고빈도 지표를 월별 산업생산지수 예측에 활용할 수 있음.
    \item MIDAS-AR(1) 모형의 기본 구조는 다음과 같음:
    \begin{align}
    y_t &= \lambda y_{t-1} + \beta_0 + \beta_1 Z_t(K,\theta) + \varepsilon_t
    \end{align}
    여기서 $y_t$는 월별 종속변수(전산업생산지수 성장률), $\lambda$는 AR(1) 계수, $Z_t(K,\theta)$는 고빈도 성장률의 가중 합임.
    \item exp-Almon 가중치는 다음과 같이 정의됨:
    \begin{align}
    Z_t(K,\theta) &= \sum_{k=1}^K w_k(\theta) x_{t,k} \\
    w_k(\theta_1,\theta_2) &= \frac{\exp(\theta_1 k + \theta_2 k^2)}{\sum_{j=1}^K \exp(\theta_1 j + \theta_2 j^2)}, \quad k=1,\dots,K
    \end{align}
    여기서 $x_{t,k}$는 고빈도 설명변수(주별 전력거래량 성장률 등)의 래그 변수이며, $K$는 사용하는 고빈도 래그 개수임. $\theta_2 < 0$ 제약을 통해 오래된 래그의 가중치가 감소하도록 유도함.
    \item Clements \& Galvão (2008)의 추정 절차를 따름:
    \begin{enumerate}
        \item 1단계: Standard MIDAS (AR 없음)로 $(\beta_0,\beta_1,\theta)$ 추정하여 초기값 획득
        \item 2단계: 잔차의 AR(1) 계수 $\lambda^{(0)}$ 추정
        \item 3단계: $\lambda^{(0)}$ 고정하여 MIDAS-AR 재추정
        \item 4단계: Full MIDAS-AR 공동 추정으로 $(\lambda,\beta_0,\beta_1,\theta)$ 최종 추정
    \end{enumerate}
\end{itemize}

\subsubsection{MAMBA}
\begin{itemize}
    \item MAMBA는 시계열 모델링을 위한 최신 딥러닝 아키텍처로, 선택적 상태 공간 모델(Selective State Space Model)을 기반으로 선형 시간 복잡도로 장기 의존성을 효과적으로 포착함 \cite{gu2024mamba}. DFM과 동일한 데이터를 이용하여 MAMBA 모형으로 nowcasting을 수행함. MAMBA는 상태 공간 모델의 구조적 특성을 유지하면서도 비선형 선택 메커니즘을 통해 입력에 따라 상태 전이를 동적으로 조정하여, 시계열 데이터의 복잡한 패턴을 학습할 수 있음.
\end{itemize}

\subsection{실험 구성}

\subsubsection{예측 실험}
\begin{itemize}
    \item 과거 데이터로 미래 값 예측. 각 모형 훈련 후 1--22개월에 대해 예측 생성.
    \item \textbf{전통적 선형 모델:} ARIMA와 VAR 모형을 포함함. ARIMA와 VAR은 재귀적(recursive) 방식으로 다단계 예측을 수행함. 1-step ahead 예측값을 다음 단계의 입력으로 사용하여 순차적으로 예측을 생성하므로, 예측 오차가 누적되어 장기 예측에서 불안정성이 증가함.
    \item \textbf{동적요인 모형:} DFM과 DDFM 모형을 포함함. DFM과 DDFM은 state-space 구조를 활용하여 잠재 요인 상태를 업데이트한 후 직접 다단계 예측을 생성함 \cite{bok2019frbny}. 칼만 필터가 데이터를 재귀적으로 처리하여 예측을 업데이트하되, 각 예측 시점에서 요인의 품질과 시의성에 기반한 가중치를 부여하므로 오차 누적이 완화됨 \cite{banbura2012nowcasting}.
\end{itemize}

\subsubsection{Nowcasting 실험}
\begin{itemize}
    \item 공식 통계 발표 전 현재 시점 거시경제 변수 추정 \cite{banbura2012nowcasting}. 각 목표 월에 대해 4주 전, 1주 전 시점에서 예측을 수행하며, 시리즈별 발표 시차(publication lag)를 기준으로 미발표 데이터를 마스킹함.
    \item DFM, DDFM, MAMBA는 Kalman filter 또는 state-space 구조를 통해 이러한 비동기적 데이터 발표와 결측치를 자연스럽게 처리할 수 있어 nowcasting에 특히 적합함.
\end{itemize}

\subsubsection{고빈도 변수 실험}
\begin{itemize}
    \item 주요 모형(DFM, DDFM)은 다변량 고차원 데이터를 활용하는 반면, 고빈도 지표(전력거래량, BSI)의 단일변수 예측에서의 활용 가능성을 탐색하기 위해 실험을 수행함.
    \item 이 실험은 전산업생산지수(KOIPALL.G) 단일변수에 대해 MIDAS-AR(1), AR(1) 벤치마크, 선형 ARX, 그리고 XGBoost 기반 비선형 모형을 비교함. 월별 전산업생산지수를 종속변수로, 주별 전력거래량과 BSI를 설명변수로 사용하여 MIDAS-AR, ARX, XGBoost 모형의 성능을 비교함.
\end{itemize}



\section{실험 결과}
\label{sec:experiment_result}

\subsection{예측 실험 결과}
\label{sec:results_forecasting}


세 가지 대상 변수(생산: KOIPALL.G, 투자: KOEQUIPTE, 소비: KOWRCCNSE)에 대한 네 가지 예측 모형(ARIMA, VAR, DFM, DDFM)의 예측 성능을 비교함.

\begin{itemize}
    \item \textbf{실험 범위:} 1개월부터 22개월까지의 시점에 대해 수행
    \item \textbf{결과 제시:} 표~\ref{tab:forecasting_results}에는 모든 시점(1-22개월)에 대한 평균값을 제시
\end{itemize} 


\textbf{결과 요약}
\begin{itemize}
    \item 표~\ref{tab:forecasting_results}는 모형별 타겟별로 모든 시점(1-22개월)에 대한 평균 표준화된 MAE와 MSE를 제시함
    \item 각 셀은 해당 모형-타겟 조합의 평균 지표값을 나타내며, 각 지표에서 최소값(최고 성능)은 굵은 글씨로 표시됨
    \item 상세한 시점별 결과는 부록에 제시됨
\end{itemize}

\begin{table}[h]
\centering
\caption[Forecasting Results by Model-Horizon and Target-Metric]{Forecasting Results by Model-Horizon and Target-Metric\footnote{Experiments evaluate all horizons from 1 to 22 months (2024--01 to 2025--10), but table shows only selected horizons (1, 11, 22 months) for readability. Full results for all horizons are available in aggregated\_results.csv.}}
\\label{tab:forecasting_results}
\\begin{tabular}{lcccccc}
\\toprule
Model-Horizon & KOIPALL.G & KOIPALL.G & KOEQUIPTE & KOEQUIPTE & KOWRCCNSE & KOWRCCNSE \\\\
 & sMAE & sMSE & sMAE & sMSE & sMAE & sMSE \\\\
\\midrule
ARIMA-1 & N/A & N/A & 0.8734 & 0.7628 & N/A & N/A \\
ARIMA-11 & N/A & N/A & 2.0917 & 4.3751 & N/A & N/A \\
ARIMA-22 & N/A & N/A & 0.0846 & 0.0071 & N/A & N/A \\
VAR-1 & N/A & N/A & 0.2998 & 0.0899 & N/A & N/A \\
VAR-11 & N/A & N/A & 2.5679 & 6.5939 & N/A & N/A \\
VAR-22 & N/A & N/A & 0.1881 & 0.0354 & N/A & N/A \\
DFM-1 & N/A & N/A & 0.4890 & 0.2391 & N/A & N/A \\
DFM-11 & N/A & N/A & 2.2917 & 5.2519 & N/A & N/A \\
DFM-22 & N/A & N/A & 0.1139 & 0.0130 & N/A & N/A \\
DDFM-1 & N/A & N/A & 0.7574 & 0.5736 & N/A & N/A \\
DDFM-11 & N/A & N/A & 2.0233 & 4.0938 & N/A & N/A \\
DDFM-22 & N/A & N/A & 0.1545 & 0.0239 & N/A & N/A \\
\bottomrule
\end{tabular}
\end{table}


\textbf{전체 시점 평균 성능 (표~\ref{tab:forecasting_results})}
\begin{itemize}
    \item \textbf{KOIPALL.G:} DDFM이 가장 낮은 sMAE(0.6865, 21개 시점 평균)와 sMSE(0.61)를 보여 우수한 성능을 보임. VAR(sMAE=0.94, sMSE=1.11)도 양호한 성능을 보이며, DFM(sMAE=14.9689, sMSE=225.30)은 매우 높은 오차를 보여 KOIPALL.G에 대해서는 부적합함.
    \item \textbf{KOEQUIPTE:} DFM과 DDFM이 거의 동일한 성능을 보이며(sMAE: DFM=1.1439, DDFM=1.1441, 평균 차이 0.000187, 21개 시점; sMSE: DFM=2.115, DDFM=2.115, 차이 0.0003), VAR(sMAE=1.37, sMSE=2.97)이 상대적으로 높은 오차를 보임.
    \item \textbf{KOWRCCNSE:} VAR이 가장 낮은 sMAE(0.32)와 sMSE(0.20)를 보여 우수한 성능을 보이며, DDFM(sMAE=0.4961, sMSE=0.49)도 양호한 성능을 보임. DFM(sMAE=2.7848, sMSE=8.21)은 상대적으로 높은 오차를 보임.
\end{itemize}

\textbf{시점별 성능 패턴}
\begin{itemize}
    \item \textbf{KOIPALL.G:} DDFM은 단기(1-6개월)에서 매우 우수한 성능을 보이며, 장기(13-21개월)에서도 안정적임. 반면 DFM은 모든 시점에서 극단적으로 높은 오차를 보임.
    \item \textbf{KOWRCCNSE:} VAR은 단기에서 우수하나 일부 시점에서 오차가 급증하며, DDFM은 대부분의 시점에서 안정적임.
    \item \textbf{KOEQUIPTE:} DFM과 DDFM은 모든 시점에서 거의 동일한 성능을 보임.
\end{itemize}

\subsection{Nowcasting 실험 결과}

\textbf{Nowcasting 결과}
\begin{itemize}
    \item \textbf{생산 모형(전산업생산지수):} DFM, DDFM, MAMBA 세 모형 모두 유사한 정확도를 보였음. DFM의 nowcasting 평균 오차는 발표 8주전 1.2\%p, 4주전 0.6\%p, 1주전 0.6\%p이며 1~8주 전 오차 평균값은 0.9\%p임. DDFM과 MAMBA도 유사한 성능을 보였으며, MAMBA의 nowcasting 평균 오차는 발표 8주전 0.9\%p, 4주전 0.8\%p, 1주전 0.8\%p이며 1~8주 전 오차 평균값은 0.8\%p임. MAMBA의 월별 전망값 변동이 DFM 모형보다 작게 나타남.
    \item \textbf{투자 모형(설비투자지수):} DFM, DDFM, MAMBA 세 모형 모두 유사한 성능을 보였음. DFM의 nowcasting 평균 오차는 발표 8주전 6.4\%p, 4주전 6.3\%p, 1주전 6.3\%p이며 1~8주 전 오차 평균값은 6.3\%p임. MAMBA 모형의 nowcasting 성과가 DFM 대비 소폭 부진하였으며, 평균 절대 예측오차는 8주전 6.7\%p, 4주전 6.6\%p, 1주전 6.6\%p이며 1~8주 평균값은 6.6\%p로 DFM 대비 0.3\%p 증가함.
\end{itemize}

\begin{figure}[h]
\centering
\includegraphics[width=0.9\textwidth]{images/nowcast/production_nowcast_compare.png}
\caption{생산 모형(전산업생산지수) Nowcasting 비교: DFM, DDFM, MAMBA 모형의 예측값과 실제값 비교.}
\label{fig:production_nowcast_compare}
\end{figure}

\begin{figure}[h]
\centering
\includegraphics[width=0.9\textwidth]{images/nowcast/production_nowcast_ensemble.png}
\caption{생산 모형(전산업생산지수) Nowcasting 앙상블: 모형별 예측값과 앙상블 결과.}
\label{fig:production_nowcast_ensemble}
\end{figure}

\begin{figure}[h]
\centering
\includegraphics[width=0.9\textwidth]{images/nowcast/investment_nowcast_compare.png}
\caption{투자 모형(설비투자지수) Nowcasting 비교}
\label{fig:investment_nowcast_compare}
\end{figure}

\begin{figure}[h]
\centering
\includegraphics[width=0.9\textwidth]{images/nowcast/investment_nowcast_ensemble.png}
\caption{투자 모형(설비투자지수) Nowcasting 앙상블: 모형별 예측값과 앙상블 결과.}
\label{fig:investment_nowcast_ensemble}
\end{figure}

\subsection{고빈도 데이터 실험 결과}

고빈도 데이터(주별 전력거래량, BSI)를 활용한 MIDAS-AR 및 XGBoost 모형의 예측 성능을 보고함. 다변량 고차원 모형(DFM, DDFM)과는 달리, 고빈도 지표의 단일변수 예측에서의 활용 가능성을 탐색하기 위한 실험임.

\textbf{실험 설계}
\begin{itemize}
    \item \textbf{종속변수:} 월별 전산업생산지수(계절조정)의 전월대비 성장률 및 전년동월비
    \item \textbf{설명변수:} 주별 전력거래량(로그--STL 계절조정 후 주간 성장률), 월별 BSI(수준 및 전년동월비)
    \item \textbf{표본 분할:} Train(2002--2020년), Validation(2021--2022년), Test(2023--2024년)
    \item \textbf{Vintage:} h0(전월 말), h1--h4(당월 1--4주)
\end{itemize}

\textbf{MIDAS-AR 모형 결과}
\begin{itemize}
    \item \textbf{전월대비 성장률:} ADF 검정 결과 정상성 가정 가능. $h0$--$h3$에서 MIDAS-AR(1)의 RMSE가 AR(1)보다 약간 열악하며, $h4$에서만 약 0.5\% RMSE 감소로 소폭 개선. AR(1)만으로도 단기 예측력이 높으며, 주별 전력거래량 추가는 full month 정보($h4$)에서만 약간의 개선을 보임.
    \item \textbf{전년동월비:} ADF 검정 결과 정상성 가정 가능. $h0$, $h3$에서 MIDAS-AR(1)의 RMSE가 AR(1)보다 약 0.7\% 악화, $h1$에서 약 7.4\% 악화, $h2$에서 약 1.3\% 개선(크기 작음), $h4$에서 두 모형 RMSE 동일. 대부분의 vintage에서 AR(1) 대비 개선이 없거나 악화됨.
    \item 상세 결과는 부록 표~\ref{tab:midasar_rmse_table}, 표~\ref{tab:midasar_rmse_yoy} 참조.
\end{itemize}

\begin{figure}[htbp]
\centering
\begin{subfigure}[b]{0.48\textwidth}
\centering
\includegraphics[width=\textwidth]{images/midas/midasar_rmse.png}
\caption{전월대비 성장률}
\label{fig:midasar_rmse}
\end{subfigure}
\hfill
\begin{subfigure}[b]{0.48\textwidth}
\centering
\includegraphics[width=\textwidth]{images/midas/midasar_mom_rmse.png}
\caption{전년동월비}
\label{fig:midasar_mom_rmse}
\end{subfigure}

\caption{Vintage별 테스트 RMSE 비교: 전산업생산지수. AR(1)과 MIDAS-AR(1) 모형의 vintage별 예측 성능을 비교함.}
\label{fig:midasar_rmse_comparison}
\end{figure}

\textbf{XGBoost 비선형 확장 결과}
\begin{itemize}
    \item \textbf{모형 구성:} (1) 선형 ARX: AR(1) + 고빈도 feature의 선형 효과, (2) AR(1)+XGB\_residual: AR(1) 잔차에 대한 XGBoost 보정, (3) XGB-direct: $(y_{t-1}, x_{t,h})$를 입력으로 $y_t$ 직접 예측
    \item \textbf{전월대비 성장률:} ARX는 대부분 vintage에서 AR(1) 대비 비슷하거나 약간 열악하며, $h4$에서만 약 1.4\% RMSE 감소. AR(1)+XGB\_residual는 모든 vintage에서 RMSE가 AR(1)보다 7--11\% 증가하여 과적합 경향. XGB-direct는 대부분 vintage에서 AR(1) 대비 2.6--4.5\% 성능 저하, $h3$, $h4$에서 각각 약 0.3\%, 0.2\% 개선(크기 매우 작음).
    \item \textbf{전년동월비:} ARX는 모든 vintage에서 AR(1) 대비 RMSE 감소율이 음수(약 $-1.6\%\sim -6.4\%$). AR(1)+XGB\_residual는 모든 vintage에서 감소율이 약 $-2.3\%\sim -6.7\%$. XGB-direct는 $h0$에서만 AR(1) 대비 약 4.4\% RMSE 감소, $h1$--$h4$에서는 모두 음수(약 $-2.3\%\sim -8.6\%$).
    \item 상세 결과는 부록 표~\ref{tab:rmse-xgb}, 표~\ref{tab:xgb_rmse_yoy}, 표~\ref{tab:arx_bsi} 참조.
\end{itemize}

\textbf{변수 중요도 및 요약}
\begin{itemize}
    \item \textbf{전력거래량:} 대부분의 모형에서 계수·Gain 모두 작고 비유의, AR(1) 대비 RMSE 개선 거의 없음. 다양한 변환·모형에도 불구하고 한계적 기여에 머묾.
    \item \textbf{BSI:} 선형 ARX와 XGBoost에서 전월·동월 BSI 변수들의 계수 및 Gain이 상대적으로 큼. 인샘플 적합과 feature importance 관점에서 의미 있는 정보 제공. 다만 테스트 RMSE 기준으로는 AR(1) 대비 뚜렷한·일관된 예측력 개선까지는 이어지지 않음.
    \item \textbf{핵심 변수:} 두 종속변수 모두에서 가장 일관된 설명력은 1기 시차 종속변수 $y_{t-1}$에서 나옴. Nowcasting에서 핵심 변수는 $y_{t-1}$과 BSI 계열 변수이며, 전력 변수는 부차적 설명 변수로 정리됨.
\end{itemize}



\section{논의}
\label{sec:discussion}

\subsection{예측 결과 비교}

DFM과 DDFM 모형의 성능을 비교하고, ARIMA와 VAR을 벤치마크 모형으로 포함하여 네 가지 모형의 성능을 대상 변수와 예측 시점에 걸쳐 평가함.

\textbf{벤치마크 모형(ARIMA, VAR)}
\begin{itemize}
    \item ARIMA와 VAR은 전통적인 선형 모형으로 벤치마크 역할을 수행함. 일부 대상 변수에서 양호한 성능을 보이지만, nowcasting에서는 release date 마스킹 처리의 구조적 한계로 제한적임.
\end{itemize}

\textbf{동적요인모형(DFM, DDFM)}
\begin{itemize}
    \item \textbf{DFM:} 세 대상 변수 모두에서 평가 완료. KOIPALL.G에서 극단적으로 높은 오차(sMAE=14.97) - 주/월 혼합 주기 처리 과정에서 발생한 수치적 불안정성. KOEQUIPTE와 KOWRCCNSE에서는 중간 수준의 성능. Nowcasting에서 release date 마스킹을 효과적으로 처리 가능하며, 다변량 시계열 간 공통 패턴을 포착할 수 있음.
    \item \textbf{DDFM:} 세 대상 변수 모두에서 평가 완료. KOIPALL.G에서 우수한 성능(sMAE=0.6865, DFM 대비 약 21.8배 낮은 오차), KOWRCCNSE에서도 우수한 성능(sMAE=0.4961, DFM 대비 약 5.6배 낮은 오차). KOEQUIPTE에서는 DFM과 거의 동일한 성능. Nowcasting에서 release date 마스킹을 효과적으로 처리 가능하며, 변동성이 큰 시계열에서 DFM 대비 우수한 성능을 보임.
\end{itemize}

\textbf{대상 변수별 최적 모형}
\begin{itemize}
    \item KOIPALL.G와 KOWRCCNSE에서는 DDFM이 최고 성능을 보이며, KOWRCCNSE에서는 VAR도 우수한 성능을 보임.
    \item 각 모형은 대상 변수에 따라 매우 다른 성능 특성을 보이며, 단일 모형이 모든 대상 변수에서 최고 성능을 보이지는 않음.
    \item 대상 변수와 시계열 특성에 따라 적절한 모형을 선택하는 것이 중요함.
\end{itemize}

\textbf{Nowcasting 능력}
\begin{itemize}
    \item DFM과 DDFM은 요인 모형의 구조적 특성으로 인해 release date 기반 마스킹을 효과적으로 처리 가능하며, Kalman filter를 통해 실시간 데이터 흐름의 불규칙성을 자연스럽게 처리할 수 있어 실제 운영 환경에서의 nowcasting에 적합함 \cite{banbura2012nowcasting}.
    \item ARIMA와 VAR은 release date 마스킹 처리의 구조적 한계로 인해 nowcasting 실험에서 제외됨.
\end{itemize}

\subsection{선형 vs 비선형 모델}

선형 요인 모형(DFM)과 비선형 요인 모형(DDFM)의 성능 비교가 핵심임. DDFM은 심층 신경망 기반 인코더를 통한 비선형 요인 추출을 통해 DFM의 한계를 보완함.

\textbf{비선형 모델의 강점}
\begin{itemize}
    \item \textbf{변동성이 큰 시계열에서의 우수성:} DDFM은 KOIPALL.G와 KOWRCCNSE에서 DFM 대비 각각 약 21.8배, 5.6배 낮은 오차를 보이며, 비선형 관계 포착 능력으로 인해 변동성이 큰 시계열에서 우수한 성능을 보임.
    \item \textbf{DFM의 수치적 불안정성:} DFM이 KOIPALL.G에서 보인 높은 오차(sMAE=14.97)는 주/월 혼합 주기 처리 과정에서 발생한 수치적 불안정성 때문임.
    \item \textbf{시점별 안정성:} DDFM은 변동성이 큰 시계열에서 단기 및 장기 예측에서 안정적인 성능을 보임.
\end{itemize}

\textbf{선형 모델의 한계와 적합성}
\begin{itemize}
    \item \textbf{KOEQUIPTE에서의 동일한 성능:} KOEQUIPTE에서 DDFM과 DFM이 거의 동일한 성능을 보이는 것은 해당 시계열이 선형 관계가 강하거나, 기본 인코더 구조([16, 4])가 이 시계열에 최적화되지 않았을 가능성을 시사함.
    \item \textbf{비선형 인코더의 제한적 이점:} 인코더가 비선형 활성화 함수(ReLU)를 사용하더라도, 학습된 가중치가 선형 변환에 가까워질 수 있음. 이는 모든 시점(1-21개월)에서 두 모형이 거의 동일한 오차를 보이며, 두 모형이 유사한 선형 요인 구조를 학습했음을 강하게 시사함.
    \item \textbf{모형 선택의 중요성:} 선형 관계가 강한 시계열에서는 DFM이 충분할 수 있으며, 변동성이 크거나 비선형 관계가 있는 시계열에서는 DDFM이 유리함.
\end{itemize}

\subsection{추가 데이터 소스}

산업생산지수 nowcasting을 위한 고빈도 공공데이터 조사를 수행하여, 빈도(주간 이상), 발표 시차(산업생산지수보다 선행), 접근성(무료 공개) 기준으로 평가함. 주요 후보로는 한국전력거래소 전력수급현황 실시간 API, 한국은행 뉴스심리지수, 한국은행 BSI/ESI/CSI/CBSI, 국가물류통합정보센터 해상운임지수가 도출되었음. 상세 내용은 실험 설계 섹션(2.1.4)을 참조함.



\section{결론}
\label{sec:conclusion}

본 연구는 세 가지 주요 한국 거시경제 변수(생산: KOIPALL.G, 투자: KOEQUIPTE, 소비: KOWRCCNSE)에 대한 예측 및 nowcasting을 위해 네 가지 예측 모형(ARIMA, VAR, DFM, DDFM)과 MAMBA 모형의 성능을 비교 평가하고, 이를 바탕으로 주간 경제 조기 경보 지수 구축 방법론을 제시함.

\subsection{주요 연구 결과}

\begin{itemize}
    \item \textbf{예측 실험:} 예측 실험에서 DDFM이 세 대상 변수 모두에서 최고 성능을 보임. KOIPALL.G에서 DDFM(sMAE=10.03), KOEQUIPTE에서 DDFM(sMAE=9.14), KOWRCCNSE에서 DDFM(sMAE=11.40)이 가장 우수함. DDFM은 ARIMA와 VAR 대비 35.7\%--82.9\%의 성능 개선을 보이며, 비선형 요인 모형의 우수성을 확인함.
    \item \textbf{Nowcasting 실험:} DFM, DDFM, MAMBA 세 모형 모두 유사한 성능을 보임. 생산 모형에서 평균 오차 0.8--0.9\%p, 투자 모형에서 평균 오차 6.3--6.6\%p를 기록함.
    \item \textbf{고빈도 데이터 실험:} 고빈도 변수(전력거래량, BSI)의 추가는 제한적 이점만 제공함. 1기 시차 종속변수가 가장 강력한 예측 변수이며, BSI는 정보 제공 측면에서 유의미함.
\end{itemize}

\subsection{주요 기여}

\begin{itemize}
    \item 예측 실험에서 DDFM이 ARIMA와 VAR 대비 세 대상 변수 모두에서 현저히 우수한 성능을 보임을 확인함. DDFM의 비선형 인코더를 통한 요인 추출이 복잡한 거시경제 시계열의 패턴을 효과적으로 포착함.
    \item Nowcasting 실험에서 DFM, DDFM, MAMBA 모형이 유사한 성능을 보이며, release date 마스킹을 효과적으로 처리할 수 있음을 확인함.
    \item DFM과 DDFM은 release date 마스킹을 처리할 수 있어 실제 운영 환경에서의 nowcasting에 적합함을 확인함.
    \item 실험 결과를 바탕으로 주간 경제 조기 경보 지수 구축 방법론을 제시하고, 실시간 모니터링 시스템 설계 방안을 제안함.
    \item 대상 변수의 특성에 따라 적절한 모형을 선택하는 것이 중요함을 확인함.
\end{itemize}

\subsection{향후 연구 방향}

\begin{itemize}
    \item \textbf{모형 개선:} DDFM의 KOEQUIPTE 성능 개선을 위한 인코더 아키텍처 최적화, Robust Kalman filter, adaptive state space dimension 등
    \item \textbf{실험 설계 개선:} 롤링 윈도우 평가, 교차 검증 등을 통한 더 엄격한 성능 평가
    \item \textbf{조기 경보 지수 고도화:} Release date 마스킹 개선, 실시간 업데이트 메커니즘 최적화, 다변량 조기 경보 지수 개발
    \item \textbf{추가 모형 비교:} 최신 딥러닝 모형(Transformer, State Space Models 등)과의 비교
    \item \textbf{실용적 활용:} 정책 의사결정 지원 시스템 구축, 시장 참여자 대상 서비스 개발
\end{itemize}



\section*{Appendix}

\subsection*{A. 추가 데이터 소스}

아래 표는 관측 빈도, 발표 시차, 접근성을 기준으로 주요 후보들을 요약한 것임. 여기서는 전체 후보 데이터를 상세히 제시함.

\begin{table}[htbp]
\centering
\small
\setlength{\tabcolsep}{4pt}
\renewcommand{\arraystretch}{1.0}

\begin{tabular}{p{2.5cm} p{3cm} p{3.5cm} p{3cm} p{3cm}}
\toprule
유형 & 데이터 & 빈도/발표시차 & 접근성/비용 & 비고 \\
\midrule
기업 실적 &
상장사 재무데이터 &
연간, 공시 후 지연 &
유료 상용 &
nowcasting 부적합 \\
\midrule
전력 &
KEPCO 전력판매량 &
월별, 업데이트 불규칙 &
무료 xlsx &
선행지표로 한계 \\
\midrule
전력 &
KPX 전력계량(1회성) &
시간별(과거 2013--23) &
무료 csv &
모형 학습용 \\
\midrule
전력 &
KPX 전력거래량 &
시간별, 2001--24,
차기 업로드 2026 예정 &
무료 csv,
발전기 단위 비공개 &
실시간 갱신 부재 \\
\midrule
전력 &
KPX 전력수급현황 API &
5분 실시간 &
무료, 코딩 필요 &
핵심 고빈도 후보 \\
\midrule
ESG/환경 &
굴뚝 TMS &
실시간, 사업장 단위 &
무료 API &
보조 변수로 유망 \\
\midrule
ESG/환경 &
에어코리아 &
실시간, 측정소 단위 &
무료 API &
주요 변수 비권장 \\
\midrule
ESG/환경 &
해양수질자동측정망 &
실시간, 정점 단위 &
무료 API &
특정 연안 산업에 한정 \\
\midrule
텍스트 &
뉴스심리지수 &
일별, 거의 동시 공표 &
무료, 공개 &
핵심 텍스트 지표 \\
\midrule
심리지수 &
BSI/ESI/CSI/CBSI &
월별, 참조월 말 발표 &
무료 &
산업생산보다 3--5주 선행 \\
\midrule
물류 &
항만 물동량 &
월별, 익월 22일 전후 &
무료, 엑셀 &
공표 일정 확인 필요 \\
\midrule
물류 &
화물 운송량 &
연간, 공표 시차 큼 &
무료 &
nowcasting 부적합 \\
\bottomrule
\end{tabular}

\caption{관측 빈도, 발표 시차, 접근성 기준 주요 데이터 요약 (1)}
\label{tab:high_freq_summary_1}
\end{table}

\begin{table}[htbp]
\centering
\small
\setlength{\tabcolsep}{4pt}
\renewcommand{\arraystretch}{1.0}

\begin{tabular}{p{2.5cm} p{3cm} p{3.5cm} p{3cm} p{3cm}}
\toprule
유형 & 데이터 & 빈도/발표시차 & 접근성/비용 & 비고 \\
\midrule
텍스트/뉴스 &
BIGKinds
(뉴스 빅데이터) &
기사 실시간 수집,
키워드 트렌드 일/주/월 집계 &
웹 서비스 무료,
일부 공공 API,
원문 크롤링 제약 &
키워드별 주간 기사수 지수,
뉴스량 선행지표 \\
\midrule
물류 &
국내 해상운임 지수 &
월별, 매월 업데이트 &
무료·공개 (NLIC) &
국내 해상운송 비용 구조 반영,
월간 경기 보조지표 \\
\midrule
물류 &
국외 해상운임 지수 &
주별, 매주 업데이트,
IPI보다 크게 선행 &
무료·공개 (NLIC) &
글로벌 교역·물동량 선행지표,
주간 nowcasting 후보 \\
\midrule
물류 &
항공화물 물동량
(공항별·수출입) &
월별, 전월 자료가
익월 중 공표 &
무료·공개 (NLIC),
엑셀 다운로드 &
수출지향 제조업 활동 보조지표,
IPI보다 비슷하거나 약간 선행 \\
\bottomrule
\end{tabular}

\caption{관측 빈도, 발표 시차, 접근성 기준 주요 데이터 요약 (2)}
\label{tab:high_freq_summary_2}
\end{table}





\subsection*{B. 경제변수 예측 실험 세부사항}

본 부록에서는 모든 예측 시점(1개월부터 22개월까지)에 대한 예측 결과를 요약함. 상세 결과는 본문의 결과 섹션과 표~\ref{tab:appendix_forecasting_all}에 제시되어 있음.

\subsubsection*{All Targets (Averaged)}

\begin{table}[h]
\centering
\caption{Forecasting Results: All Targets (Average)}
\label{tab:appendix_forecasting_all}
\small
\begin{tabular}{lcccccccc}
\toprule
Horizon & \multicolumn{2}{c}{ARIMA} & \multicolumn{2}{c}{VAR} & \multicolumn{2}{c}{DFM} & \multicolumn{2}{c}{DDFM} \\
 & sMAE & sMSE & sMAE & sMSE & sMAE & sMSE & sMAE & sMSE \\
\midrule
1 & 0.42 & 0.28 & 0.59 & 0.47 & 3.37 & 19.02 & 0.36 & 0.34 \\
2 & 1.44 & 2.14 & 1.01 & 1.07 & 5.96 & 61.80 & 1.27 & 1.74 \\
3 & 0.89 & 0.83 & 1.19 & 1.82 & 11.53 & 261.88 & 1.06 & 1.18 \\
4 & 0.59 & 0.50 & 0.43 & 0.40 & 10.03 & 261.67 & 0.42 & 0.28 \\
5 & 0.34 & 0.16 & 0.65 & 0.58 & 12.44 & 443.98 & 0.51 & 0.32 \\
6 & 0.25 & 0.16 & 0.31 & 0.13 & 15.03 & 614.58 & 0.30 & 0.12 \\
7 & 0.96 & 2.12 & 1.19 & 2.61 & 17.93 & 831.60 & 1.03 & 1.95 \\
8 & 0.83 & 0.99 & 1.00 & 1.59 & 18.61 & 1007.50 & 0.82 & 1.02 \\
9 & 0.48 & 0.45 & 0.57 & 0.44 & 21.50 & 1285.34 & 0.55 & 0.39 \\
10 & 0.44 & 0.20 & 0.43 & 0.25 & 22.99 & 1514.38 & 0.37 & 0.16 \\
11 & 0.41 & 0.24 & 0.64 & 0.70 & 25.46 & 1849.10 & 0.52 & 0.40 \\
12 & 1.02 & 1.38 & 1.06 & 1.49 & 26.87 & 2016.16 & 0.91 & 1.02 \\
13 & 1.45 & 3.46 & 1.90 & 6.02 & 29.46 & 2445.11 & 1.56 & 3.97 \\
14 & 1.65 & 4.36 & 1.66 & 5.09 & 31.30 & 2650.48 & 1.54 & 3.96 \\
15 & 0.51 & 0.39 & 0.43 & 0.30 & 32.06 & 2932.40 & 0.38 & 0.21 \\
16 & 0.30 & 0.12 & 0.41 & 0.26 & 34.12 & 3329.40 & 0.36 & 0.18 \\
17 & 0.52 & 0.40 & 0.90 & 1.13 & 35.38 & 3674.27 & 0.68 & 0.65 \\
18 & 0.71 & 0.70 & 0.87 & 0.95 & 36.18 & 3845.26 & 0.70 & 0.58 \\
19 & 1.05 & 1.27 & 0.86 & 0.95 & 38.64 & 4224.81 & 0.94 & 1.14 \\
20 & 0.41 & 0.43 & 0.44 & 0.21 & 40.14 & 4585.16 & 0.52 & 0.41 \\
21 & 1.08 & 1.94 & 1.12 & 2.11 & 41.56 & 4829.65 & 0.93 & 1.58 \\
22 & 1.60 & 2.55 & 1.46 & 2.54 & 0.91 & 0.83 & 1.68 & 2.83 \\
\bottomrule
\end{tabular}
\end{table}




\subsection*{C. 나우캐스팅 실험 세부사항}

본 부록에서는 나우캐스팅 실험의 세부사항을 제시함.

\subsubsection*{혼합주기 DFM 모형의 어려움}

주/월 데이터 혼합주기 DFM 모형은 상태변수 개수가 증가하지만, 주간 데이터를 전월대비 증가율로 변환하고 월간 데이터를 주간 상태변수의 4주 평균값으로 처리함으로써 모형 복잡성을 관리함. 주간 데이터와 월간 데이터를 함께 활용하는 혼합 주기 모형을 사용하며, DFM의 기본 frequency를 주간('w')으로 설정함.

따라서 공동요인 5개, 잔차항의 자기회귀, 5개 주간 지표, 30개 월간 지표, 6개 분기 데이터로 구성된 DFM 모형을 가정하면, $20 \times 5 + (5 + 4 \times 30 + 20 \times 6) = 345$개의 상태변수가 필요하여 모형 추정이 어려워짐.

\subsubsection*{Big 데이터 DFM 모형 구성}

\begin{table}[htbp]
\centering
\small
\begin{tabular}{lcc}
\toprule
구분 & 변수 & 개수 \\
\midrule
거시 & 실질 GDP, 소비, 민간총투자, 건설투자, 설비투자, 정부지출 & 8 \\
투자 & 설비투자, 기계수주액, 건설기성액 & 7 \\
산업생산 & 산업생산(총계, 주요 산업, 서비스업), 가동률, 출하/재고지수 & 23 \\
소비 & 소매판매, 신용카드매출액 & 6 \\
수출입 & 수출금액, 수입금액, 품목별 수출입 금액 & 7 \\
노동 & 실업률, 고용자수, 근로시간 & 7 \\
물가 & 소비자 물가, 생산자 물가, 수출입 물가, 원자재 가격 & 8 \\
서베이 & 기업경기조사, 소비자동향조사, 기업경기동향조사(한경협) & 35 \\
금융 & 가계/기업대출 잔액, 금리, 단기금리, 신용스프레드, 주가, 환율 & 9 \\
\midrule
합계 & & 110 \\
\bottomrule
\end{tabular}
\caption{Big 데이터 DFM 모형 변수 구성 (110개)}
\label{tab:big_data_variables}
\end{table}


\subsubsection*{생산 부문 모형 변수 구성}

\begin{table}[htbp]
\centering
\small
\begin{tabular}{lllll}
\toprule
분류 & 데이터 이름 & 주기 & 변환 & 시차 \\
\midrule
금융 & 주가지수 & 주 & 전월차 & 1 \\
금융 & 국채금리 & 주 & 전월차 & 1 \\
금융 & 회사채금리 & 주 & 전월차 & 1 \\
금융 & 원달러환율 & 주 & 전월차 & 1 \\
기업경기 & 뉴스심리지수 & 주 & 전월차 & 7 \\
기업경기 & 미국 경제정책 불확실성 지수 & 주 & 전월차 & 5 \\
금융 & 코스피 전기·전자업 섹터 지수 & 월 & 전월차 & 1 \\
고용/노동 & 실업률 & 월 & 전월차 & 11 \\
고용/노동 & 취업자 수 & 월 & 전월차 & 11 \\
고용/노동 & 경제활동인구 & 월 & 전월차 & 11 \\
고용/노동 & 근로자 주당 평균 노동시간 & 월 & 전월차 & 11 \\
수출입 & 수출(FOB, 달러) & 월 & 전월차 & 1 \\
수출입 & 대중국 수출(달러) & 월 & 전월차 & 1 \\
수출입 & 수출 물량 : 반도체 & 월 & 전월차 & 28 \\
수출입 & 수출 물량 : 자동차 & 월 & 전월차 & 28 \\
수출입 & 수입(CIF, 관세기준. 달러) & 월 & 전월차 & 1 \\
수출입 & 순상품교역조건 & 월 & 전월차 & 14 \\
소비/지출 & 소매판매액지수(계절조정) & 월 & 전월차 & 28 \\
물가 & 소비자물가지수 & 월 & 전월차 & 3 \\
물가 & 생산자물가지수 & 월 & 전월차 & 20 \\
물가 & 소비자물가 : 식료품·에너지 제외 & 월 & 전월차 & 3 \\
설비투자 & 설비투자지수 & 월 & 전월차 & 30 \\
산업생산 & 제조업 출하지수 & 월 & 전월차 & 30 \\
산업생산 & 제조업 재고지수 & 월 & 전월차 & 30 \\
산업생산 & 서비스업 활동지수 & 월 & 전월차 & 30 \\
산업생산 & 전산업생산지수 & 월 & 전월차 & 30 \\
산업생산 & 광공업생산지수 & 월 & 전월차 & 30 \\
산업생산 & 생산 : 화학제품·의약·재외 & 월 & 전월차 & 30 \\
산업생산 & 생산 : 전자부품·컴퓨터·영상·통신 & 월 & 전월차 & 30 \\
산업생산 & 생산 : 자동차 및 트레일러 & 월 & 전월차 & 30 \\
산업생산 & 생산 : 기타 운송장비·조선 & 월 & 전월차 & 30 \\
산업생산 & 생산 : 건설업 & 월 & 전월차 & 30 \\
산업생산 & 경기선행지수 & 월 & 전월차 & 30 \\
산업생산 & 경기동행지수 & 월 & 전월차 & 30 \\
기업경기 & 기업경기실사지수(BSI) 종합 & 월 & 전월차 & -5 \\
기업경기 & 기업경기실사지수(BSI) 기업경기전망 & 월 & 전월차 & -5 \\
기업경기 & 경기실적(전산업) & 월 & 전월차 & -5 \\
기업경기 & 경기전망(전산업) & 월 & 전월차 & -35 \\
기업경기 & FKI 기업경기지수(전산업, 계절조정) & 월 & 전월차 & -5 \\
기업경기 & 내수 실적(전산업) & 월 & 전월차 & -5 \\
기업경기 & 경기전망(전산업, 계절조정) & 월 & 전월차 & -35 \\
기업경기 & 제조업 PMI 지수 & 월 & 전월차 & 3 \\
기업경기 & 제조업 PMI 생산 & 월 & 전월차 & 3 \\
소비자동향 & 소비자심리지수(종합) & 월 & 전월차 & -5 \\
소비자동향 & 향후 경기전망 & 월 & 전월차 & -5 \\
\midrule
합계 & & & & 41 \\
\bottomrule
\end{tabular}
\caption{생산 부문 모형 변수 구성 (41개)}
\label{tab:production_variables}
\end{table}



\subsubsection*{투자 부문 모형 변수 구성}

\begin{table}[htbp]
\centering
\small
\begin{tabular}{lllll}
\toprule
분류 & 데이터 이름 & 주기 & 변환 & 시차 \\
\midrule
금융 & 주가지수 & 주 & 전월차 & 1 \\
금융 & 국채금리 & 주 & 전월차 & 1 \\
금융 & 회사채금리 & 주 & 전월차 & 1 \\
금융 & 원달러환율 & 주 & 전월차 & 1 \\
금융 & 원자재가격 & 주 & 전월차 & 1 \\
기업경기 & 뉴스심리지수 & 주 & 전월차 & 7 \\
기업경기 & 미국 경제정책 불확실성 지수 & 주 & 전월차 & 5 \\
금융 & 코스피 건설업 지수 & 주 & 전월차 & 1 \\
금융 & 코스피 기계업 지수 & 주 & 전월차 & 1 \\
고용/노동 & 취업자 수(공공업) & 월 & 전월차 & 11 \\
고용/노동 & 취업자 수(건설업) & 월 & 전월차 & 11 \\
수출입 & 수입(자본재, 달러) & 월 & 전월차 & 14 \\
물가 & 생산자물가지수 & 월 & 전월차 & 20 \\
물가 & 생산자물가지수 : 원재료 & 월 & 전월차 & 20 \\
설비투자 & 설비투자지수(계절조정) & 월 & 전월차 & 30 \\
설비투자 & 설비투자 : 기계류 & 월 & 전월차 & 30 \\
설비투자 & 설비투자 : 운송장비 & 월 & 전월차 & 30 \\
건설 & 건설 수주액(총액, 원) & 월 & 전월차 & 30 \\
건설 & 건축 인허가 면적 & 월 & 전월차 & 31 \\
건설 & 건설 착공 면적 & 월 & 전월차 & 31 \\
건설 & 건설 준공액(총액, 원) & 월 & 전월차 & 30 \\
산업생산 & 제조업 출하 : 자본재 & 월 & 전월차 & 30 \\
산업생산 & 제조업 재고 : 자본재 & 월 & 전월차 & 30 \\
산업생산 & 서비스업 : 부동산·임대업 & 월 & 전월차 & 30 \\
산업생산 & 서비스업 : 사업시설·사업지원 & 월 & 전월차 & 30 \\
산업생산 & 광공업생산지수 & 월 & 전월차 & 30 \\
산업생산 & 생산 : 건설업 & 월 & 전월차 & 30 \\
산업생산 & 생산 : 자본재 & 월 & 전월차 & 30 \\
산업생산 & 생산 : 내구재, 계절조정 & 월 & 전월차 & 30 \\
산업생산 & 경기선행지수 & 월 & 전월차 & 30 \\
기업경기 & 기업경기실사지수(BSI) 종합 & 월 & 전월차 & -5 \\
기업경기 & 기업경기실사지수(BSI) 기업경기전망 & 월 & 전월차 & -5 \\
기업경기 & 설비투자(제조업 실적) & 월 & 전월차 & -5 \\
기업경기 & 설비투자(제조업 전망) & 월 & 전월차 & -35 \\
기업경기 & FKI 기업경기지수(전산업, 계절조정) & 월 & 전월차 & -5 \\
기업경기 & 투자 실적(전산업) & 월 & 전월차 & -5 \\
기업경기 & 투자 전망(전산업) & 월 & 전월차 & -35 \\
소비자동향 & 소비자심리지수(종합) & 월 & 전월차 & -5 \\
소비자동향 & 고용상황 전망 & 월 & 전월차 & -5 \\
금융 & 여신금융·상호금융 설비자금대출 & 월 & 전월차 & 45 \\
금융 & 기업대출금리(신규취급분) & 월 & 전월차 & 30 \\
\midrule
합계 & & & & 41 \\
\bottomrule
\end{tabular}
\caption{투자 부문 모형 변수 구성 (41개)}
\label{tab:investment_variables}
\end{table}



\subsubsection*{Nowcasting 시각화 결과}

본 절에서는 생산 모형(전산업생산지수)과 투자 모형(설비투자지수)에 대한 DFM과 MAMBA 모형의 nowcasting 결과를 시각화함. 각 모형별 예측값과 실제값을 비교하여 모형의 성능을 평가함.

\begin{figure}[htbp]
\centering
\begin{subfigure}[b]{0.8\textwidth}
\centering
\includegraphics[width=\textwidth]{images/nowcast/production_nowcast_dfm.png}
\caption{생산 모형 DFM}
\label{fig:appendix_production_dfm}
\end{subfigure}

\vspace{0.3cm}

\begin{subfigure}[b]{0.8\textwidth}
\centering
\includegraphics[width=\textwidth]{images/nowcast/production_nowcast_mamba.png}
\caption{생산 모형 MAMBA}
\label{fig:appendix_production_mamba}
\end{subfigure}

\caption{생산 모형(전산업생산지수)의 DFM과 MAMBA nowcasting 결과 비교. 각 플롯은 모형별 예측값과 실제값을 시간 순서로 비교한 그래프임.}
\label{fig:appendix_production_comparison}
\end{figure}

\begin{figure}[htbp]
\centering
\begin{subfigure}[b]{0.8\textwidth}
\centering
\includegraphics[width=\textwidth]{images/nowcast/investment_nowcast_dfm.png}
\caption{투자 모형 DFM}
\label{fig:appendix_investment_dfm}
\end{subfigure}

\vspace{0.3cm}

\begin{subfigure}[b]{0.8\textwidth}
\centering
\includegraphics[width=\textwidth]{images/nowcast/investment_nowcast_mamba.png}
\caption{투자 모형 MAMBA}
\label{fig:appendix_investment_mamba}
\end{subfigure}

\caption{투자 모형(설비투자지수)의 DFM과 MAMBA nowcasting 결과 비교. 각 플롯은 모형별 예측값과 실제값을 시간 순서로 비교한 그래프임.}
\label{fig:appendix_investment_comparison}
\end{figure}

\textbf{생산 모형(전산업생산지수) 결과}
\begin{itemize}
    \item \textbf{DFM 모형:} 그림~\ref{fig:appendix_production_dfm}에서 DFM 모형의 nowcasting 결과를 확인할 수 있음. 모형은 전반적으로 실제값을 잘 추적하며, 발표 시점에 가까워질수록 예측 정확도가 향상됨.
    \item \textbf{MAMBA 모형:} 그림~\ref{fig:appendix_production_mamba}에서 MAMBA 모형의 nowcasting 결과를 확인할 수 있음. MAMBA 모형은 DFM과 유사한 성능을 보이며, 월별 전망값 변동이 DFM 모형보다 작게 나타남.
\end{itemize}

\textbf{투자 모형(설비투자지수) 결과}
\begin{itemize}
    \item \textbf{DFM 모형:} 그림~\ref{fig:appendix_investment_dfm}에서 DFM 모형의 nowcasting 결과를 확인할 수 있음. 투자 지수는 생산 지수에 비해 변동성이 크며, 모형의 예측이 일부 구간에서 실제값과 차이를 보임.
    \item \textbf{MAMBA 모형:} 그림~\ref{fig:appendix_investment_mamba}에서 MAMBA 모형의 nowcasting 결과를 확인할 수 있음. MAMBA 모형의 성과가 DFM 대비 소폭 부진하였으나, 전반적으로 유사한 패턴을 보임.
\end{itemize}



\subsection*{D. 혼합주기 예측 실험 세부사항}

\subsubsection*{Vintage별 테스트 RMSE: AR(1) vs MIDAS-AR(1)}

\begin{table}[htbp]
\centering
\small
\begin{tabular}{lcc}
\toprule
Vintage & AR(1) & MIDAS-AR(1) \\
\midrule
h0 & 0.950 (0.0) & 0.952 (-0.2) \\
h1 & 0.950 (0.0) & 0.951 (-0.1) \\
h2 & 0.950 (0.0) & 0.952 (-0.2) \\
h3 & 0.950 (0.0) & 0.951 (-0.1) \\
h4 & 0.950 (0.0) & 0.945 (0.5)  \\
\bottomrule
\end{tabular}
\caption{Vintage별 테스트 RMSE 및 AR(1) 대비 RMSE 감소율: AR(1) vs MIDAS-AR(1) (2023--2024). 종속변수: 전산업생산지수 성장률. 괄호 안 숫자는 AR(1) 대비 RMSE 감소율(\%)임.}
\label{tab:midasar_rmse_table}
\end{table}

\subsubsection*{MIDAS-AR 모형 적합 결과: 전산업생산지수 성장률 (전월대비)}

\begin{figure}[htbp]
\centering
\includegraphics[width=0.7\textwidth]{images/midas/midasar_sample_fit.png}
\caption{MIDAS-AR 모형 인샘플 적합 결과: 전산업생산지수 성장률 (전월대비). 훈련 기간(2002--2022년)에서의 모형 적합도를 보여줌.}
\label{fig:midasar_sample_fit}
\end{figure}

\begin{figure}[htbp]
\centering
\includegraphics[width=0.7\textwidth]{images/midas/midasar_test_fit.png}
\caption{MIDAS-AR 모형 테스트 적합 결과: 전산업생산지수 성장률 (전월대비). 테스트 기간(2023--2024년)에서의 예측값과 실제값 비교.}
\label{fig:midasar_test_fit}
\end{figure}

\begin{figure}[htbp]
\centering
\includegraphics[width=0.7\textwidth]{images/midas/midasar_weight.png}
\caption{MIDAS-AR 모형 exp-Almon 가중치: 전산업생산지수 성장률 (전월대비). Vintage별로 선택된 고빈도 래그에 대한 가중치 분포를 보여줌.}
\label{fig:midasar_weight}
\end{figure}

\begin{table}[htbp]
\centering
\small
\begin{tabular}{lcc}
\toprule
Vintage & AR(1) & MIDAS-AR(1) \\
\midrule
h0 & 1.49 (0.0) & 1.50 (-0.7) \\
h1 & 1.49 (0.0) & 1.60 (-7.4) \\
h2 & 1.49 (0.0) & 1.47 (1.3)  \\
h3 & 1.49 (0.0) & 1.50 (-0.7) \\
h4 & 1.49 (0.0) & 1.49 (0.0)  \\
\bottomrule
\end{tabular}
\caption{Vintage별 테스트 RMSE 및 AR(1) 대비 RMSE 감소율: 전년동월비 (2023--2024). 괄호 안 숫자는 AR(1) 대비 RMSE 감소율(\%)임.}
\label{tab:midasar_rmse_yoy}
\end{table}

\subsubsection*{MIDAS-AR 모형 적합 결과: 전산업생산지수 전년동월비}

\begin{figure}[htbp]
\centering
\includegraphics[width=0.7\textwidth]{images/midas/midasar_mom_sample_fit.png}
\caption{MIDAS-AR 모형 인샘플 적합 결과: 전산업생산지수 전년동월비. 훈련 기간(2002--2022년)에서의 모형 적합도를 보여줌.}
\label{fig:midasar_mom_sample_fit}
\end{figure}

\begin{figure}[htbp]
\centering
\includegraphics[width=0.7\textwidth]{images/midas/midasar_mom_test_fit.png}
\caption{MIDAS-AR 모형 테스트 적합 결과: 전산업생산지수 전년동월비. 테스트 기간(2023--2024년)에서의 예측값과 실제값 비교.}
\label{fig:midasar_mom_test_fit}
\end{figure}

\begin{figure}[htbp]
\centering
\includegraphics[width=0.7\textwidth]{images/midas/midasar_mom_weight.png}
\caption{MIDAS-AR 모형 exp-Almon 가중치: 전산업생산지수 전년동월비. Vintage별로 선택된 고빈도 래그에 대한 가중치 분포를 보여줌.}
\label{fig:midasar_mom_weight}
\end{figure}

\subsubsection*{Vintage별 테스트 RMSE: XGBoost 모형 비교}

\begin{table}[htbp]
\centering
\small
\begin{tabular}{lcccc}
\toprule
Vintage & AR(1) & ARX (linear) & AR(1)+XGB\_residual & XGB-direct \\
\midrule
h0 & 0.952 (0.0)         & \textbf{0.950} (0.2) & 1.110 (-10.3) & 1.030 (-4.5) \\
h1 & \textbf{0.953} (0.0)& 0.964 (-1.2)        & 1.040 (-11.2) & 0.979 (-2.6) \\
h2 & \textbf{0.953} (0.0)& 0.964 (-1.2)        & 1.040 (-10.2) & 1.000 (-4.4) \\
h3 & \textbf{0.953} (0.0)& 0.964 (-1.2)        & 1.000 (-7.0)  & 1.000 (-7.0)  \\
h4 & 0.953 (0.0)         & \textbf{0.940} (1.4)& 1.040 (-7.0)  & 0.951 (0.2)  \\
\bottomrule
\end{tabular}
\caption{Vintage별 테스트 RMSE 및 AR(1) 대비 RMSE 감소율 (2023--2024). 종속변수: 전산업생산지수 성장률. 각 셀은 2023--2024년 테스트 구간에서의 RMSE와, 괄호 안의 AR(1) 대비 RMSE 감소율(\%)을 함께 보고함. 감소율은 $100 \times (1 - \text{RMSE}_{m,h} / \text{RMSE}_{\text{AR(1)},h})$로 정의되며, 양수 값은 동일한 vintage에서 AR(1) 모형보다 예측 오차가 작다는 것을 의미함.}
\label{tab:rmse-xgb}
\end{table}

\begin{figure}[htbp]
\centering
\includegraphics[width=0.8\textwidth]{images/midas/xgboost_test_plots.png}
\caption{XGBoost 모형 테스트 결과: 전산업생산지수 성장률 (전월대비). AR(1), ARX, AR(1)+XGB\_residual, XGB-direct 모형의 vintage별 예측값과 실제값 비교.}
\label{fig:xgboost_test_plots}
\end{figure}

\begin{table}[htbp]
\centering
\small
\begin{tabular}{lcccc}
\toprule
Vintage & AR(1) & ARX (linear) & AR(1)+XGB\_residual & XGB-direct \\
\midrule
h0 & 1.49 (0.0)         & 1.51 (-1.6)        & 1.52 (-2.3)        & \textbf{1.42} (4.4) \\
h1 & \textbf{1.48} (0.0)& 1.58 (-6.4)        & 1.55 (-4.3)        & 1.61 (-8.6)         \\
h2 & \textbf{1.48} (0.0)& 1.58 (-6.4)        & 1.58 (-6.7)        & 1.55 (-4.3)         \\
h3 & \textbf{1.48} (0.0)& 1.58 (-6.4)        & 1.55 (-4.3)        & 1.58 (-6.7)         \\
h4 & \textbf{1.48} (0.0)& 1.53 (-2.9)        & 1.53 (-2.9)        & 1.52 (-2.3)         \\
\bottomrule
\end{tabular}
\caption{Vintage별 테스트 RMSE 및 AR(1) 대비 RMSE 감소율: 전년동월비 (2023--2024). 각 셀은 2023--2024년 테스트 구간에서의 RMSE와, 괄호 안의 AR(1) 대비 RMSE 감소율(\%)을 함께 보고함. 감소율은 $100 \times (1 - \mathrm{RMSE}_{m,h}/\mathrm{RMSE}_{\mathrm{AR(1)},h})$로 정의되며, 양수 값은 동일한 vintage에서 AR(1) 모형보다 예측 오차가 작다는 것을 의미함.}
\label{tab:xgb_rmse_yoy}
\end{table}

\subsubsection*{변수 중요도 히트맵}

\begin{figure}[htbp]
\centering
\includegraphics[width=0.7\textwidth]{images/midas/heatmap_iip.png}
\caption{변수 중요도 히트맵: 전산업생산지수 성장률 (전월대비). 각 변수의 vintage별 중요도를 시각화함.}
\label{fig:heatmap_iip}
\end{figure}

\begin{figure}[htbp]
\centering
\includegraphics[width=0.7\textwidth]{images/midas/heatmap_mom.png}
\caption{변수 중요도 히트맵: 전산업생산지수 전년동월비. 각 변수의 vintage별 중요도를 시각화함.}
\label{fig:heatmap_mom}
\end{figure}

\subsubsection*{ARX 모형 추정 결과}

\begin{table}[htbp]
\centering
\small
\begin{tabular}{lrrr}
\toprule
변수 & 계수 추정치 & 표준오차 & t값 \\
\midrule
상수항          & -1.699        & 1.906 & -0.89 \\
$y_{t-1}$       & -0.403$^{***}$& 0.071 & -5.67 \\
$\text{pw}_{t-1,w1}$     &  0.000        & 0.000 &  0.21 \\
$\text{BSI}_{t-1}$       & -0.076$^{*}$  & 0.035 & -2.20 \\
$\text{BSI}_{t-1}^{\text{YoY}}$ & -0.051$^{*}$  & 0.022 & -2.33 \\
$\text{pw}_{t,w1}$       &  0.000        & 0.000 & -0.21 \\
$\text{pw}_{t,w1}^{\text{YoY}}$ & -0.000       & 0.005 & -0.09 \\
$\text{BSI}_{t}$         &  0.096$^{**}$ & 0.035 &  2.76 \\
$\text{BSI}_{t}^{\text{YoY}}$   &  0.053$^{*}$  & 0.022 &  2.47 \\
\midrule
$R^{2}$       & \multicolumn{3}{r}{0.239} \\
조정 $R^{2}$  & \multicolumn{3}{r}{0.205} \\
관측치 수     & \multicolumn{3}{r}{188} \\
\bottomrule
\end{tabular}
\caption{ARX 모형 추정 결과: 월별 IIP 성장률에 대한 BSI 및 전력거래량의 영향. 종속변수는 전산업생산지수 월별 성장률($y_t$)이며, $y_{t-1}$은 1기 시차, $\text{pw}$는 월별(또는 주별) 전력거래량 관련 변수, $\text{BSI}$는 기업경기실사지수, ``YoY''는 전년동월 대비 변화를 의미함. 두 번째 열은 계수 추정치, 세 번째 열은 표준오차, 네 번째 열은 t값을 나타냄. $^{***}$ $p<0.01$, $^{**}$ $p<0.05$, $^{*}$ $p<0.10$.}
\label{tab:arx_bsi}
\end{table}

\begin{table}[htbp]
\centering
\small
\begin{tabular}{lrrr}
\toprule
변수 & 계수 추정치 & 표준오차 & t값 \\
\midrule
상수항                    & -11.587$^{***}$ &  2.754 & -4.21 \\
$y_{t-1}$                 &   0.398$^{***}$ &  0.066 &  6.07 \\
$\text{pw}_{t-1,w1}$      &  -0.000         &  0.000 & -0.45 \\
$\text{BSI}_{t-1}$        &   0.113$^{*}$   &  0.046 &  2.48 \\
$\text{BSI}_{t-1}^{\text{YoY}}$ &  -0.076$^{**}$  &  0.029 & -2.64 \\
$\text{pw}_{t,w1}$        &   0.000         &  0.000 &  1.58 \\
$\text{pw}_{t,w1}^{\text{YoY}}$ &  -0.008         &  0.006 & -1.28 \\
$\text{BSI}_{t}$          &   0.004         &  0.045 &  0.08 \\
$\text{BSI}_{t}^{\text{YoY}}$   &   0.123$^{***}$ &  0.028 &  4.40 \\
\midrule
$R^{2}$       & \multicolumn{3}{r}{0.788} \\
조정 $R^{2}$  & \multicolumn{3}{r}{0.779} \\
관측치 수     & \multicolumn{3}{r}{188}   \\
\bottomrule
\end{tabular}
\caption{ARX 모형 추정 결과: 전년동월비에 대한 BSI 및 전력거래량의 영향. 종속변수는 전산업생산지수 월별 전년동월비($y_t$)이며, $y_{t-1}$은 1기 시차, $\text{pw}$는 전력거래량 관련 변수, $\text{BSI}$는 기업경기실사지수, ``YoY''는 전년동월 대비 변화를 의미함. 두 번째 열은 계수 추정치, 세 번째 열은 표준오차, 네 번째 열은 t값을 나타냄. $^{***}$ $p<0.01$, $^{**}$ $p<0.05$, $^{*}$ $p<0.10$.}
\label{tab:arx_mom}
\end{table}



\renewcommand{\refname}{}
\section*{References}
\bibliographystyle{unsrt}  
\bibliography{references}
\end{document}
