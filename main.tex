\documentclass[12pt]{report}

\usepackage{arxiv}
\usepackage{float}
\usepackage{graphicx}
\usepackage[utf8]{inputenc} % allow utf-8 input
\usepackage[T1]{fontenc}    % use 8-bit T1 fonts
\usepackage{hyperref}       % hyperlinks
\usepackage{url}            % simple URL typesetting
\usepackage{booktabs}       % professional-quality tables
\usepackage{amsfonts}       % blackboard math symbols
\usepackage{nicefrac}       % compact symbols for 1/2, etc.
\usepackage{microtype}      % microtypography
\usepackage{lipsum}
\usepackage{amsmath}

\title{Nowcasting Production and Investment Sector with High Frequency Data Integration}

\author{
  JaeYoung Kim\\
   affiliation\\
  address\\
  \texttt{email} \\
   \And
 SeoJung Lee \\
   affiliation\\
  address\\
  \texttt{email} \\
   \And
  YoungMin Kim \\
   affiliation\\
  address\\
  \texttt{email} \\
   \And
  Minkey Chang \\
   affiliation\\
  address\\
  \texttt{email} \\
   \And
  JunHo Hwang \\
   affiliation\\
  address\\
  \texttt{email} \\
   \And
  EunKyu Sung \\
   affiliation\\
  address\\
  \texttt{email} \\
   \And
  %% \And
  %% Coauthor \\
  %% Affiliation \\
  %% Address \\
  %% \texttt{email} \\
  %% \And
  %% Coauthor \\
  %% Affiliation \\
  %% Address \\
  %% \texttt{email} \\
}

\begin{document}
% Title without authors
\begin{center}
\Large\textbf{대한민국 거시경제 Nowcasting: 고빈도 데이터 및 동적요인모형 활용}
\end{center}
\vspace{0.2cm}

\begin{abstract}
\noindent 본 연구는 한국 거시경제 변수에 대한 nowcasting을 위해 고빈도 데이터와 동적요인모형을 활용한 네 가지 연구를 수행한다. 고빈도 데이터를 활용한 단일변수 예측 실험, 거시경제 변수 딥러닝 예측 실험(변수별로 최우수 모형이 다르게 나타남), 실시간 경기 진단 및 위기 알림 모형 구축, 거시경제 예측을 위한 고빈도 공공 데이터 소스 조사를 포함한다. 본 연구는 고빈도 데이터와 동적요인모형을 결합한 실용적인 nowcasting 시스템 구축 방법론을 제공한다.
\end{abstract}

\vspace{0.1cm}
\textbf{키워드:} 나우캐스팅, 동적요인모형, 고빈도 데이터, 거시경제 예측, 딥러닝

\vspace{0.1cm}
\setcounter{tocdepth}{-1}
\tableofcontents

% ============================================
% Part I: 고빈도 데이터를 활용한 단일변수 예측 실험
% ============================================
\part{고빈도 데이터를 활용한 단일변수 예측 실험}

\setcounter{section}{0}
\section{데이터 및 빈도 불일치 자료의 구성}

\subsection{분석 개요 및 데이터 출처}

\begin{itemize}
  \item 목적:
    \begin{itemize}
      \item 월별 산업활동지표(전산업생산지수)를 종속변수로 사용.
      \item 주별 전력거래량을 고빈도 설명변수로 사용.
      \item MIDAS-AR 및 AR(1) 벤치마크, XGBoost 확장을 통해 nowcasting 성능 비교.
    \end{itemize}
  \item 데이터 출처:
    \begin{itemize}
      \item 월별 전산업생산지수 (계절조정): 국가 통계 자료.
      \item 전력거래량: 공공데이터포털에서 시간대별·지역별 거래량 자료를 수집 후 일·주 단위로 집계.
      \item 기업경기실사지수(BSI): 월별 심리지표.
    \end{itemize}
  \item 소프트웨어:
    \begin{itemize}
      \item R 사용.
      \item 주요 패키지: \texttt{readxl}, \texttt{dplyr}, \texttt{lubridate}, \texttt{forecast}, \texttt{xgboost} 등.
    \end{itemize}
\end{itemize}

\subsection{종속변수: 월별 산업생산 지표}

\begin{itemize}
  \item 표기:
    \begin{itemize}
      \item $I^{\text{tot}}_t$: 전산업생산지수(계절조정, 월 $t$).
      \item $y^{\text{tot}}_t$: 월간 로그 성장률.
      \item $y^{\text{tot,YoY}}_t$: 전년동월비.
      \item $t$: 월(month) 인덱스.
      \item 사용 가능 기간: 2000년 1월–2025년 9월 (실제 추정에선 샘플 제약 존재).
    \end{itemize}
  \item 정의:
    \begin{align}
      y^{\text{tot}}_t 
        &= 100\left( \log I^{\text{tot}}_t - \log I^{\text{tot}}_{t-1} \right), \\
      y^{\text{tot,YoY}}_t
        &= 100\left( \frac{I^{\text{tot}}_t}{I^{\text{tot}}_{t-12}} - 1 \right)
          \approx 100\left( \log I^{\text{tot}}_t - \log I^{\text{tot}}_{t-12} \right).
    \end{align}
  \item 실제 추정:
    \begin{itemize}
      \item $y^{\text{tot}}_t$ 또는 $y^{\text{tot,YoY}}_t$ 중 하나를 선택하여 $y_t$로 표기.
      \item 두 경우 모두 계절조정된 전산업생산지수를 기반으로 계산.
    \end{itemize}
\end{itemize}

\subsection{설명변수: 주별/일별 전력거래량 성장률}

\begin{itemize}
  \item 표기:
    \begin{itemize}
      \item $P_s$: 주 $s$에서의 전력거래량.
      \item $\ell_s = \log P_s$: 로그 전력거래량.
      \item $\ell^{\text{SA}}_s$: 계절조정된 로그 전력거래량.
      \item $x_s$: 계절조정된 로그 성장률.
    \end{itemize}
  \item 계절조정 및 성장률:
    \begin{align}
      \ell_s &= \log P_s, \\
      \ell^{\text{SA}}_s &= \text{seasadj}\big( \text{STL}(\ell_s) \big), \\
      x_s &= 100\left( \ell^{\text{SA}}_s - \ell^{\text{SA}}_{s-1} \right).
    \end{align}
  \item $x_s$를 MIDAS 회귀의 고빈도 설명변수로 사용.
\end{itemize}

\section{MIDAS-AR 모형 설정과 추정 방법}

\subsection{MIDAS-AR(1) 모형}

\begin{itemize}
  \item 기본 구조:
    \begin{align}
      y_t
        &= \lambda y_{t-1}
           + \beta_0 + \beta_1 Z_t(K,\theta)
           + \varepsilon_t.
    \end{align}
  \item 구성 요소:
    \begin{itemize}
      \item $\lambda$: AR(1) 계수.
      \item $\beta_0, \beta_1$: 상수항 및 MIDAS 회귀자 계수.
      \item $\theta = (\theta_1,\theta_2)$: exp–Almon 가중치 모수.
      \item $K$: 사용하는 고빈도 래그 개수.
      \item $Z_t(K,\theta)$: 고빈도 성장률의 가중 합.
    \end{itemize}
  \item 안정성 제약:
    \begin{itemize}
      \item 재파라미터화 예: $\lambda = \tanh(\lambda^{\text{raw}})$ 등.
      \item $\lvert \lambda \rvert < 1$ 자동 충족.
    \end{itemize}
\end{itemize}

\subsection{exp–Almon 가중치와 MIDAS regressor}

\begin{itemize}
  \item 월별 cutoff:
    \begin{itemize}
      \item $T_t$: 월 $t$의 말일.
      \item cutoff까지의 고빈도 성장률 $\{x_s : s \le T_t\}$ 중 최근 $K$개를 선택:
      \[
        x_{t,1},\dots,x_{t,K},
      \]
      \item $x_{t,1}$: 가장 최근 주, $x_{t,K}$: 가장 오래된 래그.
    \end{itemize}
  \item exp–Almon 가중치:
    \begin{align}
      Z_t(K,\theta) 
        &= \sum_{k=1}^K w_k(\theta)\, x_{t,k}, \\
      w_k(\theta_1,\theta_2)
        &= \frac{\exp(\theta_1 k + \theta_2 k^2)}
                {\sum_{j=1}^K \exp(\theta_1 j + \theta_2 j^2)}, \quad k=1,\dots,K.
    \end{align}
  \item 제약 및 재파라미터화:
    \begin{itemize}
      \item $w_k(\theta)\ge 0$, $\sum_k w_k(\theta)=1$.
      \item $\theta_2 < 0$ 제약을 통해 오래된 래그 가중치 감소 유도.
      \item 구현:
      \begin{equation*}
        \theta_1 = 300 \tanh(\theta_{1,\text{raw}}), \quad
        \theta_2 = -10 \frac{\exp(\theta_{2,\text{raw}})}{1 + \exp(\theta_{2,\text{raw}})}.
      \end{equation*}
      \item 결과: $|\theta_1|<300$, $-10<\theta_2<0$ 보장.
      \item 목적: 수치적 안정성 확보, corner solution 방지.
    \end{itemize}
  \item $K$ 선택:
    \begin{itemize}
      \item 후보: $K \in \{8,13,26,52\}$.
      \item Validation RMSE 최소 기준으로 최적 $K^*$ 선택.
    \end{itemize}
\end{itemize}

\subsection{\citet{clements2008macroeconomic} 추정 절차}

\begin{itemize}
  \item 샘플 분할:
    \begin{itemize}
      \item Train: 2002–2020년.
      \item Validation: 2021–2022년.
      \item Test: 2023–2024년.
      \item COVID 구간 포함 (제외 여부에 따른 성능 변화 미미).
    \end{itemize}
  \item 1단계 (Standard MIDAS, AR 없음):
    \begin{align}
      y_t = \beta_0 + \beta_1 Z_t(K,\theta) + \varepsilon_t.
    \end{align}
    \begin{itemize}
      \item NLS로 $(\beta_0,\beta_1,\theta)$ 추정 → 초기값 획득.
    \end{itemize}
  \item 2단계 (잔차의 AR 추정):
    \begin{align}
      \hat{\varepsilon}_t
        &= \lambda^{(0)} \hat{\varepsilon}_{t-1} + u_t.
    \end{align}
    \begin{itemize}
      \item OLS로 $\lambda^{(0)}$ 추정.
    \end{itemize}
  \item 3단계 ($\lambda$ 고정 MIDAS-AR):
    \begin{align}
      y_t
        &= \lambda^{(0)} y_{t-1}
           + \beta_0 + \beta_1 \widetilde{Z}_t(K,\theta;\lambda^{(0)})
           + \varepsilon_t.
    \end{align}
    \begin{itemize}
      \item $y_t$의 자기회귀 구조를 반영하여 고빈도 성장률을 한 번 조정한 뒤 가중치 부여.
      \item $y_{t-1}$이 설명하는 저빈도 움직임과 고빈도 정보의 분리를 목표.
      \item NLS로 $(\beta_0,\beta_1,\theta)$ 재추정.
    \end{itemize}
  \item 4단계 (Full MIDAS-AR 공동 추정):
    \begin{itemize}
      \item $(\lambda,\beta_0,\beta_1,\theta)$를 동시에 NLS로 추정.
      \item 초기값: 3단계 추정치.
      \item 최적화 알고리즘: BFGS.
    \end{itemize}
\end{itemize}

\subsection{K 선택과 예측 설계}

\begin{itemize}
  \item 튜닝 단계:
    \begin{itemize}
      \item 후보 $K$: $\{8,13,26,52\}$.
      \item 각 $K$에 대해 위 절차(1–4단계)를 수행.
      \item Validation 구간에서 one-step-ahead 예측치 $\hat{y}_t$ 계산.
      \item Validation RMSE:
      \begin{align}
        \text{RMSE}_{\text{val}}(K,1)
          = \sqrt{\frac{1}{T_{\text{val}}}
                  \sum_{t\in\text{val}} (y_t - \hat{y}_t)^2 }.
      \end{align}
      \item RMSE 최소가 되는 $K^*$ 선택.
    \end{itemize}
  \item 최종 예측:
    \begin{itemize}
      \item Train+Val(2002–2022) 전체로 재추정 → $(\hat{\lambda},\hat{\beta}_0,\hat{\beta}_1,\hat{\theta})$.
      \item one-step-ahead 예측:
      \begin{align}
        \hat{y}_t
          &= \hat{\lambda} y_{t-1} 
             + \hat{\beta}_0 + \hat{\beta}_1 Z_t(K^*,\hat{\theta}).
      \end{align}
      \item Test RMSE:
      \begin{align}
        \text{RMSE}_{\text{test}}
          &= \sqrt{\frac{1}{T_{\text{test}}}
                  \sum_{t\in\text{test}} (y_t - \hat{y}_t)^2 }.
      \end{align}
    \end{itemize}
\end{itemize}

\section{AR(1) 벤치마크와 Vintage별 MIDAS–AR(1)}

\subsection{공통 샘플 및 AR(1) 벤치마크}

\begin{itemize}
  \item 공통 샘플:
    \begin{itemize}
      \item $y_t$ 사용 가능 시점: 2002-03–2024-12 (가정).
      \item 주별 전력 성장률 $x_s$: 2001-04 이후.
      \item MIDAS 래그가 정의 가능한 월만 남긴 공통 집합:
      \[
        \mathcal{T}_\text{common} \subset \{2002\text{-03},\dots,2024\text{-12}\}.
      \]
    \end{itemize}
  \item AR(1) 모형:
    \begin{align}
      y_t &= \alpha + \phi y_{t-1} + e_t.
    \end{align}
  \item 추정 및 예측:
    \begin{itemize}
      \item 2002–2022년 (공통 샘플 내)에서 OLS로 $(\hat{\alpha},\hat{\phi})$ 추정.
      \item one-step-ahead 예측:
      \begin{align}
        \hat{y}_t^{\text{AR(1)}} &= \hat{\alpha} + \hat{\phi} y_{t-1}.
      \end{align}
      \item Test RMSE:
      \begin{align}
        \text{RMSE}_\text{test}^{\text{AR(1)}}
          &= \sqrt{\frac{1}{T_\text{test}} \sum_{t \in \mathcal{T}_\text{test}}
                    ( y_t - \hat{y}_t^{\text{AR(1)}} )^2}.
      \end{align}
      \item $\mathcal{T}_\text{test} = \{ t \in \mathcal{T}_\text{common} : 2023\text{-01} \le t \le 2024\text{-12}\}$.
      \item AR(1)은 고빈도 정보를 사용하지 않으므로 vintage와 무관하게 동일한 RMSE를 가짐.
    \end{itemize}
\end{itemize}

\subsection{Vintage 정보세트 정의}

\begin{itemize}
  \item 정의:
    \begin{itemize}
      \item $m_t$: 월 $t$의 첫날.
      \item $T_t$: 월 $t$의 말일.
      \item 각 주 $s$에 대해 월(month) 및 주차(week-in-month) 계산.
    \end{itemize}
  \item cutoff date $C_t^{(h)}$:
    \begin{itemize}
      \item h0: $C_t^{(\text{h0})} = m_t - 1$ (전월 말까지).
      \item h1: $t$월 week $\le 1$인 주 중 마지막 주 날짜.
      \item h2: $t$월 week $\le 2$인 주 중 마지막 주 날짜.
      \item h3: $t$월 week $\le 3$인 주 중 마지막 주 날짜.
      \item h4: $C_t^{(\text{h4})} = T_t$ (당월 전체).
    \end{itemize}
  \item 각 vintage별 MIDAS 래그:
    \begin{itemize}
      \item $\{x_s : s \le C_t^{(h)}\}$ 중 최근 $K$개 선택:
      \[
        x^{(h)}_{t,1},\dots,x^{(h)}_{t,K}.
      \]
      \item exp–Almon 가중치 적용:
      \[
        Z_t^{(h)}(K,\theta) = \sum_{k=1}^K w_k(\theta)\, x^{(h)}_{t,k}.
      \]
    \end{itemize}
\end{itemize}

\subsection{Vintage별 MIDAS–AR(1) 추정 및 요약}

\begin{itemize}
  \item 모형:
    \begin{align}
      y_t = \lambda^{(h)} y_{t-1}
           + \beta_0^{(h)} + \beta_1^{(h)} Z_t^{(h)}(K,\theta^{(h)})
           + \varepsilon_t^{(h)}.
    \end{align}
  \item 모수:
    \begin{itemize}
      \item $\lambda^{(h)}$: AR(1) 계수.
      \item $\beta_0^{(h)},\beta_1^{(h)}$: 상수 및 MIDAS 계수.
      \item $\theta^{(h)}$: exp–Almon 모수.
      \item $K$: 래그 개수 (vintage별 최적 $K^*(h)$ 선택).
    \end{itemize}
  \item 제약 및 최적화:
    \begin{itemize}
      \item 재파라미터화로 $|\lambda^{(h)}|<1$, $|\theta_1^{(h)}|<300$, $-10<\theta_2^{(h)}<0$ 보장.
      \item BFGS로 NLS 추정.
    \end{itemize}
  \item 튜닝 및 최종 예측:
    \begin{itemize}
      \item 각 $h$에 대해 $K \in \{8,13,26,52\}$ 중 Validation RMSE 최소인 $K^*(h)$ 선택.
      \item 2002–2022년 전체로 재추정 → $(\hat{\lambda}^{(h)},\hat{\beta}_0^{(h)},\hat{\beta}_1^{(h)},\hat{\theta}^{(h)})$.
      \item 테스트 구간 예측:
      \[
        \hat{y}_t^{(h)}
          = \hat{\lambda}^{(h)} y_{t-1}
           + \hat{\beta}_0^{(h)} 
           + \hat{\beta}_1^{(h)} Z_t^{(h)}(K^*(h),\hat{\theta}^{(h)}).
      \]
      \item Test RMSE:
      \[
        \text{RMSE}_\text{test}^{(h)}
          = \sqrt{\frac{1}{T_\text{test}} \sum_{t \in \mathcal{T}_\text{test}}
             ( y_t - \hat{y}_t^{(h)} )^2 }.
      \]
    \end{itemize}
  \item exp–Almon 가중치 해석:
    \begin{itemize}
      \item 각 vintage에서 $w_k^{(h)} = w_k(\hat{\theta}_1^{(h)},\hat{\theta}_2^{(h)})$ 계산.
      \item $k=1$: 가장 최근 주, $k=K^*(h)$: 가장 오래된 래그.
      \item $w_k^{(h)}$ 패턴으로 어떤 시점의 전력 정보가 중요하게 사용되는지 확인.
    \end{itemize}
\end{itemize}

\section{실증 결과 요약: AR(1) vs MIDAS–AR(1)}

\subsection{종속변수: 전산업생산지수 성장률}

\paragraph{정태성 검정}

\begin{itemize}
  \item ADF 검정:
    \begin{itemize}
      \item R 함수: \texttt{adf.test()}.
      \item lag 선택: $k = \lfloor (T-1)^{1/3} \rfloor = 6$.
      \item 검정통계량: 약 $-7.59$.
      \item p-value: $< 0.01$.
      \item 결론: 1\% 유의수준에서 단위근 귀무가설 기각 → 수준에서 정상성(stationarity) 가정 가능.
    \end{itemize}
\end{itemize}

\paragraph{Vintage별 MIDAS–AR(1) 인샘플 적합}

\begin{figure}[H]
    \centering
    \includegraphics[width=\textwidth]{midas/images/midasar_sample_fit.png}
    \caption{MIDAS–AR(1) 인샘플 적합: 실제값 vs 예측값}
    \label{fig:midasar_insample}
\end{figure}

\begin{itemize}
  \item Figure~\ref{fig:midasar_insample}:
    \begin{itemize}
      \item 2002–2022년 구간 인샘플 적합.
      \item 모든 vintage에서 인샘플 RMSE: 약 1.24–1.26.
      \item AR(1)이 설명하는 저빈도 움직임 위에 MIDAS 회귀자가 세부 변동을 추가 설명.
      \item $K^*(h)$:
        \begin{itemize}
          \item $h0$, $h1$: 상대적으로 짧은 래그($K=8$).
          \item $h2$, $h4$: $K=52$ 등 긴 래그 선택.
        \end{itemize}
    \end{itemize}
\end{itemize}

\paragraph{Vintage별 exp–Almon 가중치}

\begin{figure}[H]
    \centering
    \includegraphics[width=\textwidth]{midas/images/midasar_weight.png}
    \caption{exp–Almon 가중치}
    \label{fig:midasar_weights}
\end{figure}

\begin{itemize}
  \item Figure~\ref{fig:midasar_weights}:
    \begin{itemize}
      \item 전반적으로 최근 몇 주(1–3주)에 가중치 집중.
      \item $h0$: (t-1)월 정보만 사용 → 최근 주 중심 우하향 패턴.
      \item $h2$, $h4$: $K^*(h)=52$이나 실질 가중치는 최근 1–4주에 집중, 나머지는 0에 근접.
      \item $h3$: 가장 최근 1주에 거의 전 가중치 집중.
    \end{itemize}
\end{itemize}

\paragraph{Vintage별 Test RMSE: AR(1) vs MIDAS–AR(1)}

\begin{table}[H]
\centering
\begin{tabular}{lcc}
\toprule
Vintage & AR(1) & MIDAS-AR(1) \\
\midrule
h0 & 0.950 (0.0) & 0.952 (-0.2) \\
h1 & 0.950 (0.0) & 0.951 (-0.1) \\
h2 & 0.950 (0.0) & 0.952 (-0.2) \\
h3 & 0.950 (0.0) & 0.951 (-0.1) \\
h4 & 0.950 (0.0) & 0.945 (0.5)  \\
\bottomrule
\end{tabular}
\caption{테스트 RMSE: AR(1) vs MIDAS-AR(1)}
\label{tab:midasar_rmse_table}
\end{table}

\begin{itemize}
  \item 표 해석:
    \begin{itemize}
      \item 괄호 안 숫자: AR(1) 대비 RMSE 감소율(\%).
      \item $h0$–$h3$: MIDAS–AR(1)의 감소율 $\approx -0.2\%$~$-0.1\%$ (AR(1)보다 약간 열악).
      \item $h4$: MIDAS–AR(1)의 RMSE가 약 $0.5\%$ 감소 (소폭 개선).
    \end{itemize}
\end{itemize}

\paragraph{테스트 기간 예측 경로}

\begin{figure}[H]
    \centering
    \includegraphics[width=\textwidth]{midas/images/midasar_test_fit.png}
    \caption{테스트 기간 예측 경로: AR(1) vs MIDAS-AR(1)}
    \label{fig:midasar_test}
\end{figure}

\begin{itemize}
  \item Figure~\ref{fig:midasar_test}:
    \begin{itemize}
      \item 2023–2024년 테스트 구간에서 AR(1) vs MIDAS–AR(1) 예측 경로 비교.
      \item 전 vintage에서 두 모형의 궤적이 거의 동일.
      \item $h0$–$h3$: 일부 달에서 미세한 차이 있으나 RMSE 개선으로 연결되지 않음.
      \item $h4$: 몇몇 국면에서 MIDAS–AR(1)이 실제값에 조금 더 근접.
    \end{itemize}
  \item 결론(월간 성장률):
    \begin{itemize}
      \item AR(1)만으로도 단기 예측력이 높음.
      \item 주별 전력거래량을 추가한 MIDAS–AR(1)은 full month 정보(h4)에서만 약간의 개선.
      \item 월 중(h0–h2) 정보만으로는 추가 예측력 거의 없음.
    \end{itemize}
\end{itemize}

\subsection{종속변수: 전년동월비}

\paragraph{정태성 검정}

\begin{itemize}
  \item ADF 검정:
    \begin{itemize}
      \item R \texttt{adf.test()} 사용.
      \item lag: 자동 선택 ($k \approx 6$).
      \item 검정통계량: $-5.56$.
      \item p-value: $<0.01$.
      \item 결론: 수준에서 정상(stationary)으로 판단.
    \end{itemize}
\end{itemize}

\paragraph{인샘플 적합 및 exp–Almon 가중치}

\begin{figure}[H]
    \centering
    \includegraphics[width=\textwidth]{midas/images/midasar_mom_sample_fit.png}
    \caption{MIDAS–AR(1) 인샘플 적합: 전년동월비}
    \label{fig:midasar_yoy_insample}
\end{figure}

\begin{figure}[H]
    \centering
    \includegraphics[width=\textwidth]{midas/images/midasar_mom_weight.png}
    \caption{exp–Almon 가중치: 전년동월비 모형}
    \label{fig:midasar_yoy_weights}
\end{figure}

\begin{itemize}
  \item Figure~\ref{fig:midasar_yoy_insample}:
    \begin{itemize}
      \item 전년동월비의 변동성이 크고 스파이크가 존재.
      \item MIDAS–AR(1) 적합치는 각 vintage에서 실제값을 비교적 잘 따라감.
      \item 인샘플 RMSE: 대략 1.78–1.88 (월간 성장률 모형보다 크다).
    \end{itemize}
  \item Figure~\ref{fig:midasar_yoy_weights}:
    \begin{itemize}
      \item $h0$: $K^*(h0)=52$, 실제 가중치는 최근 5–7주에 집중.
      \item $h1$, $h2$: 종 모양(hump-shaped) 패턴; cutoff 이전 4–6주 전 래그에 큰 가중치.
      \item $h3$, $h4$: 짧은 래그($K^*=8$) 선택, 단조 감소형 가중치.
      \item 전반적으로 최근 1–2개월 내 전력 사용 정보가 중요, 일부 vintage에서 약간 더 이전 주에 비중.
    \end{itemize}
\end{itemize}

\paragraph{Vintage별 Test RMSE (전년동월비)}

\begin{table}[H]
\centering
\begin{tabular}{lcc}
\toprule
Vintage & AR(1) & MIDAS-AR(1) \\
\midrule
h0 & 1.49 (0.0) & 1.50 (-0.7) \\
h1 & 1.49 (0.0) & 1.60 (-7.4) \\
h2 & 1.49 (0.0) & 1.47 (1.3)  \\
h3 & 1.49 (0.0) & 1.50 (-0.7) \\
h4 & 1.49 (0.0) & 1.49 (0.0)  \\
\bottomrule
\end{tabular}
\caption{테스트 RMSE: 전년동월비 (MIDAS-AR)}
\label{tab:midasar_rmse_yoy}
\end{table}

\begin{itemize}
  \item 해석:
    \begin{itemize}
      \item $h0$, $h3$: MIDAS–AR(1)의 RMSE가 AR(1)보다 약 $0.7\%$ 악화.
      \item $h1$: RMSE 약 $7.4\%$ 악화.
      \item $h2$: RMSE 약 $1.3\%$ 개선(크기 작음).
      \item $h4$: 두 모형 RMSE 동일.
    \end{itemize}
\end{itemize}

\paragraph{테스트 기간 예측 경로(전년동월비)}

\begin{figure}[H]
    \centering
    \includegraphics[width=\textwidth]{midas/images/midasar_mom_test_fit.png}
    \caption{테스트 기간 예측 경로: 전년동월비}
    \label{fig:midasar_yoy_test}
\end{figure}

\begin{itemize}
    \item Figure~\ref{fig:midasar_yoy_test}:
    \begin{itemize}
      \item 전반적으로 AR(1)과 MIDAS–AR(1)의 궤적 유사.
      \item $h1$: MIDAS–AR(1)이 일부 구간에서 노이즈를 과도하게 따라가며 RMSE 악화.
      \item $h2$: 일부 구간(예: 2023년·2024년 초 상승 국면)에서 MIDAS–AR(1)이 실제값에 더 근접.
      \item $h0$, $h3$, $h4$: 두 모형 간 차이가 미미.
    \end{itemize}
  \item 결론(전년동월비):
    \begin{itemize}
      \item AR(1)만으로도 상당한 단기 예측력 확보.
      \item 주별 전력거래량을 활용한 MIDAS–AR(1)은 대부분의 vintage에서 AR(1) 대비 개선 없음.
      \item $h2$에서만 소폭의 RMSE 감소, 그 외 vintage에서는 개선 불충분 또는 악화.
    \end{itemize}
\end{itemize}

\section{XGBoost를 활용한 비선형 확장}\label{sec:xgb}

\subsection{기본 아이디어 및 표본 분할}

\begin{itemize}
  \item 기본 가정:
    \begin{itemize}
      \item $y_{t-1}$만으로도 강한 자기회귀 구조 존재 → AR(1) 벤치마크.
      \item 전력거래량, BSI 등 고빈도·심리지표는 비선형 추가정보 가능.
    \end{itemize}
  \item 비교 대상:
    \begin{itemize}
      \item (1) 선형 ARX: AR(1) + 고빈도 feature의 선형 효과.
      \item (2) AR(1) 잔차에 대한 XGBoost 보정: $e_t$를 비선형 함수로 설명.
      \item (3) XGBoost 직접 예측: $(y_{t-1}, x_{t,h})$를 입력으로 $y_t$ 직접 예측.
    \end{itemize}
  \item 표본 분할:
    \begin{itemize}
      \item 데이터: $y_t$, 주별 전력, BSI가 모두 존재하는 2003년 이후.
      \item Train: 2003–2020년.
      \item Validation: 2021–2022년.
      \item Test: 2023–2024년.
      \item vintage $h0$–$h4$: 모두 동일한 분할, feature만 다름.
    \end{itemize}
\end{itemize}

\subsection{feature 구성 및 vintage별 정보세트(요약)}

\begin{table}[H]
  \centering
  \label{tab:vintage_features}
  \begin{tabular}{lccccc}
    \toprule
    Feature & $h0$ & $h1$ & $h2$ & $h3$ & $h4$ \\
    \midrule
    \multicolumn{6}{l}{\textbf{(t-1)월 주별 전력거래량}} \\
    \quad pw\_tm1\_w1 & \checkmark & \checkmark & \checkmark & \checkmark & \checkmark \\
    \quad pw\_tm1\_w2 & \checkmark & \checkmark & \checkmark & \checkmark & \checkmark \\
    \quad pw\_tm1\_w3 & \checkmark & \checkmark & \checkmark & \checkmark & \checkmark \\
    \quad pw\_tm1\_w4 & \checkmark & \checkmark & \checkmark & \checkmark & \checkmark \\
    \addlinespace[0.5ex]
    \multicolumn{6}{l}{\textbf{(t-1)월 BSI 지수}} \\
    \quad bsi\_tm1      & \checkmark & \checkmark & \checkmark & \checkmark & \checkmark \\
    \quad bsi\_tm1\_yoy & \checkmark & \checkmark & \checkmark & \checkmark & \checkmark \\
    \addlinespace[0.5ex]
    \multicolumn{6}{l}{\textbf{t월 주별 전력거래량}} \\
    \quad pw\_t\_w1 &           & \checkmark & \checkmark & \checkmark & \checkmark \\
    \quad pw\_t\_w2 &           &            & \checkmark & \checkmark & \checkmark \\
    \quad pw\_t\_w3 &           &            &            & \checkmark & \checkmark \\
    \quad pw\_t\_w4 &           &            &            &            & \checkmark \\
    \addlinespace[0.5ex]
    \multicolumn{6}{l}{\textbf{t월 주별 전력거래량 전년동월대비}} \\
    \quad pwyoy\_t\_w1 &        & \checkmark & \checkmark & \checkmark & \checkmark \\
    \quad pwyoy\_t\_w2 &        &            & \checkmark & \checkmark & \checkmark \\
    \quad pwyoy\_t\_w3 &        &            &            & \checkmark & \checkmark \\
    \quad pwyoy\_t\_w4 &        &            &            &            & \checkmark \\
    \addlinespace[0.5ex]
    \multicolumn{6}{l}{\textbf{t월 BSI 지수 (동행)}} \\
    \quad bsi\_t      &         &            &            &            & \checkmark \\
    \quad bsi\_t\_yoy &         &            &            &            & \checkmark \\
    \bottomrule
  \end{tabular}
\caption{Vintage별 설명변수}
\end{table}

\begin{itemize}
  \item 공통 전월 정보:
    \begin{itemize}
      \item $(t-1)$월 주별 전력: \texttt{pw\_tm1\_w1}–\texttt{pw\_tm1\_w4}.
      \item $(t-1)$월 BSI: \texttt{bsi\_tm1}, \texttt{bsi\_tm1\_yoy}.
    \end{itemize}
  \item 현재 월 주별 전력 및 YoY:
    \begin{itemize}
      \item vintage별로 사용 가능한 주차까지 \texttt{pw\_t\_w$j$}, \texttt{pwyoy\_t\_w$j$} 추가.
      \item h1: 1주, h2: 1–2주, h3: 1–3주, h4: 1–4주.
    \end{itemize}
  \item 동월 BSI:
    \begin{itemize}
      \item h4: \texttt{bsi\_t}, \texttt{bsi\_t\_yoy} 포함.
    \end{itemize}
  \item feature 벡터:
    \[
      x_{t,h} = (\text{pw\_tm1\_w1-w4},\; \text{bsi\_tm1}, \text{bsi\_tm1\_yoy},
      \text{pw\_t\_w1-w4}, \text{pwyoy\_t\_w1-w4}, \text{bsi\_t},\text{bsi\_t\_yoy})'.
    \]
\end{itemize}


\subsection{모형별 정의}

\paragraph{(1) 선형 ARX}

\begin{itemize}
  \item 모형:
    \begin{equation}
      y_t = \alpha + \phi y_{t-1} + \beta_h' x_{t,h} + \varepsilon_{t,h}.
    \end{equation}
  \item 추정:
    \begin{itemize}
      \item 2003–2022년 전체로 OLS.
      \item 테스트 구간에서 out-of-sample 예측.
    \end{itemize}
\end{itemize}

\paragraph{(2) AR(1)+XGB\_residual}

\begin{itemize}
  \item 1단계: AR(1) 적합.
    \begin{itemize}
      \item 모형: $y_t = \alpha + \phi y_{t-1} + u_t$.
      \item 잔차: $e_t = y_t - \hat{y}_t^{AR(1)}$.
    \end{itemize}
  \item 2단계: 잔차에 대한 XGBoost:
    \begin{equation}
      e_t = f_h(x_{t,h}) + \eta_{t,h}.
    \end{equation}
  \item 테스트 예측:
    \[
      \hat y_{t,h}^{AR(1)+XGB\_res}
      = \hat y_t^{AR(1)} + \hat f_h(x_{t,h}).
    \]
\end{itemize}

\paragraph{(3) XGB\_direct}

\begin{itemize}
  \item 입력:
    \[
      z_{t,h} = (y_{t-1}, x_{t,h})'.
    \]
  \item 모형:
    \[
      y_t = g_h(z_{t,h}) + \xi_{t,h}.
    \]
  \item 테스트 예측:
    \[
      \hat y_{t,h}^{XGB\_direct} = \hat g_h(z_{t,h}).
    \]
\end{itemize}

\subsection{하이퍼파라미터 및 롤링 검증}

\begin{itemize}
  \item XGBoost 설정:
    \begin{itemize}
      \item \texttt{objective} = "reg:squarederror".
      \item \texttt{eval\_metric} = "rmse".
      \item $\eta = 0.05$, \texttt{max\_depth} = 3.
      \item \texttt{subsample} = 0.8, \texttt{colsample\_bytree} = 0.8.
    \end{itemize}
  \item 선택 파라미터:
    \begin{itemize}
      \item 부스팅 반복 횟수 $n_{\text{round}} \in \{50,100,150,200\}$.
    \end{itemize}
  \item 롤링 교차검증:
    \begin{itemize}
      \item 2003–2022년 중 뒤쪽 24개월을 네 개의 6개월 validation 윈도우로 분할.
      \item 각 윈도우 $j$: $t < v_j$는 train, $v_j \le t \le v_j+5$는 validation.
      \item 각 $n_{\text{round}}$에 대해 네 윈도우의 평균 RMSE 계산.
      \item 평균 RMSE 최소의 $n_{\text{round}}$ 선택.
    \end{itemize}
  \item 최종 학습 및 평가:
    \begin{itemize}
      \item 선택된 $n_{\text{round}}$로 2003–2022년 전체로 재학습.
      \item 2023–2024년 테스트 구간에 대한 예측 수행.
      \item 성능지표:
      \[
        \text{RMSE}_{h}^{(m)} =
          \sqrt{\frac{1}{T_{\text{test}}}
              \sum_{t \in \text{test}}
              (y_t - \hat y_{t,h}^{(m)})^2}.
      \]
    \end{itemize}
\end{itemize}


\section{실증 결과 요약}

\subsection{종속변수: 전산업생산지수 성장률}

\subsubsection{예측 성능: RMSE 및 상대 개선율}

\begin{table}[H]
\centering
\label{tab:rmse-xgb}
\begin{tabular}{lcccc}
\toprule
Vintage & AR(1) & ARX (linear) & AR(1)+XGB\_residual & XGB-direct \\
\midrule
h0 & 0.952 (0.0)         & \textbf{0.950} (0.2) & 1.110 (-10.3) & 1.030 (-4.5) \\
h1 & \textbf{0.953} (0.0)& 0.964 (-1.2)        & 1.040 (-11.2) & 0.979 (-2.6) \\
h2 & \textbf{0.953} (0.0)& 0.964 (-1.2)        & 1.040 (-10.2) & 1.000 (-4.4) \\
h3 & \textbf{0.953} (0.0)& 0.964 (-1.2)        & 1.000 (-7.0)  & 1.000 (-7.0)  \\
h4 & 0.953 (0.0)         & \textbf{0.940} (1.4)& 1.040 (-7.0)  & 0.951 (0.2)  \\
\bottomrule
\end{tabular}
\begin{flushleft}
\caption{테스트 RMSE: XGBoost 모형}
\footnotesize
\textit{주}: 각 셀은 2023–2024년 테스트 구간에서의 RMSE와,
괄호 안의 AR(1) 대비 RMSE 감소율(\%)을 함께 보고한다.
감소율은 $100 \times (1 - \text{RMSE}_{m,h} / \text{RMSE}_{\text{AR(1)},h})$로 정의되며,
양수 값은 동일한 vintage에서 AR(1) 모형보다 예측 오차가 작다는 것을 의미한다.
\end{flushleft}
\end{table}

\begin{itemize}
  \item ARX:
    \begin{itemize}
      \item 대부분의 vintage에서 AR(1) 대비 감소율 $\approx -1.2\%$ 수준 (비슷하거나 약간 열악).
      \item $h4$: 약 1.4\% RMSE 감소 (소폭 개선).
    \end{itemize}
  \item AR(1)+XGB\_residual:
    \begin{itemize}
      \item 모든 vintage에서 RMSE가 AR(1)보다 7–11\% 증가.
      \item 잔차에 대한 부스팅이 노이즈를 과적합하는 경향.
    \end{itemize}
  \item XGB-direct:
    \begin{itemize}
      \item 대부분의 vintage에서 AR(1) 대비 2.6–4.5\% 수준 성능 저하.
      \item $h3$, $h4$: 각각 약 0.3\%, 0.2\% 개선(크기는 매우 작음).
    \end{itemize}
\end{itemize}

\subsubsection{시각적 비교와 feature importance}

\paragraph{테스트 예측 경로}

\begin{figure}[H]
    \centering
    \includegraphics[width=\textwidth]{midas/images/xgboost_test_plots.png}
    \caption{테스트 기간 예측 경로: XGBoost 모형}
    \label{fig:xgboost_test_plots}
\end{figure}

\begin{itemize}
  \item Figure~\ref{fig:xgboost_test_plots}:
    \begin{itemize}
      \item 각 vintage별 테스트 구간에서 AR(1), ARX, AR(1)+XGB\_res, XGB\_direct 예측 경로 비교.
      \item 네 모형 모두 AR(1) 궤적과 매우 유사한 패턴.
      \item AR(1)+XGB\_residual: 일부 시점에서 진폭 과대 → RMSE 악화와 일치.
    \end{itemize}
\end{itemize}

\paragraph{Feature importance (Gain 기준)}

\begin{figure}[H]
    \centering
    \includegraphics[width=\textwidth]{midas/images/heatmap_iip.png}
    \caption{변수 중요도 히트맵}
    \label{fig:xgb-imp-heatmap_iip}
\end{figure}

\begin{itemize}
  \item 정의:
    \begin{itemize}
      \item Gain: 해당 변수로 분할했을 때의 손실 감소량 $\Delta L$을 모든 노드에서 합산한 값.
      \item vintage별로 합이 1이 되도록 정규화해 feature importance로 사용.
    \end{itemize}
  \item Figure~\ref{fig:xgb-imp-heatmap_iip}:
    \begin{itemize}
      \item XGB-direct와 AR(1)+XGB\_residual의 normalized Gain을 히트맵으로 비교.
    \end{itemize}
  \item XGB-direct:
    \begin{itemize}
      \item 모든 vintage에서 $y_{t-1}$의 중요도가 가장 높음.
      \item XGBoost가 AR(1) 구조를 재현하는 방향으로 작동.
      \item 보조적으로 $(t-1)$월 BSI 및 $h4$의 동월 BSI가 사용됨.
      \item 현재 달 1주 전력거래량 및 YoY는 일부 vintage에서만 제한적으로 중요.
    \end{itemize}
  \item AR(1)+XGB\_residual:
    \begin{itemize}
      \item \texttt{bsi\_tm1\_yoy}, \texttt{pw\_tm1\_w1}의 중요도가 상대적으로 큼.
      \item AR(1)이 설명하지 못한 잔차 패턴이 전월 BSI YoY와 전월 초 전력거래량과 관련됨을 시사.
      \item 잔차의 신호대잡음비가 낮아 RMSE 개선으로는 이어지지 않음.
    \end{itemize}
\end{itemize}

\subsubsection{ARX 모형: BSI와 전력거래량의 역할}

\begin{table}[H]
\centering
\label{tab:arx_bsi}
\begin{tabular}{lrrr}
\toprule
변수 & 계수 추정치 & 표준오차 & t값 \\
\midrule
상수항          & -1.699        & 1.906 & -0.89 \\
$y_{t-1}$       & -0.403$^{***}$& 0.071 & -5.67 \\
$\text{pw}_{t-1,w1}$     &  0.000        & 0.000 &  0.21 \\
$\text{BSI}_{t-1}$       & -0.076$^{*}$  & 0.035 & -2.20 \\
$\text{BSI}_{t-1}^{\text{YoY}}$ & -0.051$^{*}$  & 0.022 & -2.33 \\
$\text{pw}_{t,w1}$       &  0.000        & 0.000 & -0.21 \\
$\text{pw}_{t,w1}^{\text{YoY}}$ & -0.000       & 0.005 & -0.09 \\
$\text{BSI}_{t}$         &  0.096$^{**}$ & 0.035 &  2.76 \\
$\text{BSI}_{t}^{\text{YoY}}$   &  0.053$^{*}$  & 0.022 &  2.47 \\
\midrule
$R^{2}$       & \multicolumn{3}{r}{0.239} \\
조정 $R^{2}$  & \multicolumn{3}{r}{0.205} \\
관측치 수     & \multicolumn{3}{r}{188} \\
\bottomrule
\end{tabular}
\begin{flushleft}
\caption{ARX 모형 추정 결과}
\footnotesize
\textit{주}: 종속변수는 전산업생산지수 월별 성장률($y_t$)이며,
$y_{t-1}$은 1기 시차, $\text{pw}$는 월별(또는 주별) 전력거래량 관련 변수,
$\text{BSI}$는 기업경기실사지수, ``YoY''는 전년동월 대비 변화를 의미한다.
각 열은 순서대로 계수 추정치, 표준오차, t값을 포함한다.
유의수준: \textit{***} $p<0.01$, \textit{**} $p<0.05$, \textit{*} $p<0.10$.
\end{flushleft}
\end{table}

\begin{itemize}
  \item $y_{t-1}$:
    \begin{itemize}
      \item 계수 약 $-0.40$, 1\% 유의수준에서 유의.
      \item 기본적인 평균회귀 패턴 반영.
    \end{itemize}
  \item 전력거래량 변수:
    \begin{itemize}
      \item 계수 크기 매우 작고 모두 비유의.
      \item 본 사양에서는 추가 설명력 제한적.
    \end{itemize}
  \item BSI 관련 변수:
    \begin{itemize}
      \item $t-1$, $t$ 시점의 BSI 수준 및 전년동월비 모두 5\% 수준 내에서 유의.
      \item 경기국면 및 실물활동에 대한 선행·동행 정보 제공.
      \item AR(1) 대비 추가적인 예측력의 상당 부분을 BSI가 제공하는 것으로 해석 가능.
    \end{itemize}
\end{itemize}

\subsection{종속변수: 전년동월비}

\subsubsection{예측 성능: RMSE 및 상대 개선율}

\begin{table}[H]
\centering
\begin{tabular}{lcccc}
\toprule
Vintage & AR(1) & ARX (linear) & AR(1)+XGB\_residual & XGB-direct \\
\midrule
h0 & 1.49 (0.0)         & 1.51 (-1.6)        & 1.52 (-2.3)        & \textbf{1.42} (4.4) \\
h1 & \textbf{1.48} (0.0)& 1.58 (-6.4)        & 1.55 (-4.3)        & 1.61 (-8.6)         \\
h2 & \textbf{1.48} (0.0)& 1.58 (-6.4)        & 1.58 (-6.7)        & 1.55 (-4.3)         \\
h3 & \textbf{1.48} (0.0)& 1.58 (-6.4)        & 1.55 (-4.3)        & 1.58 (-6.7)         \\
h4 & \textbf{1.48} (0.0)& 1.53 (-2.9)        & 1.53 (-2.9)        & 1.52 (-2.3)         \\
\bottomrule
\end{tabular}
\begin{flushleft}
\caption{테스트 RMSE: 전년동월비 (XGBoost)}
\footnotesize\textit{주}: 각 셀은 2023–2024년 테스트 구간에서의 RMSE와, 괄호 안의 AR(1) 대비 RMSE 감소율(\%)을 함께 보고한다. 감소율은 $100 \times (1 - \mathrm{RMSE}_{m,h}/\mathrm{RMSE}_{\mathrm{AR(1)},h})$로 정의되며, 양수 값은 동일한 vintage에서 AR(1) 모형보다 예측 오차가 작다는 것을 의미한다.
\end{flushleft}
\label{tab:xgb_rmse_yoy}
\end{table}

\begin{itemize}
  \item \textbf{벤치마크(AR(1)) 성능}
  \begin{itemize}
    \item 모든 vintage에서 AR(1)의 테스트 RMSE는 약 1.48--1.49 수준으로 거의 동일함.
    \item 전년동월비 기준에서도 단순 AR(1)이 안정적인 기준 예측력을 제공함.
  \end{itemize}

  \item \textbf{ARX (linear)}
  \begin{itemize}
    \item 모든 vintage에서 AR(1) 대비 RMSE 감소율이 음수(약 $-1.6\%\sim -6.4\%$)로, 선형으로 고빈도 변수를 추가하면 예측력이 일관되게 악화됨.
    \item 특히 $h1$--$h3$에서는 약 $6\%$ 수준의 성능 저하가 반복적으로 관찰됨.
  \end{itemize}

  \item \textbf{AR(1)+XGB\_residual}
  \begin{itemize}
    \item 모든 vintage에서 감소율이 약 $-2.3\%\sim -6.7\%$로, AR(1)보다 RMSE가 항상 더 큼.
    \item AR(1)이 이미 설명한 구조 위에 잔차에 대해 부스팅을 적용하면, 전년동월비 기준에서는 잔차의 노이즈를 과적합하는 경향이 강함.
  \end{itemize}

  \item \textbf{XGB-direct}
  \begin{itemize}
    \item $h0$에서만 AR(1) 대비 약 $4.4\%$의 RMSE 감소가 관찰되며, 전월까지의 정보만 사용 가능한 시점에서는 비선형 모형이 일부 개선 효과를 보임.
    \item $h1$--$h4$에서는 감소율이 모두 음수(약 $-2.3\%\sim -8.6\%$)로 나타나, 고빈도 정보를 비선형으로 활용하더라도 AR(1)보다 예측오차가 더 큼.
  \end{itemize}

  \item \textbf{종합}
  \begin{itemize}
    \item 전년동월비를 종속변수로 사용한 경우, 대부분의 vintage에서 AR(1) 벤치마크가 가장 안정적인 선택으로 나타남.
    \item $h0$에서의 XGB-direct만 제한적인 개선을 보이며, 그 외의 경우에는 고빈도 변수의 선형·비선형 확장이 예측력을 유의미하게 개선하지 못함.
  \end{itemize}
\end{itemize}

\subsubsection{시각적 비교와 feature importance}

\paragraph{테스트 예측 경로}

\begin{figure}
    \centering
    \includegraphics[width=\textwidth]{midas/images/xgboost_test_mom_plot.png}
    \caption{테스트 기간 예측 경로: 전년동월비 (XGBoost)}
    \label{fig:xgboost_test_mom_plots}
\end{figure}

\begin{itemize}
    \item AR(1)+XGB\_residual 모형은 일부 시점에서 진폭을 과도하게 확대하여 실제값 주변의 노이즈까지 추종하는 현상이 관찰되며, 이는 앞서 확인된 RMSE 결과의 성능 악화와 일치하는 패턴임.
    \item XGB-direct 모형은 일부 vintage에서 국지적으로 실제값에 더 근접하는 구간이 있으나, 전반적인 궤적은 여전히 AR(1)과 유사하며 전체 RMSE 관점에서는 뚜렷한 우위를 보이지 않음.
    \item 종합적으로, 전년동월비 기준 테스트 구간에서도 고빈도 변수를 활용한 비선형 확장이 AR(1) 벤치마크 대비 예측 경로를 체계적으로 개선하지 못함.
\end{itemize}

\paragraph{Feature importance (Gain 기준)}

\begin{figure}[H]
    \centering
    \includegraphics[width=\textwidth]{midas/images/heatmap_mom.png}
    \caption{변수 중요도 히트맵: 전년동월비}
    \label{fig:xgb-imp-heatmap_iip}
\end{figure}
\begin{itemize}
  \item XGB-direct:
    \begin{itemize}
      \item 모든 vintage에서 1기 시차 종속변수 $y_{t-1}$의 중요도가
            가장 높음.
      \item XGBoost가 기본적으로 강한 자기회귀 구조를 먼저 학습하고 있음을 시사.
      \item 다음으로는 전월 BSI 수준(\texttt{bsi\_tm1})과 전월 BSI 전년동월비(\texttt{bsi\_tm1\_yoy})의 중요도가 상대적으로 크게 나타나, 경기심리지표가 보조적인 예측 정보를 제공하고 있음을 보여줌
      \item 현재 월 주별 전력거래량 및 YoY 변수(\texttt{pw\_t\_w1},
            \texttt{pwyoy\_t\_w1} 등)는 일부 vintage에서만 제한적인
            중요도를 가지며, 전체적으로는 기여도가 크지 않음.
    \end{itemize}
  \item AR(1)+XGB\_residual:
    \begin{itemize}
      \item 잔차를 설명하는 단계에서는 전월 BSI 전년동월비 (\texttt{bsi\_tm1\_yoy})와 전월 1주 전력거래량 (\texttt{pw\_tm1\_w1})의 중요도가 상대적으로 크게 나타남.
      \item 이는 AR(1)이 설명하지 못한 부분이 전월 경기심리의 변화와 전월 초 전력 사용 변화와 연관되어 있음을 시사하지만, 잔차의 신호대잡음비가 낮아 전체 예측 RMSE 개선으로 이어지지는 않음.
      \item 동월 BSI(\texttt{bsi\_t}, \texttt{bsi\_t\_yoy})는 주로 모든 정보가 이용 가능한 $h4$에서만 의미 있는 중요도를 보임.
    \end{itemize}
  \item 전반적으로, XGBoost 모형에서 가장 핵심적인 설명변수는 종속변수의 래그값과 BSI이며, 주별 전력거래량 변수의 한계적 기여는 제한적인 것으로 해석됨.
\end{itemize}


\subsubsection{ARX 모형: BSI와 전력거래량의 역할}

\begin{table}[H]
\centering
\begin{tabular}{lrrr}
\toprule
변수 & 계수 추정치 & 표준오차 & t값 \\
\midrule
상수항                    & -11.587$^{***}$ &  2.754 & -4.21 \\
$y_{t-1}$                 &   0.398$^{***}$ &  0.066 &  6.07 \\
$\text{pw}_{t-1,w1}$      &  -0.000         &  0.000 & -0.45 \\
$\text{BSI}_{t-1}$        &   0.113$^{*}$   &  0.046 &  2.48 \\
$\text{BSI}_{t-1}^{\text{YoY}}$
                          &  -0.076$^{**}$  &  0.029 & -2.64 \\
$\text{pw}_{t,w1}$        &   0.000         &  0.000 &  1.58 \\
$\text{pw}_{t,w1}^{\text{YoY}}$
                          &  -0.008         &  0.006 & -1.28 \\
$\text{BSI}_{t}$          &   0.004         &  0.045 &  0.08 \\
$\text{BSI}_{t}^{\text{YoY}}$
                          &   0.123$^{***}$ &  0.028 &  4.40 \\
\midrule
$R^{2}$       & \multicolumn{3}{r}{0.788} \\
조정 $R^{2}$  & \multicolumn{3}{r}{0.779} \\
관측치 수     & \multicolumn{3}{r}{188}   \\
\bottomrule
\end{tabular}
\caption{ARX 모형 추정 결과}
\begin{flushleft}
\footnotesize
\textit{주}: 종속변수는 전산업생산지수 월별 성장률($y_t$)이며,
$y_{t-1}$은 1기 시차, $\text{pw}$는 전력거래량 관련 변수,
$\text{BSI}$는 기업경기실사지수, ``YoY''는 전년동월 대비 변화를 의미한다.
각 열은 순서대로 계수 추정치, 표준오차, t값을 포함한다.
유의수준: \textit{***} $p<0.01$, \textit{**} $p<0.05$, \textit{*} $p<0.10$.
\end{flushleft}
\label{tab:arx_mom}
\end{table}

\begin{itemize}
  \item $y_{t-1}$:
    \begin{itemize}
      \item 계수 약 $0.40$, 1\% 유의수준에서 매우 유의.
      \item 전년동월비가 상당한 자기상관(지속성)을 갖고 있음을 보여줌.
    \end{itemize}

  \item 전력거래량 변수:
    \begin{itemize}
      \item 전월·당월 1주 전력 및 그 전년동월비 계수는 모두 0에 가깝고 비유의.
      \item 이 사양에서 전력거래량만으로는 추가적인 설명력이 거의 없는 것으로 나타남.
    \end{itemize}

  \item BSI 관련 변수:
    \begin{itemize}
      \item 전월 BSI 수준($\text{BSI}_{t-1}$)과 전월 BSI 전년동월비($\text{BSI}^\text{YoY}_{t-1}$)는 각각 10\%, 5\% 수준에서 유의하며, 과거 경기심리가 향후 전년동월비 변화에 정보를 제공함을 시사.
      \item 동월 BSI 수준($\text{BSI}_{t}$)은 비유의이지만, 동월 BSI 전년동월비($\text{BSI}^\text{YoY}_{t}$)는 1\% 수준에서 유의한 양(+)의 계수:
      \item 전반적으로, AR(1)에서 포착하지 못한 부분 중 상당 부분은 BSI 수준·증감에 의해 설명되고, 전력 변수의 추가 기여는 제한적인 것으로 해석 가능.
    \end{itemize}

  \item 적합도:
    \begin{itemize}
      \item $R^{2}\approx 0.79$, 조정 $R^{2}\approx 0.78$로 비교적 높은 설명력.
      \item 다만 앞선 테스트 RMSE 결과와 함께 볼 때, 높은 인샘플 적합에 비해 out-of-sample 예측력 개선은 제한적이라는 점을 시사.
    \end{itemize}
\end{itemize}

\section{요약: 변수 선택 관점에서 본 결과}

\begin{itemize}
    \item \textbf{전월대비 성장률 기준}
        \begin{itemize}
            \item 전력거래량
            \begin{itemize}
                \item 시간대별 자료를 로그–STL 계절조정 후 주간 로그 성장률로 변환.
                \item (t-1)월, t월 주별 성장률 및 전년동월비를 MIDAS 가중합·ARX·XGB feature로 사용.
                \item 대부분의 모형에서 계수·Gain 모두 작고 비유의, AR(1) 대비 RMSE 개선 거의 없음.
            \end{itemize}
            \item BSI
            \begin{itemize}
                \item 수준(\texttt{bsi\_tm1}, \texttt{bsi\_t})과 전년동월비(\texttt{bsi\_tm1\_yoy}, \texttt{bsi\_t\_yoy})로 분해해 사용.
                \item 선형 ARX와 XGBoost에서 전월·동월 BSI 변수들의 계수 및 Gain이 상대적으로 큼.
                \item 다만 전월대비 기준 out-of-sample RMSE는 AR(1) 대비 소폭·불안정한 개선에 그침.
            \end{itemize}
        \end{itemize}

    \item \textbf{전년동월비 기준}
    \begin{itemize}
        \item 전력거래량
        \begin{itemize}
            \item 동일한 변환(로그–STL 계절조정 후 주간 성장률·전년동월비)을 적용해 feature 구성.
            \item ARX, AR(1)+XGB\_residual, XGB-direct에서 중요도 낮고, 대부분 vintage에서 AR(1)보다 RMSE 악화.
            \item 예외적으로 $h0$–XGB-direct만 약 4\% 수준의 RMSE 개선, 다른 vintage로 일반화되지 않음.
        \end{itemize}
        \item BSI
        \begin{itemize}
            \item 전월 수준/전년동월비, 동월 전년동월비가 유의한 계수 및 높은 Gain을 보임.
            \item 과거·현재 경기심리 변화가 전년동월 기준 실물 변동에 일정 정보 제공.
            \item 그럼에도 대부분 vintage에서 AR(1)이 여전히 가장 안정적인 예측 성능 유지.
        \end{itemize}
    \end{itemize}

    \item \textbf{종합}
    \begin{itemize}
        \item 두 종속변수 모두에서 가장 일관된 설명력은 1기 시차 종속변수 $y_{t-1}$에서 나옴.
        \item 전력거래량 고빈도 변수는 다양한 변환·모형에도 불구하고 한계적 기여(보조 변수 수준)에 머묾.
        \item BSI 수준·전년동월비는 인샘플 적합과 feature importance 관점에서 의미 있는 정보 제공.
        \item 그러나 테스트 RMSE 기준으로는 AR(1) 벤치마크 대비 뚜렷한·일관된 예측력 개선까지는 이어지지 않음.
        \item 현재 변환 설정(주별 전력 성장률, BSI 수준·전년동월비)을 기준으로 볼 때, nowcasting에서 핵심 변수는 $y_{t-1}$과 BSI 계열 변수, 전력 변수는 부차적 설명 변수로 정리됨.
    \end{itemize}
    \end{itemize}



% ============================================
% Part II: 거시경제 변수 딥러닝 예측 실험
% ============================================
\part{거시경제 변수 딥러닝 예측 실험}

\setcounter{section}{0}
\section{개요}
\label{sec:overview}

\subsection{연구 목적}
\label{sec:macro_forecast_necessity}

효과적인 거시경제변수 예측은 정책 수립 및 의사결정의 기본이 됨. \cite{stock2002forecasting, banbura2012nowcasting}. 본 연구에서는 한국 거시경제의 대표 변수인 생산, 투자, 소비에 대해 고빈도 정보를 활용한 예측(nowcasting/forecasting) 문제를 설정하고, 상태공간 모형과 딥러닝 모형의 성능을 동일한 데이터와 평가 기준으로 비교함 \cite{banbura2012nowcasting, bok2019frbny}.

\subsection{거시 경제 변수 예측 주요 이슈}
\label{sec:forecast_issues}

거시경제 nowcasting/forecasting에 대표적인 과제는 다음과 같음:
\begin{itemize}
    \item \textbf{고차원 공변량:} 많은 거시·금융·서베이 변수를 동시에 사용시 과적합과 계산 부담 증가\cite{stock2002forecasting}.
    \item \textbf{혼합 주기/비동기 발표:} 주·월·분기 데이터가 섞이고 발표시점이 달라(jagged edges) 결측이 구조적으로 발생 \cite{banbura2012nowcasting}.
    \item \textbf{비선형/구조변화:} 위기·팬데믹 구간 등에서 선형 가정이 성능을 제한할 수 있음 \cite{huber2020nowcasting, andreini2020deep}.
\end{itemize}

\section{데이터와 전처리}
\label{sec:data_introduction}

\subsubsection{기본 데이터 탐색}
\label{subsec:data_exploration}

\begin{itemize}
  \item 시계열 간 스케일 비율이 크게 달라 수치적 정밀도 문제를 야기.
  \item 일부 시계열이 매우 낮은 분산을 보여 수치적 불안정성을 유발할 수 있음.
  \item 완전한 관측값(complete cases)의 비율이 낮아 대부분의 관측값이 상당부분 결측치를 포함.
\end{itemize}

데이터 품질 및 통계량 대시보드는 그림~\ref{fig:data_quality_dashboard}에 제시됨.
\begin{figure}[htbp]
    \centering
    \includegraphics[width=\textwidth]{forecast/images/data_quality_dashboard.png}
    \caption{데이터 품질 및 통계량 대시보드}
    \label{fig:data_quality_dashboard}
\end{figure}

\subsubsection{목표 변수 및 설명변수 구성}
\label{subsec:key_variables}

본 파트에서는 세 가지 주요 거시경제 변수를 목표 변수로 설정함:
\begin{itemize}
    \item \textbf{생산:} 전산업생산지수(KOIPALL.G)
    \item \textbf{투자:} 설비투자지수(KOEQUIPTE)
    \item \textbf{소비:} 도소매판매액(KOWRCCNSE)
\end{itemize}

\begin{itemize}
  \item 세 부문 모형 모두 총 41개 변수로 구성됨.
  \item 포함 변수:
  \begin{itemize}
    \item 고용, 산업생산, 서베이(기업경기, 소비자 동향) 등 주요 월간 지수.
    \item 주간 데이터.
    \item 주가지수 등 금융변수, 뉴스심리지수, 미국 경제정책불확실성 지수.
  \end{itemize}
  \item 기업경기동향 조사는 해당월 중 발표되어 속보성이 높음.
\end{itemize}

\subsubsection{전처리}
\label{subsec:preprocessing}

본 연구에서는 모든 모형에 동일한 전처리 파이프라인을 적용하여 공정한 비교를 보장함.

\begin{itemize}
  \item \textbf{변환(Transformation):} 각 시계열의 특성에 맞는 변환을 적용함. 변환 유형: lin(수준값), log(로그), chg(전기대비 차분), ch1(전년동기대비 차분), pch(전기대비 성장률), pc1(전년동기대비 성장률), cha(연율화 차분), pca(연율화 성장률).
  \item \textbf{결측치 처리(Imputation):} 다음 순서로 처리함:
  \begin{enumerate}
    \item forward-fill: 이전 값으로 채움.
    \item backward-fill: 이후 값으로 채움.
    \item naive forecaster: 마지막 관측값으로 채움.
  \end{enumerate}
  \item \textbf{표준화(Scaling):} 모든 모형에 RobustScaler를 적용함. 중앙값(median)을 0으로, 사분위수 범위(IQR)를 1로 조정하여 이상치에 강건한 표준화를 수행함.
  \item \textbf{데이터 단위:} 원본 데이터는 주간 단위로 제공됨. 모든 모형은 주간 데이터로 학습하고, 동일한 단위(주간)로 예측을 생성함(리샘플링 없음).
\end{itemize} 

모형별 주간-월간 변환 방식은 다음과 같음:
\begin{itemize}
    \item \textbf{DFM/DDFM:} 주간 데이터를 기본으로 하며, 혼합주기 옵션을 통해 tent kernel이 자동으로 적용되어 주간/월간 데이터를 통합 처리함 \cite{mariano2003new}. 예측 생성 시 horizon은 개월 단위로 지정되며, 모형 내부에서 자동으로 주 단위로 변환됨(1개월 = 4주). 예측 결과는 주간 단위로 생성되며, 평가를 위해 월간으로 평균 집계함.
    
    \item \textbf{딥러닝 모형(TFT, Chronos, LSTM):} 모든 딥러닝 모형은 주간 데이터로 학습하고, 같은 단위(주간)로 예측을 생성함. 예측 생성 후 평가를 위해 주간 예측을 월간으로 변환하는데, 이는 월별로 주간 예측값을 평균 집계하는 방식으로 수행됨. 구체적으로, 각 월에 해당하는 주간 예측값들을 평균하여 월간 예측값을 생성함.
\end{itemize}

\begin{itemize}
  \item 예측 평가 시 모든 모형의 주간 예측을 월간으로 평균 집계한 후, 원본 목표 변수(월간 단위)와 비교함.
  \item 모든 모형을 동일한 기준으로 평가하기 위한 것으로, 주간 예측의 세부 패턴보다는 월간 집계 수준에서의 예측 정확도를 중시함.
\end{itemize}

세 대상 변수에 대한 전처리 결과는 그림~\ref{fig:preprocessed_targets}에 제시됨.
\begin{figure}[htbp]
    \centering
    \includegraphics[width=\textwidth]{forecast/images/preprocessed_targets.png}
    \caption{전처리된 목표 변수 시계열}
    \label{fig:preprocessed_targets}
\end{figure}

\section{모형과 실험 설계}
\label{sec:methodology}

\subsection{예측 모형}
\label{sec:forecasting_models}

\subsubsection{딥러닝 시계열 모형}
\label{subsec:deep_learning_models}

\begin{itemize}
  \item \textbf{예측 생성 방식:}
  \begin{itemize}
    \item 직접 장기 예측(direct long-horizon forecasting): 전체 예측 시점(88주 = 22개월)을 한 번에 예측.
    \item 재귀적 예측(recursive forecasting): 짧은 구간씩 예측을 반복하여 전체 시점에 도달.
  \end{itemize}
  \item \textbf{Temporal Fusion Transformer (TFT):}
  \begin{itemize}
    \item Attention 기반 아키텍처로, 다중 시점 예측과 해석 가능성을 결합한 모형 \cite{lim2021temporal}.
    \item LSTM을 지역 처리에 사용하고 self-attention을 장기 의존성에 사용.
    \item Variable Selection Networks를 통해 변수별 중요도를 해석할 수 있어 경제 예측에 유용 \cite{lim2021temporal}.
    \item 본 연구에서는 재귀적 예측 방식으로 사용하며, 24주 구간을 반복 예측하여 전체 88주(22개월) 예측을 생성. 모형은 50개의 공변량과 함께 학습되며, 예측 시에도 동일한 공변량을 제공하여 학습 시와 일관된 조건에서 예측을 수행.
  \end{itemize}
  \item \textbf{Chronos:}
  \begin{itemize}
    \item 사전 훈련된 foundation model로, 대규모 시계열 데이터로 사전 훈련되어 다양한 시계열 패턴을 학습함 \cite{ansari2024chronos}.
    \item Transformer 기반 아키텍처를 사용하여 장기 의존성을 포착함.
    \item 본 연구에서는 직접 장기 예측 방식으로 사용하며, 전체 88주(22개월)를 한 번에 예측함.
  \end{itemize}
  \item \textbf{LSTM:}
  \begin{itemize}
    \item 순환 신경망(RNN)의 변형으로, forget gate, input gate, output gate를 통해 정보의 흐름을 제어하며 장기 의존성을 학습할 수 있음 \cite{hochreiter1997long}.
    \item Gradient vanishing 문제를 완화하여 긴 시계열에서도 효과적으로 학습.
    \item 본 연구에서는 직접 장기 예측 방식으로 사용하며, 전체 88주(22개월)를 한 번에 예측함.
  \end{itemize}
\end{itemize}

\subsubsection{상태공간 모형}
\label{subsec:state_space_models}

\begin{itemize}
  \item \textbf{동적요인모형(DFM):}
  \begin{itemize}
    \item 많은 시계열에서 공통 요인을 추출해 소수의 동태적 요인으로 설명하는 차원축소 기법 \cite{stock2002forecasting}.
    \item 관측식과 상태식을 갖는 state-space 형태로 표현됨.
    \item EM 알고리즘으로 파라미터를 추정하고, Kalman filter를 통해 요인을 추정함 \cite{bok2019frbny}.
    \item 혼합주기 데이터와 비동기적 데이터 발표(jagged edges)를 처리하는 데 강점이 있음 \cite{banbura2012nowcasting, bok2019frbny}.
    \item EM 알고리즘의 수치적 안정성을 위해 적응적 정규화 기법을 적용함: 요인 공분산 행렬의 조건수가 $10^8$ 이상일 때 정규화 계수를 조건수에 비례하여 조정하여 ill-conditioned 행렬 역행렬 문제를 완화함.
    \item 전이 행렬의 스펙트럴 반경이 1보다 큰 경우 예측 발산이 발생할 수 있어, 본 실험에서는 재귀적 예측(6개월 구간 단위)을 적용하여 안정성을 확보함.
  \end{itemize}
  \item \textbf{심층 동적요인모형(DDFM):}
  \begin{itemize}
    \item 오토인코더 기반 비선형 인코더를 사용해 요인 구조를 학습함으로써 전통적 DFM의 선형 가정을 완화함 \cite{andreini2020deep}.
    \item 비선형 인코더는 고차원 거시 데이터의 복잡한 상호작용을 더 적은 요인으로 포착함.
    \item 요인층 뒤에는 선형 state-space를 두어 필터링·스무딩 안정성을 유지함.
    \item 학습은 두 단계로 구성됨:
    \begin{enumerate}
      \item 오토인코더를 통해 재구성 오차를 최소화하여 요인 구조를 학습.
      \item 학습된 요인을 사용하여 전이 행렬을 추정하고 Kalman filter를 통해 최종 스무딩을 수행함.
    \end{enumerate}
    \item 전통적 DFM의 요인 식별 제약 문제를 자연스럽게 해결하며, 혼합주기 데이터와 대규모 변수 집합을 효율적으로 처리할 수 있음.
  \end{itemize}
\end{itemize}

\subsection{실험 구성}
\label{sec:experiment_design}

\subsubsection{평가 기준}
\label{subsec:evaluation_criteria}

본 연구에서는 데이터를 세 구간으로 분할하여 모형 학습 및 평가를 수행함:

\begin{itemize}
    \item \textbf{훈련 기간(Train):} 1985년 1월부터 2019년 12월까지 (35년간). 모든 모형은 이 기간의 데이터를 사용하여 학습함. 이 기간은 충분히 긴 시계열을 제공하여 모형이 장기적 패턴과 계절성을 학습할 수 있도록 함.

    \item \textbf{테스트 기간(Test):} 2024년 1월부터 2025년 10월까지 (22개월). 모든 모형의 예측 성능을 평가하는 기간으로, 실제 예측 상황을 시뮬레이션. 각 모형은 훈련 기간 데이터로 학습한 후, 테스트 기간에 대해 예측을 생성함. 구체적으로, DFM은 발산 문제를 완화하기 위해 재귀적 예측(6개월 구간 단위)을 사용하며, DDFM은 발산 문제가 발생하지 않아 재귀적 예측 없이 한 번에 전체 22개월을 예측하는 일괄 예측 방식을 사용함.
\end{itemize}

\begin{itemize}
  \item 모든 모형은 주간 단위로 예측을 생성하며, 원본 목표 변수가 월간 단위이므로 주간 예측을 월간으로 평균 집계하여 비교함.
  \item 주간 예측을 월간으로 평균 집계한 후, 각 예측 시점(1--22개월)에 대한 지표를 평균하여 최종 성능 지표로 사용함.
\end{itemize}

\subsubsection{하이퍼 패러미터}
\label{subsec:hyperparameters}

\begin{itemize}
  \item \textbf{CHRONOS:} amazon/chronos-t5-tiny (pre-trained foundation model), prediction length 24 weeks, robust scaler.
  \item \textbf{DDFM:} encoder layers [64, 32], num factors 3, epochs 50, learning rate 0.005, batch size 100, factor order 2, robust scaler.
  \item \textbf{DFM:} max EM iterations 5000, convergence threshold $1.0 \times 10^{-5}$, 3 factors, AR lag 1, mixed frequency enabled, robust scaler.
  \item \textbf{LSTM:} input size 96 weeks, hidden size 64, 2 layers, learning rate 0.001, epochs 50, batch size 32, robust scaler.
  \item \textbf{TFT:} input size 96 weeks, hidden size 64, 4 attention heads, dropout 0.1, learning rate 0.001, max epochs 10, batch size 256, max covariates 50, robust scaler.
\end{itemize}

\subsubsection{성능 지표}
\label{subsec:performance_metrics}

\begin{itemize}
  \item \textbf{표준화된 지표:} sMAE, sMSE. 두 지표 모두 훈련 데이터의 표준편차로 정규화하여 계산함. 변수 간 스케일 차이를 제거하여 비교 가능하게 하며, 특히 거시경제 변수처럼 단위와 크기가 다른 변수들을 비교할 때 유용 \cite{stock2002forecasting}.
  \item \textbf{지표 정의:} 월간 실측값을 $y_{t+h}$, 월간 예측값을 $\hat{y}_{t+h}$로 두고, 훈련 구간 목표변수의 표준편차를 $\sigma_{\text{train}}$로 두면 각 시점(예측지평 $h$)에서의 지표는 다음과 같음:
  \begin{align}
    \mathrm{MAE}_h &= \frac{1}{T_h}\sum_{t \in \mathcal{T}_h} \left|y_{t+h} - \hat{y}_{t+h}\right|, \\
    \mathrm{MSE}_h &= \frac{1}{T_h}\sum_{t \in \mathcal{T}_h} \left(y_{t+h} - \hat{y}_{t+h}\right)^2, \\
    \mathrm{sMAE}_h &= \frac{\mathrm{MAE}_h}{\sigma_{\text{train}}}, \\
    \mathrm{sMSE}_h &= \frac{\mathrm{MSE}_h}{\sigma_{\text{train}}^2}.
  \end{align}
  여기서 $\mathcal{T}_h$는 예측지평 $h$에서 평가 가능한 시점들의 집합이며, $T_h = |\mathcal{T}_h|$임.
  \item \textbf{평가 절차:} 모든 모형은 주간 데이터로 학습하고 주간 단위로 예측을 생성함. 평가는 주간 예측을 월간 평균으로 집계한 뒤, 월간 실측치와 비교하여 수행함. 최종 성능 지표는 $h=1,\dots,22$에 대해 계산한 $\mathrm{sMAE}_h$, $\mathrm{sMSE}_h$를 평균하여 요약함.
\end{itemize}

\section{실험 결과}
\label{sec:experiment_result}

\subsection{예측 결과 요약}
\label{sec:forecasting_results_comparison}

\begin{itemize}
  \item 대상 변수: 생산(KOIPALL.G), 투자(KOEQUIPTE), 소비(KOWRCCNSE).
  \item 비교 모형: DFM, DDFM, TFT, Chronos, LSTM.
  \item 모든 모형은 주간 단위로 예측을 생성하며, 주간 예측을 월간으로 평균 집계한 후 1개월부터 22개월까지의 시점에 대해 평가함.
  \item 예측 결과는 아래 전체 시점 평균 성능 섹션에서 요약되며, 시점별 상세 결과는 부록 표들(표~\ref{tab:koipallg_forecasts}, 표~\ref{tab:koequipte_forecasts}, 표~\ref{tab:kowrccnse_forecasts})에 포함됨.
\end{itemize}

\subsubsection{전체 시점 평균 성능}

변수별로 최우수 모형이 다르게 나타났으며, 특히 예측 시점(단기 vs 장기)에 따라 강점이 달라지는 양상이 관찰됨. 이는 (i) 목표 변수의 변동성·구조변화 정도, (ii) 공변량의 정보량, (iii) 모형의 예측 방식(재귀 vs 일괄) 차이가 동시에 작용한 결과로 해석할 수 있음. 한편 DFM은 전이 행렬의 불안정성으로 인해 테스트 구간에서 오차가 크게 나타났음(논의 섹션 참조).

\begin{itemize}
  \item \textbf{KOIPALL.G(생산):} Chronos(sMAE=1.44) > LSTM(sMAE=1.87) > DDFM(sMAE=1.72) > TFT(sMAE=2.07) > DFM(sMAE=3.55). 단기 예측에서는 DDFM이 가장 우수하나, 장기 예측에서는 Chronos와 LSTM이 우수.
  
  \item \textbf{KOEQUIPTE(투자):} TFT(sMAE=0.53) > DDFM(sMAE=1.50) > Chronos(sMAE=2.60) > DFM(sMAE=3.99) > LSTM(sMAE=5.04). 단기 예측에서는 DDFM이 가장 우수하나, 장기 예측에서는 TFT가 크게 우수함.
  
  \item \textbf{KOWRCCNSE(소비):} DDFM(sMAE=0.32) > TFT(sMAE=1.48) > Chronos(sMAE=2.44) > LSTM(sMAE=2.87) > DFM(sMAE=3.82). 단기 및 장기 예측 모두에서 DDFM이 가장 우수함.
\end{itemize}

테스트 기간 동안의 예측값과 실제값 비교는 그림~\ref{fig:forecast_vs_actual_all}에 제시됨. KOIPALL.G에서는 Chronos와 LSTM이 실제값을 비교적 잘 추적하며, KOEQUIPTE에서는 TFT가 실제값을 가장 정확하게 추적함. KOWRCCNSE에서는 DDFM이 실제값을 가장 정확하게 추적함. DFM은 재귀적 예측으로 인해 예측값이 들쭉날쭉한 패턴을 보임.
\begin{figure}[htbp]
    \centering
    \includegraphics[width=\textwidth]{forecast/images/forecast_vs_actual_all.png}
    \caption{예측값 vs 실제값 비교: 목표 변수 월별 실제값과 모델 예측값}
    \label{fig:forecast_vs_actual_all}
\end{figure}

\subsubsection{시점별 성능 패턴}
\label{subsec:horizon_performance}

\begin{itemize}
  \item \textbf{단기(1--6개월):} 전반적으로 DDFM이 상대적으로 강한 경향이 관찰됨. 이는 고차원 공변량의 공통요인을 비선형 인코더로 압축한 뒤 상태공간 구조로 단기 변동을 안정적으로 추적하는 메커니즘이 단기 예측에서 유리하게 작동했을 가능성을 시사함.
  \item \textbf{중기(7--12개월):} 모형 간 격차가 점차 확대되는 구간으로, 목표 변수에 따라 ``단기 우위''가 유지되기도 하고 약화되기도 함. 특히 예측 방식(재귀 vs 일괄), 공변량 활용 방식(선택/가중 vs 요인 압축), 그리고 주간 예측을 월간 평균으로 집계하는 평가 방식이 결합되면서, 특정 모형의 상대적 장점이 중기부터 달라질 수 있음.
  \item \textbf{장기(13--22개월):} 생산·투자에서는 TFT/Chronos/LSTM이 상대적으로 안정적인 경우가 있었고, 소비에서는 DDFM 우위가 유지됨. 장기 예측에서는 잠재요인 동학의 단순화, 구조변화, 그리고 누적 오차가 더 크게 작용할 수 있으며, 이때 장기 의존성 학습(Transformer/RNN)이나 공변량 선택 구조(TFT)가 일부 변수에서 더 안정적으로 나타날 수 있음.
  \item \textbf{변수별 요약}: (i) 생산은 장기 구간에서 Chronos/LSTM이 상대적으로 강했고, (ii) 투자는 TFT가 전반적으로 우세했으며, (iii) 소비는 DDFM의 우위가 비교적 일관되게 관찰됨.
  \item 시점별 상세 수치는 부록 표(표~\ref{tab:koipallg_forecasts}--\ref{tab:kowrccnse_forecasts})에 제시됨.
\end{itemize}

\subsubsection{결과 해석(변수별)}

\begin{itemize}
  \item \textbf{생산(KOIPALL.G):} 생산지수는 비교적 완만한 추세·계절성이 존재하고, 다양한 공변량(서베이/금융/대외지표)이 광범위하게 연동되는 경향이 있어, 장기 의존성을 학습하거나(Transformer/RNN) 사전 학습 분포의 일반 패턴을 활용하는 모형(Chronos)이 유리할 수 있음.
  \item \textbf{투자(KOEQUIPTE):} 투자지수는 경기·금융여건 변화에 민감하고 변동성이 커서, 공변량의 단기 신호(금리/스프레드/서베이)를 잘 선택·가중하는 구조(TFT의 변수 선택/attention)가 유리할 수 있음.
  \item \textbf{소비(KOWRCCNSE):} 소비는 다변량 공통요인 구조의 영향이 비교적 뚜렷할 수 있어, 고차원 공변량을 잠재요인으로 압축하고 상태공간 업데이트로 노이즈를 완화하는 요인 모형 계열(DDFM)이 안정적으로 작동할 가능성이 있음.
\end{itemize}

\subsection{논의}
\label{sec:discussion}

\subsubsection{모형별 강점과 한계}

\begin{itemize}
  \item \textbf{DFM:} Kalman filter 기반 state-space 모형은 혼합주기/결측/비동기 발표를 구조적으로 처리할 수 있어 nowcasting에 적합함 \cite{banbura2012nowcasting, bok2019frbny}. 다만 본 실험에서는 전이 행렬 안정화가 충분하지 않아(스펙트럴 반경 $>1$) 예측 발산 또는 과도한 수렴이 관찰되었고, 이를 완화하기 위해 DFM에만 재귀 예측(구간 단위) 적용함. 이로 인해 예측 경로의 불연속/들쭉날쭉함이 나타날 수 있으며, 모형 간 비교의 정합성에도 영향.
  \item \textbf{DDFM:} 비선형 인코더로 잠재요인을 학습해 선형 가정의 한계를 일부 완화할 수 있으나 \cite{andreini2020deep}, 장기 예측에서는 잠재요인 동학(transition)의 단순화, 그리고 구조변화 구간에서의 일반화 한계가 성능 저하로 나타날 수 있음. 반대로 단기 예측에서는 요인 추출의 이점이 두드러질 수 있음.
  \item \textbf{TFT/LSTM/Chronos:} 딥러닝 계열은 비선형성과 장기 의존성을 직접 학습할 수 있으나, (i) 데이터 양/전처리, (ii) 공변량의 가용성(실시간 예측 시점에 관측 가능한지), (iii) 직접 장기 예측 vs 재귀 예측 선택에 따라 성능 변동이 커질 수 있음.
\end{itemize}

\subsubsection{DFM 선형 동학 수렴 이슈}

본 실험에서는 DFM에만 재귀적 예측(6개월 구간 단위)을 적용하여 발산을 완화했으나, 이는 예측 경로의 불연속성을 야기하고 모형 간 비교의 정합성을 저해함. 발견한 이슈는 다음과 같음:

\begin{itemize}
  \item 전이 행렬(transition matrix)의 스펙트럴 반경이 1보다 크면 상태 변수가 시간에 따라 기하급수적으로 발산하여 예측이 불안정.
  \item 이는 EM 알고리즘의 M-step에서 전이 행렬을 추정할 때 안정성 제약이 충분히 강화되지 않아 발생할 수 있음.
  \item EM 알고리즘의 M-step에서 요인 공분산 행렬($\sum E[Z_t Z_t']$)의 고유값이 반복에 따라 증가하면서 행렬 역행렬 계산 시 수치적 불안정성이 발생. 특히 고정 정규화 계수(예: $10^{-6}$)는 초기 반복에서는 충분하지만, 조건수가 $10^8$ 이상으로 증가하면 불충분해져서 ill-conditioned 행렬 역행렬로 인한 수치적 오차가 누적.
  \item 본 실험에서는 조건수에 비례하여 정규화 계수를 조정하는 적응적 정규화(adaptive regularization) 기법을 적용했으나, 일부 변수에서는 여전히 수렴 문제가 관찰.
  \item 결측치가 많은 경우 유효 관측값이 부족하여 요인 공분산 행렬 추정이 불안정해질 수 있음
  \item 선형 상태공간 모형의 구조적 한계로 인해 비선형 동학이나 구조변화를 포함한 데이터에서 근본적인 수렴 문제가 발생.
\end{itemize}


\subsubsection{평가 설계}

\begin{itemize}
  \item \textbf{주간→월간 평균 집계}: 모든 모형이 주간 단위로 예측을 생성한 뒤 월간 평균으로 평가되므로, 주간 내 변동을 잘 맞추는 모형의 이점이 평가에서 약화될 수 있음. 반대로 월간 평균 수준을 잘 맞추는 모형이 유리할 수 있음.
  \item \textbf{결측치 처리의 누수 가능성}: forward/backward fill 및 naive 채움은 구현 방식에 따라 미래 정보가 암묵적으로 섞일 수 있음. 특히 테스트 구간에서 backward-fill이 사용되었다면, 실시간 nowcasting 관점에서 부적절할 수 있으므로(미래값 사용) 추후 파이프라인 점검 필요.
  \item \textbf{재귀 vs 일괄 예측}: DFM은 구간 단위 재귀 예측, 다른 모형은 일괄 장기 예측 또는 다른 구간 길이로 운영되었기 때문에, ``모형 구조''뿐 아니라 ``예측 운영 방식''이 성능 차이의 일부를 설명할 수 있음. 후속 연구에서는 동일 모형에 대해 두 방식을 모두 비교하는 것이 바람직.
\end{itemize}

\subsection{의의 및 한계}
\label{sec:contribution_limitations}

\subsubsection{의의}
\begin{itemize}
  \item \textbf{동일 전처리 기반 비교}: state-space(DFM/DDFM)과 딥러닝(TFT/Chronos/LSTM)을 동일한 전처리 및 평가 방식 하에서 비교하여, 한국 거시변수 예측에서의 상대적 강점이 변수별로 달라질 수 있음을 제시함.
  \item \textbf{실용적 관찰 제공}: DFM의 수치적 불안정(전이행렬 안정성)과 같은 구현·운영 이슈가 실제 예측 성능을 크게 훼손할 수 있음을 확인하고, 안정화 강화 필요성 제기함.
  \item \textbf{변수별 모델 선택의 근거}: 생산/투자/소비의 성격 차이에 따라 (i) 사전학습 기반 일반화, (ii) 공변량 선택/가중 구조, (iii) 잠재요인 기반 다변량 압축이 각기 다른 성능을 낼 수 있음을 실증 결과와 함께 정리함.
\end{itemize}

\subsubsection{한계 및 개선 방향}
\begin{itemize}
  \item \textbf{실시간(nowcast) 설정의 불완전성}: 예측 시점별로 사용 가능한 정보집합(vintage)을 엄밀히 구성하지 못했으며, 공변량이 예측 시점에 관측 가능한지에 대한 검증이 충분하지 않음.
  \item \textbf{전처리·결측치 처리의 영향}: 결측치 채움 방식은 성능에 큰 영향을 줄 수 있으며, 실시간 예측 관점에서 허용되지 않는 정보 사용(backward-fill 등) 여부 점검 필요.
  \item \textbf{하이퍼파라미터 튜닝의 비대칭}: 모형별 튜닝 자원이 동일하지 않으면 성능 비교가 왜곡될 수 있음. 동일한 튜닝 예산(탐색 공간/반복/검증 프로토콜) 적용이 바람직함.
  \item \textbf{예측 방식 차이}: 재귀/일괄 예측의 차이는 장기 예측에서 성능과 안정성에 직접 영향 있음. 동일 모형에 대해 재귀 구간 길이와 업데이트 전략을 체계적으로 비교할 필요 있음.
  \item \textbf{DFM 수렴 안정성 개선}: 전이 행렬의 스펙트럴 반경 제약 강화, EM 알고리즘의 적응적 정규화 파라미터 튜닝, 그리고 데이터 품질 개선(결측 패턴 분석, 공선성 감소)을 통해 수렴 안정성을 향상시킬 수 있음. 또한 비선형 동학을 포착할 수 있는 확장 모형(예: 비선형 전이 함수, regime-switching)을 고려할 수 있음.
\end{itemize}

\section{결론}
\label{sec:conclusion}

본 파트의 비교 실험에서는 변수별로 우수한 모형이 달랐으며(생산: Chronos, 투자: TFT, 소비: DDFM), 시점(단기/장기)에 따라서도 상대적 우위가 달라질 수 있음을 확인함. 특히 단기(1--6개월)에서는 DDFM의 강점이 관찰된 반면, 장기(13--22개월)에서는 생산·투자에서 딥러닝 계열(TFT/Chronos/LSTM)이 상대적으로 안정적인 경우가 나타났고, 소비에서는 DDFM 우위가 비교적 유지되는 경향이 확인되었음. 다만 시간적 제약으로 인해 DDFM의 재귀적 예측을 구현하지 못했음을 감안하면 DDFM 역시 비슷하게 안정적인 결과가 기대됨.

이 결과는 ``어떤 모형이 절대적으로 우수함''이라기보다, (i) 목표 변수의 변동성·구조변화 정도, (ii) 공변량의 정보량과 발표 시차, (iii) 전처리 및 결측치 처리, (iv) 예측 운영 방식(재귀/일괄)과 같은 설정이 성능을 크게 좌우함을 시사함. 따라서 실시간 nowcasting 시스템 구축 관점에서는 모형 선택만큼이나 \emph{정보집합의 정의}, \emph{데이터 누수 방지}, \emph{예측 운영 방식 표준화}가 중요.

마지막으로 본 연구는 상태공간 모형이 데이터가 부족한 거시 금융 예측 환경에서 안정적으로 잠재 상태를 학습하고 예측에 활용될 수 있다는 가능성을 확인하였음. 다만 선형 상태 공간 모형은 비선형 동학에 의한 수렴 안정성 문제가 나타나 이를 보완할 수 있는 딥러닝 기반 상태 모형 구축이 권장.

\appendix
\section{부록: 상세 예측 결과}
\label{sec:forecast_appendix}

본 부록에서는 본문에서 제시한 예측 결과의 상세 통계를 제공함. 모든 평가지표에 대한 전체 모형 비교 결과와 각 목표 변수별 시점별 상세 결과를 포함함.

\subsection{전체 평가지표 비교}
\label{subsec:appendix_all_metrics}

모든 모형에 대한 평가지표(sMAE, sMSE)를 시점별로 제시한 결과는 각 목표 변수별 상세 결과 표에서 확인할 수 있음.

\subsection{목표 변수별 시점별 상세 결과}
\label{subsec:appendix_horizon_wise}

각 목표 변수에 대한 시점별 상세 예측 결과는 아래 표들에 제시됨. 각 표는 1개월부터 22개월까지의 예측 시점에 대해 sMAE와 sMSE를 모형별로 제시함.

\subsubsection{KOIPALL.G (생산)}
\label{subsubsec:appendix_koipall_g}

생산 지수(KOIPALL.G)에 대한 시점별 예측 결과는 아래 표에 제시됨.

\begin{table}[htbp]
    \centering
    \normalsize
    \begin{tabular}{lrrrrrrrrrr}
        \toprule
        \multirow{2}{*}{Month} & \multicolumn{2}{c}{CHRONOS} & \multicolumn{2}{c}{DDFM} & \multicolumn{2}{c}{DFM} & \multicolumn{2}{c}{LSTM} & \multicolumn{2}{c}{TFT} \\
         & sMAE & sMSE & sMAE & sMSE & sMAE & sMSE & sMAE & sMSE & sMAE & sMSE \\
        \midrule
        2024-01 & 1.421 & 2.019 & 0.090 & 0.008 & 2.179 & 4.748 & 1.657 & 2.747 & 2.071 & 4.291 \\
        2024-02 & 1.475 & 2.177 & 0.191 & 0.036 & 2.125 & 4.515 & 1.855 & 3.440 & 2.101 & 4.415 \\
        2024-03 & 1.396 & 1.948 & 0.442 & 0.195 & 2.204 & 4.859 & 1.808 & 3.270 & 2.022 & 4.090 \\
        2024-04 & 1.455 & 2.116 & 0.553 & 0.306 & 2.146 & 4.604 & 1.913 & 3.660 & 2.083 & 4.340 \\
        2024-05 & 1.421 & 2.019 & 0.737 & 0.543 & 4.701 & 22.100 & 2.066 & 4.266 & 2.051 & 4.206 \\
        2024-06 & 1.413 & 1.996 & 0.914 & 0.835 & 6.986 & 48.804 & 1.996 & 3.984 & 2.059 & 4.241 \\
        2024-07 & 1.383 & 1.913 & 1.112 & 1.236 & 2.217 & 4.915 & 1.983 & 3.933 & 2.015 & 4.061 \\
        2024-08 & 1.429 & 2.043 & 1.215 & 1.477 & 2.171 & 4.712 & 2.073 & 4.299 & 2.055 & 4.223 \\
        2024-09 & 1.417 & 2.007 & 1.396 & 1.949 & 2.183 & 4.767 & 2.082 & 4.335 & 2.045 & 4.181 \\
        2024-10 & 1.442 & 2.079 & 1.539 & 2.370 & 2.158 & 4.657 & 2.137 & 4.566 & 2.071 & 4.290 \\
        2024-11 & 1.383 & 1.913 & 1.748 & 3.055 & 4.663 & 21.744 & 1.913 & 3.660 & 2.013 & 4.053 \\
        2024-12 & 1.467 & 2.152 & 1.832 & 3.357 & 7.040 & 49.568 & 1.783 & 3.181 & 2.117 & 4.483 \\
        2025-01 & 1.392 & 1.936 & 2.076 & 4.310 & 2.209 & 4.878 & 1.740 & 3.028 & 2.017 & 4.069 \\
        2025-02 & 1.425 & 2.031 & 2.192 & 4.805 & 2.175 & 4.730 & 1.475 & 2.176 & 2.052 & 4.212 \\
        2025-03 & 1.477 & 2.182 & 2.310 & 5.337 & 2.125 & 4.515 & 1.561 & 2.436 & 2.104 & 4.427 \\
        2025-04 & 1.446 & 2.091 & 2.512 & 6.309 & 2.158 & 4.657 & 1.976 & 3.905 & 2.072 & 4.292 \\
        2025-05 & 1.387 & 1.924 & 2.720 & 7.401 & 4.663 & 21.744 & 1.963 & 3.854 & 2.022 & 4.090 \\
        2025-06 & 1.459 & 2.128 & 2.817 & 7.937 & 7.028 & 49.392 & 1.699 & 2.886 & 2.104 & 4.427 \\
        2025-07 & 1.479 & 2.189 & 2.965 & 8.790 & 2.125 & 4.515 & 1.860 & 3.461 & 2.124 & 4.512 \\
        2025-08 & 1.467 & 2.152 & 3.146 & 9.895 & 2.137 & 4.568 & 2.075 & 4.304 & 2.112 & 4.461 \\
        2025-09 & 1.517 & 2.300 & 3.264 & 10.655 & 2.088 & 4.358 & 1.958 & 3.833 & 2.163 & 4.677 \\
        \bottomrule
    \end{tabular}
    \caption{Forecast metrics by month for KOIPALL.G (2024-01 to 2025-09).}
    \label{tab:koipallg_forecasts}
\end{table}

\subsubsection{KOEQUIPTE (투자)}
\label{subsubsec:appendix_koequipte}

투자 지수(KOEQUIPTE)에 대한 시점별 예측 결과는 아래 표에 제시됨.

\begin{table}[htbp]
    \centering
    \normalsize
    \begin{tabular}{lrrrrrrrrrr}
        \toprule
        \multirow{2}{*}{Month} & \multicolumn{2}{c}{CHRONOS} & \multicolumn{2}{c}{DDFM} & \multicolumn{2}{c}{DFM} & \multicolumn{2}{c}{LSTM} & \multicolumn{2}{c}{TFT} \\
         & sMAE & sMSE & sMAE & sMSE & sMAE & sMSE & sMAE & sMSE & sMAE & sMSE \\
        \midrule
        2024-01 & 1.396 & 1.950 & 0.207 & 0.043 & 1.985 & 3.941 & 3.782 & 14.301 & 0.250 & 0.063 \\
        2024-02 & 2.565 & 6.579 & 0.060 & 0.004 & 6.227 & 38.781 & 4.382 & 19.201 & 0.614 & 0.377 \\
        2024-03 & 2.455 & 6.027 & 0.053 & 0.003 & 3.792 & 14.379 & 4.369 & 19.088 & 0.456 & 0.208 \\
        2024-04 & 2.553 & 6.518 & 0.123 & 0.015 & 3.070 & 9.427 & 4.315 & 18.618 & 0.514 & 0.265 \\
        2024-05 & 2.531 & 6.407 & 0.118 & 0.014 & 3.204 & 10.264 & 5.767 & 33.259 & 0.379 & 0.144 \\
        2024-06 & 2.522 & 6.362 & 0.370 & 0.137 & 3.112 & 9.686 & 6.288 & 39.541 & 0.389 & 0.151 \\
        2024-07 & 2.921 & 8.532 & 0.952 & 0.906 & 2.478 & 6.140 & 4.924 & 24.241 & 0.801 & 0.642 \\
        2024-08 & 2.591 & 6.711 & 0.823 & 0.677 & 6.176 & 38.149 & 5.708 & 32.584 & 0.588 & 0.346 \\
        2024-09 & 2.742 & 7.521 & 1.172 & 1.374 & 4.096 & 16.777 & 6.082 & 36.992 & 0.761 & 0.578 \\
        2024-10 & 2.638 & 6.960 & 1.255 & 1.574 & 2.929 & 8.581 & 6.860 & 47.066 & 0.655 & 0.428 \\
        2024-11 & 2.581 & 6.659 & 1.372 & 1.881 & 2.980 & 8.881 & 6.079 & 36.956 & 0.603 & 0.363 \\
        2024-12 & 2.768 & 7.661 & 1.769 & 3.131 & 2.772 & 7.684 & 4.391 & 19.281 & 0.673 & 0.453 \\
        2025-01 & 2.161 & 4.668 & 1.395 & 1.945 & 1.843 & 3.396 & 3.602 & 12.977 & 0.222 & 0.049 \\
        2025-02 & 2.752 & 7.576 & 2.142 & 4.589 & 6.408 & 41.062 & 4.323 & 18.688 & 0.826 & 0.683 \\
        2025-03 & 2.768 & 7.660 & 2.319 & 5.376 & 4.147 & 17.195 & 4.622 & 21.365 & 0.810 & 0.657 \\
        2025-04 & 2.754 & 7.583 & 2.492 & 6.212 & 2.792 & 7.796 & 5.508 & 30.333 & 0.791 & 0.626 \\
        2025-05 & 2.573 & 6.619 & 2.494 & 6.218 & 2.961 & 8.769 & 5.015 & 25.155 & 0.543 & 0.295 \\
        2025-06 & 2.411 & 5.812 & 2.554 & 6.522 & 3.093 & 9.564 & 4.908 & 24.088 & 0.156 & 0.024 \\
        2025-07 & 2.624 & 6.888 & 2.964 & 8.786 & 2.306 & 5.318 & 7.345 & 53.951 & 0.274 & 0.075 \\
        2025-08 & 2.585 & 6.683 & 3.117 & 9.713 & 6.254 & 39.111 & 5.764 & 33.224 & 0.205 & 0.042 \\
        2025-09 & 2.945 & 8.672 & 3.668 & 13.455 & 3.806 & 14.484 & 4.982 & 24.816 & 0.560 & 0.314 \\
        \bottomrule
    \end{tabular}
    \caption{Forecast metrics by month for KOEQUIPTE (2024-01 to 2025-09).}
    \label{tab:koequipte_forecasts}
\end{table}

\subsubsection{KOWRCCNSE (소비)}
\label{subsubsec:appendix_kowrccnse}

소비 지수(KOWRCCNSE)에 대한 시점별 예측 결과는 아래 표에 제시됨.

\begin{table}[htbp]
    \centering
    \normalsize
    \begin{tabular}{lrrrrrrrrrr}
        \toprule
        \multirow{2}{*}{Month} & \multicolumn{2}{c}{CHRONOS} & \multicolumn{2}{c}{DDFM} & \multicolumn{2}{c}{DFM} & \multicolumn{2}{c}{LSTM} & \multicolumn{2}{c}{TFT} \\
         & sMAE & sMSE & sMAE & sMSE & sMAE & sMSE & sMAE & sMSE & sMAE & sMSE \\
        \midrule
        2024-01 & 2.084 & 4.344 & 0.018 & 0.000 & 6.101 & 37.226 & 2.941 & 8.649 & 1.377 & 1.897 \\
        2024-02 & 2.180 & 4.754 & 0.137 & 0.019 & 5.977 & 35.723 & 2.943 & 8.659 & 1.322 & 1.747 \\
        2024-03 & 2.739 & 7.502 & 0.056 & 0.003 & 2.343 & 5.488 & 3.067 & 9.407 & 1.394 & 1.944 \\
        2024-04 & 2.401 & 5.764 & 0.028 & 0.001 & 3.198 & 10.227 & 3.002 & 9.009 & 1.387 & 1.924 \\
        2024-05 & 2.218 & 4.920 & 0.017 & 0.000 & 3.191 & 10.182 & 1.760 & 3.097 & 1.394 & 1.942 \\
        2024-06 & 2.241 & 5.021 & 0.065 & 0.004 & 3.187 & 10.160 & 2.191 & 4.800 & 1.336 & 1.786 \\
        2024-07 & 2.200 & 4.841 & 0.117 & 0.014 & 6.036 & 36.436 & 3.080 & 9.487 & 1.357 & 1.841 \\
        2024-08 & 2.077 & 4.316 & 0.168 & 0.028 & 6.047 & 36.561 & 3.067 & 9.406 & 1.406 & 1.977 \\
        2024-09 & 2.110 & 4.452 & 0.238 & 0.057 & 2.384 & 5.683 & 3.100 & 9.612 & 1.438 & 2.069 \\
        2024-10 & 2.083 & 4.339 & 0.260 & 0.068 & 3.170 & 10.050 & 2.849 & 8.118 & 1.415 & 2.001 \\
        2024-11 & 2.144 & 4.596 & 0.276 & 0.076 & 3.194 & 10.204 & 2.845 & 8.094 & 1.390 & 1.932 \\
        2024-12 & 2.614 & 6.830 & 0.339 & 0.115 & 3.177 & 10.094 & 2.941 & 8.650 & 1.316 & 1.731 \\
        2025-01 & 2.523 & 6.363 & 0.364 & 0.133 & 6.019 & 36.226 & 2.934 & 8.607 & 1.367 & 1.869 \\
        2025-02 & 2.553 & 6.520 & 0.467 & 0.218 & 6.081 & 36.977 & 3.096 & 9.586 & 1.447 & 2.093 \\
        2025-03 & 2.724 & 7.419 & 0.479 & 0.229 & 2.360 & 5.569 & 2.882 & 8.303 & 1.415 & 2.001 \\
        2025-04 & 2.615 & 6.840 & 0.490 & 0.240 & 3.205 & 10.271 & 2.848 & 8.110 & 1.380 & 1.904 \\
        2025-05 & 2.631 & 6.924 & 0.530 & 0.281 & 3.205 & 10.271 & 3.029 & 9.173 & 1.543 & 2.381 \\
        2025-06 & 2.605 & 6.784 & 0.597 & 0.356 & 3.184 & 10.138 & 3.054 & 9.326 & 1.811 & 3.280 \\
        2025-07 & 2.637 & 6.953 & 0.738 & 0.545 & 6.128 & 37.558 & 3.151 & 9.927 & 1.900 & 3.608 \\
        2025-08 & 2.532 & 6.413 & 0.699 & 0.488 & 6.043 & 36.519 & 2.902 & 8.424 & 1.807 & 3.266 \\
        2025-09 & 2.537 & 6.439 & 0.741 & 0.549 & 1.431 & 2.049 & 2.886 & 8.332 & 1.801 & 3.244 \\
        \bottomrule
    \end{tabular}
    \caption{Forecast metrics by month for KOWRCCNSE (2024-01 to 2025-09).}
    \label{tab:kowrccnse_forecasts}
\end{table}


% ============================================
% Part III: 실시간 경기 진단 및 위기 알림 모형 구축
% ============================================
\part{실시간 경기 진단 및 위기 알림 모형 구축}

\setcounter{section}{0}
\includepdf[pages=-]{nowcast/nowcast_part3.pdf}


% ============================================
% Part IV: 거시경제 예측을 위한 데이터 활용 방안 조사
% ============================================
\part{거시경제 예측을 위한 데이터 활용 방안 조사}

\setcounter{section}{0}
\section{연구 목적 및 평가 기준}

본 보고서는 한국 경제의 \textbf{산업생산지수(월별)}를 nowcasting하기 위한 고빈도\slash 미시 데이터 후보들을 검토하고,
각 데이터의 \textbf{접근성(공개 여부, 유료 여부, 법적 제약)}, \textbf{발표 시차}, \textbf{빈도(주별 이상 여부)}를 기준으로
실질적으로 활용 가능한지 평가하는 것을 목적으로 한다.

특히 다음 두 가지 조건을 핵심 평가 기준으로 삼는다.

\begin{enumerate}
  \item \textbf{빈도 기준}
        \begin{itemize}
          \item 원칙적으로 \textbf{주간(weekly) 이상 고빈도} 데이터를 선호한다.
          \item 일\slash 5분\slash 실시간 데이터는 주간으로 집계하여 활용 가능하다.
        \end{itemize}

  \item \textbf{발표 시차 기준}
        \begin{itemize}
          \item 종속변수인 산업생산지수(산업활동동향)는 참조월의 \textbf{다음 달 말}에 발표된다.
          \item 같은 참조월에 대해 그보다 먼저 발표되는 월별 데이터는 유효한 predictor로 활용 가능하다.
          \item 한국은행의 BSI/ESI/CSI 등 심리지수는 해당월 말 06:00에 발표되어,
                산업생산 및 설비투자 지수보다 \textbf{약 3--5주 선행}한다.
        \end{itemize}
\end{enumerate}

아울러, 일반 국민과 연구기관이 \textbf{법\slash 제도상 접근할 수 없는 데이터, 유료 상용 서비스, 비정기\slash 1회성 데이터}는
모형 구축의 지속가능성이 낮다고 보고 별도로 구분한다.

\section{데이터 유형별 검토}

\subsection{기업 실적 데이터}

\subsubsection{상장사 재무데이터 (NICE V-se, KIS-Value, Value Search 등)}

\begin{itemize}
  \item \textbf{내용}: 상장사 매출, 재고, 생산량 등 재무제표 기반 분기\slash 연간 데이터.
  \item \textbf{빈도}: 분기 및 연간.
  \item \textbf{접근성}:
        \begin{itemize}
          \item 구 KIS-Value, 현 Value Search 등 상용 서비스로 \textbf{유료 구독}이 필요하다.
          \item 일반 국민과 공공 연구기관이 무료 또는 실시간으로 이용하기 어렵다.
        \end{itemize}
  \item \textbf{평가}:
        \begin{itemize}
          \item 분기 빈도이므로 월별 산업생산지수 nowcasting에는 직접 활용이 곤란하다.
          \item 또한 유료 상용 서비스로, 정책 연구용 공개 모형에서 재현성과 접근성 확보가 어렵다.
        \end{itemize}
\end{itemize}

\noindent\textbf{결론}: 산업생산지수 nowcasting용 핵심 후보에서는 제외한다. 다만 장기 구조모형에서
기업 재무건전성과 투자행태 분석용 보조 변수로는 활용 가능성이 있다.

\subsection{설비가동률 및 전력 관련 데이터}

\subsubsection{한국전력공사(KEPCO) 전력판매량}

\begin{itemize}
  \item \textbf{내용}: 시군구, 용도업종, 계약종별 전력판매량(\texttt{kWh}).
  \item \textbf{기간 및 형식}: 2004--2025년 월별 데이터, 엑셀(\texttt{xlsx}) 형식.
  \item \textbf{업데이트 주기}: ``대략 3개월에 한 번'' 수준으로 \textbf{정기성이 떨어짐}.
  \item \textbf{접근성}: 한국전력공사 홈페이지에서 무료 다운로드 가능.
  \item \textbf{평가}:
        \begin{itemize}
          \item 공간 및 업종 분해 수준은 우수하나, \textbf{월별}이며 업데이트가 불규칙적이다.
          \item 참조월 종료 직후 시점에서 신속히 확보하기 어렵고,
                산업생산지수 발표 시점(다음 달 말)에 비해 명확한 선행성을 확보하기 어렵다.
        \end{itemize}
\end{itemize}

\noindent\textbf{결론}: 모형 추정과 검증용 후행 자료로는 의미가 있으나,
실시간 nowcasting용 선행지표로서의 활용성은 낮다.

\subsubsection{한국전력거래소(KPX) 시계열 데이터}

\paragraph{(1) 시도별 시간대별 전력 계량 데이터(2013--2023, 1회성)}

\begin{itemize}
  \item \textbf{내용}: 2013--2023년 시도별\slash 시간대별 전력데이터(GWh).
  \item \textbf{특징}: 공공데이터포털에 1회성 데이터로 업로드되었으며, 차기 등록 예정은 미정이다.
  \item \textbf{접근성}: CSV 형식으로 무상 제공(일반인 다운로드 가능).
  \item \textbf{평가}:
        \begin{itemize}
          \item 시간 단위 고빈도이므로 주\slash 월 단위로 집계해 모형 추정에 활용 가능하다.
          \item 그러나 \textbf{실시간 업데이트 계획이 없어}, 과거 구간에서의 모형 학습과 백테스트 용도에 한정된다.
        \end{itemize}
\end{itemize}

\paragraph{(2) 시도별 시간대별 전력거래량(2001--2024)}

\begin{itemize}
  \item \textbf{내용}: 2001--2024년 시도별\slash 시간대별 전력거래량(MWh), CSV 형식.
  \item \textbf{업데이트 계획}: 공공데이터포털에 ``차기 등록 예정일 2026-06-30''으로 명시.
  \item \textbf{법적 제약}:
        \begin{itemize}
          \item 개별 발전기 단위 전력거래량 및 주소 등은 \textbf{사업자의 영업정보}로,
                정보공개법 제9조에 따라 제3자 제공이 불가하다.
        \end{itemize}
  \item \textbf{평가}:
        \begin{itemize}
          \item 시도 단위 집계는 무료로 제공되어 접근성이 양호하다.
          \item 일별 시간대별 전력거래량이므로 산업 생산 활동이 보다 활발한 주중, 주간 시간대와 주택 전기 사용 비중이 큰 주말 및 야간 시간대로 구분한다면 보다 예측력을 높일 수 있을 것으로 보인다.
          \item 그러나 대규모 일괄 갱신 방식이라, 월별 nowcasting에 필요한
                ``전월 데이터의 신속한 제공''과는 거리가 있다.
        \end{itemize}
\end{itemize}

\paragraph{(3) KPX 전력수급현황 실시간 API(5분 단위)}

\begin{itemize}
  \item \textbf{내용}: 전국 단위 aggregate 전력수급현황(5분 단위), 짧은 시계열, 오픈 API 제공.
  \item \textbf{포맷}: XML 및 JSON(\LaTeX\ 코드가 아닌 프로그래밍 환경에서 처리 필요).
  \item \textbf{접근성}:
        \begin{itemize}
          \item API 키 발급 후 무료 사용 가능하다.
        \end{itemize}
  \item \textbf{평가}:
        \begin{itemize}
          \item 5분 빈도 자료는 주간 또는 월간 지표로 쉽게 집계할 수 있다.
          \item KPX 전력거래량 데이터는 nowcasting 모형 추정에 사용하고 실제 nowcasting에는 KPX 전력수급현황 실시간 데이터를 사용하는 방안이 있다.
          \item 발표 시차가 사실상 실시간에 가까워, 산업생산지수에 대해 \textbf{강한 선행성}을 가진다.
          \item 전력수요는 제조업 가동과 밀접히 연동되어 있어 산업생산의 선행 또는 동행지표로 활용 여지가 크다.
        \end{itemize}
\end{itemize}

\noindent\textbf{결론}: 실시간 nowcasting을 위한 \textbf{핵심 고빈도 후보}로 평가된다.
다만 aggregate 수준이므로 업종별 및 용도별 식별은 불가능하다. 전력거래량 데이터와 마찬가지로 산업 생산 활동이 활발한 시간대 별로 구분하여 사용할 필요가 있다.

\subsection{ESG/탄소 및 환경 센서 데이터}

\subsubsection{굴뚝자동측정기기(TMS) 데이터 – 한국환경공단}

\begin{itemize}
  \item \textbf{내용}: 굴뚝 원격감시체계(CleanSYS)에 연결된 대형 배출사업장(제철·제강, 민간발전, 석유화학 등)의
        굴뚝 배출가스 농도 측정값을 실시간으로 제공한다. 먼지, 황산화물(SOx), 질소산화물(NOx), 염화수소,
        불화수소, 암모니아, 일산화탄소 등 7개 오염물질에 대한 측정값이 제공된다.
  \item \textbf{데이터 빈도 및 업데이트 주기}
        \begin{itemize}
          \item OpenAPI(\texttt{rltmMesureResult})는 배출구별 측정시간(\texttt{mesure\_dt})이 포함된
                \textbf{시각별 실시간 측정값}을 반환한다. 예시 응답은 \verb|2020-12-10 14:00| 등
                1시간 단위 타임스탬프를 사용한다.
            \item 정확한 빈도를 알기 위해서는 실제로 OpenAPI를 사용해볼 필요가 있다.
          \item 공공데이터포털에는 별도로 ``굴뚝자동측정기기 측정결과공개시스템\_실시간공개'' 등의 CSV가 제공되나, 과거 데이터의 시계열 길이를 정확히 알기 어렵다.
        \end{itemize}
  \item \textbf{접근성}:
        \begin{itemize}
          \item OpenAPI는 무료이며, 인증키 발급 후 REST 방식(JSON, XML)으로 호출 가능하다.
          \item 다만 실시간값을 장기간 축적하려면 자체적으로 크롤러 및 DB를 구축해 저장해야 하며, 과거 데이터의 유무가 불확실하므로, nowcasting 모형 추정에 한계가 있다.
        \end{itemize}
  \item \textbf{nowcasting 활용성 평가}
        \begin{itemize}
          \item \textbf{장점}: 대형 발전·제조 사업장의 굴뚝 배출량은 설비 가동률과 밀접하게 연동되므로,
                산업생산지수 중 특히 \emph{중화학·에너지 집약 산업}에 대한 정보를 담고 있을 가능성이 크다.
                시간 단위 데이터를 주간 또는 월간 합·평균·분산 등으로 집계해 nowcasting에 투입할 수 있다.
          \item \textbf{한계}:
                \begin{itemize}
                  \item TMS 부착 사업장은 전체 사업장의 일부로, \textbf{표본 편의}가 존재한다.
                  \item 일부 사업장은 비정기적 가동을 하거나, 환경 규제 강화 및 방지시설 운전 등
                        \textbf{정책·규제 요인}에 의해 배출량이 크게 변동할 수 있어,
                        순수한 생산량 신호로만 보기 어렵다.
                  \item 실시간 API를 이용한 \emph{자체적인 데이터 축적} 없이는 과거 장기 시계열을 완전하게 확보하기 어렵다.
                \end{itemize}
        \end{itemize}
\end{itemize}

\noindent\textbf{결론}: 고빈도·사업장 패널 구조를 가진다는 점에서 \textbf{가치 있는 보조 지표}이며,
전력수급 데이터와 결합해 에너지 집약 산업의 가동률을 추정하는 데 특히 유용할 수 있다.
다만 접근·구축 비용과 표본 편의 문제를 감안하면, 산업생산지수 nowcasting 모형의
\emph{핵심 단일 지표}라기보다는 \emph{보완 변수}로 활용하는 것이 적절하다. 무엇보다, nowcasting 모형 추정을 위해서는 과거 데이터의 확보가 필수적이나, 현재로서는 확인이 되지 않으므로 본 보고서에서의 활용은 제한적이다.

\subsubsection{에어코리아 대기오염정보}

\begin{itemize}
  \item \textbf{내용}: 전국 대기오염 측정소별로 미세먼지(PM10), 초미세먼지(PM2.5), 오존(O$_3$),
        이산화질소(NO$_2$), 일산화탄소(CO), 아황산가스(SO$_2$) 등 6개 항목의 농도와
        통합대기환경지수(CAI), 등급 정보 등을 제공한다.
  \item \textbf{데이터 빈도 및 업데이트 주기}
        \begin{itemize}
          \item 실시간 대기정보는 \textbf{해당 시각을 기준으로 직전 1시간 동안 측정한 값의 평균}을
                매시(hourly) 단위로 제공한다.
          \item OpenAPI ``대기오염정보'' 서비스는 측정소별\slash 시도별 실시간 측정정보 조회,
                통합대기환경지수 나쁨 이상 측정소 목록조회, 대기질 예보 및 초미세먼지 주간예보 조회 등으로
                구성되어 있으며, 실시간 측정값은 \textbf{1시간 간격으로 수시 갱신}된다.
          \item 일부 통계·요약 파일 데이터(XLSX)는 ``수시(1회성 업로드)'' 형식으로 제공되지만,
                nowcasting에는 주로 \textbf{실시간 OpenAPI}를 사용하게 된다.
        \end{itemize}
  \item \textbf{접근성}:
        \begin{itemize}
          \item 공공데이터포털에서 무료 API 키를 발급받으면, 별도 비용 없이 활용 가능하다.
          \item 기술문서와 예제가 잘 정리되어 있어, 기본적인 프로그래밍 역량이 있다면
                시계열 패널 데이터를 자동 수집하는 것이 비교적 용이한 편이다.
        \end{itemize}
  \item \textbf{nowcasting 활용성 평가}
        \begin{itemize}
          \item \textbf{장점}:
                \begin{itemize}
                  \item 전국 수백 개 측정소의 \textbf{고빈도 패널} 구조를 가지고 있어,
                        시·도별 또는 산업단지 주변의 평균, 분산, 극값 등을 추출해
                        주간·월간 환경 스트레스 지표로 만들 수 있다.
                  \item 교통량, 난방 수요, 일부 산업 활동 등과 관련된 신호를 일정 부분 담고 있어
                        \emph{도시 서비스업·교통 관련 활동}과 연관된 경기 상황을 보조적으로 포착할 수 있다.
                \end{itemize}
          \item \textbf{한계}:
                \begin{itemize}
                  \item 대기오염 수준은 기상 조건(풍향, 강수, 기온 역전 등), 국외 유입 등
                        \textbf{비경제적 요인}의 영향을 크게 받는다.
                  \item 공장 가동, 난방, 교통량 등 다양한 요인이 혼합되어 있어,
                        제조업 \emph{산업생산지수}와의 직접적인 연관성은 상대적으로 약할 수 있다.
                \end{itemize}
        \end{itemize}
\end{itemize}

\noindent\textbf{결론}: 에어코리아 데이터는 시계열·패널 구조와 업데이트 빈도가 우수하여
기술적으로는 nowcasting에 활용이 용이하다. 다만 신호 해석 시 기상 변수와의 동시 고려가 필수이며,
\textbf{전국 제조업 생산의 직접적인 대용변수라기보다는, 도시 활동 수준 및 환경 스트레스의 보조 지표}로
위치시키는 것이 적절하다.

\subsubsection{해양수질자동측정망 관측정보(해양수산부·해양환경공단)}

\begin{itemize}
  \item \textbf{내용}: 연안 오염우심해역(시화·마산·광양만, 부산 수영만, 새만금, 울산, 하구역 등)에 설치된
        해양수질자동측정소에서 수온, 염분, 용존산소, pH, 탁도, 전기전도도 등 수질 항목과
        일부 기상 항목을 \textbf{상시(continuous)} 측정하는 관측망이다.
  \item \textbf{데이터 빈도 및 업데이트 주기}
        \begin{itemize}
          \item 해양환경정보포털에 따르면,
                수질측정장치 자료는 \textbf{1시간 간격}, 수질·기상측정센서 자료는 \textbf{5분 간격}으로 생산된다.
          \item 관측소 위치·운영현황을 담은 ``해양수질자동측정소 정보'' 메타데이터는
                \textbf{연간 주기로 업데이트}되며, 관측 자료 자체는 포털에서 실시간 시계열로 조회 가능하다.
          \item 공공데이터포털 OpenAPI(예: 하구 및 만 정점조회, 관측서비스)를 통해 정점별 코드·위치 및
                관측값을 JSON/XML 형식으로 조회할 수 있다.
        \end{itemize}
  \item \textbf{접근성}:
        \begin{itemize}
          \item API 및 파일 데이터 모두 무료 공개이며, 인증키 발급 후 사용 가능하다.
          \item 다만 해역·정점 코드 체계를 이해해야 하고, 원하는 기간·정점을 반복 호출하여
                자체적으로 시계열을 구축해야 하므로, 실질 활용에는 일정 수준의 데이터 처리 역량이 필요하다.
        \end{itemize}
  \item \textbf{nowcasting 활용성 평가}
        \begin{itemize}
          \item \textbf{장점}:
                \begin{itemize}
                  \item 시화·마산·광양만, 울산·부산 등 \textbf{연안 산업단지 인근 해역}의 수질 변화를 고빈도로 관측하므로,
                        석유화학, 제철, 조선 등 수출·중화학 공업의 활동과 관련된 방류 패턴을
                        간접적으로 포착할 수 있다.
                  \item 시간해상도가 5분~1시간 수준이므로, 주간·월간 집계뿐 아니라
                        변동성, 극값, 첨두 빈도 등 다양한 통계량을 구성할 수 있다.
                \end{itemize}
          \item \textbf{한계}:
                \begin{itemize}
                  \item 특정 하구·만 해역에 공간적으로 집중되어 있어,
                        전국 산업생산지수 전체를 대표하기에는 \textbf{공간 범위가 제한적}이다.
                  \item 강우·하천유입, 조석, 해류 등 \textbf{자연환경 요인}의 영향이 매우 커서,
                        순수한 공장 가동 신호를 분리하기 위해서는 기상·수문 자료와의 통합 분석이 필요하다.
                \end{itemize}
        \end{itemize}
\end{itemize}

\noindent\textbf{결론}: 해양수질자동측정망 데이터는 특정 연안 산업단지(예: 시화·광양·울산 등)에 대한
\textbf{지역별 산업활동 nowcasting}에는 의미 있는 보조 지표가 될 수 있다.
다만 전국 산업생산지수 nowcasting에서 핵심 설명변수로 사용하기에는 공간 커버리지와 자연환경 영향이 크므로,
전력수급·심리지표와 결합한 \emph{보완적 지표}로 활용하는 방향이 바람직하다.



\subsection{텍스트\slash 뉴스 및 심리지수}

\subsubsection{한국은행 뉴스심리지수}

\begin{itemize}
  \item \textbf{내용}: 인터넷 포털의 경제뉴스 텍스트를 바탕으로 구축한 뉴스심리지수.
  \item \textbf{기간 및 빈도}: 2005--2025년 일별 시계열 데이터.
  \item \textbf{접근성}: 한국은행 ECOS 및 대시보드를 통해 공개(무료).
  \item \textbf{평가}:
        \begin{itemize}
          \item 일별 자료이므로 주 및 월 단위로 집계가 가능하다.
          \item 발표 시차가 거의 없고, 뉴스 자체가 경제 이벤트에 선행 또는 동행하므로
                산업생산 nowcasting에 매우 적합하다.
          \item 공공 지표이므로 재현성과 접근성 측면에서도 우수하다.
        \end{itemize}
\end{itemize}

\noindent\textbf{결론}: 산업생산지수 nowcasting 모형의 \textbf{핵심 텍스트\slash 심리 지표}로 활용을 권장한다.

\subsubsection{한국은행 BSI\slash ESI\slash CSI 등 월간 심리지수}

\begin{itemize}
  \item \textbf{내용}: 경제심리지수(ESI), 소비자심리지수(CCSI), 현재 및 전망 CSI,
        기업심리지수(CBSI) 등 여러 월간 설문지표.
  \item \textbf{발표 시점}:
        \begin{itemize}
          \item BSI, ESI, CSI는 참조월 말일 전후 06:00에 발표된다.
          \item 산업활동동향(산업생산 및 설비투자)은 참조월의 다음 달 말에 발표된다.
          \item 결과적으로 동일 참조월에 대해 약 3--5주 선행한다.
        \end{itemize}
  \item \textbf{접근성}: ECOS 및 보도자료를 통해 무료 제공된다.
  \item \textbf{평가}:
        \begin{itemize}
          \item 빈도는 월간이지만, 발표 시점이 산업생산보다 빠르므로 nowcasting에 활용 가능하다.
          \item 기업과 가계의 경기 인식 및 전망을 반영하여 산업생산의 방향성 탐지에 유용하다.
        \end{itemize}
\end{itemize}

\noindent\textbf{결론}: ``월간이지만 선행 발표되는'' 대표적인 선행지표로,
산업생산지수 nowcasting 모형에 포함할 것을 권장한다.


\subsubsection{BIGKinds(뉴스 빅데이터, 한국언론진흥재단)}

\begin{itemize}
  \item \textbf{내용}: BIGKinds는 한국언론진흥재단이 운영하는 뉴스 빅데이터 플랫폼으로,
        주요 일간지 및 방송사 등 100여 개 언론사의 기사를 수집하여 검색 및 분석 기능을 제공한다.
        사용자 검색어에 따라 기사 목록, 클러스터링 기반 ``오늘의 이슈'', 개체명 분석 기반
        ``오늘의 키워드'' 등을 시각화하여 보여준다.
  \item \textbf{데이터 빈도 및 업데이트 주기}
        \begin{itemize}
          \item \emph{오늘의 이슈}: 매일 수집된 뉴스에 대해 뉴스클러스터링을 수행하여
                상위 10개 이슈를 도출하며, 하루 2회(예: 08시, 17시) 정기 분석이 이루어진다.
          \item \emph{오늘의 키워드}: 매일 수집된 뉴스에서 주요 인물, 기관, 장소를 추출하여
                해당 키워드가 포함된 뉴스 건수 순으로 시각화한다.
          \item \emph{키워드 트렌드}: 검색 단계에서 특정 키워드를 입력하면,
                해당 키워드가 포함된 뉴스 건수를 \textbf{일간/주간/월간/연간} 단위로 집계하여
                그래프로 보여주는 기능이 제공된다.
                기간(예: 최근 1개월, 3개월, 1년)과 집계 단위(일/주/월/년)를 선택할 수 있어,
                특정 키워드에 대한 \textbf{주간 기사 건수 추이}를 직접 확인할 수 있다.
          \item 분석 결과 화면(Step 03)에서는 키워드 트렌드, 연관어 분석, 관계도 분석 등과 함께
                결과값을 파일로 내려받는 ``데이터 다운로드'' 기능도 제공되므로,
                주간 기사 건수 시계열을 엑셀 등의 형식으로 추출해 별도의 nowcasting 지표로
                활용할 수 있다.
        \end{itemize}
  \item \textbf{접근성 및 법적 제약}
        \begin{itemize}
          \item BIGKinds 웹 사이트는 무료 회원가입 후 이용 가능한 \textbf{웹 기반 서비스}로,
                키워드 검색, 키워드 트렌드 그래프, 이슈 리포트 등을 브라우저 상에서 확인하고
                일부 결과를 엑셀/이미지로 다운로드할 수 있다.
          \item 다만 각 기사 원문은 저작권법의 보호를 받으며, 사이트 하단 이용약관에 따라
                무단 전재, 복제, 대량 수집은 제한된다. 따라서 \textbf{기사 텍스트 전체를
                크롤링하여 자체 DB를 구축하는 방식}은 법적\slash 약관상 제약이 크다.
          \item 한국언론진흥재단은 BIGKinds 기반 분석 정보를 공공데이터포털을 통해 일부 개방하고 있다.
                예를 들어 ``오늘의 이슈 Top 10''과 같은 요약 지표는 파일 및 Open API 형태로
                제공되며, 인증키 발급 후 REST 방식으로 활용 가능하다. 다만 이는 전 언론사 기사 원문이
                아니라, \textbf{분석·요약된 통계 지표}라는 점에서 범위가 제한된다.
          \item BIGKinds의 검색\slash 분석 기능 일부를 외부에서 호출하기 위한 Open API는
                주로 공공기관 및 언론사와의 협약을 통해 제공되는 것으로 알려져 있으며,
                일반 연구자\slash 개인에게 일괄 개방된 형태는 아니다.
        \end{itemize}
  \item \textbf{키워드별 주간 기사 건수 시계열 구축 가능성}
        \begin{itemize}
          \item BIGKinds 검색창에서 특정 키워드를 입력하고 기간을 설정하면,
                해당 키워드가 포함된 뉴스 기사 건수를 \textbf{일간 또는 주간 단위}로 집계한
                트렌드 그래프를 확인할 수 있다.
          \item 집계 단위를 ``주간''으로 선택하면, 선택한 기간 동안의 \textbf{주별 기사 건수}가
                막대그래프 혹은 선그래프 형태로 표시되며, 이 그래프에 대응하는 데이터는
                분석 결과 다운로드 기능을 통해 엑셀 파일로 저장할 수 있다.
          \item 예를 들어 ``산업생산'', ``설비투자'', 특정 업종(자동차, 반도체 등) 또는
                특정 기업명을 키워드로 설정하고 주간 기사 건수 시계열을 추출하면,
                해당 주에 대한 \textbf{뉴스 노출 강도}를 나타내는 주간 텍스트 기반 지표를
                구축할 수 있다.
        \end{itemize}
  \item \textbf{산업생산지수 nowcasting 관점 평가}
        \begin{itemize}
          \item \textbf{빈도 측면}에서, BIGKinds의 키워드 트렌드 기능은 최소 일 단위,
                선택에 따라 주 단위 기사 건수 시계열을 제공하므로, 월별 산업생산지수를
                nowcasting하기에 충분한 시간 해상도를 가진다.
          \item \textbf{접근성 측면}에서는, 웹 인터페이스와 공공데이터포털을 통해
                일부 분석 지표를 누구나 무료로 이용할 수 있지만, 전체 기사 원문 및
                고빈도 메타데이터에 대한 API 접근은 재단과의 협약 또는 별도 계약이
                필요할 가능성이 크다.
          \item \textbf{법적\slash 제도적 측면}에서, 정책 연구 보고서에서는
                포털\slash 언론사 웹페이지를 비공식 크롤링하는 방식보다는,
                BIGKinds 웹 서비스와 공공데이터포털이 제공하는 \textbf{공식 분석 지표}
                (예: 키워드별 주간 기사 건수, 오늘의 이슈 Top 10 등)를 활용하여
                텍스트 기반 선행지수를 구축하는 방향을 제안하는 것이 타당하다.
          \item 실제 모형에서는, BIGKinds에서 추출한 \textbf{키워드별 주간 기사 건수 지수}를
                산업생산지수의 선행변수로 포함하고, 한국은행 뉴스심리지수와 결합하여
                뉴스의 양(Volume)과 감성(Sentiment)을 동시에 반영하는 텍스트 기반
                nowcasting 변수를 구성하는 전략이 유효하다.
        \end{itemize}
\end{itemize}

\noindent\textbf{결론}: BIGKinds는 뉴스 데이터를 기반으로 한 \textbf{고빈도 텍스트 지표의 원천}으로,
키워드별 주간 기사 건수 시계열을 비교적 손쉽게 구축할 수 있다는 점에서 산업생산지수
nowcasting에 중요한 후보 데이터이다. 다만 기사 원문 자체의 대량 수집에는 저작권 및
이용약관 제약이 존재하므로, 공공데이터포털 및 BIGKinds가 공식적으로 제공하는
분석 지표와 다운로드 기능을 활용하여 \emph{키워드별 주간 뉴스량 지수}를 만드는 방향이
현실적이다.


\subsection{운송 및 물류 관련 데이터}

\subsubsection{항만 물동량 통계 (국가물류통합정보센터, PORT\textendash MIS)}

국가물류통합정보센터(NLIC)는 통합 PORT\textendash MIS 자료를 활용하여 항만별 월간 물동량 통계를 제공한다. 항만명(부산, 광양 등)과 수출입/환적/연안, 입출항 구분별 화물 처리실적(톤)을 월별로 조회할 수 있으며, 기준년월 선택 메뉴를 통해 \textbf{2010년 1월 부터 2025년 10월까지}의 자료가 제공되고 있어 \textbf{\emph{월 빈도의 시계열}} 확보가 가능하다.\footnote{국가물류통합정보센터 ``항만별 물동량 통계'' 화면에서 기준년월 선택 및 항만별 월간/누계 물동량을 제공하고 있으며, 메타정보에 업데이트 주기 ``매월''로 명시되어 있음.} 자료 설명에 따르면 해당 통계는 통합 PORT\textendash MIS의 화물처리실적(화물 반출입 신고정보)을 기초로 작성된다.

메타정보상 업데이트 주기는 \emph{매월}로 표시되어 있고, 기준년월이 2025년 10월까지 제공되는 점을 고려할 때, 전월 자료가 비교적 짧은 시차로 집계·공표되고 있음을 알 수 있다.\footnote{동일 화면의 자료설명 표에서 ``업데이트 주기 : 매월''로 기재.} PORT\textendash MIS 원자료에 대한 해양수산 관련 보고서에서는 ``전월 물동량 통계가 매월 22일경 확정·공표된다''고 언급하고 있어, 실무적으로는 \emph{전월 자료가 익월 하순(22일 전후)에 확정되는 것으로 추정}된다.\footnote{한국해양수산개발원 KMI의 해양수산 지표 보고서에서 Port\textendash MIS 물동량 통계에 대해 ``매월 22일에 전월 통계가 확정 공표되므로 2019년 10월 통계까지 포함''이라는 설명이 제시됨. }

한편, 산업생산지수는 광업·제조업동향조사 결과를 포함한 ``산업활동동향'' 공표를 통해 \emph{매월 30일경} 전월 통계가 발표된다.\footnote{광업·제조업동향조사 메타정보에 ``매월 30일경 생산·소비·투자·경기부문을 종합하여 산업활동동향으로 공표''라고 명시.} 최근 공표일을 보면 2025년 10월치 산업활동동향이 11월 28일, 9월치는 10월 31일에 발표되는 등, 통상 \emph{익월 말}에 공표되고 있다.\footnote{산업활동동향 보도자료 목록에서 2025년 4\textasciitilde 10월 자료의 공표일이 4월 30일, 5월 30일, 7월 1일, 7월 31일, 8월 29일, 10월 27일, 10월 31일, 11월 28일로 나타남.}

이를 종합하면, 항만 물동량 통계는 산업생산지수에 비해 \emph{약 1주일 정도 이른 시점(익월 22일 전후)}에 전월 값이 확정·제공되는 것으로 보이며, 월별 산업생산지수를 nowcasting하는 데 있어 시차 측면의 이점을 갖는다. 다만,
\begin{itemize}
  \item 현재 시점에서는 공식적인 ``정식 공표일''이 별도의 공표계획표로 제공되지는 않고 있으며,
  \item NLIC 포털 화면에서 과거 자료를 일괄 다운로드하는 과정에서 일부 개편·지연 가능성이 존재하므로
\end{itemize}
정책용 nowcasting 모델에 활용하기 위해서는 PORT\textendash MIS 및 NLIC 운영기관과의 추가적인 공표일 확인이 필요하다.

접근성 측면에서는, 항만 물동량 통계는 NLIC 누리집에서 \emph{무료·공개}로 제공되며, 엑셀 다운로드 기능을 통해 별도의 승인절차 없이 이용 가능하다. 따라서 ``일반인 접근이 어려운 데이터''에는 해당하지 않으며, 실무 연구자가 비교적 손쉽게 확보 가능한 고빈도 물류변수로 평가된다.

\subsubsection{해상운임 지수 (국내·국외)}

국가물류통합정보센터는 국내 및 국외 해상운임 지수를 별도 통계로 제공한다. 국내 해상운임 지수는 국내 항로 운임을 집계한 월별 지수이며, 자료설명에서 업데이트 주기가 ``매월''로 명시되어 있다.\footnote{NLIC ``국내 해상운임 지수'' 메타정보에서 업데이트 주기를 매월로 제시.} 2019년 1월 부터 최근 2025 10월까지 제공되고 있다. 또한 수출 수입 데이터가 별개로 존재하며 대상 국가별로 나누어져있다 (미국 동서부, 일본, EU, 중국 등)

국외 해상운임 지수는 상하이해운거래소(Shanghai Shipping Exchange) 등 해외 기관의 컨테이너·벌크 운임지수를 취합한 것으로, NLIC 메타정보상 \emph{업데이트 주기가 ``매주''}로 표시되어 있어 주간 단위의 시계열 데이터를 제공한다.\footnote{NLIC ``국외 해상운임 지수'' 자료설명 표에 업데이트 주기 ``매주''로 명시.} 실제 화면상 관측일자는 주 단위(예: 2025년 4월 18일, 4월 25일 등)로 구성되어 있어, 우리가 요구하는 \emph{최소 주별 빈도} 조건을 충족한다. 2014년부터 제공하고 있다.

민간 부문의 글로벌 컨테이너 항만 물동량·운임 지수(Drewry Port Throughput Index 등)는 국제 해운시장 분석을 위해 널리 활용되지만, 대부분 \emph{유료 구독형 서비스}로 제공된다.\footnote{KMI의 글로벌 컨테이너 항만물동량 분석 보고서는 Drewry Port Throughput Index 등 민간지수를 인용하며, 해당 지수는 민간기관이 매월 발표하는 상용 통계임을 언급.} nowcasting 모형에서 사용하기 위해서는 별도의 구독·이용계약이 필요하다.

반면, NLIC가 제공하는 국내·국외 해상운임 지수는 공공 포털을 통해 무료로 제공되므로 접근성 측면에서 유리하다. 산업생산지수 nowcasting 측면에서는,
\begin{itemize}
  \item 국내 제조업·수출입 구조가 해상운송에 크게 의존하고 있고,
  \item 국외 해상운임 지수가 글로벌 물동량 및 교역 경기의 선행지표로 작용할 수 있다는 점에서,
\end{itemize}
주간·월간 운임 지수를 설명변수로 포함하는 것이 유의미하다. 특히 국외 해상운임 지수는 \emph{주별 갱신}이 이루어지므로, 월별 산업생산 공표 이전에 상대적으로 풍부한 정보 집합을 제공할 수 있다는 장점이 있다. 다만, 운임지수는 공급측 요인(선복량 조정 등)과 금융적 요인도 반영하므로, 단순 수준값보다는 변화율·이동평균 등으로 변환하여 사용하고, 모형 내에서 경제적 해석 가능성을 별도로 검토하는 것이 바람직하다.

\subsubsection{항공화물 및 수출입 화물 물동량 통계}

NLIC의 항공화물통계 메뉴에서는 ``공항별 물동량 통계''와 ``수출입화물 물동량 통계''를 제공한다. 공항별 물동량 통계는 국내 주요 공항(인천, 김포 등)의 화물처리 실적을 월별로 제공하며, 메타정보에 따르면 업데이트 주기는 \emph{매월}이다.\footnote{국가물류통합정보센터 ``공항별 물동량 통계'' 자료설명에서 업데이트 주기를 매월로 명시.} 수출입화물 물동량 통계 역시 항공 수출입 화물을 대상로, 국가·노선별 물동량을 월별로 집계하고, 메타정보상 ``업데이트 주기 매월''로 표시되어 있다.\footnote{NLIC ``수출입화물 물동량 통계'' 자료설명에서 업데이트 주기 매월로 제시.} 그러나 정확한 공표일을 확인하지는 못했다.

이들 통계는 통상 관세청 통관자료 및 항공사·공항운영기관의 화물 처리실적을 기초로 작성되며, 전월 자료가 익월 중에 업데이트되는 구조이다. 공식적인 공표일(예: 매월 몇 일)이 산업활동동향처럼 명시되어 있지는 않지만, 현재(12월 초) 2013년 1월부터 2025년 10월까지 제공되는 점을 고려하면, 산업생산지수(익월 말 공표)와 유사하거나 다소 이른 시점에 접근 가능할 것으로 판단된다. 즉,
\begin{itemize}
  \item 월빈도이지만,
  \item 업데이트가 산업생산지수 공표 이전에 이루어질 가능성이 있어,
\end{itemize}
제조업 중 수출지향 업종(전자, 기계, 운송장비 등)의 생산을 설명하는 보조지표로 nowcasting 모형에 포함할 여지가 있다.

데이터 접근성은 항만 물동량 통계와 마찬가지로 \emph{무료·공개}이며, 엑셀 다운로드 기능을 통한 일괄 수집이 가능하다. 다만, 국가·노선별 세부 분류까지 모두 활용할 경우 차원이 급격히 증가하므로, nowcasting 모형에서는 인천공항 총 수출입 화물량, 주요 교역국(미국, 중국, EU 등) 상대 화물량, 또는 적절한 가중평균 지수 등으로 축약하여 사용하는 것이 현실적이다.

\subsubsection{내륙화물(도로) 수송실적 및 생활물류(택배) 통계}

내륙 운송과 관련하여, 국토교통부는 도로운송 수송실적을 국가데이터포털을 통해 제공한다. 해당 자료는 전국 시도 간 화물 운송량(O/D별 수송실적), 품목별 화물수송실적, 품목별 O/D 물동량 정보를 연도별로 집계한 통계로, 단위는 톤이다.\footnote{공공데이터포털 ``국토교통부\_도로운송수송실적'' 설명에 따르면, 도로운송수송실적은 시도 간 O/D별 및 품목별 수송실적을 제공하는 연도별 데이터이며, 데이터 기준연도는 2022년, 최종 취합 및 등록은 2025년에 이루어짐.} 메타정보에 따르면 업데이트 주기는 ``수시(1회성 데이터)''로 표기되어 있으며, 실제 데이터 기준일은 2022년, 공개는 2025년에 이루어진 것으로 명시되어 있다.\footnote{같은 메타정보에서 업데이트 주기 ``수시(1회성 데이터)'', 데이터 기준일 2025년, 내용은 2022년 실적이라는 설명이 제시됨.}

이는 도로운송 수송실적 통계가 \emph{행정자료를 활용한 사후 집계 통계}로서, 연간 단위로 상당한 시차를 두고 공표되는 성격임을 의미한다. 따라서,
\begin{itemize}
  \item 빈도가 연간(연도별) 수준에 머물고,
  \item 공표 시차도 산업생산지수에 비해 훨씬 길기 때문에,
\end{itemize}
월별 산업생산지수 nowcasting에는 적합하지 않다. 이 통계는 중장기 물류 인프라 계획 수립, 지역 간 물류흐름 분석 등 구조적 분석에 적합한 데이터로 보인다.

생활물류(택배)와 관련해서는 NLIC의 생활물류통계 메뉴에서 연도별·월별 택배 물동량 및 매출 통계를 제공하고 있으나, 자료설명에 따르면 연간 업데이트 주기를 갖는 연간 통계가 중심이다.\footnote{NLIC 생활물류통계의 ``연도별 생활물류실적'' 자료설명에서 업데이트 주기를 연간으로 명시하고, 택배 물동량·매출 통계를 연도별로 제공. :contentReference[oaicite:12]{index=12}} 월별 통계표도 제공되지만, 실제로는 과거 연도의 실적을 사후 집계한 형태로, 통상 전년도 자료가 다음 해에 일괄 업데이트되는 구조로 보인다. 이 경우,
\begin{itemize}
  \item 택배 물동량이 소비·전자상거래 관련 경기지표로서 의미는 있으나,
  \item 월별 산업생산지수를 \emph{실시간에 가깝게 Nowcasting}하는 용도에는 시차가 너무 길어 실용성이 제한된다.
\end{itemize}

요약하면, 운송 및 물류 영역에서 nowcasting에 실질적으로 활용 가능한 공공 데이터는
\begin{enumerate}
  \item \textbf{항만별 월간 물동량 통계(PORT\textendash MIS 기반)}: 월빈도, 익월 22일경 전월 값 확정, 산업생산지수 공표(익월 말)보다 약간 빠른 공표.
  \item \textbf{국내 해상운임 지수}: 월빈도, 매월 업데이트.
  \item \textbf{국외 해상운임 지수}: \emph{주빈도}, 매주 업데이트.
  \item \textbf{항공화물(공항별·수출입 화물 물동량)}: 월빈도, 매월 업데이트.
\end{enumerate}
정도로 정리할 수 있다. 이들 지표는 모두 무료·공개 데이터이며, 산업생산지수보다 짧거나 유사한 시차로 공표되므로, 산업생산지수 nowcasting 모형의 설명변수 후보로 유의미하다. 반면,
\begin{itemize}
  \item 도로운송 수송실적(O/D별 연도별 화물 물동량),
  \item 연도별 생활물류(택배) 통계,
  \item 민간 유료 서비스(글로벌 컨테이너 항만 물동량 지수 등)
\end{itemize}
은 공표 시차·빈도 또는 비용 측면에서 nowcasting용 실시간 지표로 활용하기에는 한계가 있어, 구조적 분석 또는 보조적 참고자료로 활용하는 것이 적절하다.


\section{종합 평가 및 권고}

\subsection{요약 표}

아래 표는 관측 빈도, 발표 시차, 접근성을 기준으로 주요 후보들을 요약한 것이다.

\begin{table}[htbp]
\centering
\small
\setlength{\tabcolsep}{4pt}      % 열 간격 약간 줄이기
\renewcommand{\arraystretch}{1.0}% 행 높이 기본값

\begin{tabular}{p{2.5cm} p{3cm} p{3.5cm} p{3cm} p{3cm}}
\toprule
유형 & 데이터 & 빈도/발표시차 & 접근성/비용 & 비고 \\
\midrule
기업 실적 &
상장사 재무데이터 &
분기/연간, 공시 후 지연 &
유료 상용 &
nowcasting 부적합 \\
\midrule
전력 &
KEPCO 전력판매량 &
월별, 업데이트 불규칙 &
무료 xlsx &
선행지표로 한계 \\
\midrule
전력 &
KPX 전력계량(1회성) &
시간별(과거 2013--23) &
무료 csv &
모형 학습용 \\
\midrule
전력 &
KPX 전력거래량 &
시간별, 2001--24,
차기 업로드 2026 예정 &
무료 csv,
발전기 단위 비공개 &
실시간 갱신 부재 \\
\midrule
전력 &
KPX 전력수급현황 API &
5분 실시간 &
무료, 코딩 필요 &
핵심 고빈도 후보 \\
\midrule
ESG/환경 &
굴뚝 TMS &
실시간, 사업장 단위 &
무료 API &
보조 변수로 유망 \\
\midrule
ESG/환경 &
에어코리아 &
실시간, 측정소 단위 &
무료 API &
주요 변수 비권장 \\
\midrule
ESG/환경 &
해양수질자동측정망 &
실시간, 정점 단위 &
무료 API &
특정 연안 산업에 한정 \\
\midrule
텍스트 &
뉴스심리지수 &
일별, 거의 동시 공표 &
무료, 공개 &
핵심 텍스트 지표 \\
\midrule
심리지수 &
BSI/ESI/CSI/CBSI &
월별, 참조월 말 발표 &
무료 &
산업생산보다 3--5주 선행 \\
\midrule
물류 &
항만 물동량 &
월별, 익월 22일 전후 &
무료, 엑셀 &
공표 일정 확인 필요 \\
\midrule
물류 &
화물 운송량 &
연간, 공표 시차 큼 &
무료 &
nowcasting 부적합 \\
\bottomrule
\end{tabular}

\caption{관측 빈도, 발표 시차, 접근성 기준 주요 데이터 요약 (1)}
\end{table}

\begin{table}[htbp]
\centering
\small
\setlength{\tabcolsep}{4pt}
\renewcommand{\arraystretch}{1.0}

\begin{tabular}{p{2.5cm} p{3cm} p{3.5cm} p{3cm} p{3cm}}
\toprule
유형 & 데이터 & 빈도/발표시차 & 접근성/비용 & 비고 \\
\midrule
텍스트/뉴스 &
BIGKinds
(뉴스 빅데이터) &
기사 실시간 수집,
키워드 트렌드 일/주/월 집계 &
웹 서비스 무료,
일부 공공 API,
원문 크롤링 제약 &
키워드별 주간 기사수 지수,
뉴스량 선행지표 \\
\midrule
물류 &
국내 해상운임 지수 &
월별, 매월 업데이트 &
무료·공개 (NLIC) &
국내 해상운송 비용 구조 반영,
월간 경기 보조지표 \\
\midrule
물류 &
국외 해상운임 지수 &
주별, 매주 업데이트,
IPI보다 크게 선행 &
무료·공개 (NLIC) &
글로벌 교역·물동량 선행지표,
주간 nowcasting 후보 \\
\midrule
물류 &
항공화물 물동량
(공항별·수출입) &
월별, 전월 자료가
익월 중 공표 &
무료·공개 (NLIC),
엑셀 다운로드 &
수출지향 제조업 활동 보조지표,
IPI보다 비슷하거나 약간 선행 \\
\bottomrule
\end{tabular}

\caption{관측 빈도, 발표 시차, 접근성 기준 주요 데이터 요약 (2)}
\end{table}


\subsection{우선 활용 권고}

\begin{enumerate}
  \item \textbf{최우선 고빈도 후보(주간 이상 집계 가능)}
        \begin{itemize}
          \item 한국전력거래소 전력수급현황 API(5분).
          \item 한국은행 뉴스심리지수(일별).
          \item KLIC 주별 해상운임지수 데이터
        \end{itemize}

        위 세 축을 이용해 전력, 해상운임임, 뉴스심리에 기반한 \textbf{주간 합성 경기지수}를 구성하고,
        이를 산업생산지수 nowcasting 모형(DFM, MIDAS, Bayesian VAR 등)에 투입하는 것이 바람직하다.

  \item \textbf{월간 선행지표로서의 활용}
        \begin{itemize}
          \item 한국은행 BSI, ESI, CSI, CBSI: 산업생산지수보다 3--5주 먼저 발표되므로
                월말 기준으로 산업생산에 대한 선행 정보를 제공한다.
          \item 항만 물동량 통계: 공표 시점이 산업활동동향보다 빠를 경우
                수출 주도 제조업의 실질생산을 반영하는 보완 지표로 활용 가능하다.
        \end{itemize}

  \item \textbf{제외 또는 제한적 활용 대상}
        \begin{itemize}
          \item 상장사 재무데이터(유료, 분기/연간 빈도).
          \item KEPCO 전력판매량(월별, 불규칙 업데이트).
          \item 화물 운송량(연간).
          \item 에어코리아 대기오염(비경제 요인 영향 과대).
        \end{itemize}
\end{enumerate}

\section{향후 과제 및 결론}

\subsection{향후 과제}

\begin{enumerate}
  \item \textbf{공개 데이터 공표 일정의 체계적 정리}:
        항만 물동량, KEPCO 전력판매량 등 월별 지표의 실제 공표일을 추가로 수집하여
        산업생산지수 대비 선행성 여부를 정량적으로 확인할 필요가 있다.

  \item \textbf{API 기반 데이터 수집 인프라 구축}:
        KPX, 환경공단, 해양수산부, 한국은행 등 각 기관 API에서
        일\slash 주간 단위의 패널을 자동으로 축적하는 수집\slash 정제 파이프라인을
        KDI 혹은 기획재정부 내부 시스템에 구축하는 것이 중요하다.

  \item \textbf{예측력 검증 및 변수 선정}:
        위 후보 데이터로 산업생산지수를 nowcasting하는 파일럿 모형을 구축한 뒤,
        DFM, MIDAS 회귀, 머신러닝(랜덤포레스트, LASSO 등)을 활용한
        변수 중요도 평가를 통해 실제 예측력이 높은 지표를 선별해야 한다.
\end{enumerate}

\subsection{결론}

본 조사를 통해, 일반 국민 및 공공 연구기관이 비용과 법적 제약 없이 활용할 수 있으면서
주간 이상 고빈도 또는 산업생산지수보다 선행 발표되는 데이터로는
\begin{enumerate}
  \item 한국전력거래소 전력수급현황 실시간 데이터,
  \item 한국은행 뉴스심리지수,
  \item 한국은행 BSI, ESI, CSI, CBSI,
  \item (보조적으로) 국가물류통합정보센터 해상운임지수
\end{enumerate}
가 유효 후보로 도출되었다.

이들 데이터를 기반으로 산업생산지수의 nowcasting 모형을 설계하고,
필요 시 추가 데이터(텍스트 뉴스 원자료, 지역별 세부 전력 및 환경 지표 등)를 단계적으로 확장하는 전략이
KDI 및 기획재정부 차원의 실시간 경기 진단 체계 구축에 가장 현실적이고 효율적인 접근으로 판단된다.


\renewcommand{\refname}{}
\bibliographystyle{unsrt}  
\bibliography{references}
\end{document}
