\begin{table}[h]
\centering
\caption[목표 변수별 모형 성능 비교 (표준화된 RMSE)]{목표 변수별 모형 성능 비교 (표준화된 RMSE)\footnote{ARIMA와 VAR은 모든 목표 변수에 대한 결과가 있음. DFM의 KOCNPER.D(민간 소비)는 수치적 불안정성으로 인해 결과가 없음. DDFM은 모든 목표 변수에 대해 1일 및 7일 예측 결과가 있음.}}
\label{tab:overall_metrics_by_target}
\begin{tabular}{lccc}
\toprule
모형 & GDP & 민간 소비 & 총고정자본형성 \\
\midrule
ARIMA & 0.3140 & 0.2293 & 0.5555 \\
VAR & 0.0563 & 0.0549 & 0.0281 \\
DFM & 0.5335 & N/A\footnotemark[1] & 8.4175 \\
DDFM & 0.5335 & 0.6286 & 1.7368 \\
\bottomrule
\end{tabular}
\footnotetext[1]{DFM의 KOCNPER.D는 EM 알고리즘의 수치적 불안정성으로 인해 모든 예측 기간에서 실패함.}
\end{table}
