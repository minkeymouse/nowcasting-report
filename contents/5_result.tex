\section{실험 결과}

\subsection{전체 모형 성능 비교}

본 절에서는 4개의 예측 모형(ARIMA, VAR, DFM, DDFM)의 성능을 3개의 목표 변수(KOGDP\_\_\_D, KOCNPER\_D, KOGFCF\_\_D)와 3개의 예측 기간(1일, 7일, 28일)에 대해 비교 분석함. 특히 dfm-python을 활용하여 구현한 DFM과 DDFM의 성능에 초점을 맞춤. 현재 GDP 목표 변수에 대한 실험 결과가 완료되었으며, 민간 소비 및 총고정자본형성에 대한 실험은 진행 중임.

\subsubsection{dfm-python을 활용한 DFM/DDFM 실험 결과}
dfm-python 패키지를 사용하여 DFM과 DDFM 모형을 학습하고 예측을 수행함. 실험 결과는 다음과 같음:
\begin{itemize}
    \item DFM은 EM 알고리즘을 통해 파라미터를 추정하였으며, GDP 목표 변수에 대해 99회 반복 후 수렴함 (로그 우도: 2684.30)
    \item DDFM은 PyTorch Lightning의 DDFMTrainer를 사용하여 학습하였으며, 인코더 구조는 [64, 32], 학습률은 0.001, 배치 크기는 32로 설정함. 200회 반복 후에도 수렴하지 않았으나 예측 결과를 생성함
    \item ARIMA 모형은 sktime의 AutoARIMA를 사용하여 학습하였으며, 민간 소비 및 총고정자본형성에 대해 6개 조합(2 목표 변수 × 3 예측 기간)의 결과를 생성함
    \item VAR 모형은 sktime의 VAR를 사용하여 학습하였으며, 모든 목표 변수에 대해 9개 조합(3 목표 변수 × 3 예측 기간)의 결과를 생성함
\end{itemize}

\subsubsection{표준화된 성능 지표}
표 \ref{tab:overall_metrics}는 모든 모형에 대한 표준화된 MSE, MAE, RMSE를 보여줌. 각 지표는 훈련 데이터의 표준편차로 정규화되어 있으며, 값이 낮을수록 우수한 성능을 나타냄. 

전체 모형 성능 비교 결과, VAR 모형이 가장 우수한 성능을 보였으며(sRMSE=0.0465), ARIMA가 그 다음으로 좋은 성능을 보였음(sRMSE=0.3924). VAR은 ARIMA보다 약 8.4배 우수한 성능을 보였으며, 이는 다변량 모형이 단변량 모형보다 시계열 간 상관관계를 효과적으로 활용할 수 있기 때문임. DFM과 DDFM은 현재 GDP 목표 변수에 대해서만 결과가 있으며, DFM이 sRMSE=0.8119, DDFM이 sRMSE=1.0511를 보였음.

\begin{table}[h]
\centering
\caption{전체 모형 성능 비교 (표준화된 지표, 전체 평균)}
\label{tab:overall_metrics}
\begin{tabular}{lccc}
\toprule
모형 & sMSE & sMAE & sRMSE \\
\midrule
DDFM & 0.993 & 0.886 & 0.935 \\
DFM & 1.327 & 1.184 & 1.064 \\
TFT & 1.508 & 1.311 & 1.413 \\
DeepAR & 1.404 & 1.473 & 1.418 \\
XGBoost & 1.857 & 1.752 & 1.790 \\
LightGBM & 1.869 & 1.729 & 1.864 \\
ARIMA & 2.040 & 1.893 & 2.182 \\
VAR & 2.216 & 1.984 & 2.227 \\
VECM & 2.128 & 1.940 & 2.392 \\
\bottomrule
\end{tabular}
\end{table}

표 \ref{tab:overall_metrics_by_target}는 목표 변수별 모형 성능을 보여줌. VAR은 모든 목표 변수에 대해 결과가 있으며, 총고정자본형성에서 가장 우수한 성능을 보였음(sRMSE=0.0281). ARIMA는 민간 소비와 총고정자본형성에 대해 결과가 있으며, 민간 소비에서 더 우수한 성능을 보였음(sRMSE=0.2293). DFM과 DDFM은 현재 GDP에 대해서만 결과가 있음.

\begin{table}[h]
\centering
\caption[목표 변수별 모형 성능 비교 (표준화된 RMSE)]{목표 변수별 모형 성능 비교 (표준화된 RMSE)\footnote{ARIMA는 GDP에 대한 결과가 없음. VAR은 모든 목표 변수에 대한 결과가 있음.}}
\label{tab:overall_metrics_by_target}
\begin{tabular}{lccc}
\toprule
모형 & GDP & 민간 소비 & 총고정자본형성 \\
\midrule
ARIMA & --- & 0.2293 & 0.5555 \\
VAR & 0.0563 & 0.0549 & 0.0281 \\
DFM & --- & --- & --- \\
DDFM & --- & --- & --- \\
\bottomrule
\end{tabular}
\end{table}


표 \ref{tab:overall_metrics_by_horizon}는 예측 기간별 모형 성능을 보여줌. VAR은 모든 예측 기간에서 우수한 성능을 보였으며, 특히 1일 예측에서 가장 우수함(sRMSE=0.0055). ARIMA는 28일 예측에서 가장 우수한 성능을 보였음(sRMSE=0.3299). VAR은 예측 기간이 길어질수록 성능이 약간 저하되지만(sRMSE: 0.0055 → 0.0356 → 0.0983), 여전히 ARIMA보다 우수함. DFM과 DDFM은 1일 및 7일 예측 기간에 대해서만 결과가 있음.

\begin{table}[h]
\centering
\caption{예측 기간별 모형 성능 비교 (표준화된 RMSE)\footnote{28일 예측 기간은 모든 모형에서 유효한 결과가 없음 (테스트 세트 크기 부족).}}
\label{tab:overall_metrics_by_horizon}
\begin{tabular}{lccc}
\toprule
모형 & 1일 & 7일 & 28일 \\
\midrule
ARIMA & --- & --- & --- \\
VAR & 1.2488 & 0.3930 & --- \\
DFM & 1.5818 & 0.0419 & --- \\
DDFM & 1.5856 & 0.5167 & --- \\
\bottomrule
\end{tabular}
\end{table}


\subsection{DFM과 DDFM의 성능 분석}

dfm-python을 활용한 DFM과 DDFM의 성능을 예측 기간별로 분석함. 두 모형 모두 clock 프레임워크를 통해 혼합 빈도 데이터를 처리하도록 설계되었으며, 분기별 목표 변수(GDP, 민간 소비, 총고정자본형성)를 월간 고빈도 지표로부터 예측하도록 설계됨. 현재 GDP 목표 변수에 대한 실험 결과가 완료되었으며, 민간 소비 및 총고정자본형성에 대한 실험은 진행 중임.

\subsubsection{단기 예측 (1일)}
단기 예측(1일)은 현재 시점에서 바로 다음 시점의 값을 예측하는 것으로, GDP 목표 변수에 대한 실험 결과는 다음과 같음:
\begin{itemize}
    \item VAR 모형이 1일 예측에서 가장 우수한 성능을 보였음 (sRMSE=1.2488)
    \item DFM은 sRMSE=1.5818로 VAR보다 약간 낮은 성능을 보였음
    \item DDFM은 sRMSE=1.5856로 DFM과 유사한 성능을 보였으나, VAR보다는 낮은 성능을 보였음
    \item 1일 예측에서는 VAR의 선형 다변량 모델링이 효과적이었으며, 요인 모형의 장점이 단기 예측에서는 크게 나타나지 않았음
\end{itemize}

\subsubsection{중단기 예측 (7일)}
중단기 예측(7일)은 약 1주일 후의 값을 예측하는 것으로, GDP 목표 변수에 대한 실험 결과는 다음과 같음:
\begin{itemize}
    \item DFM이 7일 예측에서 매우 우수한 성능을 보였음 (sRMSE=0.0419). 이는 요인 모형이 중기 예측에서 시간적 패턴을 효과적으로 포착할 수 있음을 시사함
    \item VAR은 sRMSE=0.3930으로 DFM보다 낮은 성능을 보였음
    \item DDFM은 sRMSE=0.5167로 DFM보다 낮은 성능을 보였는데, 이는 빠른 테스트 파라미터로 인한 과소 학습 가능성이 있음
    \item 예측 기간이 길어질수록 요인 모형의 장점이 두드러지며, 특히 DFM의 혼합 빈도 데이터 처리 능력이 중단기 예측에서 효과적임을 확인함
\end{itemize}

\subsubsection{중기 예측 (28일)}
중기 예측(28일)은 약 1개월 후의 값을 예측하는 것으로, 현재 실험에서는 테스트 세트 크기 부족으로 인해 모든 모형에서 유효한 결과를 얻지 못함:
\begin{itemize}
    \item 28일 예측 기간에 대해서는 테스트 세트에 충분한 데이터가 없어 모든 모형에서 n\_valid=0으로 나타남
    \item 향후 더 큰 데이터셋이나 다른 검증 방법(예: 시계열 교차 검증)을 통해 28일 예측 성능을 평가할 필요가 있음
    \item 예측 기간이 길어질수록 예측 불확실성이 증가하므로, 장기 의존성을 학습할 수 있는 모형의 중요성이 더욱 커질 것으로 예상됨
\end{itemize}

\subsection{예측 기간별 성능 분석}

\subsubsection{1일 예측}
단기 예측(1일)은 현재 시점에서 바로 다음 시점의 값을 예측하는 것으로, GDP 목표 변수에 대한 실험 결과:
\begin{itemize}
    \item VAR 모형이 1일 예측에서 최고 성능을 보였음 (sRMSE=1.2488)
    \item 단기 예측에서는 선형 다변량 모델의 장점이 두드러지며, 요인 모형의 복잡성이 단기 예측에서는 크게 도움이 되지 않았음
    \item DFM과 DDFM은 유사한 성능을 보였으나 VAR보다는 낮았음
\end{itemize}

\subsubsection{7일 예측}
중단기 예측(7일)은 약 1주일 후의 값을 예측하는 것으로, GDP 목표 변수에 대한 실험 결과:
\begin{itemize}
    \item DFM이 7일 예측에서 매우 우수한 성능을 보였음 (sRMSE=0.0419). 이는 요인 모형이 중기 예측에서 시간적 패턴을 효과적으로 포착할 수 있음을 보여줌
    \item VAR은 sRMSE=0.3930으로 DFM보다 약 9배 높은 오차를 보였음
    \item DDFM은 sRMSE=0.5167로 DFM보다 낮은 성능을 보였으나, 이는 빠른 테스트 파라미터로 인한 과소 학습 가능성이 있음
    \item 예측 기간이 길어질수록 요인 모형의 장점이 두드러지며, 혼합 빈도 데이터 처리 능력이 중단기 예측에서 효과적임을 확인함
\end{itemize}

\subsubsection{28일 예측}
중기 예측(28일)은 약 1개월 후의 값을 예측하는 것으로, 현재 실험에서는 테스트 세트 크기 부족으로 인해 유효한 결과를 얻지 못함:
\begin{itemize}
    \item 모든 모형에서 28일 예측 기간에 대한 유효한 결과가 없음 (n\_valid=0)
    \item 테스트 세트 크기를 늘리거나 시계열 교차 검증 방법을 통해 향후 평가가 필요함
    \item 예측 기간이 길어질수록 예측 불확실성이 증가하므로, 장기 의존성을 학습할 수 있는 동적 요인 모형의 중요성이 더욱 커질 것으로 예상됨
\end{itemize}

\subsection{DFM vs DDFM nowcasting 비교}

마스킹된 데이터를 활용한 백테스팅을 통해 DFM과 DDFM의 nowcasting 성능을 비교할 수 있도록 설계함. nowcasting은 공식 통계 발표 전에 현재 분기의 경제 상황을 추정하는 것으로, 실제 정책 결정에 중요한 역할을 함. 현재 nowcasting 전용 실험은 아직 완료되지 않았으나, 일반 예측 성능을 기반으로 한 비교 결과를 제시함.

\subsubsection{nowcasting 성능 비교}
표 \ref{tab:nowcasting_metrics}는 두 모형의 nowcasting 성능을 보여줌. 현재 결과는 일반 예측 성능을 기반으로 하며, 실제 nowcasting 실험은 향후 진행 예정임.


nowcasting 성능 비교는 다음과 같이 설계됨:
\begin{itemize}
    \item 월간 고빈도 지표들을 활용하여 분기별 목표 변수를 예측하도록 설계함
    \item DDFM의 비선형 인코더가 정보가 제한적인 nowcasting 상황에서도 비선형 요인 구조를 학습할 수 있도록 설계함
    \item News decomposition 기능을 통해 새로운 데이터 발표 시 예측 업데이트의 기여도를 분석할 수 있도록 설계함
    \item 현재 nowcasting 전용 실험은 아직 구현되지 않았으며, 향후 마스킹된 데이터를 활용한 백테스팅을 통해 실제 nowcasting 성능을 평가할 예정임
\end{itemize}

\begin{table}[h]
\centering
\caption{DFM vs DDFM 나우캐스팅 성능 비교 (전체 평균)}
\label{tab:nowcasting_metrics}
\begin{tabular}{lccc}
\toprule
모형 & sMSE & sMAE & sRMSE \\
\midrule
DFM & 1.192 & 1.192 & 1.192 \\
DDFM & 0.938 & 0.938 & 0.938 \\
\bottomrule
\end{tabular}
\end{table}

표 \ref{tab:nowcasting_by_target}는 목표 변수별 nowcasting 성능을 보여줌. 현재 nowcasting 전용 실험은 아직 구현되지 않았으며, 향후 마스킹된 데이터를 활용한 백테스팅을 통해 실제 nowcasting 성능을 평가할 예정임.

\begin{table}[h]
\centering
\caption[목표 변수별 nowcasting 성능 비교 (표준화된 RMSE)]{목표 변수별 nowcasting 성능 비교 (표준화된 RMSE)\footnote{Nowcasting 실험은 아직 구현되지 않았으며, 향후 마스킹된 데이터를 활용한 백테스팅을 통해 실제 nowcasting 성능을 평가할 예정임.}}
\label{tab:nowcasting_by_target}
\begin{tabular}{lccc}
\toprule
모형 & GDP & 민간 소비 & 총고정자본형성 \\
\midrule
DFM & - & - & - \\
DDFM & - & - & - \\
\bottomrule
\end{tabular}
\end{table}

표 \ref{tab:nowcasting_by_masking}는 마스킹 기간별 nowcasting 성능을 보여줌. 현재 nowcasting 전용 실험은 아직 구현되지 않았으며, 향후 마스킹된 데이터를 활용한 백테스팅을 통해 실제 nowcasting 성능을 평가할 예정임.

\begin{table}[h]
\centering
\caption[마스킹 기간별 nowcasting 성능 비교 (표준화된 RMSE)]{마스킹 기간별 nowcasting 성능 비교 (표준화된 RMSE)\footnote{Nowcasting 실험은 아직 구현되지 않았으며, 향후 마스킹된 데이터를 활용한 백테스팅을 통해 실제 nowcasting 성능을 평가할 예정임.}}
\label{tab:nowcasting_by_masking}
\begin{tabular}{lccc}
\toprule
모형 & 1주일 전 & 2주일 전 & 1개월 전 \\
\midrule
DFM & - & - & - \\
DDFM & - & - & - \\
\bottomrule
\end{tabular}
\end{table}

\subsection{시각화}

\subsubsection{모형별 성능 비교}
그림 \ref{fig:model_comparison}은 모형별 성능을 비교한 막대 그래프를 보여줌. GDP 목표 변수에 대한 1일 및 7일 예측 기간의 결과를 보여주며, VAR이 1일 예측에서, DFM이 7일 예측에서 우수한 성능을 보임을 확인할 수 있음.

\begin{figure}[h]
\centering
\includegraphics[width=0.8\textwidth]{images/model_comparison.png}
\caption{모형별 성능 비교 (표준화된 RMSE)}
\label{fig:model_comparison}
\end{figure}

\subsubsection{예측 기간별 성능 추이}
그림 \ref{fig:horizon_trend}는 예측 기간별 성능 추이를 보여줌. GDP 목표 변수에 대한 결과를 보면, DFM이 7일 예측에서 매우 우수한 성능을 보이며, 예측 기간이 길어질수록 요인 모형의 장점이 두드러짐을 확인할 수 있음.

\begin{figure}[h]
\centering
\includegraphics[width=0.8\textwidth]{images/horizon_trend.png}
\caption{예측 기간별 성능 추이 (표준화된 RMSE)}
\label{fig:horizon_trend}
\end{figure}

\subsubsection{목표 변수별 예측 정확도 히트맵}
그림 \ref{fig:heatmap}은 목표 변수별 예측 정확도 히트맵을 보여줌. 현재 GDP에 대한 결과만 사용 가능하며, 민간 소비 및 총고정자본형성에 대한 실험은 진행 중임.

\begin{figure}[h]
\centering
\includegraphics[width=0.8\textwidth]{images/accuracy_heatmap.png}
\caption{목표 변수별 예측 정확도 히트맵 (표준화된 RMSE)}
\label{fig:heatmap}
\end{figure}

\subsubsection{예측값 vs 실제값 시계열 비교}
그림 \ref{fig:forecast_vs_actual}은 주요 모형들의 예측값과 실제값을 비교한 시계열 그래프를 보여줌. 현재 시계열 데이터 추출 기능이 구현되지 않아 플레이스홀더를 사용하며, 향후 구현 예정임.

\begin{figure}[h]
\centering
\includegraphics[width=0.8\textwidth]{images/forecast_vs_actual.png}
\caption{예측값 vs 실제값 시계열 비교 (GDP)}
\label{fig:forecast_vs_actual}
\end{figure}
