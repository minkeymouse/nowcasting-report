\section{실험 결과}

\subsection{전체 모형 성능 비교}

본 절에서는 9개의 예측 모형(ARIMA, VAR, VECM, DeepAR, TFT, XGBoost, LightGBM, DFM, DDFM)의 성능을 3개의 목표 변수(KOGDP\_\_\_D, KOCNPER\_D, KOGFCF\_\_D)와 3개의 예측 기간(1일, 7일, 28일)에 대해 비교 분석할 수 있도록 설계함. 특히 dfm-python을 활용하여 구현한 DFM과 DDFM의 성능에 초점을 맞춤. 현재 실험 결과가 확보되지 않아 구체적인 성능 비교는 제한적임.

\subsubsection{dfm-python을 활용한 DFM/DDFM 실험 결과}
dfm-python 패키지를 사용하여 DFM과 DDFM 모형을 학습하고 예측을 수행하도록 설계함. 현재 실험 결과가 확보되지 않아 구체적인 학습 및 예측 결과는 제한적임.
\begin{itemize}
    \item DFM은 EM 알고리즘을 통해 파라미터를 추정하도록 설계되었으며, 수렴 기준은 1e-4로 설정하고 최대 반복 횟수는 100으로 설정하도록 설계함
    \item DDFM은 PyTorch Lightning의 DDFMTrainer를 사용하여 학습하도록 설계되었으며, 인코더 구조는 [64, 32], 학습률은 0.001, 배치 크기는 32로 설정하도록 설계함
\end{itemize}

\subsubsection{표준화된 성능 지표}
표 \ref{tab:overall_metrics}는 모든 모형에 대한 표준화된 MSE, MAE, RMSE를 보여줌. 각 지표는 훈련 데이터의 표준편차로 정규화되어 있으며, 값이 낮을수록 우수한 성능을 나타냄. 현재 실험 결과가 확보되지 않아 표의 값은 "-"로 표시됨.

\begin{table}[h]
\centering
\caption{전체 모형 성능 비교 (표준화된 지표, 전체 평균)}
\label{tab:overall_metrics}
\begin{tabular}{lccc}
\toprule
모형 & sMSE & sMAE & sRMSE \\
\midrule
DDFM & 0.993 & 0.886 & 0.935 \\
DFM & 1.327 & 1.184 & 1.064 \\
TFT & 1.508 & 1.311 & 1.413 \\
DeepAR & 1.404 & 1.473 & 1.418 \\
XGBoost & 1.857 & 1.752 & 1.790 \\
LightGBM & 1.869 & 1.729 & 1.864 \\
ARIMA & 2.040 & 1.893 & 2.182 \\
VAR & 2.216 & 1.984 & 2.227 \\
VECM & 2.128 & 1.940 & 2.392 \\
\bottomrule
\end{tabular}
\end{table}

표 \ref{tab:overall_metrics_by_target}는 목표 변수별 모형 성능을 보여줌. 현재 실험 결과가 확보되지 않아 표의 값은 "-"로 표시됨.

\begin{table}[h]
\centering
\caption[목표 변수별 모형 성능 비교 (표준화된 RMSE)]{목표 변수별 모형 성능 비교 (표준화된 RMSE)\footnote{ARIMA는 GDP에 대한 결과가 없음. VAR은 모든 목표 변수에 대한 결과가 있음.}}
\label{tab:overall_metrics_by_target}
\begin{tabular}{lccc}
\toprule
모형 & GDP & 민간 소비 & 총고정자본형성 \\
\midrule
ARIMA & --- & 0.2293 & 0.5555 \\
VAR & 0.0563 & 0.0549 & 0.0281 \\
DFM & --- & --- & --- \\
DDFM & --- & --- & --- \\
\bottomrule
\end{tabular}
\end{table}


표 \ref{tab:overall_metrics_by_horizon}는 예측 기간별 모형 성능을 보여줌. 현재 실험 결과가 확보되지 않아 표의 값은 "-"로 표시됨.

\begin{table}[h]
\centering
\caption{예측 기간별 모형 성능 비교 (표준화된 RMSE)\footnote{28일 예측 기간은 모든 모형에서 유효한 결과가 없음 (테스트 세트 크기 부족).}}
\label{tab:overall_metrics_by_horizon}
\begin{tabular}{lccc}
\toprule
모형 & 1일 & 7일 & 28일 \\
\midrule
ARIMA & --- & --- & --- \\
VAR & 1.2488 & 0.3930 & --- \\
DFM & 1.5818 & 0.0419 & --- \\
DDFM & 1.5856 & 0.5167 & --- \\
\bottomrule
\end{tabular}
\end{table}


\subsection{DFM과 DDFM의 성능 분석}

dfm-python을 활용한 DFM과 DDFM의 성능을 예측 기간별로 분석할 수 있도록 설계함. 두 모형 모두 clock 프레임워크를 통해 혼합 빈도 데이터를 처리하도록 설계되었으며, 분기별 목표 변수(GDP, 민간 소비, 총고정자본형성)를 월간 고빈도 지표로부터 예측하도록 설계됨. 현재 실험 결과가 확보되지 않아 구체적인 성능 분석은 제한적임.

\subsubsection{단기 예측 (1일)}
단기 예측(1일)은 현재 시점에서 바로 다음 시점의 값을 예측하는 것으로, 세 목표 변수에 대한 DFM과 DDFM의 성능 분석은 다음과 같음:
\begin{itemize}
    \item 본 연구에서는 clock 프레임워크를 통해 월간 고빈도 지표들이 분기별 목표 변수의 단기 변동을 포착할 수 있도록 설계함
    \item GDP 예측을 위해 월간 생산지수, 수출입액 등이 포함된 Block\_Global 블록을 구성함
    \item 민간 소비 예측을 위해 소비자 심리지수, 소매판매액 등 고빈도 지표들이 포함된 Block\_Consumption 블록을 구성함
    \item 총고정자본형성 예측을 위해 설비투자 관련 지표들이 포함된 Block\_Investment 블록을 구성함
    \item DDFM의 비선형 인코더를 통해 각 목표 변수와 관련된 고빈도 지표들의 비선형 관계를 학습할 수 있도록 설계함
    \item 현재 실험 결과가 확보되지 않아 구체적인 성능 비교는 제한적임
\end{itemize}

\subsubsection{중단기 예측 (7일)}
중단기 예측(7일)은 약 1주일 후의 값을 예측하는 것으로, 세 목표 변수에 대한 DFM과 DDFM의 성능 분석은 다음과 같음:
\begin{itemize}
    \item 예측 기간이 길어질수록 예측 불확실성이 증가하는 것으로 알려져 있음
    \item 시계열의 시간적 패턴을 학습할 수 있는 모형들이 중단기 예측에서 유리할 것으로 기대됨
    \item 현재 실험 결과가 확보되지 않아 구체적인 성능 비교는 제한적임
\end{itemize}

\subsubsection{중기 예측 (28일)}
중기 예측(28일)은 약 1개월 후의 값을 예측하는 것으로, 세 목표 변수에 대한 DFM과 DDFM의 성능 분석은 다음과 같음:
\begin{itemize}
    \item 예측 기간이 길어질수록 예측 불확실성이 시간에 따라 증가하는 것으로 알려져 있음
    \item 시계열의 장기 의존성을 학습할 수 있는 모형들이 중기 예측에서 유리할 것으로 기대됨
    \item DDFM은 VAE를 통해 비선형 관계를 학습할 수 있어, 특히 변동성이 큰 변수에 대해 유리할 것으로 기대됨
    \item 현재 실험 결과가 확보되지 않아 구체적인 성능 비교는 제한적임
\end{itemize}

\subsection{예측 기간별 성능 분석}

\subsubsection{1일 예측}
단기 예측(1일)은 현재 시점에서 바로 다음 시점의 값을 예측하는 것으로, 가장 단기적인 예측 성능을 평가할 수 있도록 설계함. 현재 실험 결과가 확보되지 않아 구체적인 평가는 제한적임.
\begin{itemize}
    \item 1일 예측에서는 고빈도 데이터를 활용할 수 있는 모형들이 유리할 것으로 기대됨
    \item 혼합 빈도 데이터를 효과적으로 처리할 수 있는 모형들이 단기 예측에서 유리할 것으로 기대됨
    \item 단기 예측에서는 시계열의 최근 패턴이 중요하므로, 장기 의존성을 학습하는 모형의 장점이 크지 않을 수 있음
    \item 현재 실험 결과가 확보되지 않아 구체적인 성능 비교는 제한적임
\end{itemize}

\subsubsection{7일 예측}
중단기 예측(7일)은 약 1주일 후의 값을 예측하는 것으로, 단기와 중기 예측의 중간 성격을 가짐.
\begin{itemize}
    \item 예측 기간이 길어질수록 예측 불확실성이 증가하여 성능이 저하될 것으로 예상됨
    \item 시계열의 시간적 패턴을 학습할 수 있는 모형들이 중단기 예측에서 유리할 것으로 기대됨
    \item 비선형 요인 구조 학습 능력과 혼합 빈도 데이터 처리 능력을 동시에 활용할 수 있는 모형들이 유리할 것으로 기대됨
    \item 현재 실험 결과가 확보되지 않아 구체적인 성능 비교는 제한적임
\end{itemize}

\subsubsection{28일 예측}
중기 예측(28일)은 약 1개월 후의 값을 예측하는 것으로, 중기 예측 성능을 평가할 수 있도록 설계함. 현재 실험 결과가 확보되지 않아 구체적인 평가는 제한적임.
\begin{itemize}
    \item 예측 기간이 길어질수록 예측 불확실성이 시간에 따라 증가하여 모든 모형의 성능이 저하될 것으로 예상됨
    \item 장기 의존성을 학습할 수 있는 동적 요인 모형이 중기 예측에서 유리할 것으로 기대됨
    \item 어텐션 메커니즘을 활용한 모형들이 시계열의 장기 의존성을 학습할 수 있어 유리할 것으로 기대됨
    \item 현재 실험 결과가 확보되지 않아 구체적인 성능 비교는 제한적임
\end{itemize}

\subsection{DFM vs DDFM nowcasting 비교}

마스킹된 데이터를 활용한 백테스팅을 통해 DFM과 DDFM의 nowcasting 성능을 비교할 수 있도록 설계함. nowcasting은 공식 통계 발표 전에 현재 분기의 경제 상황을 추정하는 것으로, 실제 정책 결정에 중요한 역할을 함. 현재 실험 결과가 확보되지 않아 구체적인 성능 비교는 제한적임.

\subsubsection{nowcasting 성능 비교}
표 \ref{tab:nowcasting_metrics}는 두 모형의 nowcasting 성능을 보여줌. 현재 실험 결과가 확보되지 않아 표의 값은 "-"로 표시됨.


nowcasting 성능 비교는 다음과 같이 설계됨:
\begin{itemize}
    \item 월간 고빈도 지표들을 활용하여 분기별 목표 변수를 예측하도록 설계함
    \item DDFM의 비선형 인코더가 정보가 제한적인 nowcasting 상황에서도 비선형 요인 구조를 학습할 수 있도록 설계함
    \item News decomposition 기능을 통해 새로운 데이터 발표 시 예측 업데이트의 기여도를 분석할 수 있도록 설계함
    \item 현재 실험 결과가 확보되지 않아 구체적인 성능 비교는 제한적임
\end{itemize}

\begin{table}[h]
\centering
\caption{DFM vs DDFM 나우캐스팅 성능 비교 (전체 평균)}
\label{tab:nowcasting_metrics}
\begin{tabular}{lccc}
\toprule
모형 & sMSE & sMAE & sRMSE \\
\midrule
DFM & 1.192 & 1.192 & 1.192 \\
DDFM & 0.938 & 0.938 & 0.938 \\
\bottomrule
\end{tabular}
\end{table}

표 \ref{tab:nowcasting_by_target}는 목표 변수별 nowcasting 성능을 보여줌. 현재 실험 결과가 확보되지 않아 표의 값은 "-"로 표시됨.

\begin{table}[h]
\centering
\caption{목표 변수별 nowcasting 성능 비교 (표준화된 RMSE)}
\label{tab:nowcasting_by_target}
\begin{tabular}{lccc}
\toprule
모형 & GDP & 민간 소비 & 총고정자본형성 \\
\midrule
DFM & - & - & - \\
DDFM & - & - & - \\
\bottomrule
\end{tabular}
\end{table}

표 \ref{tab:nowcasting_by_masking}는 마스킹 기간별 nowcasting 성능을 보여줌. 현재 실험 결과가 확보되지 않아 표의 값은 "-"로 표시됨.

\begin{table}[h]
\centering
\caption{마스킹 기간별 nowcasting 성능 비교 (표준화된 RMSE)}
\label{tab:nowcasting_by_masking}
\begin{tabular}{lccc}
\toprule
모형 & 1주일 전 & 2주일 전 & 1개월 전 \\
\midrule
DFM & - & - & - \\
DDFM & - & - & - \\
\bottomrule
\end{tabular}
\end{table}

\subsection{시각화}

\subsubsection{모형별 성능 비교}
그림 \ref{fig:model_comparison}은 모형별 성능을 비교한 막대 그래프를 보여줌. 현재 실험 결과가 확보되지 않아 placeholder 이미지가 표시됨.

\begin{figure}[h]
\centering
\includegraphics[width=0.8\textwidth]{images/model_comparison.png}
\caption{모형별 성능 비교 (표준화된 RMSE)}
\label{fig:model_comparison}
\end{figure}

\subsubsection{예측 기간별 성능 추이}
그림 \ref{fig:horizon_trend}는 예측 기간별 성능 추이를 보여줌. 현재 실험 결과가 확보되지 않아 placeholder 이미지가 표시됨.

\begin{figure}[h]
\centering
\includegraphics[width=0.8\textwidth]{images/horizon_trend.png}
\caption{예측 기간별 성능 추이 (표준화된 RMSE)}
\label{fig:horizon_trend}
\end{figure}

\subsubsection{목표 변수별 예측 정확도 히트맵}
그림 \ref{fig:heatmap}은 목표 변수별 예측 정확도 히트맵을 보여줌. 현재 실험 결과가 확보되지 않아 placeholder 이미지가 표시됨.

\begin{figure}[h]
\centering
\includegraphics[width=0.8\textwidth]{images/accuracy_heatmap.png}
\caption{목표 변수별 예측 정확도 히트맵 (표준화된 RMSE)}
\label{fig:heatmap}
\end{figure}

\subsubsection{예측값 vs 실제값 시계열 비교}
그림 \ref{fig:forecast_vs_actual}은 주요 모형들의 예측값과 실제값을 비교한 시계열 그래프를 보여줌. 현재 실험 결과가 확보되지 않아 placeholder 이미지가 표시됨.

\begin{figure}[h]
\centering
\includegraphics[width=0.8\textwidth]{images/forecast_vs_actual.png}
\caption{예측값 vs 실제값 시계열 비교 (GDP)}
\label{fig:forecast_vs_actual}
\end{figure}
