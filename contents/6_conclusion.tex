\section{결론}

본 연구는 세 가지 주요 한국 거시경제 변수에 대한 나우캐스팅을 위해 네 가지 예측 모형(ARIMA, VAR, DFM, DDFM)의 성능을 여러 예측 수평선에서 비교한다: 생산(KOIPALL.G), 투자(KOEQUIPTE), 소비(KOWRCCNSE).

주요 결과는 세 가지 대상과 세 가지 예측 수평선(1일, 7일, 28일)에 걸쳐 모형 성능을 비교한 실험 결과를 기반으로 제시된다. 전반적 성능 지표는 표~\ref{tab:overall_metrics}에 요약되어 있으며, 대상별 및 수평선별 비교는 각각 표~\ref{tab:overall_metrics_by_target} 및 표~\ref{tab:overall_metrics_by_horizon}에 제시된다. 모든 36개 조합(3개 대상 × 4개 모형 × 3개 수평선)에 대한 상세 결과는 표~\ref{tab:metrics_36_rows}에 제공된다. 성능은 서로 다른 시계열 규모 간 공정한 비교를 가능하게 하기 위해 표준화된 지표(sMSE, sMAE, sRMSE)를 사용하여 평가된다.

\begin{table}[h]
\centering
\caption{전체 모형 성능 비교 (표준화된 지표, 전체 평균)}
\label{tab:overall_metrics}
\begin{tabular}{lccc}
\toprule
모형 & sMSE & sMAE & sRMSE \\
\midrule
DDFM & 0.993 & 0.886 & 0.935 \\
DFM & 1.327 & 1.184 & 1.064 \\
TFT & 1.508 & 1.311 & 1.413 \\
DeepAR & 1.404 & 1.473 & 1.418 \\
XGBoost & 1.857 & 1.752 & 1.790 \\
LightGBM & 1.869 & 1.729 & 1.864 \\
ARIMA & 2.040 & 1.893 & 2.182 \\
VAR & 2.216 & 1.984 & 2.227 \\
VECM & 2.128 & 1.940 & 2.392 \\
\bottomrule
\end{tabular}
\end{table}

\begin{table}[h]
\centering
\caption[목표 변수별 모형 성능 비교 (표준화된 RMSE)]{목표 변수별 모형 성능 비교 (표준화된 RMSE)\footnote{ARIMA는 GDP에 대한 결과가 없음. VAR은 모든 목표 변수에 대한 결과가 있음.}}
\label{tab:overall_metrics_by_target}
\begin{tabular}{lccc}
\toprule
모형 & GDP & 민간 소비 & 총고정자본형성 \\
\midrule
ARIMA & --- & 0.2293 & 0.5555 \\
VAR & 0.0563 & 0.0549 & 0.0281 \\
DFM & --- & --- & --- \\
DDFM & --- & --- & --- \\
\bottomrule
\end{tabular}
\end{table}


\begin{table}[h]
\centering
\caption{예측 기간별 모형 성능 비교 (표준화된 RMSE)\footnote{28일 예측 기간은 모든 모형에서 유효한 결과가 없음 (테스트 세트 크기 부족).}}
\label{tab:overall_metrics_by_horizon}
\begin{tabular}{lccc}
\toprule
모형 & 1일 & 7일 & 28일 \\
\midrule
ARIMA & --- & --- & --- \\
VAR & 1.2488 & 0.3930 & --- \\
DFM & 1.5818 & 0.0419 & --- \\
DDFM & 1.5856 & 0.5167 & --- \\
\bottomrule
\end{tabular}
\end{table}


\begin{table}[h]
\centering
\caption[Standardized MSE and MAE by Target, Model, and Horizon]{Standardized MSE and MAE by Target, Model, and Horizon\footnote{36 combinations: 3 targets (KOEQUIPTE, KOWRCCNSE, KOIPALL.G) × 4 models (ARIMA, VAR, DFM, DDFM) × 3 horizons (1, 7, 28 days). Missing values marked as N/A.}}
\label{tab:metrics_36_rows}
\begin{tabular}{lccccc}
\toprule
Target & Model & Horizon & sMSE & sMAE & sRMSE \\
\midrule
KOEQUIPTE & ARIMA & 1 & 0.0993 & 0.3151 & 0.3151 \\
KOEQUIPTE & ARIMA & 7 & 2.5190 & 1.5871 & 1.5871 \\
KOEQUIPTE & ARIMA & 28 & 2.8019 & 1.6739 & 1.6739 \\
KOEQUIPTE & VAR & 1 & 3.6540e-09 & 6.0448e-05 & 6.0448e-05 \\
KOEQUIPTE & VAR & 7 & 5.7463e+27 & 7.5805e+13 & 7.5805e+13 \\
KOEQUIPTE & VAR & 28 & 1.4143e+120 & 1.1892e+60 & 1.1892e+60 \\
KOEQUIPTE & DFM & 1 & 17.6861 & 4.2055 & 4.2055 \\
KOEQUIPTE & DFM & 7 & 37.2803 & 6.1058 & 6.1058 \\
KOEQUIPTE & DFM & 28 & N/A & N/A & N/A \\
KOEQUIPTE & DDFM & 1 & 1.0611e-04 & 0.0103 & 0.0103 \\
KOEQUIPTE & DDFM & 7 & 3.6503 & 1.9106 & 1.9106 \\
KOEQUIPTE & DDFM & 28 & N/A & N/A & N/A \\
KOWRCCNSE & ARIMA & 1 & 0.6586 & 0.8116 & 0.8116 \\
KOWRCCNSE & ARIMA & 7 & 0.4198 & 0.6479 & 0.6479 \\
KOWRCCNSE & ARIMA & 28 & 0.4579 & 0.6767 & 0.6767 \\
KOWRCCNSE & VAR & 1 & 5.7942e-09 & 7.6119e-05 & 7.6119e-05 \\
KOWRCCNSE & VAR & 7 & 1.1671e+23 & 3.4162e+11 & 3.4162e+11 \\
KOWRCCNSE & VAR & 28 & 2.9220e+117 & 5.4055e+58 & 5.4055e+58 \\
KOWRCCNSE & DFM & 1 & 85.6010 & 9.2521 & 9.2521 \\
KOWRCCNSE & DFM & 7 & 50.0605 & 7.0753 & 7.0753 \\
KOWRCCNSE & DFM & 28 & N/A & N/A & N/A \\
KOWRCCNSE & DDFM & 1 & 0.6682 & 0.8174 & 0.8174 \\
KOWRCCNSE & DDFM & 7 & 1.8477 & 1.3593 & 1.3593 \\
KOWRCCNSE & DDFM & 28 & N/A & N/A & N/A \\
KOIPALL.G & ARIMA & 1 & 0.0034 & 0.0584 & 0.0584 \\
KOIPALL.G & ARIMA & 7 & 2.2818 & 1.5106 & 1.5106 \\
KOIPALL.G & ARIMA & 28 & 0.3866 & 0.6218 & 0.6218 \\
KOIPALL.G & VAR & 1 & 3.5484e-09 & 5.9568e-05 & 5.9568e-05 \\
KOIPALL.G & VAR & 7 & 6.4035e+22 & 2.5305e+11 & 2.5305e+11 \\
KOIPALL.G & VAR & 28 & 1.6938e+117 & 4.1155e+58 & 4.1155e+58 \\
KOIPALL.G & DFM & 1 & 35.0342 & 5.9190 & 5.9190 \\
KOIPALL.G & DFM & 7 & 27.8662 & 5.2788 & 5.2788 \\
KOIPALL.G & DFM & 28 & N/A & N/A & N/A \\
KOIPALL.G & DDFM & 1 & 0.2132 & 0.4617 & 0.4617 \\
KOIPALL.G & DDFM & 7 & 0.0318 & 0.1784 & 0.1784 \\
KOIPALL.G & DDFM & 28 & N/A & N/A & N/A \\
\bottomrule
\end{tabular}
\end{table}

\begin{figure}[h]
\centering
\includegraphics[width=0.8\textwidth]{images/accuracy_heatmap.png}
\caption{정확도 히트맵: 모형 및 대상 변수별 표준화된 RMSE. 낮은 값(어두운 색상)은 더 나은 성능을 나타낸다.}
\label{fig:accuracy_heatmap}
\end{figure}

\begin{figure}[h]
\centering
\includegraphics[width=0.8\textwidth]{images/horizon_trend.png}
\caption{예측 수평선별 성능 추세: 각 모형에 대한 예측 수평선(1일, 7일, 28일)에 걸친 표준화된 RMSE.}
\label{fig:horizon_trend}
\end{figure}

대상 변수에 걸친 모형 정확도의 시각적 비교는 그림~\ref{fig:accuracy_heatmap}에 제시되며, 예측 수평선에 걸친 성능 추세는 그림~\ref{fig:horizon_trend}에 설명되어 있다.

\subsection{주요 결과}

실험 결과는 몇 가지 중요한 결과를 보여준다:

\textbf{ARIMA 성능:} ARIMA는 세 가지 대상과 예측 수평선에 걸쳐 가장 일관되고 신뢰할 수 있는 성능을 보인다. 산업생산(KOIPALL.G)의 경우, ARIMA는 우수한 1일 예측(sRMSE = 0.058)을 달성하며 28일 수평선까지 합리적인 성능을 확장한다(sRMSE = 0.62). 소비(KOWRCCNSE)의 경우, ARIMA는 모든 수평선에 걸쳐 0.65-0.81 사이의 sRMSE 값으로 특히 강한 성능을 보인다. 투자 예측(KOEQUIPTE)은 더 도전적이며, ARIMA는 0.32-1.67의 sRMSE 값을 달성하여 설비투자의 더 높은 변동성을 반영한다.

\textbf{VAR 제한사항:} VAR은 탁월한 1일 예측을 생성하지만(모든 대상에 대해 sRMSE $<$ 10$^{-4}$), 더 긴 수평선에 대해 심각한 수치적 불안정성을 겪는다. 7일 및 28일 예측의 경우, VAR 오차는 비현실적인 크기로 폭발한다(sRMSE $>$ 10$^{11}$), 이는 다단계 앞 예측에 모형을 부적합하게 만든다. 이 불안정성은 매우 짧은 수평선을 넘어 예측할 때 VAR 모형의 근본적인 제한사항이며, 오차 누적 및 잠재적 비정상성 문제 때문일 가능성이 높다.

\textbf{DFM 성능:} DFM은 모든 대상과 수평선에 걸쳐 낮은 성능을 보이며, 1일 예측에 대해 4.2에서 9.3 사이의 sRMSE 값과 7일 예측에 대해 5.3에서 7.1 사이의 값을 보인다. 모형은 특히 소비(KOWRCCNSE) 및 생산(KOIPALL.G)과 어려움을 겪으며, EM 알고리즘 수렴 중 수치적 불안정성 경고를 보인다. 투자(KOEQUIPTE)의 경우, DFM 성능이 더 나지만 여전히 ARIMA보다 낮다. 28일 수평선은 테스트 데이터 부족으로 인해 모든 DFM 모형에서 사용할 수 없다.

\textbf{DDFM 성능:} DDFM은 혼합된 성능을 보인다. 투자(KOEQUIPTE)의 경우, DDFM은 탁월한 1일 예측 정확도(sRMSE = 0.0103)를 달성하여 모든 다른 모형을 능가한다. 생산(KOIPALL.G)의 경우, DDFM은 1일(sRMSE = 0.46) 및 7일(sRMSE = 0.18) 예측 모두에서 우수한 성능을 보이며, ARIMA를 크게 능가한다. 그러나 소비(KOWRCCNSE)의 경우, DDFM의 성능은 1일 예측에 대해 ARIMA와 유사하지만 7일 예측에 대해서는 더 낮다. 28일 수평선은 테스트 데이터 부족으로 인해 모든 DDFM 모형에서 사용할 수 없다. 투자 및 생산에 대한 DDFM의 우수한 성능은 딥러닝 인코더가 이러한 시계열의 복잡한 패턴을 효과적으로 포착함을 시사한다.

\subsection{제한사항}

몇 가지 제한사항을 인정해야 한다: (1) 계획된 36개 모형-대상-수평선 조합 중 30개만 유효한 결과를 가지며, DFM/DDFM 28일 예측은 80/20 훈련-테스트 분할 후 테스트 데이터 부족으로 사용할 수 없다; (2) 평가는 각 수평선당 단일 테스트 포인트(n\_valid = 1)를 사용하여 통계적 신뢰성을 제한한다; (3) 더 긴 수평선에 대한 VAR의 수치적 불안정성은 정규화 또는 대안 추정 방법을 통해 해결되지 않았다; (4) DFM은 EM 알고리즘 수렴 중 일부 대상(KOWRCCNSE, KOIPALL.G)에 대해 수치적 불안정성 경고를 보이지만 결과는 여전히 생성된다; (5) 제한된 테스트 데이터는 DFM/DDFM 모형의 28일 예측 평가를 방지한다.

\subsection{향후 연구 방향}

향후 연구는 다음을 통해 이러한 제한사항을 해결해야 한다: (1) 더 긴 수평선에 대한 수치적 불안정성을 해결하기 위한 정규화 기법 또는 대안 VAR 사양 조사; (2) 더 신뢰할 수 있는 통계적 평가를 위해 각 수평선당 여러 테스트 포인트를 얻기 위해 평가 기간 확장; (3) 개선된 EM 알고리즘 수렴 기준 또는 대안 추정 방법을 통해 DFM 수치적 불안정성 문제 해결; (4) ARIMA의 안정성과 특정 대상에 대한 DDFM의 우수한 성능을 결합한 앙상블 방법 탐색; (5) 나우캐스팅 정확도 향상을 위해 추가 고빈도 데이터 소스 통합; (6) 구조적 단절을 감지하고 그에 따라 모형을 적응시키는 방법 개발; (7) DDFM이 투자 및 생산에 대해서는 ARIMA를 능가하지만 소비에 대해서는 그렇지 않은 이유 조사.
