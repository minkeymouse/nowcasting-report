\section{결론}

본 연구는 고빈도 데이터를 활용하여 한국의 주요 거시경제 변수(생산, 투자, 소비)를 예측하는 동적 요인 모형과 딥러닝 모형의 성능을 비교 분석함.

주요 결과는 다음과 같음: (1) DFM 모형은 분기 데이터로부터 월간 지수를 추정하는 데 양호한 성과를 보였으며, 추정된 월간 지수는 관측 지수와 높은 상관관계를 보임. (2) 고빈도 DFM 모형은 실시간 경기진단에 효과적이며, 1주전 예측에서 양호한 성능을 보임. (3) 딥러닝 모형은 DFM 모형 대비 예측오차가 개선되었으나, 월간 변동폭을 과소 추정하는 경향이 있어 두 모형의 평균값 사용이 바람직함.

본 연구의 기여는 다음과 같음: (1) 생산, 투자, 소비 부문별 nowcasting 시스템 구축, (2) 고빈도 데이터를 활용한 실시간 경기진단 프레임워크 제시, (3) DFM과 딥러닝 모형의 앙상블 전략 제안.

향후 연구 방향으로는 더 많은 고빈도 데이터(실시간 소비 데이터, 카드 거래 데이터 등)의 통합, 구조적 변화 시점 감지 및 모형 적응, 외생 충격 고려 등이 제안됨.
