\subsubsection{Nowcasting}

Nowcasting은 공식 통계가 발표되기 전에 현재 시점의 거시경제 변수를 추정하는 기법임 \cite{banbura2012nowcasting}. 본 연구에서는 각 목표 월에 대해 4주 전과 1주 전 시점에서 nowcasting을 수행하며, 시리즈별 발표 시차를 기준으로 미발표 데이터를 마스킹하여 실제 운영 환경을 시뮬레이션함. DFM과 DDFM은 state-space 구조와 칼만 필터를 통해 이러한 불규칙성을 자연스럽게 처리할 수 있음. ARIMA와 VAR 모형은 release date 마스킹 처리의 구조적 한계로 인해 nowcasting 실험에서 제외되었으며, DFM과 DDFM 모형만 평가 대상에 포함됨.

\begin{table}[h]
\centering
\caption[Nowcasting Backtest Results by Model-Timepoint and Target-Metric]{Nowcasting Backtest Results by Model-Timepoint and Target-Metric\footnote{Train with data from 1985 to 2019, nowcast from Jan 2024 to Dec 2024. For each target month, perform nowcasting at multiple time points (4 weeks before, 1 week before month end). By masking unavailable data based on release dates, generate 1 horizon forecast at each time point. Calculate sMSE, sMAE for each month and time point, then average across 12 months.}}
\label{tab:nowcasting_backtest}
\begin{tabular}{lcccccc}
\toprule
Model-Timepoint & KOIPALL.G & KOIPALL.G & KOEQUIPTE & KOEQUIPTE & KOWRCCNSE & KOWRCCNSE \\
 & sMAE & sMSE & sMAE & sMSE & sMAE & sMSE \\
\midrule
ARIMA-4weeks & N/A & N/A & N/A & N/A & N/A & N/A \\
ARIMA-1weeks & N/A & N/A & N/A & N/A & N/A & N/A \\
VAR-4weeks & N/A & N/A & N/A & N/A & N/A & N/A \\
VAR-1weeks & N/A & N/A & N/A & N/A & N/A & N/A \\
DFM-4weeks & N/A & N/A & N/A & N/A & N/A & N/A \\
DFM-1weeks & N/A & N/A & N/A & N/A & N/A & N/A \\
DDFM-4weeks & N/A & N/A & N/A & N/A & N/A & N/A \\
DDFM-1weeks & N/A & N/A & N/A & N/A & N/A & N/A \\
\bottomrule
\end{tabular}
\end{table}

\textbf{실험 상태:} Nowcasting 백테스트 실험에서 DFM과 DDFM 모형 모두 CUDA 텐서 변환 오류로 인해 모든 시점(2024-01 ~ 2025-10, 22개월)에서 실패함. 총 6개 백테스트 JSON 파일(3개 대상 변수 × 2개 모형)이 생성되었으며, 모든 132개 월별 예측 포인트가 "status": "failed" 상태임. 오류 원인은 예측값이 CUDA 디바이스에 있는 텐서를 numpy 배열로 변환할 때 CPU로 먼저 이동하지 않아 발생한 것으로 확인되었으며, 텐서 변환 코드를 수정하여 해결되었음. 표~\ref{tab:nowcasting_backtest}의 모든 값이 현재 N/A로 표시되며, 코드 수정은 완료되었으나 수정사항의 효과를 검증하기 위해서는 백테스트를 재실행해야 함.

그림~\ref{fig:nowcasting_comparison_koipallg}, 그림~\ref{fig:nowcasting_comparison_koequipte}, 그림~\ref{fig:nowcasting_comparison_kowrccnse}는 Nowcasting 시점별 비교 플롯임. 각 대상 변수별로 "4주 전 nowcasting"과 "1주 전 nowcasting"을 나란히 비교하며, 22개월(2024-01 ~ 2025-10)의 예측값과 실제값을 시간 순서로 연결한 선 그래프임. 현재 플롯은 CUDA 텐서 변환 오류로 인해 placeholder 이미지로 표시되며, 코드 수정 후 재실행 시 유효한 데이터로 플롯을 생성할 수 있음. Forecasting 실험 결과를 바탕으로 1주 전 예측이 4주 전 예측보다 더 정확할 것으로 예상되며, DDFM이 DFM보다 전반적으로 우수한 성능을 보일 것으로 예상됨.

\begin{figure}[h]
\centering
\includegraphics[width=0.9\textwidth]{images/nowcasting_comparison_koipall_g.png}
\caption{Nowcasting 시점별 비교: 전산업생산지수 (KOIPALL.G). 왼쪽: 4주 전 nowcasting, 오른쪽: 1주 전 nowcasting. 각 플롯은 22개월(2024-01 ~ 2025-10)의 예측값과 실제값을 시간 순서로 연결한 선 그래프임. 파란선은 실제값, 주황색 점선은 DFM 예측값, 빨간색 점선은 DDFM 예측값을 나타냄.}
\label{fig:nowcasting_comparison_koipallg}
\end{figure}

\begin{figure}[h]
\centering
\includegraphics[width=0.9\textwidth]{images/nowcasting_comparison_koequipte.png}
\caption{Nowcasting 시점별 비교: 설비투자지수 (KOEQUIPTE). 왼쪽: 4주 전 nowcasting, 오른쪽: 1주 전 nowcasting. 각 플롯은 22개월(2024-01 ~ 2025-10)의 예측값과 실제값을 시간 순서로 연결한 선 그래프임. 파란선은 실제값, 주황색 점선은 DFM 예측값, 빨간색 점선은 DDFM 예측값을 나타냄.}
\label{fig:nowcasting_comparison_koequipte}
\end{figure}

\begin{figure}[h]
\centering
\includegraphics[width=0.9\textwidth]{images/nowcasting_comparison_kowrccnse.png}
\caption{Nowcasting 시점별 비교: 도소매판매액 (KOWRCCNSE). 왼쪽: 4주 전 nowcasting, 오른쪽: 1주 전 nowcasting. 각 플롯은 22개월(2024-01 ~ 2025-10)의 예측값과 실제값을 시간 순서로 연결한 선 그래프임. 파란선은 실제값, 주황색 점선은 DFM 예측값, 빨간색 점선은 DDFM 예측값을 나타냄.}
\label{fig:nowcasting_comparison_kowrccnse}
\end{figure}

