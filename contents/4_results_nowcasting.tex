\subsubsection{Nowcasting}

Nowcasting은 공식 통계가 발표되기 전에 현재 시점의 거시경제 변수를 추정하는 기법임 \cite{banbura2012nowcasting}. 경제 지표는 분기 종료 후 수주에서 수개월 뒤에야 공식 발표되므로, 정책 의사결정 시점에서 현재 경제 상황을 파악하기 어려움. Nowcasting은 이러한 발표 시차(publication lag) 문제를 해결하기 위해 개발된 기법으로, 다양한 고빈도 지표를 활용하여 현재 시점의 거시경제 변수를 실시간으로 추정함.

Nowcasting의 핵심 과제는 실시간 데이터 흐름(data flow)의 불규칙성을 처리하는 것임. 경제 지표들은 서로 다른 발표 일정과 빈도를 가지며, 특정 시점에서 일부 지표는 이미 발표되었고 다른 지표는 아직 발표되지 않은 상태(jagged edges)가 발생함 \cite{bok2019frbny}. DFM과 DDFM은 state-space 구조와 칼만 필터를 통해 이러한 불규칙성을 자연스럽게 처리할 수 있음. 칼만 필터는 새로운 데이터가 도착할 때마다 예측을 재귀적으로 업데이트하며, 각 데이터의 품질과 시의성에 기반한 가중치를 부여함 \cite{banbura2012nowcasting}.

본 연구에서는 각 목표 월에 대해 4주 전과 1주 전 시점에서 nowcasting을 수행함. 각 시점에서 시리즈별 발표 시차를 기준으로 미발표 데이터를 마스킹하여 실제 운영 환경을 시뮬레이션함. 시간이 지날수록 더 많은 데이터가 사용 가능해지므로, 1주 전 예측이 4주 전 예측보다 일반적으로 더 정확할 것으로 기대됨.

\begin{table}[h]
\centering
\caption[Nowcasting Backtest Results by Model-Timepoint and Target-Metric]{Nowcasting Backtest Results by Model-Timepoint and Target-Metric\footnote{Train with data from 1985 to 2019, nowcast from Jan 2024 to Dec 2024. For each target month, perform nowcasting at multiple time points (4 weeks before, 1 week before month end). By masking unavailable data based on release dates, generate 1 horizon forecast at each time point. Calculate sMSE, sMAE for each month and time point, then average across 12 months.}}
\label{tab:nowcasting_backtest}
\begin{tabular}{lcccccc}
\toprule
Model-Timepoint & KOIPALL.G & KOIPALL.G & KOEQUIPTE & KOEQUIPTE & KOWRCCNSE & KOWRCCNSE \\
 & sMAE & sMSE & sMAE & sMSE & sMAE & sMSE \\
\midrule
ARIMA-4weeks & N/A & N/A & N/A & N/A & N/A & N/A \\
ARIMA-1weeks & N/A & N/A & N/A & N/A & N/A & N/A \\
VAR-4weeks & N/A & N/A & N/A & N/A & N/A & N/A \\
VAR-1weeks & N/A & N/A & N/A & N/A & N/A & N/A \\
DFM-4weeks & N/A & N/A & N/A & N/A & N/A & N/A \\
DFM-1weeks & N/A & N/A & N/A & N/A & N/A & N/A \\
DDFM-4weeks & N/A & N/A & N/A & N/A & N/A & N/A \\
DDFM-1weeks & N/A & N/A & N/A & N/A & N/A & N/A \\
\bottomrule
\end{tabular}
\end{table}

그림~\ref{fig:nowcasting_comparison_koipallg}, 그림~\ref{fig:nowcasting_comparison_koequipte}, 그림~\ref{fig:nowcasting_comparison_kowrccnse}는 Nowcasting 시점별 비교 플롯임. 각 대상 변수별로 "4주 전 nowcasting"과 "1주 전 nowcasting"을 나란히 비교하는 플롯으로, 총 3쌍(6개 플롯, 대상 변수별 1쌍)으로 구성됨. 각 플롯은 22개월(2024-01 ~ 2025-10)의 예측값과 실제값을 시간 순서로 연결한 선 그래프임. 파란선(실제값), 주황색 점선(DFM 예측값), 빨간색 점선(DDFM 예측값)을 비교함. X축은 월별 타임스탬프(2024.01 ~ 2025.10), Y축은 대상 변수 값(원본 스케일)임. 모든 값은 원본 데이터 스케일로 표시되며, 변환된 값(chg 등)은 역변환을 통해 원본 스케일로 복원됨. 이 플롯은 시간이 지날수록(1주 전이 4주 전보다) 더 많은 데이터를 사용할 수 있어 예측 정확도가 향상될 수 있음을 나타냄.

\begin{figure}[h]
\centering
\includegraphics[width=0.9\textwidth]{images/nowcasting_comparison_koipall_g.png}
\caption{Nowcasting 시점별 비교: 전산업생산지수 (KOIPALL.G). 왼쪽: 4주 전 nowcasting, 오른쪽: 1주 전 nowcasting. 각 플롯은 22개월(2024-01 ~ 2025-10)의 예측값과 실제값을 시간 순서로 연결한 선 그래프임. 파란선은 실제값, 주황색 점선은 DFM 예측값, 빨간색 점선은 DDFM 예측값을 나타냄.}
\label{fig:nowcasting_comparison_koipallg}
\end{figure}

\begin{figure}[h]
\centering
\includegraphics[width=0.9\textwidth]{images/nowcasting_comparison_koequipte.png}
\caption{Nowcasting 시점별 비교: 설비투자지수 (KOEQUIPTE). 왼쪽: 4주 전 nowcasting, 오른쪽: 1주 전 nowcasting. 각 플롯은 22개월(2024-01 ~ 2025-10)의 예측값과 실제값을 시간 순서로 연결한 선 그래프임. 파란선은 실제값, 주황색 점선은 DFM 예측값, 빨간색 점선은 DDFM 예측값을 나타냄.}
\label{fig:nowcasting_comparison_koequipte}
\end{figure}

\begin{figure}[h]
\centering
\includegraphics[width=0.9\textwidth]{images/nowcasting_comparison_kowrccnse.png}
\caption{Nowcasting 시점별 비교: 도소매판매액 (KOWRCCNSE). 왼쪽: 4주 전 nowcasting, 오른쪽: 1주 전 nowcasting. 각 플롯은 22개월(2024-01 ~ 2025-10)의 예측값과 실제값을 시간 순서로 연결한 선 그래프임. 파란선은 실제값, 주황색 점선은 DFM 예측값, 빨간색 점선은 DDFM 예측값을 나타냄.}
\label{fig:nowcasting_comparison_kowrccnse}
\end{figure}

