\subsubsection{Nowcasting}

Nowcasting은 공식 통계가 발표되기 전에 현재 시점의 거시경제 변수를 추정하는 기법임. 이는 실시간 경제 모니터링의 핵심으로, 중앙은행과 정책기관에서 정책 의사결정을 위해 널리 활용됨. 본 연구에서는 DFM과 DDFM 모형에 대해서만 Nowcasting 백테스트를 수행하여 실제 운영 환경에서의 성능을 평가함. 

ARIMA와 VAR 모형은 release date 마스킹을 효과적으로 처리하기 어려워 Nowcasting 실험에서 제외되었으며, 이는 단변량/다변량 시계열 모형의 구조적 한계 때문임. 반면 DFM과 DDFM은 state-space 형태로 표현되며, Kalman filter를 통해 실시간 데이터 흐름의 불규칙성(jagged edges)을 자연스럽게 처리할 수 있음. 특히 요인 공간에서 마스킹을 처리한 후 관측 공간으로 변환하는 구조적 유연성을 가지고 있어, 비동기적 데이터 발표와 결측치를 효과적으로 다룰 수 있음.

Nowcasting 백테스트 실험은 DFM과 DDFM 모형에 대해 수행되었으나, 현재 시점에서는 유효한 결과가 생성되지 않았음. 이는 데이터 마스킹 과정에서 발생하는 기술적 문제나 데이터 가용성 이슈로 인한 것으로 보임. 향후 실험에서 이러한 문제를 해결하여 유효한 결과를 생성할 예정임.

Nowcasting 실험은 다음과 같이 구성됨: 각 목표 월(2024-01 ~ 2025-10, 22개월)에 대해 여러 시점에서 예측을 수행함. 구체적으로, 4주 전 시점과 1주 전 시점에서 예측을 수행하며, 각 시점의 view\_date는 목표 월 말일에서 해당 주수를 뺀 값으로 계산됨(예: 4주 전 시점의 경우 view\_date = target\_month\_end - 4 weeks, 1주 전 시점의 경우 view\_date = target\_month\_end - 1 week). 

각 시점에서 시리즈별 발표 시차(release date)를 기준으로 미발표 데이터를 NaN으로 마스킹함. 이는 실제 운영 환경에서 특정 시점에 사용 가능한 데이터만을 사용하여 예측하는 상황을 시뮬레이션함. DFM과 DDFM은 요인 모형의 구조적 특성으로 인해 이러한 마스킹을 요인 공간에서 처리한 후 관측 공간으로 변환할 수 있어, ARIMA/VAR과 달리 효과적으로 작동함. 각 시점에서 1 horizon forecast를 생성하며, 시점별 예측 정확도를 비교함. 시간이 지날수록 더 많은 데이터가 사용 가능해지므로, 1주 전 예측이 4주 전 예측보다 일반적으로 더 정확할 것으로 기대됨.

표~\ref{tab:nowcasting_backtest}는 DFM과 DDFM 모형의 2024-2025년 월별 백테스트 결과를 시점별로 제시함. 훈련 기간은 1985년부터 2019년까지이며, Nowcasting 기간은 2024년 1월부터 2025년 10월까지(22개월)임. 표의 행은 모형-시점 조합(4개 행: DFM-4weeks, DFM-1week, DDFM-4weeks, DDFM-1week)을 나타내며, 열은 대상 변수-지표 조합(6개 열: KOIPALL.G\_sMAE, KOIPALL.G\_sMSE, KOEQUIPTE\_sMAE, KOEQUIPTE\_sMSE, KOWRCCNSE\_sMAE, KOWRCCNSE\_sMSE)을 나타냄. 총 4개 행 $\times$ 7개 열(모형-시점 열 포함)로 구성되며, 각 셀은 해당 모형-시점-대상 조합에 대한 평균 sMSE 또는 sMAE를 나타냄. 현재 시점에서는 유효한 결과가 생성되지 않아 모든 값이 N/A로 표시됨.

\begin{table}[h]
\centering
\caption[Nowcasting Backtest Results by Model-Timepoint and Target-Metric]{Nowcasting Backtest Results by Model-Timepoint and Target-Metric\footnote{Train with data from 1985 to 2019, nowcast from Jan 2024 to Dec 2024. For each target month, perform nowcasting at multiple time points (4 weeks before, 1 week before month end). By masking unavailable data based on release dates, generate 1 horizon forecast at each time point. Calculate sMSE, sMAE for each month and time point, then average across 12 months.}}
\label{tab:nowcasting_backtest}
\begin{tabular}{lcccccc}
\toprule
Model-Timepoint & KOIPALL.G & KOIPALL.G & KOEQUIPTE & KOEQUIPTE & KOWRCCNSE & KOWRCCNSE \\
 & sMAE & sMSE & sMAE & sMSE & sMAE & sMSE \\
\midrule
ARIMA-4weeks & N/A & N/A & N/A & N/A & N/A & N/A \\
ARIMA-1weeks & N/A & N/A & N/A & N/A & N/A & N/A \\
VAR-4weeks & N/A & N/A & N/A & N/A & N/A & N/A \\
VAR-1weeks & N/A & N/A & N/A & N/A & N/A & N/A \\
DFM-4weeks & N/A & N/A & N/A & N/A & N/A & N/A \\
DFM-1weeks & N/A & N/A & N/A & N/A & N/A & N/A \\
DDFM-4weeks & N/A & N/A & N/A & N/A & N/A & N/A \\
DDFM-1weeks & N/A & N/A & N/A & N/A & N/A & N/A \\
\bottomrule
\end{tabular}
\end{table}

그림~\ref{fig:nowcasting_comparison_koipallg}, 그림~\ref{fig:nowcasting_comparison_koequipte}, 그림~\ref{fig:nowcasting_comparison_kowrccnse}는 Nowcasting 시점별 비교 플롯임. 각 대상 변수별로 "4주 전 nowcasting"과 "1주 전 nowcasting"을 나란히 비교하는 플롯으로, 총 3쌍(6개 플롯, 대상 변수별 1쌍)으로 구성됨. 각 플롯은 22개월(2024-01 ~ 2025-10)의 예측값과 실제값을 시간 순서로 연결한 선 그래프임. 파란선(실제값)과 빨간 점선(모형 평균 예측값)을 비교함. X축은 월별 타임스탬프(2024.01 ~ 2025.10), Y축은 대상 변수 값(\%)임. 이 플롯은 시간이 지날수록(1주 전이 4주 전보다) 더 많은 데이터를 사용할 수 있어 예측 정확도가 향상될 수 있음을 나타냄.

\begin{figure}[h]
\centering
\includegraphics[width=0.9\textwidth]{images/nowcasting_comparison_koipall_g.png}
\caption{Nowcasting 시점별 비교: 전산업생산지수 (KOIPALL.G). 왼쪽: 4주 전 nowcasting, 오른쪽: 1주 전 nowcasting. 각 플롯은 22개월(2024-01 ~ 2025-10)의 예측값과 실제값을 시간 순서로 연결한 선 그래프임. 파란선은 실제값, 빨간 점선은 모형 평균 예측값을 나타냄.}
\label{fig:nowcasting_comparison_koipallg}
\end{figure}

\begin{figure}[h]
\centering
\includegraphics[width=0.9\textwidth]{images/nowcasting_comparison_koequipte.png}
\caption{Nowcasting 시점별 비교: 설비투자지수 (KOEQUIPTE). 왼쪽: 4주 전 nowcasting, 오른쪽: 1주 전 nowcasting. 각 플롯은 22개월(2024-01 ~ 2025-10)의 예측값과 실제값을 시간 순서로 연결한 선 그래프임. 파란선은 실제값, 빨간 점선은 모형 평균 예측값을 나타냄.}
\label{fig:nowcasting_comparison_koequipte}
\end{figure}

\begin{figure}[h]
\centering
\includegraphics[width=0.9\textwidth]{images/nowcasting_comparison_kowrccnse.png}
\caption{Nowcasting 시점별 비교: 도소매판매액 (KOWRCCNSE). 왼쪽: 4주 전 nowcasting, 오른쪽: 1주 전 nowcasting. 각 플롯은 22개월(2024-01 ~ 2025-10)의 예측값과 실제값을 시간 순서로 연결한 선 그래프임. 파란선은 실제값, 빨간 점선은 모형 평균 예측값을 나타냄.}
\label{fig:nowcasting_comparison_kowrccnse}
\end{figure}

