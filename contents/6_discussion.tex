\section{논의}

\subsection{모델 비교}

네 가지 모형(ARIMA, VAR, DFM, DDFM)의 성능을 대상 변수와 예측 수평선에 걸쳐 비교:

\textbf{ARIMA:}
\begin{itemize}
    \item 세 대상 변수 모두에서 성공적으로 평가 완료
    \item 특징: 단순성, 해석 가능성, 안정적 성능
    \item 일부 수평선에서 우수한 성능을 보임 (예: KOEQUIPTE 3개월에서 최소 sMAE/sMSE)
    \item Nowcasting에서는 release date 마스킹 처리의 어려움으로 인해 제한적임
\end{itemize}

\textbf{VAR:}
\begin{itemize}
    \item 세 대상 변수 모두에서 성공적으로 평가 완료
    \item 단기 수평선에서 상대적으로 안정적인 성능을 보임
    \item Nowcasting에서는 release date 마스킹 처리의 어려움으로 인해 제한적임
\end{itemize}

\textbf{DFM:}
\begin{itemize}
    \item 세 대상 변수 모두에서 성공적으로 평가 완료
    \item Nowcasting에서 release date 마스킹을 효과적으로 처리 가능
\end{itemize}

\textbf{DDFM:}
\begin{itemize}
    \item 세 대상 변수 모두에서 성공적으로 평가 완료
    \item 중간 수평선(11개월)에서 우수한 성능을 보임
    \item Nowcasting에서 release date 마스킹을 효과적으로 처리 가능
\end{itemize}

\subsection{원인 분석}

\subsubsection{모형별 제한사항}
\begin{itemize}
    \item VAR: 긴 수평선에서 공분산 행렬 특이성으로 인한 수치적 불안정성
    \item DFM: EM 알고리즘 수렴 중 수치적 불안정성 발생, 수치 안정화 기법 적용으로 해결
\end{itemize}

\subsubsection{DDFM의 성능 특성}

DDFM은 중간 수평선(11개월)에서 우수한 성능을 보이나, 단기(1개월)와 장기(22개월) 수평선에서는 다른 모형보다 높은 오차를 보임. 가능한 원인: 데이터 양 부족, 선형 관계에서 비선형 인코더의 과도한 복잡성, 하이퍼파라미터 최적화 부족 \cite{andreini2020deep}. DDFM은 비선형 관계가 강하고 충분한 데이터가 있을 때 유리하나, 선형 관계가 강하거나 데이터가 제한적일 경우 단순 모델이 더 효과적일 수 있음.

\subsection{Nowcasting 시점별 분석}

Nowcasting 실험 구성:
\begin{itemize}
    \item 모형: DFM, DDFM (2개) - ARIMA와 VAR은 release date 마스킹 처리의 구조적 한계로 인해 제외
    \item 대상 변수: 3개
    \item 목표 월: 2024--01 ~ 2025--10 (22개월)
    \item 예측 시점: 4주 전, 1주 전
    \item 결과: DFM과 DDFM 모두 모든 대상 변수에서 성공적으로 결과를 생성함
\end{itemize}

\textbf{예상 성능 향상:} \cite{banbura2012nowcasting}
\begin{itemize}
    \item 시간 경과에 따라 더 많은 데이터 사용 가능 $\to$ 예측 정확도 향상
    \item DFM/DDFM: 많은 시계열 활용, 요인 추출 $\to$ 데이터 증가 시 성능 향상 기대
    \item Release date 마스킹: DFM/DDFM은 요인 모형의 구조적 특성으로 인해 release date 기반 마스킹을 효과적으로 처리 가능. Kalman filter는 각 시점의 데이터 발표를 재귀적으로 처리하여 예측을 업데이트하며, 데이터의 시의성과 품질을 자동으로 고려함
    \item 실시간 데이터 흐름: 비동기적 데이터 발표로 인한 불규칙성(jagged edges)을 DFM/DDFM이 자연스럽게 처리할 수 있어, 실제 운영 환경에서의 nowcasting에 적합함
\end{itemize}

