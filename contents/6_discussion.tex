\section{논의}

\subsection{모델 비교}

네 가지 모형(ARIMA, VAR, DFM, DDFM)의 성능을 대상 변수와 예측 수평선에 걸쳐 비교:

\textbf{ARIMA:}
\begin{itemize}
    \item 일관된 성능, 다양한 수평선에 걸친 안정성
    \item KOIPALL.G: 1개월 sRMSE = 0.058, 28개월까지 합리적 성능
    \item KOWRCCNSE: 0.65--0.81 sRMSE
    \item KOEQUIPTE: 0.32--1.67 sRMSE (설비투자 변동성 반영)
    \item 특징: 단순성, 해석 가능성, 안정적 성능
\end{itemize}

\textbf{VAR:}
\begin{itemize}
    \item 1개월 예측: 매우 낮은 오차 (sMAE $<$ 10$^{-4}$, sMSE $<$ 10$^{-8}$)
    \item 7개월/28개월: 수치적 불안정성 (sMSE $>$ 10$^{22}$, sMAE $>$ 10$^{10}$)
    \item 제한: 다단계 예측에 사용 제한
\end{itemize}

\textbf{DFM:}
\begin{itemize}
    \item 높은 오차: 1개월 4.2--9.3 sRMSE, 7개월 5.3--7.1 sRMSE
    \item KOWRCCNSE, KOIPALL.G에서 어려움, EM 알고리즘 수렴 중 수치적 불안정성
    \item KOEQUIPTE: 상대적으로 낮은 오차이나 ARIMA보다 높음
\end{itemize}

\textbf{DDFM:}
\begin{itemize}
    \item KOEQUIPTE: 1개월 sRMSE = 0.0103 (최고 성능)
    \item KOIPALL.G: 1개월 sRMSE = 0.46, 7개월 sRMSE = 0.18 (ARIMA보다 우수)
    \item KOWRCCNSE: 1개월 ARIMA 유사, 7개월 더 높은 오차
    \item 해석: 딥러닝 인코더가 복잡한 패턴 학습 가능 (투자, 생산)
\end{itemize}

\subsection{원인 분석}

\subsubsection{VAR의 수치적 불안정성}
\begin{itemize}
    \item 1개월 예측: 낮은 오차
    \item 긴 수평선: 공분산 행렬 특이성으로 인한 수치적 불안정성
    \item 원인: 재귀적 예측 과정에서 수치 오차 누적 및 증폭
\end{itemize}

\subsubsection{DFM의 EM 알고리즘 수렴 문제}
\begin{itemize}
    \item 원인: MATLAB/Python 구현 차이 (R 행렬 최소값, 대칭성 강제, 칼만 필터 재귀적 업데이트)
    \item 적용 기법: 사전정규화, R 행렬 최소값 $10^{-4}$ 강제, 공분산 행렬 대칭성 강제 \cite{golub2013matrix, higham2002computing}
\end{itemize}

\subsubsection{DDFM의 비선형 패턴 포착 능력}

DDFM 성능 차이의 원인 (시계열 비선형성 정도):
\begin{itemize}
    \item \textbf{투자 (KOEQUIPTE):} 경기 순환, 정책 변화, 기업 심리 등 복잡한 비선형 요인 $\to$ DDFM 우수
    \item \textbf{생산 (KOIPALL.G):} 산업별, 부문별 이질성으로 인한 비선형 패턴 $\to$ DDFM 우수
    \item \textbf{소비 (KOWRCCNSE):} 상대적으로 안정적이고 선형적 패턴 $\to$ ARIMA 적합
\end{itemize}

\textbf{시사점:} 모형 선택 시 시계열 특성 고려 필요 (비선형 패턴: DDFM, 선형 패턴: ARIMA)

\subsection{Nowcasting 시점별 분석}

Nowcasting 실험 구성:
\begin{itemize}
    \item 모형: ARIMA, VAR, DFM, DDFM (4개)
    \item 대상 변수: 3개
    \item 목표 월: 2024--01 ~ 2024--12 (12개월)
    \item 예측 시점: 4주 전, 1주 전
\end{itemize}

\textbf{예상 성능 향상:} \cite{banbura2012nowcasting}
\begin{itemize}
    \item 시간 경과에 따라 더 많은 데이터 사용 가능 $\to$ 예측 정확도 향상
    \item DFM/DDFM: 많은 시계열 활용, 요인 추출 $\to$ 데이터 증가 시 성능 향상 기대
    \item Release date 마스킹: DFM/DDFM에서 더 정확, ARIMA/VAR은 근사화 구현으로 제한
\end{itemize}

