\section{논의}
\label{sec:discussion}

\subsection{모형 성능 해석}

본 연구의 실험 결과는 거시경제 변수 예측에서 모형 선택의 중요성과 각 모형의 특성을 명확히 보여줌. 특히 VAR 모형의 우수한 성능, DFM과 DDFM의 상대적 성과, 그리고 목표 변수별 모형 적합성의 차이에 대한 경제적 해석이 필요함.

\subsubsection{VAR 모형의 우수성: 다변량 관계의 효과적 활용}
VAR 모형의 우수한 성능은 거시경제 변수들 간의 구조적 상호의존성을 직접적으로 포착할 수 있기 때문임. GDP, 소비, 투자 등은 단기적으로 순환적 피드백 구조를 형성함: 소비 증가는 GDP 성장을 유도하고, 이는 다시 소비와 투자에 영향을 미치는 동시적 관계임. VAR은 이러한 동시적 상호작용을 벡터 자기회귀 구조로 모델링하므로, 단기 예측에서 특히 효과적임.

경제학적 관점에서, VAR의 성공은 거시경제 변수들이 독립적으로 진화하지 않고 동시에 결정되는 구조적 특성을 반영함. 반면 ARIMA는 단변량 모형으로서 이러한 구조적 관계를 활용하지 못하며, DFM과 DDFM은 요인 추출 과정에서 정보 손실이 발생할 수 있어 단기 동적 관계 포착에 한계가 있음.

\subsubsection{DFM의 성공과 실패: 데이터 구조의 영향}
DFM이 KOGDP...D에서는 양호한 성능을 보였으나 KOCNPER.D에서는 수치적 불안정성으로 실패한 이유는 데이터 구조의 차이에서 찾을 수 있음. GDP는 많은 고빈도 지표들(생산지수, 수출입액, 설비투자 등)과 안정적인 선형 관계를 가지며, 요인 구조가 명확하게 정의됨. 반면 민간 소비(KOCNPER.D)는 소비자 심리, 소득, 물가 등 다양한 요인들이 복잡하게 얽혀 있어, 선형 요인 모델로는 포착하기 어려운 비선형 관계를 가질 수 있음.

KOCNPER.D에서 발생한 수치적 불안정성은 소비 관련 변수들의 높은 공선성과 복잡한 비선형 관계에서 기인함. 소비 관련 변수들이 유사한 패턴을 보여 요인 추출 시 공선성이 발생하고, 이로 인해 EM 알고리즘의 수렴이 실패함. 이는 선형 요인 모델의 근본적 한계를 시사하며, 비선형 모델(DDFM)이 이러한 복잡한 관계를 더 잘 포착할 수 있음을 보여줌.

\subsubsection{DFM vs DDFM: 비선형성의 역할}
DDFM은 비선형 인코더를 통해 요인 구조를 학습하므로, 선형 가정에 의존하는 DFM의 EM 알고리즘보다 복잡한 관계를 포착할 수 있음. KOCNPER.D에서 DDFM이 성공적으로 예측을 수행한 것은 비선형 인코더가 소비 관련 변수들의 복잡한 관계를 더 잘 모델링할 수 있기 때문임. 반면 KOGDP...D에서는 DFM과 DDFM이 유사한 성능을 보였는데, 이는 GDP와 고빈도 지표 간의 관계가 상대적으로 선형적이어서 비선형 모델의 이점이 제한적이기 때문임.

그러나 DDFM이 항상 DFM보다 우수한 것은 아님. KOGFCF..D(총고정자본형성)에서는 두 모형 모두 높은 오차를 보였으며, 이는 투자 변수의 높은 변동성과 구조적 변화가 요인 모델의 한계를 드러내기 때문임. 비선형성은 복잡한 관계를 포착하는 데 도움이 되지만, 과도한 변동성이나 외생 충격에 대해서는 요인 구조 자체가 한계를 가짐. 따라서 비선형 모델의 이점은 목표 변수의 데이터 특성에 따라 달라짐을 확인할 수 있음.

\subsection{경제적 의미}

\subsubsection{nowcasting의 실용성: 구체적 시나리오}
nowcasting은 공식 통계 발표 전에 현재 분기의 경제 상황을 추정하는 것으로, 실제 정책 결정에 중요한 역할을 함. 구체적인 시나리오를 예로 들면, 2024년 2분기 말(6월 말) 시점에서 정책 결정자들은 해당 분기의 GDP 성장률을 추정해야 함. 그러나 공식 GDP 통계는 분기 종료 후 약 25일이 지나야 발표됨. 이 시점에서 월간 생산지수(5월, 6월), 소매판매액(5월, 6월), 수출입액(5월, 6월) 등의 고빈도 지표는 이미 사용 가능함. DFM 또는 DDFM을 활용하면 이러한 고빈도 지표들을 통해 2분기 GDP를 추정할 수 있으며, 이를 바탕으로 선제적인 정책 조치(예: 통화정책 조정, 재정정책 수립)를 취할 수 있음.

Schorfheide와 Song (2020)의 연구에서도 COVID-19 팬데믹 기간 동안 nowcasting의 중요성이 강조되었으며, 혼합 빈도 모형을 활용한 실시간 경제 모니터링이 정책 대응의 시의성을 향상시킬 수 있음을 보여줌 \cite{schorfheide2020nowcasting}. 특히 경제 위기 상황에서는 공식 통계 발표까지의 시차로 인해 정책 대응이 지연될 수 있으며, nowcasting을 통해 이러한 시차를 단축할 수 있음.

\subsubsection{모형 선택 가이드라인}
실험 결과를 바탕으로 다음과 같은 모형 선택 가이드라인을 제시할 수 있음:

\begin{itemize}
    \item \textbf{단기 예측(1-7일)}: VAR 모형이 가장 효과적임. 거시경제 변수들 간의 동시적 상호작용이 중요한 단기 예측에서는 VAR의 직접적 다변량 모델링이 우수함.
    \item \textbf{중장기 예측(28일 이상)}: ARIMA가 상대적으로 우수한 성능을 보임. 장기적으로는 각 변수의 자체 동학이 중요하며, 단변량 모델이 충분할 수 있음.
    \item \textbf{혼합 빈도 nowcasting}: DFM 또는 DDFM이 적합함. 분기별 목표 변수를 월간 고빈도 지표로부터 예측하는 경우, 요인 모델의 혼합 빈도 처리 능력이 필수적임. 다만 목표 변수의 데이터 특성에 따라 DFM(선형 관계가 명확한 경우) 또는 DDFM(비선형 관계가 중요한 경우)을 선택해야 함.
    \item \textbf{변동성이 큰 변수}: KOGFCF..D(총고정자본형성)와 같이 변동성이 큰 변수에 대해서는 요인 모델의 한계가 드러나며, VAR이나 ARIMA가 더 안정적일 수 있음.
\end{itemize}

\subsubsection{정책적 함의}
본 연구의 결과는 다음과 같은 정책적 함의를 가짐:

\begin{itemize}
    \item \textbf{실시간 경제 모니터링 시스템 구축}: VAR 모형의 우수한 단기 예측 성능을 활용하여 실시간 경제 모니터링 시스템을 구축할 수 있음. 예를 들어, 한국은행의 통화정책위원회는 월간 고빈도 지표를 활용하여 분기별 GDP를 실시간으로 추정하고, 이를 바탕으로 기준금리 결정 시점을 앞당길 수 있음.
    \item \textbf{모형 앙상블 전략}: 단일 모형에 의존하기보다는 여러 모형의 예측을 종합하는 앙상블 전략이 효과적임. VAR(단기), ARIMA(장기), DFM/DDFM(혼합 빈도)의 예측을 가중 평균하거나, 상황에 따라 선택적으로 활용할 수 있음.
    \item \textbf{고빈도 데이터 인프라 강화}: 고빈도 데이터가 예측 정확도 향상에 기여함을 확인하였으므로, 생산지수, 소매판매액, 수출입액 등의 월간 지표의 신뢰성과 시의성을 향상시키는 것이 중요함. 또한 새로운 고빈도 지표(예: 실시간 소비 데이터, 카드 거래 데이터)의 수집 및 통합도 고려할 수 있음.
\end{itemize}

\subsection{연구의 한계점}

본 연구는 다음과 같은 한계점을 가짐:

\begin{itemize}
    \item \textbf{데이터 품질 및 전처리}: 많은 변수들이 결측치를 포함하고 있어, 전처리 과정에서 선형 보간 또는 전방 채우기를 사용함. 이는 정보 손실을 초래할 수 있으며, 특히 구조적 변화 시점에서 부정확할 수 있음.
    \item \textbf{모형 해석가능성}: 딥러닝 기반 모형(DDFM)은 비선형 인코더를 통해 복잡한 관계를 학습하지만, 학습된 요인 구조의 경제적 해석이 어려움. 요인 해석 방법론(예: attention weight 분석, 요인 부하량 분석)의 개발이 필요함.
    \item \textbf{외생 충격 고려 부재}: 본 연구는 과거 데이터에 기반한 예측만을 다루며, 예상치 못한 외생 충격(예: 자연재해, 지정학적 사건, 글로벌 금융위기)을 고려하지 않음. 구조적 변화를 모델링하는 방법론(예: regime-switching models)의 통합이 필요함.
    \item \textbf{한국 데이터에 국한}: 본 연구는 한국 데이터에만 적용되었으며, 다른 국가나 지역에 대한 일반화 가능성은 추가 검증이 필요함. 특히 선진국과 신흥국 간의 데이터 구조 차이를 고려한 비교 연구가 필요함.
    \item \textbf{실험 완료율}: 총 36개 조합 중 28개 조합(77.8\%)만 완료되었으며, 8개 조합은 데이터 및 모형의 한계로 인해 평가 불가능함. 특히 DFM의 KOCNPER.D 수치적 불안정성 및 모든 모형의 28일 예측 평가 불가능 등의 한계가 있음.
\end{itemize}

\subsection{향후 연구 방향}

본 연구의 실험 결과를 바탕으로 다음과 같은 구체적인 향후 연구 방향을 제안함:

\begin{itemize}
    \item \textbf{DFM 수치적 안정성 개선}: KOCNPER.D에서 DFM의 EM 알고리즘이 수치적 불안정성을 보였으며, 다음과 같은 개선 방안을 제안함: (1) 적응형 정규화(adaptive regularization) - 조건수가 높을수록 정규화 강도 증가, (2) 더 나은 초기화 방법 - 주성분 분석(PCA) 기반 초기화 대신 정규화된 주성분 분석 사용, (3) 대안적 추정 방법 - EM 알고리즘 대신 베이지안 방법론(MCMC) 또는 변분 추론(VI) 고려.
    \item \textbf{28일 예측 기간 평가 방법론 개발}: 테스트 세트 크기 부족(80/20 분할로 인해 테스트 세트가 28개 미만)으로 28일 예측 평가가 불가능하였으나, 시계열 교차 검증(time series cross-validation) 또는 롤링 윈도우(rolling window) 방법을 통해 평가 가능함. 특히 walk-forward validation을 통해 각 시점에서 28일 예측을 수행하고 성능을 집계하는 방법을 제안함.
    \item \textbf{Nowcasting 전용 실험 설계}: 마스킹된 데이터를 활용한 백테스팅을 통해 실제 nowcasting 성능을 평가할 필요가 있음. 구체적으로, 각 시점에서 목표 변수의 최근 관측치를 마스킹하고, 사용 가능한 고빈도 데이터만을 활용하여 예측을 수행하는 실험을 설계함. 이를 통해 News decomposition 기능을 활용하여 새로운 데이터 발표 시 예측 업데이트의 기여도를 정량화할 수 있음.
    \item \textbf{변동성이 큰 변수에 대한 모형 개선}: KOGFCF..D(총고정자본형성)에서 DFM과 DDFM 모두 높은 오차를 보였으며, 변동성이 큰 변수에 대한 모형 개선이 필요함. 제안 방안: (1) 변동성 모델링 통합 - GARCH 또는 stochastic volatility 모델을 요인 모델에 통합, (2) 구조적 변화 감지 - changepoint detection을 통한 구조적 변화 시점 식별 및 모형 적응, (3) 앙상블 방법 - 요인 모델과 VAR/ARIMA의 앙상블을 통해 변동성에 강건한 예측 생성.
    \item \textbf{하이퍼파라미터 최적화}: DDFM의 하이퍼파라미터(인코더 구조, 학습률, 배치 크기 등)를 목표 변수별로 최적화하는 연구가 필요함. 베이지안 최적화(Bayesian optimization) 또는 자동 머신러닝(AutoML) 기법을 활용하여 효율적으로 탐색할 수 있음.
\end{itemize}



