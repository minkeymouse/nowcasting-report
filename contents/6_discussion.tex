\section{논의}

\subsection{모델 비교}

네 가지 모형(ARIMA, VAR, DFM, DDFM)의 성능을 대상 변수와 예측 수평선에 걸쳐 비교 분석함. Forecasting과 Nowcasting 결과를 종합하여 각 모형의 장단점을 평가함.

\textbf{ARIMA:} ARIMA는 세 가지 대상 변수와 평가된 예측 수평선에 걸쳐 일관된 성능을 보임. 산업생산(KOIPALL.G)의 경우, ARIMA는 1일 예측에서 sRMSE = 0.058을 보이며 30일 수평선까지 합리적인 성능을 유지함. 소비(KOWRCCNSE)의 경우, ARIMA는 평가된 수평선에 걸쳐 0.65-0.81 사이의 sRMSE 값을 보임. 투자 예측(KOEQUIPTE)은 더 도전적이며, ARIMA는 0.32-1.67의 sRMSE 값을 보여 설비투자의 더 높은 변동성을 반영함. ARIMA의 특징은 단순성, 해석 가능성, 그리고 다양한 수평선에 걸친 안정적인 성능임.

\textbf{VAR:} VAR은 1일 예측에서 매우 낮은 오차를 보임(평가된 대상에 대해 sMAE $<$ 10$^{-4}$, sMSE $<$ 10$^{-8}$). 그러나 더 긴 수평선에 대해 수치적 불안정성을 보임. 7일 및 30일 예측의 경우, VAR 오차는 매우 큰 값을 보임(sMSE $>$ 10$^{22}$, sMAE $>$ 10$^{10}$), 이는 다단계 앞 예측에 모형 사용을 제한함. 이 불안정성은 짧은 수평선을 넘어 예측할 때 VAR 모형의 제한사항으로 관찰됨.

\textbf{DFM:} DFM은 평가된 대상과 수평선에 걸쳐 높은 오차를 보이며, 1일 예측에 대해 4.2에서 9.3 사이의 sRMSE 값과 7일 예측에 대해 5.3에서 7.1 사이의 값을 보임. 모형은 소비(KOWRCCNSE) 및 생산(KOIPALL.G)에서 어려움을 겪으며, EM 알고리즘 수렴 중 수치적 불안정성 경고를 보임. 투자(KOEQUIPTE)의 경우, DFM 오차가 상대적으로 낮지만 여전히 ARIMA보다 높음.

\textbf{DDFM:} DDFM은 혼합된 성능을 보임. 투자(KOEQUIPTE)의 경우, DDFM은 1일 예측에서 sRMSE = 0.0103을 보여 다른 모형보다 낮은 오차를 보임. 생산(KOIPALL.G)의 경우, DDFM은 1일(sRMSE = 0.46) 및 7일(sRMSE = 0.18) 예측에서 ARIMA보다 낮은 오차를 보임. 그러나 소비(KOWRCCNSE)의 경우, DDFM의 성능은 1일 예측에 대해 ARIMA와 유사하지만 7일 예측에 대해서는 더 높은 오차를 보임. 투자 및 생산에 대한 DDFM의 성능은 딥러닝 인코더가 이러한 시계열의 복잡한 패턴을 학습할 수 있는 것으로 해석될 수 있음.

\subsection{원인 분석}

각 모형의 성능 차이는 다음과 같은 원인으로 분석됨:

\subsubsection{VAR의 수치적 불안정성}

VAR 모형은 1일 예측에서 낮은 오차를 보이지만, 더 긴 수평선에 대해 수치적 불안정성을 보임. 이는 VAR 모형의 다단계 예측에서 공분산 행렬의 특이성(singularity)이나 조건이 나쁜 행렬(ill-conditioned matrix)이 발생할 수 있기 때문임. VAR 모형은 재귀적으로 예측을 수행하므로, 작은 수치 오차가 누적되어 큰 오차로 증폭될 수 있음.

\subsubsection{DFM의 EM 알고리즘 수렴 문제}

DFM의 낮은 성능은 EM 알고리즘 수렴 중 수치적 불안정성 문제와 관련이 있음. MATLAB 참조 구현과 Python 구현 간의 차이점이 주요 원인으로 분석됨: (1) R 행렬 최소값이 MATLAB은 $10^{-4}$인데 Python은 $10^{-6}$으로 설정되어 F 행렬이 ill-conditioned가 될 수 있음; (2) 대칭성 강제가 일관되게 적용되지 않아 비대칭성 누적이 발생할 수 있음; (3) 칼만 필터의 재귀적 업데이트에서 NaN/Inf 전파로 연쇄 반응이 발생할 수 있음.

본 연구에서는 다음과 같은 수치 안정화 개선 사항을 적용함: (1) 사전정규화로 조건수 $10^8$ 이상인 경우 적응적 정규화 적용 \cite{golub2013matrix}; (2) R 행렬 최소값을 $10^{-4}$로 상향 조정하여 MATLAB과 일치; (3) 공분산 행렬 업데이트 후 즉시 대칭성 강제 \cite{higham2002computing}. 이러한 개선으로 수치 안정성을 개선하고자 함.

\subsubsection{DDFM의 비선형 패턴 포착 능력}

DDFM이 투자 및 생산에서 상대적으로 낮은 오차를 보이는 반면 소비에서는 그렇지 않은 이유는 시계열의 비선형성 정도와 관련이 있을 수 있음:

\begin{itemize}
    \item \textbf{투자 시계열:} 설비투자 지수는 경기 순환, 정책 변화, 기업 심리 등 복잡한 비선형 요인에 의해 영향을 받을 수 있음. 이러한 복잡한 패턴은 딥러닝 인코더가 학습할 수 있으며, DDFM이 ARIMA보다 낮은 오차를 보이는 것으로 관찰됨.
    \item \textbf{생산 시계열:} 전산업생산지수는 산업별, 부문별로 다른 동태를 보이며, 이러한 이질성이 비선형 패턴을 생성할 수 있음. DDFM의 비선형 인코더는 이러한 복잡한 관계를 학습할 수 있어 ARIMA보다 낮은 오차를 보이는 것으로 관찰됨.
    \item \textbf{소비 시계열:} 도소매판매액은 상대적으로 안정적이고 선형적인 패턴을 보임. 소비는 소득, 가격 등 비교적 단순한 요인에 의해 결정될 수 있으므로, 전통적인 ARIMA 모형이 더 적합할 수 있음.
\end{itemize}

이러한 결과는 모형 선택 시 시계열의 특성을 고려할 수 있음을 시사함. 복잡한 비선형 패턴을 가진 시계열에는 DDFM이 적합할 수 있고, 상대적으로 단순한 선형 패턴을 가진 시계열에는 ARIMA가 적합할 수 있음.

\subsection{Nowcasting 시점별 분석}

Nowcasting 실험은 모형(ARIMA, VAR, DFM, DDFM)과 대상 변수(3개)에 대해 수행되었으며, 각 목표 월(2024-01 ~ 2024-12)에 대해 4주 전 시점과 1주 전 시점에서 예측을 수행함. 이론적으로, 시간이 지날수록 더 많은 데이터를 사용할 수 있어 예측 정확도가 향상될 수 있음 \cite{banbura2012nowcasting}. DFM과 DDFM 모형은 많은 시계열을 활용하여 요인을 추출하기 때문에, 더 많은 데이터가 제공될 때 성능 향상이 나타날 수 있음. Release date 기반 마스킹은 DFM과 DDFM에서 더 정확하게 반영될 수 있으며, ARIMA와 VAR에서는 근사화된 구현으로 인해 제한이 있을 수 있음.

