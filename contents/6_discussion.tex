\section{논의}

\subsection{모형 성능 해석}

본 연구에서는 4개 모형(ARIMA, VAR, DFM, DDFM)을 비교 분석함. GDP 목표 변수에 대한 실험 결과를 바탕으로 다음과 같은 패턴을 관찰할 수 있음:

\subsubsection{전통적 통계 모형의 한계}
\begin{itemize}
    \item 전통적인 통계 모형인 ARIMA, VAR은 선형 관계를 가정하기 때문에 복잡한 비선형 패턴을 포착하는 데 한계가 있음
    \item 특히 COVID-19 이후와 같은 구조적 변화가 발생한 시기에는 예측 성능이 크게 저하됨
    \item ARIMA는 단변량 모형으로서 다른 변수들의 정보를 활용하지 못하는 한계가 있으며, VAR은 다변량 모형이지만 선형 가정으로 인해 복잡한 비선형 관계를 모델링하기 어려운 것으로 평가됨
    \item 또한 이러한 모형들은 정상성(stationarity) 가정을 요구하므로, 비정상 시계열에 대해서는 차분(differencing) 등의 전처리가 필요한 상황임
    \item 혼합 빈도 데이터를 처리하는 데 한계가 있어, 고빈도 데이터를 효과적으로 활용하기 어려움
\end{itemize}

\subsubsection{동적 요인 모형의 우수성}
\begin{itemize}
    \item 동적 요인 모형인 DFM과 DDFM은 혼합 빈도 데이터를 효과적으로 처리할 수 있으며, 특히 nowcasting 상황에서 강점을 보임
    \item DFM은 많은 시계열에서 공통 요인을 추출하여 차원을 축소하고, Kalman 필터를 통해 요인과 관측치를 동시에 추정할 수 있음
    \item Stock과 Watson (2002)은 주성분 분석을 활용한 요인 추출 방법이 고차원 시계열 데이터의 차원 축소에 효과적임을 보여주었으며, 이를 통해 거시경제 변수 예측의 정확도를 향상시킬 수 있음을 제시함 \cite{stock2002forecasting}
    \item 특히 대규모 예측 변수 집합에서 주성분을 추출하여 요인을 구성하는 방법은 계산 효율성과 예측 성능의 균형을 제공함
    \item DDFM은 비선형 요인 구조를 학습할 수 있는 잠재력을 보유하나, 현재 실험에서는 빠른 테스트 파라미터(epochs=1)로 인해 DFM보다 낮은 성능을 보였음. 향후 충분한 학습을 통해 개선된 성능을 기대할 수 있음
    \item 특히 자기인코더를 활용하여 잠재 상태를 생성하고 복잡한 비선형 관계를 포착할 수 있도록 설계되었으나, 현재 실험에서는 과소 학습으로 인해 기대한 성능을 발휘하지 못했음. 충분한 학습 에폭과 반복 횟수를 사용하면 더 정확한 예측을 제공할 수 있을 것으로 기대됨
    \item 동적 요인 모형의 가장 큰 장점은 혼합 빈도 데이터를 자연스럽게 처리할 수 있다는 점임
    \item 분기별 목표 변수를 월간 또는 주간 고빈도 지표로부터 예측할 수 있으며, 이는 nowcasting에 매우 유용함
\end{itemize}

\subsection{경제적 의미}

본 연구의 프레임워크는 정책 결정자와 실무진에게 다음과 같은 시사점을 제공할 것으로 기대됨:

\subsubsection{nowcasting의 실용성}
\begin{itemize}
    \item DFM과 DDFM을 활용한 nowcasting은 공식 통계 발표 전에 현재 분기의 경제 상황을 추정할 수 있어, 신속한 정책 대응이 가능함
    \item 예를 들어, 분기별 GDP가 공식 발표되기 전에 월간 생산지수, 소매판매액, 수출입액 등의 고빈도 지표를 활용하여 현재 분기의 GDP를 추정할 수 있음
    \item 이는 특히 경제 위기 상황에서 중요함
    \item COVID-19 팬데믹과 같이 경제 상황이 급격히 변화하는 시기에는 공식 통계 발표까지의 시차로 인해 정책 대응이 지연될 수 있음
    \item Schorfheide와 Song (2020)의 연구에서도 팬데믹 기간 동안 nowcasting의 중요성이 강조되었으며, 혼합 빈도 모형을 활용한 실시간 경제 모니터링이 정책 대응의 시의성을 향상시킬 수 있음을 보여줌 \cite{schorfheide2020nowcasting}
    \item nowcasting을 통해 정책 결정자들은 실시간에 가까운 경제 상황을 파악하고, 필요시 선제적인 정책 조치를 취할 수 있음
\end{itemize}

\subsubsection{고빈도 데이터의 활용}
GDP 목표 변수에 대한 실험 결과를 보면, 혼합 빈도 데이터를 효과적으로 처리할 수 있는 DFM이 7일 예측에서 매우 우수한 성능을 보였음 (sRMSE=0.0419). 이는 고빈도 데이터를 활용한 요인 모형이 중기 예측에서 효과적임을 시사함. VAR 모형도 고빈도 데이터를 활용할 수 있으나, 1일 예측에서는 우수한 성능을 보였으나 7일 예측에서는 DFM보다 낮은 성능을 보였음.

\subsubsection{모형 선택의 중요성}
\begin{itemize}
    \item 목표 변수와 예측 기간에 따라 최적의 모형이 다를 것으로 예상되며, 상황에 맞는 모형 선택이 중요함
    \item 예를 들어, 단기 예측에서는 고빈도 데이터를 효과적으로 활용할 수 있는 모형이 유리할 것으로 기대되며, 중기 예측에서는 장기 의존성을 학습할 수 있는 모형이 유리할 것으로 기대됨
    \item 또한 목표 변수의 특성에 따라 최적 모형이 달라질 것으로 예상됨
    \item GDP 목표 변수에 대한 실험 결과를 보면, VAR이 1일 예측에서 우수한 성능을 보였으나, DFM이 7일 예측에서 매우 우수한 성능을 보였음. 이는 예측 기간에 따라 최적 모형이 달라질 수 있음을 시사함. 총고정자본형성과 같이 변동성이 큰 변수에 대해서는 비선형 관계를 학습할 수 있는 모형이 필요할 것으로 예상되나, 현재 실험은 진행 중임.
\end{itemize}

\subsubsection{정책적 함의}
본 연구의 프레임워크는 다음과 같은 정책적 함의를 가질 것으로 기대됨:

\begin{itemize}
    \item \textbf{실시간 경제 모니터링}: nowcasting을 통해 실시간에 가까운 경제 상황을 파악할 수 있으므로, 정책 결정의 시의성을 향상시킬 수 있음.
    \item \textbf{데이터 기반 의사결정}: 다양한 예측 모형의 결과를 종합적으로 고려하여 더 정확한 경제 전망을 수립할 수 있음. 실험 결과를 보면 예측 기간에 따라 최적 모형이 달라지므로, 단기 예측에는 VAR을, 중기 예측에는 DFM을 활용하는 등 상황에 맞는 모형 선택이 중요함
    \item \textbf{고빈도 데이터 수집의 중요성}: 고빈도 데이터가 예측 정확도 향상에 기여함을 확인함. DFM이 7일 예측에서 매우 우수한 성능을 보인 것은 혼합 빈도 데이터 처리 능력 덕분이며, 관련 데이터 수집 및 관리 시스템의 구축이 중요함
\end{itemize}

\subsection{연구의 한계점}

본 연구는 다음과 같은 한계점을 가짐:

\begin{itemize}
    \item \textbf{데이터 품질}: 많은 변수들이 결측치를 포함하고 있어, 전처리 과정에서 정보 손실이 발생할 수 있는 상황임.
    \item \textbf{모형 해석가능성}: 딥러닝 기반 모형(DDFM)은 예측 성능이 우수할 것으로 기대되나, 예측 결과의 경제적 해석이 어려운 것으로 알려져 있음.
    \item \textbf{외생 충격 고려}: 본 연구는 과거 데이터에 기반한 예측만을 다루며, 예상치 못한 외생 충격(예: 자연재해, 지정학적 사건)을 고려하지 않음.
    \item \textbf{한국 데이터에 국한}: 본 연구는 한국 데이터에만 적용되었으며, 다른 국가나 지역에 대한 일반화 가능성은 추가 검증이 필요한 상황임.
\end{itemize}

\subsection{향후 연구 방향}

본 연구의 실험 결과를 바탕으로 다음과 같은 향후 연구 방향을 제안함:
\begin{itemize}
    \item \textbf{충분한 학습 파라미터 사용}: 현재 실험은 빠른 테스트를 위해 낮은 학습 파라미터를 사용하였으나, 향후 충분한 학습을 통해 모형 성능을 정확히 평가할 필요가 있음
    \item \textbf{민간 소비 및 투자 변수 실험 완료}: 현재 GDP에 대한 실험만 완료되었으며, 민간 소비 및 총고정자본형성에 대한 실험 완료 후 전체적인 비교 분석이 필요함
    \item \textbf{28일 예측 기간 평가}: 테스트 세트 크기 부족으로 28일 예측 기간에 대한 평가가 불가능하였으나, 시계열 교차 검증 등을 통해 향후 평가가 필요함
    \item \textbf{Nowcasting 전용 실험}: 현재 nowcasting 전용 실험은 아직 완료되지 않았으나, 마스킹된 데이터를 활용한 백테스팅을 통해 실제 nowcasting 성능을 평가할 필요가 있음
\end{itemize}



