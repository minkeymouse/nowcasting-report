\subsubsection{Forecasting}

본 절에서는 세 가지 대상 변수(생산: KOIPALL.G, 투자: KOEQUIPTE, 소비: KOWRCCNSE)에 대한 네 가지 예측 모형(ARIMA, VAR, DFM, DDFM)의 예측 성능을 비교함. 실험은 1일부터 30일까지의 예측 수평선에 대해 수행되었으나, 표에는 가독성을 위해 1일, 7일, 30일 값만 제시함.

표~\ref{tab:forecasting_results}는 모형-수평선 조합별(12개 행: ARIMA-1, ARIMA-7, ARIMA-30, VAR-1, VAR-7, VAR-30, DFM-1, DFM-7, DFM-30, DDFM-1, DDFM-7, DDFM-30)로 각 대상 변수에 대한 표준화된 MAE와 MSE를 제시함. 각 셀은 해당 모형-수평선-대상 조합에 대한 지표값을 나타냄.

\begin{table}[h]
\centering
\caption[Forecasting Results by Model-Horizon and Target-Metric]{Forecasting Results by Model-Horizon and Target-Metric\footnote{Experiments evaluate all horizons from 1 to 22 months (2024--01 to 2025--10), but table shows only selected horizons (1, 11, 22 months) for readability. Full results for all horizons are available in aggregated\_results.csv.}}
\\label{tab:forecasting_results}
\\begin{tabular}{lcccccc}
\\toprule
Model-Horizon & KOIPALL.G & KOIPALL.G & KOEQUIPTE & KOEQUIPTE & KOWRCCNSE & KOWRCCNSE \\\\
 & sMAE & sMSE & sMAE & sMSE & sMAE & sMSE \\\\
\\midrule
ARIMA-1 & N/A & N/A & 0.8734 & 0.7628 & N/A & N/A \\
ARIMA-11 & N/A & N/A & 2.0917 & 4.3751 & N/A & N/A \\
ARIMA-22 & N/A & N/A & 0.0846 & 0.0071 & N/A & N/A \\
VAR-1 & N/A & N/A & 0.2998 & 0.0899 & N/A & N/A \\
VAR-11 & N/A & N/A & 2.5679 & 6.5939 & N/A & N/A \\
VAR-22 & N/A & N/A & 0.1881 & 0.0354 & N/A & N/A \\
DFM-1 & N/A & N/A & 0.4890 & 0.2391 & N/A & N/A \\
DFM-11 & N/A & N/A & 2.2917 & 5.2519 & N/A & N/A \\
DFM-22 & N/A & N/A & 0.1139 & 0.0130 & N/A & N/A \\
DDFM-1 & N/A & N/A & 0.7574 & 0.5736 & N/A & N/A \\
DDFM-11 & N/A & N/A & 2.0233 & 4.0938 & N/A & N/A \\
DDFM-22 & N/A & N/A & 0.1545 & 0.0239 & N/A & N/A \\
\bottomrule
\end{tabular}
\end{table}

\begin{figure}[h]
\centering
\includegraphics[width=0.9\textwidth]{images/forecast_vs_actual_koipall_g.png}
\caption{예측 대 실제: 전산업생산지수 (KOIPALL.G). 30개월의 역사적 데이터와 ARIMA, VAR, DFM, DDFM 모형의 30개월 예측을 보여줌.}
\label{fig:forecast_vs_actual_koipallg}
\end{figure}

\begin{figure}[h]
\centering
\includegraphics[width=0.9\textwidth]{images/forecast_vs_actual_koequipte.png}
\caption{예측 대 실제: 설비투자지수 (KOEQUIPTE). 30개월의 역사적 데이터와 ARIMA, VAR, DFM, DDFM 모형의 30개월 예측을 보여줌.}
\label{fig:forecast_vs_actual_koequipte}
\end{figure}

\begin{figure}[h]
\centering
\includegraphics[width=0.9\textwidth]{images/forecast_vs_actual_kowrccnse.png}
\caption{예측 대 실제: 도소매판매액 (KOWRCCNSE). 30개월의 역사적 데이터와 ARIMA, VAR, DFM, DDFM 모형의 30개월 예측을 보여줌.}
\label{fig:forecast_vs_actual_kowrccnse}
\end{figure}

그림~\ref{fig:forecast_vs_actual_koipallg}, 그림~\ref{fig:forecast_vs_actual_koequipte}, 그림~\ref{fig:forecast_vs_actual_kowrccnse}는 각 대상 변수별로 30개월의 예측 및 실제 값을 비교한 플롯임. 각 플롯은 원본 시계열, ARIMA, VAR, DFM, DDFM 예측선을 포함하며(총 5개 선), X축은 월별 타임스탬프, Y축은 대상 변수 값임. X축 총 60개월로 구성되며, 왼쪽 30개월은 원본 시계열만 표시되고, 오른쪽 30개월은 실제값과 4개 모형의 예측값이 함께 표시됨.

