\section{고빈도 변수 접목을 위한 시도}
\label{sec:high_frequency}

\subsection{고빈도 데이터의 활용 전략}
\label{sec:hf_strategy}

\subsubsection{고빈도 데이터의 정의 및 특성}
고빈도 데이터는 저빈도 목표 변수(예: 분기별 GDP)보다 더 자주 관측되는 데이터를 의미한다. 본 연구에서는 월간 및 주간 지표를 고빈도 데이터로 분류한다. 고빈도 데이터의 주요 특성은 다음과 같다:

\begin{itemize}
    \item \textbf{시의성}: 고빈도 데이터는 저빈도 목표 변수보다 더 빠르게 발표되므로, 실시간 경제 상황을 파악하는 데 유용하다.
    \item \textbf{정보 함량}: 고빈도 데이터는 목표 변수의 단기 변동을 포착할 수 있는 정보를 포함하고 있다.
    \item \textbf{출시 시차}: 각 고빈도 지표는 서로 다른 출시 시차(release lag)를 가지며, 이를 고려한 모델링이 중요하다.
\end{itemize}

\subsubsection{설명 변수의 분류 및 활용}
본 연구에서 활용하는 고빈도 설명 변수들은 목표 변수와의 관련성에 따라 다음과 같이 분류된다:

\paragraph{생산 관련 지표}
생산 관련 지표는 GDP 예측에 중요한 역할을 한다. 주요 지표는 다음과 같다:
\begin{itemize}
    \item 전산업 생산지수 (KOIPALL.G): 전체 산업의 생산 활동을 종합적으로 나타내는 지표로, GDP의 산업 생산 부분을 직접적으로 반영한다.
    \item 제조업 생산지수 (KOIPMIMAG): 제조업의 생산 활동을 나타내며, 한국 경제에서 제조업이 차지하는 비중이 크므로 GDP 예측에 중요한 선행 지표이다.
    \item 서비스업 생산지수 (KOSIAI..E): 서비스업의 생산 활동을 나타내며, 최근 서비스업이 GDP에서 차지하는 비중이 증가함에 따라 중요성이 높아지고 있다.
    \item 건설 생산지수 (KOMPRI30G): 건설업의 생산 활동을 나타내며, 투자와 밀접한 관련이 있다.
\end{itemize}

이러한 지표들은 경제 활동의 실시간 변화를 반영하므로, 분기별 GDP를 예측하는 데 유용한 정보를 제공한다. 특히 생산지수는 GDP의 산업 생산 부분을 직접적으로 반영하므로, GDP 예측의 핵심 선행 지표로 활용된다.

\paragraph{소비 관련 지표}
소비 관련 지표는 민간 소비 예측에 중요한 역할을 한다. 주요 지표는 다음과 같다:
\begin{itemize}
    \item 소비자 심리지수 (KOCNFCONR, KOCSECDMR 등)
    \item 소매판매액 (KOWRCCNSE)
    \item 신용카드 거래액 (KOCREU..A)
    \item 사이버 쇼핑몰 거래액 (KOCSMTTOA)
\end{itemize}

소비자 심리지수는 소비자의 미래 소비 의도를 반영하므로, 민간 소비 예측에 선행 지표로 활용될 수 있다.

\paragraph{투자 관련 지표}
투자 관련 지표는 총고정자본형성 예측에 중요한 역할을 한다. 주요 지표는 다음과 같다:
\begin{itemize}
    \item 설비투자 지수 (KOEQUIPTE, KOEQUIMAE 등)
    \item 건설 착공 면적 (KOCONSTRP)
    \item 건축 허가 면적 (KOCONPMTH)
    \item 건설 완료 금액 (KOVALCONA)
\end{itemize}

이러한 지표들은 기업의 투자 의도를 반영하므로, 총고정자본형성 예측에 유용한 정보를 제공한다.

\paragraph{금융 지표}
금융 지표는 전체 경제 활동에 영향을 미치는 중요한 변수이다. 주요 지표는 다음과 같다:
\begin{itemize}
    \item 기준금리 (KOCALL.)
    \item 대출금리 (KOLRLTCO, KOLRLTHH)
    \item 주가지수 (KOSHRPRCF)
    \item 기업채 스프레드 (KO3YEARC)
\end{itemize}

금융 지표는 경제 활동의 선행 지표로 활용될 수 있으며, 특히 투자 결정에 영향을 미친다.

\paragraph{설문 지표}
설문 지표는 경제 주체들의 기대와 심리를 반영한다. 주요 지표는 다음과 같다:
\begin{itemize}
    \item 기업경기실사지수(BSI): 기업의 경기 전망
    \item 소비자동향지수(CSI): 소비자의 경제 전망
\end{itemize}

설문 지표는 객관적 지표보다 빠르게 경제 상황의 변화를 반영할 수 있어, 예측에 유용한 정보를 제공한다.

\subsection{출시 시차를 고려한 모델링}
\label{sec:release_lag}

\subsubsection{출시 시차의 개념}
출시 시차(release lag)는 데이터가 실제로 발생한 시점과 공식적으로 발표되는 시점 사이의 시간 차이를 의미한다. 예를 들어, 분기별 GDP는 해당 분기가 종료된 후 약 25일이 지나야 공식 발표된다. 반면, 월간 생산지수는 해당 월이 종료된 후 약 30일이 지나면 발표된다.

출시 시차를 고려한 모델링은 나우캐스팅의 핵심이다. 각 시점에서 사용 가능한 데이터만을 활용하여 목표 변수를 예측해야 하며, 아직 발표되지 않은 데이터는 사용할 수 없다.

\subsubsection{데이터 가용성 행렬}
데이터 가용성 행렬(availability matrix)은 각 시점에서 각 변수가 사용 가능한지 여부를 나타낸다. 본 연구에서는 메타데이터의 release 필드를 활용하여 데이터 가용성 행렬을 구성한다.

예를 들어, release 값이 -5인 변수는 해당 월이 종료된 후 5일이 지나면 발표되며, release 값이 25인 변수는 해당 분기가 종료된 후 25일이 지나면 발표된다. 이러한 정보를 활용하여 각 시점에서 사용 가능한 데이터만을 선택하여 모델링한다.

\subsubsection{점진적 정보 업데이트}
나우캐스팅 과정에서 새로운 데이터가 발표될 때마다 예측을 업데이트할 수 있다. 이를 점진적 정보 업데이트(incremental information update)라고 한다. DFM과 DDFM은 Kalman 필터를 통해 새로운 정보를 효율적으로 통합할 수 있어, 점진적 정보 업데이트에 적합하다.

\subsection{혼합 빈도 모델링의 효과}
\label{sec:mixed_frequency_effect}

\subsubsection{고빈도 데이터의 정보 함량}
고빈도 데이터는 저빈도 목표 변수를 예측하는 데 유용한 정보를 제공한다. 예를 들어, 분기별 GDP는 해당 분기의 3개월간의 경제 활동을 집계한 것이므로, 월간 생산지수, 소매판매액 등의 고빈도 지표들이 분기별 GDP의 구성 요소를 미리 반영할 수 있다.

본 연구에서는 고빈도 데이터를 활용한 예측 성능과 고빈도 데이터를 사용하지 않은 예측 성능을 비교하여, 고빈도 데이터의 정보 함량을 정량적으로 평가한다.

\subsubsection{텐트 커널 집계의 효과}
DFM에서 사용하는 텐트 커널(tent kernel) 집계 방식은 저빈도 시계열을 고빈도 요인으로부터 생성하는 핵심 메커니즘이다. 텐트 커널은 시간 가중치를 부여하여, 분기 내 각 월의 기여도를 다르게 설정한다.

텐트 커널의 수학적 정의는 다음과 같다. 분기 $q$에 해당하는 월간 시점들의 집합을 $S_q = \{m_1, m_2, m_3\}$라고 하면, 각 월의 가중치는 다음과 같이 계산된다:

\begin{equation}
w_{m_i} = \begin{cases}
\frac{i - 1}{2} & \text{if } i = 1, 2 \\
1 - \frac{i - 2}{2} & \text{if } i = 2, 3
\end{cases}
\end{equation}

이러한 가중치는 분기의 첫 번째 월과 세 번째 월에는 작은 가중치를, 두 번째 월(중간 월)에는 큰 가중치를 부여한다. 이는 분기의 중간 시점이 전체 분기 값을 더 잘 대표한다는 직관에 기반한다.

텐트 커널의 효과를 평가하기 위해, 다양한 집계 방식을 비교할 수 있다:
\begin{itemize}
    \item \textbf{균등 가중치}: 각 월에 동일한 가중치를 부여 ($w_{m_i} = 1/3$)
    \item \textbf{최근 가중치}: 최근 월에 더 큰 가중치를 부여
    \item \textbf{텐트 커널}: 분기 중간에 더 큰 가중치를 부여
\end{itemize}

실증 연구에 따르면, 텐트 커널이 균등 가중치나 최근 가중치보다 우수한 성능을 보이는 것으로 나타났다. 이는 분기의 중간 시점이 전체 분기 값을 더 잘 대표하며, 텐트 커널이 이러한 특성을 효과적으로 반영하기 때문이다.

\subsection{모형 비교 및 성능 분석}
\label{sec:model_comparison}

\subsubsection{전체 모형 성능 비교}
본 연구에서는 9개의 예측 모형(ARIMA, VAR, VECM, DeepAR, TFT, XGBoost, LightGBM, DFM, DDFM)을 비교 분석한다. 각 모형의 고빈도 데이터 활용 능력과 예측 성능을 평가한다.

[아직 실험 미진행]

\subsubsection{고빈도 데이터 활용 모형의 우수성}
고빈도 데이터를 효과적으로 활용할 수 있는 모형(DFM, DDFM)과 그렇지 않은 모형(ARIMA, VAR 등)의 성능을 비교하여, 고빈도 데이터 활용의 효과를 검증한다.

[아직 실험 미진행]

\subsubsection{목표 변수별 성능 분석}
각 목표 변수(GDP, 소비, 투자)에 대해 모형 성능을 분석하여, 목표 변수의 특성에 따라 최적 모형이 달라지는지 확인한다.

[아직 실험 미진행]

