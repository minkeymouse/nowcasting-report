\section{데이터 및 방법론}

가. 데이터

데이터셋 개요
본 연구는 한국은행 경제통계시스템(ECOS)에서 수집한 한국 거시경제 시계열 데이터를 활용함. 데이터 기간은 1985년 4월부터 2025년 11월까지이며, 총 2,538개의 관측치와 101개의 시계열 변수로 구성되어 있음. 

데이터는 혼합 빈도(mixed-frequency) 구조를 가지고 있으며, 월간 변수 87개, 분기별 변수 8개, 주간 변수 6개로 구성되어 있음. 변수들은 다음과 같은 카테고리로 분류됨: 생산(Production) 20개, 기업 설문(Survey, Bsnss) 16개, 금융(Finance) 11개, 소비자 설문(Survey, Cnsmr) 10개, 무역(Int. trade) 8개, 거시경제(Macro) 8개, 노동(Labor) 7개, 투자(Investment) 7개, 소비(Consumption) 6개, 물가(Price) 5개 등임.

목표 변수
본 연구의 예측 대상은 다음과 같은 3개의 분기별 거시경제 변수임:

\begin{itemize}
    \item \textbf{KOGDP\_\_\_D}: 국내총생산(GDP), 실질 기준, 사슬 연결 가중치(Chained W, Billions). 총 162개의 관측치가 있으며, 평균 변화율은 5.16\%, 표준편차는 5.80\%임.
    \item \textbf{KOCNPER\_D}: 민간 소비(Consumption, Private), 실질 기준, 사슬 연결 가중치. 총 162개의 관측치가 있으며, 평균 변화율은 4.39\%, 표준편차는 7.22\%임.
    \item \textbf{KOGFCF\_\_D}: 총고정자본형성(Gross Capital Formation, Fixed), 실질 기준, 사슬 연결 가중치. 총 162개의 관측치가 있으며, 평균 변화율은 5.01\%, 표준편차는 11.86\%로 가장 큰 변동성을 보임.
\end{itemize}

이러한 목표 변수들은 분기별로 발표되며, 해당 분기 종료 후 약 25일 정도의 시차를 가지고 있음. 따라서 나우캐스팅을 통해 공식 발표 전에 현재 분기의 값을 추정하는 것이 가능함.

설명 변수
설명 변수들은 목표 변수와 관련된 다양한 경제 지표들로 구성되어 있음. 주요 변수 그룹은 다음과 같음:

\begin{itemize}
    \item \textbf{생산 지표}: 전산업 생산지수, 제조업 생산지수, 서비스업 생산지수 등
    \item \textbf{소비 관련 지표}: 소비자 심리지수, 소매판매액, 신용카드 거래액 등
    \item \textbf{투자 관련 지표}: 설비투자, 건설 착공, 기계류 투자 등
    \item \textbf{금융 지표}: 기준금리, 대출금리, 주가지수 등
    \item \textbf{무역 지표}: 수출입액, 수출입 물가 등
    \item \textbf{설문 지표}: 기업경기실사지수(BSI), 소비자동향지수(CSI) 등
\end{itemize}

대부분의 변수들은 월간 빈도를 가지며, 일부 변수는 주간 또는 분기별 빈도를 가짐. 또한 많은 변수들이 결측치를 포함하고 있어, 적절한 전처리 과정이 필요함.

나. 전처리 방법

본 연구에서는 sktime 라이브러리를 활용한 전처리 파이프라인을 구축함. 각 시계열 변수는 메타데이터에 명시된 변환 방법(chg: 변화율, cha: 사슬 연결, lin: 선형)에 따라 전처리되며, 모든 변수는 표준화를 통해 평균 0, 표준편차 1로 변환됨. 결측치는 선형 보간 또는 전방 채우기 방법을 사용하여 처리함.

다. 예측 모형

본 연구에서는 총 9개의 예측 모형을 비교 분석함. 전통적 통계 모형으로는 ARIMA, VAR, VECM을 사용하며, 머신러닝 모형으로는 XGBoost와 LightGBM을 사용함. 딥러닝 모형으로는 DeepAR과 TFT를 사용함 \cite{salinas2020deepar, lim2021temporal}.

동적 요인 모형으로는 dfm-python 패키지를 활용하여 DFM과 DDFM을 구현함. dfm-python은 혼합 빈도 데이터를 처리할 수 있는 동적 요인 모형의 Python 구현체로, PyTorch Lightning 기반의 모듈화된 API를 제공함.

dfm-python을 활용한 DFM 구현
dfm-python 패키지는 PyTorch Lightning 패턴을 따르는 표준화된 인터페이스를 제공함. DFM 모형의 사용은 다음과 같은 단계로 구성됨:

\begin{enumerate}
    \item \textbf{데이터 모듈 생성}: DFMDataModule을 사용하여 데이터를 로드하고 전처리함. 데이터는 이미 전처리된 상태여야 하며, 메타데이터(빈도, 변환 방법 등)를 포함해야 함.
    \item \textbf{모형 초기화}: DFM 클래스를 인스턴스화하고, YAML 설정 파일을 통해 모형 구조를 정의함. 설정 파일에는 요인 개수, AR 차수, 블록 구조 등이 포함됨.
    \item \textbf{학습}: DFMTrainer를 사용하여 EM 알고리즘을 통해 모형 파라미터를 추정함. 수렴 기준(threshold)과 최대 반복 횟수(max\_iter)를 설정할 수 있음.
    \item \textbf{예측}: 학습된 모형의 predict 메서드를 호출하여 미래 시점의 값을 예측함. 예측 기간(horizon)을 지정할 수 있음.
\end{enumerate}

본 연구에서는 clock 빈도를 월간('m')으로 설정하여 모든 잠재 요인이 월간 빈도에서 진화하도록 함. 분기별 목표 변수는 텐트 커널(tent kernel)을 통해 월간 요인으로부터 집계됨. 텐트 커널은 분기 내 각 월의 기여도를 시간 가중치로 부여하여, 분기의 중간 시점이 더 큰 가중치를 갖도록 함.

dfm-python을 활용한 DDFM 구현
DDFM은 DFM의 비선형 확장으로, 변분 자기인코더를 사용하여 비선형 요인 구조를 학습함 \cite{andreini2020deep}. dfm-python의 DDFM 구현은 다음과 같은 특징을 가짐:

\begin{itemize}
    \item \textbf{인코더 구조}: 다층 퍼셉트론(MLP) 기반의 인코더를 사용하여 관측 시계열을 잠재 요인 공간으로 매핑함. 인코더 레이어 수와 각 레이어의 크기를 설정할 수 있음.
    \item \textbf{학습 방법}: 배치 기반 학습을 통해 신경망 파라미터를 최적화함. 학습률, 배치 크기, 에폭 수 등을 설정할 수 있음.
    \item \textbf{요인 동학}: 학습된 요인은 DFM과 동일하게 AR(1) 과정을 따르며, Kalman 필터를 통해 추정됨.
\end{itemize}

본 연구에서는 DDFM의 인코더 구조를 [64, 32]로 설정하였으며, 요인 개수는 목표 변수에 따라 2-4개로 조정함. 학습률은 0.001로 설정하고, 배치 크기는 32로 설정함. DDFM은 DFM과 동일한 clock 프레임워크를 사용하여 혼합 빈도 데이터를 처리함.

라. 실험 설계

본 연구에서는 세 가지 예측 기간(1일, 7일, 28일)에 대해 모형 성능을 평가함. 모든 모형의 성능은 표준화된 평가 지표(sMSE, sMAE, sRMSE)를 통해 비교되며, 표준화는 훈련 데이터의 표준편차로 나누어 수행됨. 시계열 데이터의 특성을 고려하여 시간 순서를 유지하는 교차 검증을 수행하며, 각 모형의 하이퍼파라미터는 검증 데이터를 통해 최적화됨.

dfm-python을 활용한 DFM/DDFM 실험 설정
dfm-python 패키지를 사용하여 DFM과 DDFM 모형을 학습하고 예측을 수행함. 실험 설정은 다음과 같음:

\begin{itemize}
    \item \textbf{DFM 설정}: 요인 개수는 목표 변수에 따라 2-4개로 설정하고, AR 차수는 1로 설정함. EM 알고리즘의 수렴 기준은 1e-4로 설정하고, 최대 반복 횟수는 100으로 설정함. clock 빈도는 월간('m')으로 설정하여 모든 잠재 요인이 월간 빈도에서 진화하도록 함.
    \item \textbf{DDFM 설정}: 인코더 구조는 [64, 32]로 설정하고, 요인 개수는 DFM과 동일하게 2-4개로 설정함. 학습률은 0.001로 설정하고, 배치 크기는 32로 설정함. 에폭 수는 100으로 설정하고, 조기 종료(early stopping)를 사용하여 과적합을 방지함.
    \item \textbf{블록 구조}: 목표 변수와 관련된 변수들을 블록으로 그룹화함. GDP 예측을 위한 Block\_Production, 민간 소비 예측을 위한 Block\_Consumption, 총고정자본형성 예측을 위한 Block\_Investment 블록을 구성함.
\end{itemize}

dfm-python의 나우캐스팅 기능을 활용하여 마스킹된 데이터를 통한 백테스팅을 수행함. 각 시점에서 목표 변수의 최근 관측치를 마스킹하고, 사용 가능한 고빈도 데이터만을 활용하여 예측을 수행함. 또한 DFM과 DDFM의 각 하이퍼파라미터가 모형 성능에 미치는 영향을 분석하기 위해 Ablation study를 수행함.
