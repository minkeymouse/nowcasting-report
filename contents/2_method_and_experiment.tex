\section{데이터 및 방법론}

가. 데이터

데이터셋 개요
본 연구는 한국은행 경제통계시스템(ECOS)에서 수집한 한국 거시경제 시계열 데이터를 활용함. 데이터 기간은 1985년 4월부터 2025년 11월까지이며, 총 2,538개의 관측치와 101개의 시계열 변수로 구성되어 있음. 

데이터는 혼합 빈도(mixed-frequency) 구조를 가지고 있으며, 월간 변수 87개, 분기별 변수 8개, 주간 변수 6개로 구성되어 있음. 변수들은 다음과 같은 카테고리로 분류됨: 생산(Production) 20개, 기업 설문(Survey, Bsnss) 16개, 금융(Finance) 11개, 소비자 설문(Survey, Cnsmr) 10개, 무역(Int. trade) 8개, 거시경제(Macro) 8개, 노동(Labor) 7개, 투자(Investment) 7개, 소비(Consumption) 6개, 물가(Price) 5개 등임.

목표 변수
본 연구의 예측 대상은 다음과 같은 3개의 분기별 거시경제 변수임:

\begin{itemize}
    \item \textbf{KOGDP\_\_\_D}: 국내총생산(GDP), 실질 기준, 사슬 연결 가중치(Chained W, Billions). 총 162개의 관측치가 있으며, 평균 변화율은 5.16\%, 표준편차는 5.80\%임.
    \item \textbf{KOCNPER\_D}: 민간 소비(Consumption, Private), 실질 기준, 사슬 연결 가중치. 총 162개의 관측치가 있으며, 평균 변화율은 4.39\%, 표준편차는 7.22\%임.
    \item \textbf{KOGFCF\_\_D}: 총고정자본형성(Gross Capital Formation, Fixed), 실질 기준, 사슬 연결 가중치. 총 162개의 관측치가 있으며, 평균 변화율은 5.01\%, 표준편차는 11.86\%로 가장 큰 변동성을 보임.
\end{itemize}

이러한 목표 변수들은 분기별로 발표되며, 해당 분기 종료 후 약 25일 정도의 시차를 가지고 있음. 따라서 나우캐스팅을 통해 공식 발표 전에 현재 분기의 값을 추정하는 것이 가능함.

설명 변수
설명 변수들은 목표 변수와 관련된 다양한 경제 지표들로 구성되어 있음. 주요 변수 그룹은 다음과 같음:

\begin{itemize}
    \item \textbf{생산 지표}: 전산업 생산지수, 제조업 생산지수, 서비스업 생산지수 등
    \item \textbf{소비 관련 지표}: 소비자 심리지수, 소매판매액, 신용카드 거래액 등
    \item \textbf{투자 관련 지표}: 설비투자, 건설 착공, 기계류 투자 등
    \item \textbf{금융 지표}: 기준금리, 대출금리, 주가지수 등
    \item \textbf{무역 지표}: 수출입액, 수출입 물가 등
    \item \textbf{설문 지표}: 기업경기실사지수(BSI), 소비자동향지수(CSI) 등
\end{itemize}

대부분의 변수들은 월간 빈도를 가지며, 일부 변수는 주간 또는 분기별 빈도를 가짐. 또한 많은 변수들이 결측치를 포함하고 있어, 적절한 전처리 과정이 필요함.

나. 전처리 방법

본 연구에서는 sktime 라이브러리를 활용한 전처리 파이프라인을 구축함. 각 시계열 변수는 메타데이터에 명시된 변환 방법(chg: 변화율, cha: 사슬 연결, lin: 선형)에 따라 전처리되며, 모든 변수는 표준화를 통해 평균 0, 표준편차 1로 변환됨. 결측치는 선형 보간 또는 전방 채우기 방법을 사용하여 처리함.

다. 예측 모형

본 연구에서는 총 9개의 예측 모형을 비교 분석함. 전통적 통계 모형으로는 ARIMA(단변량 기준 모형), VAR(다변량 자기회귀), VECM(벡터 오차수정 모형)을 사용함. 머신러닝 모형으로는 XGBoost와 LightGBM을 사용하며, 딥러닝 모형으로는 DeepAR(LSTM 기반)과 TFT(어텐션 메커니즘 기반)을 사용함. DeepAR은 자기회귀 순환 신경망을 활용한 확률적 예측 모형으로, 시계열의 장기 의존성을 효과적으로 학습할 수 있음 \cite{salinas2020deepar}.

동적 요인 모형으로는 DFM과 DDFM을 사용함. DFM은 많은 시계열에서 공통 요인을 추출하여 차원을 축소하고, 혼합 빈도 데이터를 효과적으로 처리할 수 있는 모형임. DFM의 기본 구조는 다음과 같음:
\begin{align}
x_t &= C z_t + \varepsilon_t, \quad \varepsilon_t \sim \mathcal{N}(0, R) \label{eq:dfm_obs} \\
z_t &= A z_{t-1} + \eta_t, \quad \eta_t \sim \mathcal{N}(0, Q) \label{eq:dfm_state}
\end{align}
여기서 $x_t$는 관측 시계열, $z_t$는 공통 요인, $C$는 요인 적재 행렬, $A$는 전이 행렬임. DFM은 clock 프레임워크와 텐트 커널 집계를 통해 혼합 빈도 데이터를 처리함.

DDFM은 딥러닝 기법을 DFM에 접목한 심층 동적 요인 모형으로, 변분 자기인코더를 활용하여 비선형 요인 구조를 학습함 \cite{andreini2020deep}. DDFM은 기존 DFM의 선형 가정을 완화하여 구조적 변화 시기에 더 우수한 성능을 보이는 것으로 나타남.

라. 실험 설계

본 연구에서는 세 가지 예측 기간(1일, 7일, 28일)에 대해 모형 성능을 평가함. 모든 모형의 성능은 표준화된 평가 지표(sMSE, sMAE, sRMSE)를 통해 비교되며, 표준화는 훈련 데이터의 표준편차로 나누어 수행됨. 시계열 데이터의 특성을 고려하여 시간 순서를 유지하는 교차 검증을 수행하며, 각 모형의 하이퍼파라미터는 검증 데이터를 통해 최적화됨.

DFM과 DDFM의 나우캐스팅 성능을 평가하기 위해 마스킹된 데이터를 활용한 백테스팅을 수행함. 각 시점에서 목표 변수의 최근 관측치를 마스킹하고, 사용 가능한 고빈도 데이터만을 활용하여 예측을 수행함. 또한 DFM과 DDFM의 각 하이퍼파라미터가 모형 성능에 미치는 영향을 분석하기 위해 Ablation study를 수행함.
