\section{논의}
\label{sec:discussion}

\subsection{예측 결과 비교}

DFM과 DDFM 모형의 성능을 비교하고, ARIMA와 VAR을 벤치마크 모형으로 포함하여 네 가지 모형의 성능을 대상 변수와 예측 시점에 걸쳐 평가함.

\textbf{벤치마크 모형(ARIMA, VAR)}
\begin{itemize}
    \item ARIMA와 VAR은 전통적인 선형 모형으로 벤치마크 역할을 수행함. 일부 대상 변수에서 양호한 성능을 보이지만, nowcasting에서는 release date 마스킹 처리의 구조적 한계로 제한적임.
\end{itemize}

\textbf{동적요인모형(DFM, DDFM)}
\begin{itemize}
    \item \textbf{DFM:} 세 대상 변수 모두에서 평가 완료. KOIPALL.G에서 극단적으로 높은 오차(sMAE=14.97) - 주/월 혼합 주기 처리 과정에서 발생한 수치적 불안정성. KOEQUIPTE와 KOWRCCNSE에서는 중간 수준의 성능. Nowcasting에서 release date 마스킹을 효과적으로 처리 가능하며, 다변량 시계열 간 공통 패턴을 포착할 수 있음.
    \item \textbf{DDFM:} 세 대상 변수 모두에서 평가 완료. KOIPALL.G에서 우수한 성능(sMAE=0.6865, DFM 대비 약 21.8배 낮은 오차), KOWRCCNSE에서도 우수한 성능(sMAE=0.4961, DFM 대비 약 5.6배 낮은 오차). KOEQUIPTE에서는 DFM과 거의 동일한 성능. Nowcasting에서 release date 마스킹을 효과적으로 처리 가능하며, 변동성이 큰 시계열에서 DFM 대비 우수한 성능을 보임.
\end{itemize}

\textbf{대상 변수별 최적 모형}
\begin{itemize}
    \item KOIPALL.G와 KOWRCCNSE에서는 DDFM이 최고 성능을 보이며, KOWRCCNSE에서는 VAR도 우수한 성능을 보임.
    \item 각 모형은 대상 변수에 따라 매우 다른 성능 특성을 보이며, 단일 모형이 모든 대상 변수에서 최고 성능을 보이지는 않음.
    \item 대상 변수와 시계열 특성에 따라 적절한 모형을 선택하는 것이 중요함.
\end{itemize}

\textbf{Nowcasting 능력}
\begin{itemize}
    \item DFM과 DDFM은 요인 모형의 구조적 특성으로 인해 release date 기반 마스킹을 효과적으로 처리 가능하며, Kalman filter를 통해 실시간 데이터 흐름의 불규칙성을 자연스럽게 처리할 수 있어 실제 운영 환경에서의 nowcasting에 적합함 \cite{banbura2012nowcasting}.
    \item ARIMA와 VAR은 release date 마스킹 처리의 구조적 한계로 인해 nowcasting 실험에서 제외됨.
\end{itemize}

\subsection{선형 vs 비선형 모델}

선형 요인 모형(DFM)과 비선형 요인 모형(DDFM)의 성능 비교가 핵심임. DDFM은 심층 신경망 기반 인코더를 통한 비선형 요인 추출을 통해 DFM의 한계를 보완함.

\textbf{비선형 모델의 강점}
\begin{itemize}
    \item \textbf{변동성이 큰 시계열에서의 우수성:} DDFM은 KOIPALL.G와 KOWRCCNSE에서 DFM 대비 각각 약 21.8배, 5.6배 낮은 오차를 보이며, 비선형 관계 포착 능력으로 인해 변동성이 큰 시계열에서 우수한 성능을 보임.
    \item \textbf{DFM의 수치적 불안정성:} DFM이 KOIPALL.G에서 보인 높은 오차(sMAE=14.97)는 주/월 혼합 주기 처리 과정에서 발생한 수치적 불안정성 때문임.
    \item \textbf{시점별 안정성:} DDFM은 변동성이 큰 시계열에서 단기 및 장기 예측에서 안정적인 성능을 보임.
\end{itemize}

\textbf{선형 모델의 한계와 적합성}
\begin{itemize}
    \item \textbf{KOEQUIPTE에서의 동일한 성능:} KOEQUIPTE에서 DDFM과 DFM이 거의 동일한 성능을 보이는 것은 해당 시계열이 선형 관계가 강하거나, 기본 인코더 구조([16, 4])가 이 시계열에 최적화되지 않았을 가능성을 시사함.
    \item \textbf{비선형 인코더의 제한적 이점:} 인코더가 비선형 활성화 함수(ReLU)를 사용하더라도, 학습된 가중치가 선형 변환에 가까워질 수 있음. 이는 모든 시점(1-21개월)에서 두 모형이 거의 동일한 오차를 보이며, 두 모형이 유사한 선형 요인 구조를 학습했음을 강하게 시사함.
    \item \textbf{모형 선택의 중요성:} 선형 관계가 강한 시계열에서는 DFM이 충분할 수 있으며, 변동성이 크거나 비선형 관계가 있는 시계열에서는 DDFM이 유리함.
\end{itemize}

\subsection{추가 데이터 소스}

산업생산지수 nowcasting을 위한 고빈도 공공데이터 조사를 수행하여, 빈도(주간 이상), 발표 시차(산업생산지수보다 선행), 접근성(무료 공개) 기준으로 평가함. 주요 후보로는 한국전력거래소 전력수급현황 실시간 API, 한국은행 뉴스심리지수, 한국은행 BSI/ESI/CSI/CBSI, 국가물류통합정보센터 해상운임지수가 도출되었음. 상세 내용은 실험 설계 섹션(2.1.4)을 참조함.

