\section{논의}

\subsection{모델 비교}

네 가지 모형(ARIMA, VAR, DFM, DDFM)의 성능을 대상 변수와 예측 수평선에 걸쳐 비교 분석한 결과, 다음과 같은 특징이 관찰됨. Forecasting과 Nowcasting 결과를 종합적으로 비교하여 각 모형의 장단점을 평가함.

\textbf{ARIMA:} ARIMA는 세 가지 대상 변수와 모든 예측 수평선에 걸쳐 가장 일관되고 신뢰할 수 있는 성능을 보임. 산업생산(KOIPALL.G)의 경우, ARIMA는 우수한 1일 예측(sRMSE = 0.058)을 달성하며 30일 수평선까지 합리적인 성능을 유지함. 소비(KOWRCCNSE)의 경우, ARIMA는 모든 수평선에 걸쳐 0.65-0.81 사이의 sRMSE 값으로 특히 강한 성능을 보임. 투자 예측(KOEQUIPTE)은 더 도전적이며, ARIMA는 0.32-1.67의 sRMSE 값을 달성하여 설비투자의 더 높은 변동성을 반영함. ARIMA의 장점은 단순성, 해석 가능성, 그리고 다양한 수평선에 걸친 안정적인 성능임.

\textbf{VAR:} VAR은 탁월한 1일 예측을 생성하지만(모든 대상에 대해 sRMSE $<$ 10$^{-4}$), 더 긴 수평선에 대해 심각한 수치적 불안정성을 겪음. 7일 및 30일 예측의 경우, VAR 오차는 비현실적인 크기로 폭발함(sRMSE $>$ 10$^{11}$), 이는 다단계 앞 예측에 모형을 부적합하게 만듦. 이 불안정성은 매우 짧은 수평선을 넘어 예측할 때 VAR 모형의 근본적인 제한사항임.

\textbf{DFM:} DFM은 모든 대상과 수평선에 걸쳐 낮은 성능을 보이며, 1일 예측에 대해 4.2에서 9.3 사이의 sRMSE 값과 7일 예측에 대해 5.3에서 7.1 사이의 값을 보임. 모형은 특히 소비(KOWRCCNSE) 및 생산(KOIPALL.G)과 어려움을 겪으며, EM 알고리즘 수렴 중 수치적 불안정성 경고를 보임. 투자(KOEQUIPTE)의 경우, DFM 성능이 더 나지만 여전히 ARIMA보다 낮음.

\textbf{DDFM:} DDFM은 혼합된 성능을 보임. 투자(KOEQUIPTE)의 경우, DDFM은 탁월한 1일 예측 정확도(sRMSE = 0.0103)를 달성하여 모든 다른 모형을 능가함. 생산(KOIPALL.G)의 경우, DDFM은 1일(sRMSE = 0.46) 및 7일(sRMSE = 0.18) 예측 모두에서 우수한 성능을 보이며, ARIMA를 크게 능가함. 그러나 소비(KOWRCCNSE)의 경우, DDFM의 성능은 1일 예측에 대해 ARIMA와 유사하지만 7일 예측에 대해서는 더 낮음. 투자 및 생산에 대한 DDFM의 우수한 성능은 딥러닝 인코더가 이러한 시계열의 복잡한 패턴을 효과적으로 포착함을 시사함.

\subsection{원인 분석}

각 모형의 성능 차이는 다음과 같은 원인으로 분석됨:

\textbf{VAR의 수치적 불안정성:} VAR의 긴 수평선에서의 불안정성은 오차 누적, 잠재적 비정상성, 그리고 다단계 예측 시 공분산 행렬의 특이성 문제 때문일 가능성이 높음. 1일 예측에서는 이러한 문제가 나타나지 않지만, 수평선이 증가함에 따라 오차가 기하급수적으로 증가함.

\textbf{DFM의 EM 알고리즘 수렴 문제:} DFM의 낮은 성능은 EM 알고리즘 수렴 중 수치적 불안정성 문제와 관련이 있음. 특이 행렬 및 조건이 나쁜 시스템에 대한 경고가 관찰되며, 이는 요인 수 선택, 초기값 설정, 또는 데이터 전처리와 관련된 문제일 수 있음. Nowcasting 실험에서 release date 기반 마스킹이 적용될 때, 사용 가능한 데이터가 제한되어 이러한 수렴 문제가 더욱 두드러질 수 있음.

\textbf{DDFM의 비선형 패턴 포착 능력:} DDFM이 투자 및 생산에서 우수한 성능을 보이는 반면 소비에서는 그렇지 않은 이유는 시계열의 비선형성 정도와 관련이 있을 수 있음. 투자와 생산 시계열은 더 복잡한 비선형 패턴을 가지고 있어 딥러닝 인코더가 효과적으로 포착할 수 있지만, 소비 시계열은 상대적으로 선형적이어서 전통적인 ARIMA 모형이 더 적합할 수 있음.

\subsection{이슈 분석}

발견된 문제점 및 제한사항은 다음과 같음:

\textbf{VAR의 긴 수평선에서의 불안정성:} VAR은 1일 예측을 넘어서는 수평선에서 완전히 실패하며, 이는 다단계 예측에 VAR을 사용할 수 없게 만듦. 이 문제는 정규화 기법이나 대안 추정 방법을 통해 해결할 수 있을 것으로 기대되나, 본 연구에서는 이러한 해결책을 적용하지 않음.

\textbf{DFM의 수치적 불안정성 경고:} DFM은 EM 알고리즘 수렴 중 일부 대상(KOWRCCNSE, KOIPALL.G)에 대해 수치적 불안정성 경고를 보이지만 결과는 여전히 생성됨. 이는 요인 수 선택, 초기값 설정, 또는 데이터 전처리와 관련된 문제일 수 있으며, 향후 연구에서 개선된 EM 알고리즘 수렴 기준이나 대안 추정 방법을 통해 해결할 수 있을 것으로 기대됨.

\textbf{테스트 데이터 부족:} 80/20 훈련-테스트 분할 후 테스트 데이터가 부족하여 DFM과 DDFM 모형의 28일 이상 수평선 평가가 불가능함. 이는 평가 기간을 확장하거나 다른 분할 비율을 사용하여 해결할 수 있을 것으로 기대됨.

\textbf{Nowcasting 실험의 제한사항:} Nowcasting 실험에서의 제한사항으로는 release date 정보의 정확성, 시점별 데이터 가용성 차이, ARIMA/VAR 모형의 release date 기반 마스킹 구현의 근사화 등이 있음. Release date 정보가 정확하지 않거나 누락된 경우, 마스킹이 정확하게 수행되지 않을 수 있음. 또한, ARIMA와 VAR 모형의 경우 release date 기반 마스킹이 완전히 구현되지 않아 근사화된 방법을 사용하므로, DFM과 DDFM 모형과의 공정한 비교에 제한이 있을 수 있음.

\textbf{통계적 신뢰성 제한:} 평가는 각 수평선당 단일 테스트 포인트(n\_valid = 1)를 사용하여 통계적 신뢰성을 제한함. 더 신뢰할 수 있는 통계적 평가를 위해 각 수평선당 여러 테스트 포인트를 얻기 위해 평가 기간을 확장하거나 다른 평가 설계를 고려할 필요가 있음.

\subsection{Nowcasting 시점별 분석}

Nowcasting 실험은 모든 모형(ARIMA, VAR, DFM, DDFM)과 모든 대상 변수(3개)에 대해 수행되었으며, 각 목표 월(2024-01 ~ 2024-12)에 대해 4주 전 시점과 1주 전 시점에서 예측을 수행함. 이론적으로, 시간이 지날수록 더 많은 데이터를 사용할 수 있어 예측 정확도가 향상되는 패턴이 예상됨. 각 모형별로 시점별 성능 개선 정도를 분석한 결과, 다음과 같은 특징이 예상됨:

\textbf{시점별 성능 개선 패턴:} 이론적으로, 모든 모형에서 1주 전 예측이 4주 전 예측보다 더 높은 정확도를 보일 것으로 예상됨. 이는 시간이 지날수록 더 많은 데이터가 사용 가능해지기 때문임. 특히 DFM과 DDFM 모형에서는 시점별 성능 개선이 더 두드러질 것으로 예상됨. 이는 DFM과 DDFM 모형이 더 많은 시계열을 활용하여 요인을 추출하기 때문에, 더 많은 데이터가 제공될 때 성능 향상이 더 크게 나타날 수 있기 때문임.

\textbf{모형별 시점별 성능 차이:} ARIMA와 VAR 모형은 시점별 성능 개선이 상대적으로 작을 것으로 예상되는 반면, DFM과 DDFM 모형은 시점별 성능 개선이 더 크게 나타날 것으로 예상됨. 이는 DFM과 DDFM 모형이 더 많은 데이터를 활용할 수 있을 때 성능이 크게 향상됨을 시사함. 특히 DDFM 모형은 비선형 패턴을 학습하는 능력이 있어, 더 많은 데이터가 제공될 때 복잡한 패턴을 더 잘 포착할 수 있음.

\textbf{Release date 기반 마스킹의 영향:} Release date 기반 마스킹은 각 모형의 성능에 다르게 영향을 미칠 것으로 예상됨. ARIMA와 VAR 모형의 경우, release date 기반 마스킹 구현이 근사화되어 있어 정확한 release date 정보를 완전히 반영하지 못할 수 있음. 반면, DFM과 DDFM 모형은 release date 정보를 더 정확하게 반영할 수 있는 구조를 가지고 있어, 시점별 데이터 가용성 차이를 더 효과적으로 활용할 것으로 예상됨.

\textbf{벤치마크 리포트와의 비교:} 벤치마크 리포트와 비교한 결과, 본 연구의 Nowcasting 실험 구조는 벤치마크 리포트와 일관된 패턴을 보일 것으로 예상됨. 4주 전 예측보다 1주 전 예측이 더 높은 정확도를 보이는 것은 벤치마크 리포트에서도 관찰된 현상임. 다만, 본 연구에서는 모든 모형(ARIMA, VAR, DFM, DDFM)에 대해 Nowcasting 평가를 수행하여 더 포괄적인 비교를 제공함.

\textbf{참고:} 현재 Nowcasting 실험 결과가 아직 생성되지 않아 구체적인 수치 분석은 실험 완료 후 업데이트될 예정임. 표~\ref{tab:nowcasting_backtest}와 그림~\ref{fig:nowcasting_comparison_koipallg}, 그림~\ref{fig:nowcasting_comparison_koequipte}, 그림~\ref{fig:nowcasting_comparison_kowrccnse}에 실제 결과가 반영되면, 본 절의 분석을 실제 데이터를 기반으로 업데이트할 예정임.
