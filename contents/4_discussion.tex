\section{논의}

가. 모형 성능 해석

본 연구에서 비교한 9개 모형의 성능을 종합적으로 분석한 결과, 다음과 같은 패턴을 관찰할 수 있음:

전통적 통계 모형의 한계
전통적인 통계 모형인 ARIMA, VAR, VECM은 선형 관계를 가정하기 때문에 복잡한 비선형 패턴을 포착하는 데 한계가 있는 것으로 나타남. 특히 COVID-19 이후와 같은 구조적 변화가 발생한 시기에는 예측 성능이 크게 저하되는 것으로 나타남. 

ARIMA는 단변량 모형으로서 다른 변수들의 정보를 활용하지 못하는 한계가 있으며, VAR과 VECM은 다변량 모형이지만 선형 가정으로 인해 복잡한 비선형 관계를 모델링하기 어려운 것으로 평가됨. 또한 이러한 모형들은 정상성(stationarity) 가정을 요구하므로, 비정상 시계열에 대해서는 차분(differencing) 등의 전처리가 필요한 상황임.

머신러닝 모형의 특성
머신러닝 기반 모형인 XGBoost와 LightGBM은 비선형 관계를 학습할 수 있어 전통적 모형보다 우수한 성능을 보였으나, 시계열의 시간적 의존성을 완전히 활용하지 못하는 한계가 있는 것으로 나타남. 이러한 모형들은 시차 변수(lagged features)를 명시적으로 생성해야 하며, 시계열의 순서 정보를 직접적으로 모델링하지 않는 것으로 평가됨.

트리 기반 모형의 장점은 특징 공학을 통해 다양한 변수 간 상호작용을 포착할 수 있다는 점이며, 특히 고빈도 데이터의 다양한 지표들을 효과적으로 활용할 수 있음. 그러나 시계열의 장기 의존성을 학습하는 데는 한계가 있는 것으로 나타남.

딥러닝 모형의 강점과 한계
딥러닝 기반 모형인 DeepAR과 TFT는 시계열의 장기 의존성을 학습할 수 있어 중장기 예측에서 우수한 성능을 보였으나, 학습에 많은 데이터와 계산 자원이 필요하다는 단점이 있는 것으로 평가됨.

DeepAR은 자기회귀 구조를 통해 시계열의 시간적 패턴을 학습하며, 확률적 예측을 제공하여 예측 불확실성을 정량화할 수 있음 \cite{salinas2020deepar}. TFT는 어텐션 메커니즘을 활용하여 시계열의 장기 의존성과 외생 변수 간의 관계를 동시에 학습하며, 변수 선택 어텐션을 통해 해석 가능한 예측을 제공함 \cite{lim2021temporal}. 또한 상태 공간 모형을 딥러닝에 접목한 Deep State Space Models도 시계열 예측에 활용되고 있으며, 장기 의존성을 효과적으로 학습할 수 있음 \cite{rangapuram2018deep}.

그러나 이러한 모형들은 대량의 데이터가 필요하며, 학습 시간이 길고 하이퍼파라미터 튜닝이 복잡하다는 단점이 있음. 또한 모형의 해석이 어려워 경제적 의미를 파악하기 어려운 것으로 평가됨.

동적 요인 모형의 우수성
동적 요인 모형인 DFM과 DDFM은 혼합 빈도 데이터를 효과적으로 처리할 수 있으며, 특히 나우캐스팅 상황에서 강점을 보인 것으로 나타남. DFM은 많은 시계열에서 공통 요인을 추출하여 차원을 축소하고, Kalman 필터를 통해 요인과 관측치를 동시에 추정할 수 있음 \cite{stock2002forecasting}. 주성분 분석을 활용한 요인 추출 방법은 고차원 시계열 데이터의 차원 축소에 효과적임 \cite{stock2002forecasting}.

DDFM은 비선형 요인 구조를 학습할 수 있어 기존 DFM보다 개선된 성능을 보인 것으로 평가됨. 특히 변분 자기인코더를 활용하여 복잡한 비선형 관계를 포착할 수 있으며, 기존 DFM의 선형 가정을 완화하여 더 정확한 예측을 제공함.

동적 요인 모형의 가장 큰 장점은 혼합 빈도 데이터를 자연스럽게 처리할 수 있다는 점임. 분기별 목표 변수를 월간 또는 주간 고빈도 지표로부터 예측할 수 있으며, 이는 나우캐스팅에 매우 유용한 것으로 평가됨.

나. 경제적 의미

본 연구의 결과는 정책 결정자와 실무진에게 다음과 같은 시사점을 제공함:

나우캐스팅의 실용성
DFM과 DDFM을 활용한 나우캐스팅은 공식 통계 발표 전에 현재 분기의 경제 상황을 추정할 수 있어, 신속한 정책 대응이 가능한 것으로 나타남. 예를 들어, 분기별 GDP가 공식 발표되기 전에 월간 생산지수, 소매판매액, 수출입액 등의 고빈도 지표를 활용하여 현재 분기의 GDP를 추정할 수 있음.

이는 특히 경제 위기 상황에서 중요한 것으로 평가됨. COVID-19 팬데믹과 같이 경제 상황이 급격히 변화하는 시기에는 공식 통계 발표까지의 시차로 인해 정책 대응이 지연될 수 있는 상황임. 나우캐스팅을 통해 정책 결정자들은 실시간에 가까운 경제 상황을 파악하고, 필요시 선제적인 정책 조치를 취할 수 있음.

고빈도 데이터의 활용
월간 및 주간 고빈도 데이터를 활용함으로써 분기별 목표 변수의 예측 정확도를 향상시킬 수 있는 것으로 나타남. 본 연구의 결과는 다양한 고빈도 지표들이 거시경제 변수 예측에 유용한 정보를 제공함을 보여줌.

특히 설문 지표(기업경기실사지수, 소비자동향지수), 생산 지표(전산업 생산지수, 제조업 생산지수), 금융 지표(기준금리, 주가지수) 등이 예측에 중요한 역할을 하는 것으로 나타남. 이러한 지표들은 경제 활동의 실시간 변화를 반영하므로, 분기별 목표 변수를 예측하는 데 유용한 것으로 평가됨.

모형 선택의 중요성
목표 변수와 예측 기간에 따라 최적의 모형이 다르므로, 상황에 맞는 모형 선택이 중요한 것으로 나타남. 예를 들어, 단기 예측에서는 고빈도 데이터를 효과적으로 활용할 수 있는 모형이 우수한 성능을 보일 수 있으며, 중기 예측에서는 장기 의존성을 학습할 수 있는 모형이 유리할 수 있음.

또한 목표 변수의 특성에 따라 최적 모형이 달라질 수 있음. GDP와 같이 상대적으로 안정적인 변수는 전통적 모형도 좋은 성능을 보일 수 있으나, 총고정자본형성과 같이 변동성이 큰 변수는 비선형 관계를 학습할 수 있는 모형이 필요할 수 있는 것으로 평가됨.

정책적 함의
본 연구의 결과는 다음과 같은 정책적 함의를 가짐:

\begin{itemize}
    \item \textbf{실시간 경제 모니터링}: 나우캐스팅을 통해 실시간에 가까운 경제 상황을 파악할 수 있으므로, 정책 결정의 시의성을 향상시킬 수 있음.
    \item \textbf{데이터 기반 의사결정}: 다양한 예측 모형의 결과를 종합적으로 고려하여 더 정확한 경제 전망을 수립할 수 있음.
    \item \textbf{고빈도 데이터 수집의 중요성}: 고빈도 데이터가 예측 정확도 향상에 기여하므로, 관련 데이터 수집 및 관리 시스템의 구축이 중요한 것으로 평가됨.
\end{itemize}

다. 연구의 한계점

본 연구는 다음과 같은 한계점을 가짐:

\begin{itemize}
    \item \textbf{데이터 품질}: 많은 변수들이 결측치를 포함하고 있어, 전처리 과정에서 정보 손실이 발생할 수 있는 상황임.
    \item \textbf{모형 해석가능성}: 딥러닝 기반 모형(DDFM, DeepAR, TFT)은 예측 성능은 우수하나, 예측 결과의 경제적 해석이 어려운 것으로 평가됨.
    \item \textbf{외생 충격 고려}: 본 연구는 과거 데이터에 기반한 예측만을 다루며, 예상치 못한 외생 충격(예: 자연재해, 지정학적 사건)을 고려하지 않음.
    \item \textbf{한국 데이터에 국한}: 본 연구는 한국 데이터에만 적용되었으며, 다른 국가나 지역에 대한 일반화 가능성은 추가 검증이 필요한 상황임.
\end{itemize}

라. 향후 연구 방향

본 연구의 결과를 바탕으로 다음과 같은 향후 연구 방향을 제안함:

\begin{itemize}
    \item \textbf{실시간 예측 시스템 구축}: 본 연구에서 개발한 모형을 실시간으로 업데이트하고 예측을 생성하는 시스템을 구축함.
    \item \textbf{비전통적 데이터 활용}: 소셜 미디어 데이터, 검색 트렌드, 위성 이미지 등 비전통적 데이터 소스를 활용한 예측 모형 개발.
    \item \textbf{해석 가능한 딥러닝 모형}: 딥러닝 모형의 예측 결과를 해석할 수 있는 방법론 개발 (예: attention weight 분석, feature importance).
    \item \textbf{앙상블 모형}: 여러 모형의 예측을 결합하는 앙상블 방법을 통해 예측 정확도를 향상시킴.
    \item \textbf{불확실성 정량화}: 예측뿐만 아니라 예측 불확실성도 함께 제공하는 확률적 예측 모형 개발.
\end{itemize}

