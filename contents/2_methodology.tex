\section{결과 비교}
\label{sec:results}

\subsection{실험 설계}
\label{subsec:experimental_design}

\subsubsection{실험 셋업}

\begin{itemize}
    \item \textbf{대상 변수:} KOEQUIPTE, KOWRCCNSE, KOIPALL.G (3개)
    \item \textbf{모형:} ARIMA, VAR, DFM, DDFM (4개)
    \item \textbf{평가:} 22개 예측 수평선(1--22개월), 모형-대상 조합별 평균 계산
\end{itemize}

\begin{table}[h]
\centering
\caption{Dataset and Model Parameters}
\label{tab:dataset_params}
\begin{tabular}{lccc}
\toprule
Target Variable & Series Count & Training Period & Forecast Period \\
\midrule
KOIPALL.G & 43 & 1985-2019 & 2024-2025 \\
KOEQUIPTE & 43 & 1985-2019 & 2024-2025 \\
KOWRCCNSE & 43 & 1985-2019 & 2024-2025 \\
\bottomrule
\end{tabular}
\end{table}


표~\ref{tab:dataset_params} 요약:
\begin{itemize}
    \item 각 대상 변수마다 평균 43개 시계열 사용
    \item 훈련 기간: 1985--2019년
    \item 예측 기간: 2024--2025년
\end{itemize}

\subsubsection{데이터 전처리}

\begin{itemize}
    \item \textbf{변환:} 시계열별 변환 유형('lin', 'log', 'chg' 등) 적용
    \item \textbf{결측치 처리:} forward-fill $\to$ backward-fill $\to$ naive forecaster 순차 적용
    \item \textbf{표준화:}
    \begin{itemize}
        \item ARIMA/VAR: 원본 스케일 유지
        \item DFM/DDFM: StandardScaler 적용 (평균 0, 표준편차 1)
    \end{itemize}
\end{itemize}

\subsubsection{데이터 품질 문제 및 시리즈 제거}

데이터 품질 개선을 위해 다음 시리즈를 제거:
\begin{itemize}
    \item 높은 상관관계(> 0.95) 시리즈
    \item 극단적 결측치(91.3\%) 시리즈 (pmiall, pmiout)
    \item 블록 구조 단일 글로벌 블록으로 단순화, 요인 수 3개 통일
\end{itemize}

\subsubsection{예측 모형}

\textbf{ARIMA:} 자기회귀 및 이동평균 성분 포착, 정상성을 위해 차분 사용, 단변량 시계열 예측. 차수 (1,1,1) 사용.

\textbf{VAR:} ARIMA를 다변량으로 확장, 여러 시계열 간 동적 관계 포착. 시차 1 사용. 긴 수평선에서 수치적 불안정성 발생 가능.

\textbf{DFM:} 많은 시계열에서 공통 요인 추출, 차원 축소, 혼합주기 데이터 처리 \cite{stock2002forecasting}. DFM은 state-space 형태로 표현되며, measurement equation과 transition equation으로 구성됨. EM 알고리즘으로 파라미터 추정, 칼만 필터와 스무더로 요인 추정. 칼만 필터는 실시간 데이터 흐름을 재귀적으로 처리하여 각 시점의 예측을 업데이트하며, 데이터의 품질과 시의성을 기반으로 가중치를 부여함. 이는 nowcasting에 특히 유용한 특성으로, 비동기적 데이터 발표와 결측치를 자연스럽게 처리할 수 있음. 혼합주기 데이터의 경우, 텐트 커널(tent kernel) 집계 방법을 사용하여 서로 다른 주기의 데이터를 통합함.

\textbf{DDFM:} 오토인코더 기반 아키텍처로 비선형 요인 관계 학습 \cite{andreini2020deep}. DDFM은 인코더를 통해 관측 변수에서 잠재 요인을 추출하고, 디코더를 통해 요인에서 관측 변수로 재구성함. 이 과정에서 선형 DFM의 제약을 완화하여 더 복잡한 요인 구조를 학습할 수 있음. 대규모 데이터셋에서도 효과적으로 작동하며, 전통적인 DFM의 계산적 한계를 극복함.

\subsubsection{Forecasting과 Nowcasting}

\textbf{Forecasting:} 과거 데이터로 미래 값 예측. 각 모형 훈련 후 1--22개월 수평선에 대해 예측 생성.

\textbf{Nowcasting:} 공식 통계 발표 전 현재 시점 거시경제 변수 추정 \cite{banbura2012nowcasting}. Nowcasting은 실시간 경제 모니터링의 핵심 기법으로, 중앙은행과 정책기관에서 널리 활용됨. 각 목표 월에 대해 4주 전, 1주 전 시점에서 예측을 수행하며, 시리즈별 발표 시차(publication lag)를 기준으로 미발표 데이터를 마스킹함. 이는 실제 운영 환경에서 특정 시점에 사용 가능한 데이터만을 사용하여 예측하는 상황을 시뮬레이션함. 시간이 지날수록 더 많은 데이터가 사용 가능해지므로, 예측 정확도가 향상될 것으로 기대됨. DFM과 DDFM은 요인 모형의 구조적 특성으로 인해 release date 기반 마스킹을 효과적으로 처리할 수 있으나, ARIMA와 VAR은 이러한 구조적 유연성이 부족하여 nowcasting에 제한적임.
