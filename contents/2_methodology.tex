\section{방법론}
\label{sec:methodology}

\subsection{예측 모형}

본 연구는 네 가지 예측 모형(ARIMA, VAR, DFM, DDFM)을 비교한다. 각 모형은 세 가지 대상 변수(KOEQUIPTE, KOWRCCNSE, KOIPALL.G)에 대해 세 가지 예측 수평선(1일, 7일, 28일)에서 평가된다. 데이터셋 세부사항과 모형 파라미터는 표~\ref{tab:dataset_params}에 요약되어 있다.

\begin{table}[h]
\centering
\caption{Dataset and Model Parameters}
\label{tab:dataset_params}
\begin{tabular}{lccc}
\toprule
Target Variable & Series Count & Training Period & Forecast Period \\
\midrule
KOIPALL.G & 43 & 1985-2019 & 2024-2025 \\
KOEQUIPTE & 43 & 1985-2019 & 2024-2025 \\
KOWRCCNSE & 43 & 1985-2019 & 2024-2025 \\
\bottomrule
\end{tabular}
\end{table}


\subsubsection{ARIMA 모형}

ARIMA(AutoRegressive Integrated Moving Average) 모형은 자기회귀 및 이동평균 성분을 포착하고 정상성을 위해 차분을 사용하는 단변량 시계열 예측 방법이다. 모형 차수는 표준 시계열 분석 절차를 통해 결정된다.

\subsubsection{VAR 모형}

벡터자기회귀(Vector Autoregression, VAR) 모형은 ARIMA를 다변량 설정으로 확장하여 여러 시계열 간의 동적 관계를 포착한다. 시차 차수는 정보 기준에 따라 선택된다.

\subsubsection{동적요인모형 (DFM)}

동적요인모형(Dynamic Factor Model, DFM)은 많은 시계열에서 공통 요인을 추출하여 차원을 축소하고 혼합주기 데이터를 효과적으로 처리한다 \cite{stock2002forecasting}. DFM은 다음과 같이 정의된다:

\begin{align}
x_t &= C z_t + \varepsilon_t, \quad \varepsilon_t \sim \mathcal{N}(0, R) \\
z_t &= A z_{t-1} + \eta_t, \quad \eta_t \sim \mathcal{N}(0, Q)
\end{align}

여기서 $x_t$는 관측된 시계열, $z_t$는 공통 요인, $C$는 요인 적재 행렬, $A$는 전이 행렬이다. 파라미터는 EM 알고리즘을 통해 추정되며, 요인은 칼만 필터와 스무더를 사용하여 추정된다.

\subsubsection{심층 동적요인모형 (DDFM)}

심층 동적요인모형(Deep Dynamic Factor Model, DDFM)은 오토인코더 기반 아키텍처를 사용하여 복잡한 요인 구조를 학습한다 \cite{andreini2020deep}. 비선형 인코더는 정교한 요인 관계 학습을 가능하게 하며, PyTorch Lightning을 사용하여 구현된다.

\subsection{혼합주기 집계}

혼합주기 데이터 처리를 위해 텐트 커널(tent kernel) 집계 방법을 사용한다 \cite{mariano2003new}. 이 방법은 집계 기간 중간에 가까운 관측값에 더 큰 가중치를 부여하여 서로 다른 주기의 데이터를 효과적으로 결합할 수 있게 한다.

\subsection{평가 지표}

나우캐스팅 성능은 표준화된 지표를 사용하여 평가되며, 이를 통해 서로 다른 시계열과 규모 간 공정한 비교가 가능하다. 표준화된 평균제곱오차(sMSE), 표준화된 평균절대오차(sMAE), 표준화된 평균제곱근오차(sRMSE)를 보고하며, 표준화는 훈련 데이터의 표준편차를 사용하여 수행된다.

\subsection{평가 설계}

평가는 단일 단계 예측 설계를 사용하며, 각 예측 수평선(1일, 7일, 28일)은 단일 테스트 포인트를 사용하여 평가된다. 이 설계 선택은 통계적 신뢰성을 제한하지만 각 수평선에서 모형 성능에 대한 집중적인 평가를 제공한다. 훈련-테스트 분할은 데이터의 80\%를 훈련에, 20\%를 테스트에 사용하며, 이로 인해 DFM 및 DDFM 모형의 28일 예측 평가를 위한 테스트 데이터가 부족하다(이 조합에 대해 n\_valid = 0). 이 제한사항은 결과 및 결론 섹션에서 인정된다.
