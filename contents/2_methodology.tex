\section{결과 비교}
\label{sec:results}

\subsection{실험 설계}
\label{subsec:experimental_design}

\subsubsection{실험 셋업}

\begin{itemize}
    \item \textbf{대상 변수:} KOEQUIPTE, KOWRCCNSE, KOIPALL.G (3개)
    \item \textbf{모형:} ARIMA, VAR, DFM, DDFM (4개)
    \item \textbf{평가:} 22개 예측 수평선(1--22개월), 모형-대상 조합별 평균 계산
\end{itemize}

\begin{table}[h]
\centering
\caption{Dataset and Model Parameters}
\label{tab:dataset_params}
\begin{tabular}{lccc}
\toprule
Target Variable & Series Count & Training Period & Forecast Period \\
\midrule
KOIPALL.G & 43 & 1985-2019 & 2024-2025 \\
KOEQUIPTE & 43 & 1985-2019 & 2024-2025 \\
KOWRCCNSE & 43 & 1985-2019 & 2024-2025 \\
\bottomrule
\end{tabular}
\end{table}


표~\ref{tab:dataset_params} 요약:
\begin{itemize}
    \item 각 대상 변수마다 평균 43개 시계열 사용
    \item 훈련 기간: 1985--2019년
    \item 예측 기간: 2024--2025년
\end{itemize}

\textbf{훈련 기간과 예측 기간 사이의 4년 간격:}
\begin{itemize}
    \item \textbf{COVID-19 시기 제외:} 2020--2023년 비정상적 패턴 제외, 정상 경제 패턴만 학습
    \item \textbf{데이터 누수 방지:} 미래 정보 누수 방지, 실제 예측 성능 공정 평가
    \item \textbf{구조 변화 검증:} COVID-19 이후 경제 구조 변화 영향 평가
\end{itemize}

\textbf{모형 파라미터:}
\begin{itemize}
    \item \textbf{ARIMA:} 차수 (1,1,1)
    \item \textbf{VAR:} 시차 1
    \item \textbf{DFM:} 요인 3개, 최대 5000회 반복
    \item \textbf{DDFM:} 64--32 레이어, 요인 3개, 100 에폭
\end{itemize}

\textbf{DFM/DDFM 공통 설정:}
\begin{itemize}
    \item 요인 수: 3개 (통일)
    \item 블록 구조: 단일 글로벌 블록 (수치적 안정성)
    \item 상태 공간 차원: 15차원 (요인 3개 $\times$ tent kernel 파라미터 5개)
\end{itemize}

\subsubsection{데이터 전처리}

\begin{itemize}
    \item \textbf{변환:} 시계열별 변환 유형('lin', 'log', 'chg' 등) 적용
    \item \textbf{결측치 처리:} forward-fill $\to$ backward-fill $\to$ naive forecaster 순차 적용
    \item \textbf{표준화:}
    \begin{itemize}
        \item ARIMA/VAR: 원본 스케일 유지
        \item DFM/DDFM: StandardScaler 적용 (평균 0, 표준편차 1)
    \end{itemize}
\end{itemize}

\subsubsection{데이터 품질 문제 및 시리즈 제거}

데이터 품질 개선을 위해 다음 시리즈를 제거:
\begin{itemize}
    \item 높은 상관관계(> 0.95) 시리즈
    \item 극단적 결측치(91.3\%) 시리즈 (pmiall, pmiout)
    \item 블록 구조 단일 글로벌 블록으로 단순화, 요인 수 3개 통일
\end{itemize}

\subsubsection{예측 모형}

\textbf{ARIMA:} 자기회귀 및 이동평균 성분 포착, 정상성을 위해 차분 사용, 단변량 시계열 예측. 차수 (1,1,1) 사용.

\textbf{VAR:} ARIMA를 다변량으로 확장, 여러 시계열 간 동적 관계 포착. 시차 1 사용. 긴 수평선에서 수치적 불안정성 발생 가능.

\textbf{DFM:} 많은 시계열에서 공통 요인 추출, 차원 축소, 혼합주기 데이터 처리 \cite{stock2002forecasting}. EM 알고리즘으로 파라미터 추정, 칼만 필터로 요인 추정.

\textbf{DDFM:} 오토인코더 기반 아키텍처로 비선형 요인 관계 학습 \cite{andreini2020deep}.

\subsubsection{Forecasting과 Nowcasting}

\textbf{Forecasting:} 과거 데이터로 미래 값 예측. 각 모형 훈련 후 1--22개월 수평선에 대해 예측 생성.

\textbf{Nowcasting:} 공식 통계 발표 전 현재 시점 거시경제 변수 추정 \cite{banbura2012nowcasting}. 각 목표 월에 대해 4주 전, 1주 전 시점에서 예측. 시리즈별 발표 시차 기준으로 미발표 데이터 마스킹.

\subsubsection{평가 지표}

표준화된 지표 사용 (서로 다른 시계열과 규모 간 공정한 비교): sMSE, sMAE, sRMSE. 표준화는 훈련 데이터의 표준편차 사용. 최종 지표는 모형-대상 조합별로 1--22개월 수평선 평균.
