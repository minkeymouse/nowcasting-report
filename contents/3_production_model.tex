\section{Production Model: KOIPALL.G}

\subsection{Target Variable}

The Industrial Production Index, All Industries (KOIPALL.G) serves as the production indicator, representing overall industrial activity in the Korean economy. This index aggregates production across all industries and is a key indicator for assessing economic activity.

\subsection{Data Composition}

The production model utilizes monthly and quarterly time series data relevant to industrial production. The dataset includes variables related to employment, industrial production, business surveys, and other economic indicators that are predictive of overall industrial activity.

\subsection{Model Comparison Results}

We compare the forecasting performance of four models (ARIMA, VAR, DFM, DDFM) on KOIPALL.G across three forecast horizons (1, 7, and 28 days). Performance metrics (standardized MSE, MAE, and RMSE) will be presented in Table~\ref{tab:overall_metrics_by_target} and visualized in Figure~\ref{fig:forecast_vs_actual_koipallg}.

\subsection{Forecast Performance}

The forecast vs actual plot (Figure~\ref{fig:forecast_vs_actual_koipallg}) shows the historical series and model forecasts over the evaluation period. Detailed performance metrics by forecast horizon are presented in Table~\ref{tab:overall_metrics_by_horizon}. Detailed metrics for all model-horizon combinations for KOIPALL.G are available in Table~\ref{tab:metrics_36_rows}.

\begin{figure}[h]
\centering
\includegraphics[width=0.9\textwidth]{images/forecast_vs_actual_koipall_g.png}
\caption{Forecast vs Actual: Industrial Production Index (KOIPALL.G). Shows 30 months of historical data followed by 30 months of forecasts from ARIMA, VAR, DFM, and DDFM models.}
\label{fig:forecast_vs_actual_koipallg}
\end{figure}

\subsection{Discussion}

Experimental results for KOIPALL.G reveal significant differences in model performance across forecast horizons. ARIMA demonstrates the most consistent performance: excellent for 1-day forecasts (sMSE = 0.0034, sRMSE = 0.0584), moderate for 7-day forecasts (sMSE = 2.28, sRMSE = 1.51), and reasonable for 28-day forecasts (sMSE = 0.39, sRMSE = 0.62). The performance degradation from 1-day to 7-day forecasts suggests that short-term patterns are more predictable than medium-term trends.

VAR shows exceptional performance for 1-day forecasts (sMSE ≈ 3.5×10⁻⁹, sRMSE ≈ 6.0×10⁻⁵) but suffers from severe numerical instability for longer horizons. For 7-day and 28-day forecasts, VAR produces extremely large errors (sRMSE > 10²² for h=7, > 10⁵⁸ for h=28), indicating that the model is not suitable for multi-step ahead forecasting on this target. This instability is a known limitation of VAR models when forecasting beyond very short horizons.

DFM shows moderate performance for 1-day forecasts (sRMSE = 5.92) but improves for 7-day forecasts (sRMSE = 5.28), indicating that the factor model captures medium-term trends better than short-term fluctuations. However, DFM performance is significantly worse than ARIMA across all horizons. DDFM demonstrates excellent performance for 1-day forecasts (sRMSE = 0.46) and exceptional performance for 7-day forecasts (sRMSE = 0.18), outperforming ARIMA for these horizons. The 28-day horizon is unavailable for both DFM and DDFM due to insufficient test data after the 80/20 train-test split.

Overall, ARIMA provides the most reliable forecasts for industrial production across all horizons, with performance degrading gracefully as the forecast horizon increases.

