\section{생산 모형: KOIPALL.G}

\subsection{대상 변수}

전산업생산지수(Industrial Production Index, All Industries: KOIPALL.G)는 생산 지표로 작용하며, 한국 경제의 전체 산업 활동을 나타낸다. 이 지수는 모든 산업의 생산을 집계하며 경제 활동 평가를 위한 핵심 지표이다.

\subsection{데이터 구성}

생산 모형은 산업생산과 관련된 월간 및 분기 시계열 데이터를 활용한다. 데이터셋에는 고용, 산업생산, 기업 서베이 및 전체 산업 활동을 예측하는 기타 경제 지표와 관련된 변수들이 포함된다.

\subsection{모형 비교 결과}

KOIPALL.G에 대해 네 가지 모형(ARIMA, VAR, DFM, DDFM)의 예측 성능을 세 가지 예측 수평선(1일, 7일, 28일)에서 비교한다. 성능 지표(표준화된 MSE, MAE, RMSE)는 표~\ref{tab:overall_metrics_by_target}에 제시되며 그림~\ref{fig:forecast_vs_actual_koipallg}에 시각화된다.

\subsection{예측 성능}

예측 대 실제 플롯(그림~\ref{fig:forecast_vs_actual_koipallg})은 평가 기간 동안의 역사적 시계열과 모형 예측을 보여준다. 예측 수평선별 상세 성능 지표는 표~\ref{tab:overall_metrics_by_horizon}에 제시된다. KOIPALL.G에 대한 모든 모형-수평선 조합의 상세 지표는 표~\ref{tab:metrics_36_rows}에서 확인할 수 있다.

\begin{figure}[h]
\centering
\includegraphics[width=0.9\textwidth]{images/forecast_vs_actual_koipall_g.png}
\caption{예측 대 실제: 전산업생산지수 (KOIPALL.G). 30개월의 역사적 데이터와 ARIMA, VAR, DFM, DDFM 모형의 30개월 예측을 보여준다.}
\label{fig:forecast_vs_actual_koipallg}
\end{figure}

\subsection{논의}

KOIPALL.G에 대한 실험 결과는 예측 수평선에 걸쳐 모형 성능의 유의한 차이를 보여준다. ARIMA는 가장 일관된 성능을 보인다: 1일 예측에서 우수(sMSE = 0.0034, sRMSE = 0.0584), 7일 예측에서 보통(sMSE = 2.28, sRMSE = 1.51), 28일 예측에서 합리적(sMSE = 0.39, sRMSE = 0.62). 1일에서 7일 예측으로의 성능 저하는 단기 패턴이 중기 추세보다 예측하기 쉬움을 시사한다.

VAR은 1일 예측에서 탁월한 성능을 보이지만(sMSE $\approx$ 3.5$\times$10$^{-9}$, sRMSE $\approx$ 6.0$\times$10$^{-5}$), 더 긴 수평선에서는 심각한 수치적 불안정성을 겪는다. 7일 및 28일 예측의 경우, VAR은 극도로 큰 오차를 생성한다(h=7일 경우 sRMSE $>$ 10$^{22}$, h=28일 경우 $>$ 10$^{58}$), 이는 이 대상에 대해 다단계 앞 예측에 적합하지 않음을 나타낸다. 이 불안정성은 매우 짧은 수평선을 넘어 예측할 때 VAR 모형의 알려진 제한사항이다.

DFM은 1일 예측에서 보통 성능을 보이지만(sRMSE = 5.92), 7일 예측에서 개선된다(sRMSE = 5.28). 이는 요인 모형이 단기 변동보다 중기 추세를 더 잘 포착함을 나타낸다. 그러나 DFM 성능은 모든 수평선에서 ARIMA보다 현저히 낮다. DDFM은 1일 예측에서 우수한 성능(sRMSE = 0.46)과 7일 예측에서 탁월한 성능(sRMSE = 0.18)을 보이며, 이러한 수평선에서 ARIMA를 능가한다. 28일 수평선은 80/20 훈련-테스트 분할 후 테스트 데이터 부족으로 인해 DFM과 DDFM 모두에서 사용할 수 없다.

전반적으로, ARIMA는 모든 수평선에 걸쳐 산업생산에 대한 가장 신뢰할 수 있는 예측을 제공하며, 예측 수평선이 증가함에 따라 성능이 점진적으로 저하된다.
