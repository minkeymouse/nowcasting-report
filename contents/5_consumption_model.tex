\section{Consumption Model: KOWRCCNSE}

\subsection{Target Variable}

The Wholesale and Retail Trade Sales (KOWRCCNSE) serves as the consumption indicator, measuring sales activity in wholesale and retail sectors. This index reflects consumer spending patterns and is a key indicator of domestic demand.

\subsection{Data Composition}

The consumption model utilizes monthly and quarterly time series data relevant to consumption and retail sales. The dataset includes variables related to employment in wholesale and retail sectors, consumer goods imports, retail sales indices (durable, semi-durable, non-durable goods), credit card transactions, online shopping transactions, consumer price indices, manufacturing shipments and inventories of consumer goods, service sector activity (wholesale/retail, accommodation/food services), business sentiment indices, consumer sentiment indices, and financial indicators such as housing loans and household lending rates.

\subsection{Model Comparison Results}

We compare the forecasting performance of four models (ARIMA, VAR, DFM, DDFM) on KOWRCCNSE across three forecast horizons (1, 7, and 28 days). Performance metrics (standardized MSE, MAE, and RMSE) will be presented in Table~\ref{tab:overall_metrics_by_target} and visualized in Figure~\ref{fig:forecast_vs_actual_kowrccnse}.

\subsection{Forecast Performance}

The forecast vs actual plot (Figure~\ref{fig:forecast_vs_actual_kowrccnse}) shows the historical series and model forecasts over the evaluation period. Detailed performance metrics by forecast horizon are presented in Table~\ref{tab:overall_metrics_by_horizon}. Detailed metrics for all model-horizon combinations for KOWRCCNSE are available in Table~\ref{tab:metrics_36_rows}.

\begin{figure}[h]
\centering
\includegraphics[width=0.9\textwidth]{images/forecast_vs_actual_kowrccnse.png}
\caption{Forecast vs Actual: Wholesale and Retail Trade Sales (KOWRCCNSE). Shows 30 months of historical data followed by 30 months of forecasts from ARIMA, VAR, DFM, and DDFM models.}
\label{fig:forecast_vs_actual_kowrccnse}
\end{figure}

\subsection{Discussion}

For KOWRCCNSE, ARIMA shows the best overall performance among available models, with standardized RMSE values of 0.81 (1-day), 0.65 (7-day), and 0.68 (28-day). Notably, performance actually improves slightly from 1-day to 7-day forecasts, suggesting that short-term noise may be more problematic than medium-term trends for this series. The 28-day forecast performance (sRMSE = 0.68) is comparable to the 7-day forecast, indicating reasonable stability across medium-term horizons.

VAR exhibits the same pattern as observed for other targets: near-perfect 1-day forecasts (sMSE $\approx$ 5.8$\times$10$^{-9}$, sRMSE $\approx$ 7.6$\times$10$^{-5}$) followed by catastrophic failure for longer horizons (sRMSE $>$ 10$^{11}$ for h=7, $>$ 10$^{58}$ for h=28). The numerical instability makes VAR unusable for consumption forecasting beyond the immediate next period.

ARIMA's superior performance for KOWRCCNSE compared to KOEQUIPTE (average sRMSE = 0.71 vs. 1.19) suggests that consumption patterns are more regular and predictable than investment patterns. This aligns with economic theory, as consumption tends to follow smoother trends driven by income and demographics, while investment is more volatile and subject to business cycle effects.

DFM shows poor performance for KOWRCCNSE, with sRMSE values of 9.25 (1-day) and 7.08 (7-day), significantly worse than ARIMA. This suggests that the factor model struggles with consumption patterns, possibly due to numerical instability issues during EM algorithm convergence (warnings about singular matrices and ill-conditioned systems were observed). DDFM demonstrates good performance, with sRMSE values of 0.82 (1-day) and 1.36 (7-day). While DDFM's 1-day performance is comparable to ARIMA (0.82 vs. 0.81), ARIMA maintains better performance for 7-day forecasts (0.65 vs. 1.36). The 28-day horizon is unavailable for both DFM and DDFM due to insufficient test data after the 80/20 train-test split. Overall, ARIMA remains the best choice for consumption forecasting, with DDFM providing a competitive alternative for short-term forecasts.

