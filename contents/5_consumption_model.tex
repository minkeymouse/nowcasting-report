\section{소비부문 모형}

\subsection{데이터 구성}

소비부문 데이터는 고용, 소비, 설문 등 주요 월간 지표를 포함하여 구성됨. 주요 변수는 고용/노동(취업자 수: 도·소매업), 수출입(수입: 소비재), 소비/지출(소매판매액, 도·소매업 판매지수: 내구재, 반내구재, 비내구재, 신용카드 거래액, 사이버쇼핑 거래액), 물가(소비자물가지수, 소비자물가: 농산물·유류 제외, 식료품·에너지 제외), 산업생산(제조업 출하/재고: 소비재, 생산: 비내구재, 소비재, 서비스업: 도·소매, 숙박·음식점), 기업경기(BSI, FKI 지수), 소비자동향(CSI: 종합, 생활형편, 가계소득 전망, 소비지출 계획, 경기판단, 고용상황 전망, 가계저축/부채 전망), 금융(주택담보대출, 가계대출금리) 등임.

\subsection{DFM 모형 추정}

4개 공통요인을 가정하고 DFM 모형을 추정하여, 이를 통해 월간 소비 지수를 산출함. 추정된 공통요인과 모수, 잔차항을 이용하여 월간 도소매판매액 지수를 추정함. 추정된 월간 소비 지수는 관측된 소매판매액 지수와 유사성을 보임.

\subsection{고빈도 DFM 모형}

고빈도 데이터를 포함하여 주, 월간 데이터로 구성된 고빈도 DFM 모형을 추정함. 고빈도 데이터는 주가, 금리, 환율 등 금융시장 데이터와 뉴스 심리지수를 활용함. 금융시장 특성을 반영하기 위해 요인 개수를 1개 추가한 5개로 가정함.

고빈도 DFM 모형은 월간 소비 지수에 대해 양호한 nowcasting 성과를 보임. 평균 절대 예측오차는 4주전 및 1주전 수준에서 양호한 성능을 보임.

\subsection{딥러닝 모형 비교}

동일한 데이터를 이용하여 딥러닝 모형으로 추정 시 nowcasting 성과가 개선됨. 생산 및 투자 모형과 마찬가지로 DFM 모형 대비 예측오차가 개선되었으나, 월간 변동폭을 과소 추정하는 경향이 있어 DFM 모형과 DNN 모형의 평균값 사용이 바람직함.

