\section{소비 모형: KOWRCCNSE}

\subsection{대상 변수}

도소매판매액(Wholesale and Retail Trade Sales: KOWRCCNSE)은 소비 지표로 작용하며, 도매 및 소매 부문의 판매 활동을 측정한다. 이 지수는 소비자 지출 패턴을 반영하며 국내 수요의 핵심 지표이다.

\subsection{데이터 구성}

소비 모형은 소비 및 소매 판매와 관련된 월간 및 분기 시계열 데이터를 활용한다. 데이터셋에는 도매 및 소매 부문 고용, 소비재 수입, 소매 판매 지수(내구재, 반내구재, 비내구재), 신용카드 거래, 온라인 쇼핑 거래, 소비자물가지수, 소비재 제조업 출하 및 재고, 서비스 부문 활동(도소매, 숙박/음식 서비스), 기업 심리 지수, 소비자 심리 지수, 주택 대출 및 가계 대출금리와 같은 금융 지표와 관련된 변수들이 포함된다.

\subsection{모형 비교 결과}

KOWRCCNSE에 대해 네 가지 모형(ARIMA, VAR, DFM, DDFM)의 예측 성능을 세 가지 예측 수평선(1일, 7일, 28일)에서 비교한다. 성능 지표(표준화된 MSE, MAE, RMSE)는 표~\ref{tab:overall_metrics_by_target}에 제시되며 그림~\ref{fig:forecast_vs_actual_kowrccnse}에 시각화된다.

\subsection{예측 성능}

예측 대 실제 플롯(그림~\ref{fig:forecast_vs_actual_kowrccnse})은 평가 기간 동안의 역사적 시계열과 모형 예측을 보여준다. 예측 수평선별 상세 성능 지표는 표~\ref{tab:overall_metrics_by_horizon}에 제시된다. KOWRCCNSE에 대한 모든 모형-수평선 조합의 상세 지표는 표~\ref{tab:metrics_36_rows}에서 확인할 수 있다.

\begin{figure}[h]
\centering
\includegraphics[width=0.9\textwidth]{images/forecast_vs_actual_kowrccnse.png}
\caption{예측 대 실제: 도소매판매액 (KOWRCCNSE). 30개월의 역사적 데이터와 ARIMA, VAR, DFM, DDFM 모형의 30개월 예측을 보여준다.}
\label{fig:forecast_vs_actual_kowrccnse}
\end{figure}

\subsection{논의}

KOWRCCNSE의 경우, ARIMA는 사용 가능한 모형 중에서 가장 우수한 전반적 성능을 보인다. 표준화된 RMSE 값은 0.81(1일), 0.65(7일), 0.68(28일)이다. 특히 성능은 1일에서 7일 예측으로 실제로 약간 개선되며, 이는 단기 노이즈가 이 시계열에 대해 중기 추세보다 더 문제가 될 수 있음을 시사한다. 28일 예측 성능(sRMSE = 0.68)은 7일 예측과 유사하여 중기 수평선에 걸쳐 합리적인 안정성을 나타낸다.

VAR은 다른 대상에서 관찰된 것과 동일한 패턴을 보인다: 거의 완벽한 1일 예측(sMSE $\approx$ 5.8$\times$10$^{-9}$, sRMSE $\approx$ 7.6$\times$10$^{-5}$) 다음에 더 긴 수평선에 대한 치명적인 실패(h=7일 경우 sRMSE $>$ 10$^{11}$, h=28일 경우 $>$ 10$^{58}$). 수치적 불안정성은 즉시 다음 기간을 넘어 소비 예측에 VAR을 사용할 수 없게 만든다.

KOWRCCNSE에 대한 ARIMA의 우수한 성능이 KOEQUIPTE와 비교하여(평균 sRMSE = 0.71 vs. 1.19) 소비 패턴이 투자 패턴보다 더 규칙적이고 예측 가능함을 시사한다. 이는 소비가 소득 및 인구통계학적 요인에 의해 구동되는 더 부드러운 추세를 따르는 반면, 투자는 더 변동적이고 경기 순환 효과의 영향을 받는다는 경제 이론과 일치한다.

DFM은 KOWRCCNSE에 대해 낮은 성능을 보이며, sRMSE 값이 9.25(1일) 및 7.08(7일)로 ARIMA보다 현저히 낮다. 이는 요인 모형이 소비 패턴과 어려움을 겪고 있음을 시사하며, EM 알고리즘 수렴 중 수치적 불안정성 문제 때문일 수 있다(특이 행렬 및 조건이 나쁜 시스템에 대한 경고가 관찰됨). DDFM은 좋은 성능을 보이며, sRMSE 값이 0.82(1일) 및 1.36(7일)이다. DDFM의 1일 성능은 ARIMA와 유사하지만(0.82 vs. 0.81), ARIMA는 7일 예측에서 더 나은 성능을 유지한다(0.65 vs. 1.36). 28일 수평선은 80/20 훈련-테스트 분할 후 테스트 데이터 부족으로 인해 DFM과 DDFM 모두에서 사용할 수 없다. 전반적으로, ARIMA는 소비 예측에 대한 최선의 선택으로 남아 있으며, DDFM은 단기 예측에 대한 경쟁력 있는 대안을 제공한다.
