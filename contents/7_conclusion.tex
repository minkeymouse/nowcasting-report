\section{결론}

\subsection{연구 요약}

본 연구는 고빈도 데이터를 활용하여 한국의 주요 거시경제 변수(GDP, 소비, 투자)를 예측하는 다양한 모형들의 성능을 체계적으로 비교 분석함. 특히 동적 요인 모형(DFM)과 심층 동적 요인 모형(DDFM)에 초점을 맞추어, 혼합 빈도 데이터 처리 능력과 예측 성능을 평가함. 총 36개 조합 중 28개 조합(77.8\%)에 대한 실험 결과가 완료되었음.

핵심 발견은 다음과 같음: (1) VAR 모형이 단기 예측에서 압도적으로 우수한 성능을 보였으며, 이는 거시경제 변수들 간의 동시적 상호작용을 직접적으로 모델링할 수 있기 때문임. (2) DFM과 DDFM은 혼합 빈도 데이터 처리 능력을 보유하나, 목표 변수의 데이터 특성에 따라 성능이 크게 달라짐. DFM은 선형 관계가 명확한 GDP에서는 효과적이나, 비선형 관계가 중요한 소비 변수에서는 수치적 불안정성을 보임. (3) DDFM의 비선형 인코더는 일부 목표 변수에서 DFM보다 안정적인 성능을 보였으나, 변동성이 큰 투자 변수에서는 한계를 드러냄. (4) 모형 선택은 목표 변수와 예측 기간에 따라 달라지며, 단기 예측에는 VAR, 혼합 빈도 nowcasting에는 DFM/DDFM이 적합함.

\subsection{연구의 기여도}

본 연구의 주요 기여는 다음과 같음:

\begin{itemize}
    \item \textbf{통합 비교 프레임워크 구축}: 전통적 통계 모형(ARIMA, VAR)과 동적 요인 모형(DFM, DDFM)을 동일한 데이터셋과 평가 기준으로 비교하는 통합 프레임워크를 제시함. 이를 통해 각 모형의 장단점을 목표 변수와 예측 기간에 따라 체계적으로 분석할 수 있음.
    \item \textbf{혼합 빈도 모형의 실증적 평가}: dfm-python 패키지를 활용하여 DFM과 DDFM의 혼합 빈도 데이터 처리 능력을 실증적으로 검증함. 특히 한국 거시경제 데이터에 대한 적용 가능성과 한계를 명확히 제시함.
    \item \textbf{모형 선택 가이드라인 수립}: 실험 결과를 바탕으로 목표 변수와 예측 기간에 따른 모형 선택 가이드라인을 제시함. 단기 예측에는 VAR, 혼합 빈도 nowcasting에는 DFM/DDFM, 변동성이 큰 변수에는 VAR/ARIMA를 권장함.
    \item \textbf{수치적 불안정성 원인 분석}: DFM의 EM 알고리즘이 특정 목표 변수에서 수치적 불안정성을 보이는 원인을 분석하고, DDFM의 비선형 인코더가 이를 완화할 수 있음을 실증적으로 보임.
\end{itemize}

\subsection{정책적 함의}

본 연구의 결과는 다음과 같은 정책적 시사점을 제공함:

\begin{itemize}
    \item \textbf{단기 예측 기반 정책 의사결정}: VAR 모형의 우수한 단기 예측 성능을 활용하여 통화정책위원회의 기준금리 결정 시점을 앞당기거나, 재정정책 수립 시 더 신속한 경제 상황 파악이 가능함. 특히 경제 위기 상황에서 공식 통계 발표 전 선제적 정책 대응이 가능함.
    \item \textbf{혼합 빈도 nowcasting 시스템 도입}: DFM과 DDFM의 혼합 빈도 처리 능력을 활용하여 한국은행이나 통계청에서 분기별 GDP 발표 전 실시간 추정 시스템을 구축할 수 있음. 이를 통해 정책 결정의 시의성을 크게 향상시킬 수 있음.
    \item \textbf{고빈도 데이터 인프라 투자}: 고빈도 데이터가 예측 정확도 향상에 기여함을 확인하였으므로, 생산지수, 소매판매액, 수출입액 등의 월간 지표의 신뢰성과 시의성 향상을 위한 데이터 인프라 투자가 필요함.
\end{itemize}

\subsection{연구의 한계점}

본 연구는 다음과 같은 한계점을 가짐 (자세한 내용은 Section \ref{sec:discussion} 참조):

\begin{itemize}
    \item \textbf{데이터 품질 및 전처리}: 많은 변수들이 결측치를 포함하고 있어, 전처리 과정에서 정보 손실이 발생할 수 있음
    \item \textbf{모형 해석가능성}: 딥러닝 기반 모형(DDFM)은 학습된 요인 구조의 경제적 해석이 어려움
    \item \textbf{외생 충격 고려 부재}: 본 연구는 과거 데이터에 기반한 예측만을 다루며, 예상치 못한 외생 충격을 고려하지 않음
    \item \textbf{한국 데이터에 국한}: 본 연구는 한국 데이터에만 적용되었으며, 다른 국가나 지역에 대한 일반화 가능성은 추가 검증이 필요함
    \item \textbf{실험 완료율}: 총 36개 조합 중 28개 조합(77.8\%)만 완료되었으며, 8개 조합은 데이터 및 모형의 한계로 인해 평가 불가능함
\end{itemize}

\subsection{향후 연구 방향}

본 연구의 결과와 한계를 바탕으로 다음과 같은 향후 연구 방향을 제안함 (자세한 내용은 Section \ref{sec:discussion} 참조):

\begin{itemize}
    \item \textbf{DFM 수치적 안정성 개선}: 적응형 정규화, 베이지안 추정 방법(MCMC), 또는 변분 추론(VI)을 통한 대안적 추정 방법 연구
    \item \textbf{28일 예측 평가 방법론 개발}: 시계열 교차 검증 또는 walk-forward validation을 통한 평가 방법론 개발
    \item \textbf{변동성이 큰 변수에 대한 모형 개선}: GARCH 또는 stochastic volatility 모델 통합, changepoint detection을 통한 구조적 변화 적응
    \item \textbf{Nowcasting 전용 실험 설계}: 마스킹된 데이터를 활용한 백테스팅 및 News decomposition 기능 활용
    \item \textbf{DDFM 하이퍼파라미터 최적화}: 베이지안 최적화 또는 AutoML 기법을 통한 목표 변수별 최적화
    \item \textbf{앙상블 모형 개발}: VAR, ARIMA, DFM/DDFM의 예측을 통합하는 앙상블 방법 개발
\end{itemize}

본 연구는 고빈도 데이터를 활용한 거시경제 변수 예측에 대한 체계적인 분석 프레임워크를 구축하였으며, 한국 거시경제 데이터에 대한 실증적 비교 분석을 통해 모형 선택 가이드라인을 제시함. 총 36개 조합 중 28개 조합(77.8\%)에 대한 실험 결과가 완료되었으며, 남은 8개 조합은 데이터 및 모형의 근본적 한계로 인해 평가 불가능함. 이러한 한계에도 불구하고, 본 연구는 거시경제 변수 예측 및 nowcasting 연구의 기초 자료로 활용될 수 있을 것으로 기대됨.
