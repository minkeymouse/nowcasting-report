\section{결론}

\subsection{연구 요약}

본 연구는 고빈도 데이터를 활용하여 한국의 주요 거시경제 변수(GDP, 소비, 투자)를 예측하는 다양한 모형들의 성능을 체계적으로 비교 분석할 수 있도록 설계함. 특히 동적 요인 모형(DFM)과 심층 동적 요인 모형(DDFM)에 초점을 맞추어, 혼합 빈도 데이터 처리 능력과 nowcasting 성능을 평가할 수 있도록 설계함. \begin{center}[아직 실험 미진행]\end{center}

주요 연구 내용은 다음과 같음:

\begin{itemize}
    \item \textbf{DFM의 혼합 빈도 처리 능력}: 본 연구에서는 DFM이 텐트 커널을 활용하여 분기별 목표 변수를 월간 고빈도 지표로부터 예측하도록 설계함. Stock과 Watson (2002)이 제안한 주성분 분석을 활용한 요인 추출 방법이 고차원 시계열 데이터의 차원 축소에 효과적임을 참고하여 모형을 구성함 \cite{stock2002forecasting}. \begin{center}[아직 실험 미진행]\end{center}
    \item \textbf{DDFM의 비선형 관계 포착}: 본 연구에서는 DDFM이 자기인코더(autoencoder)를 활용하여 잠재 상태를 생성하고 비선형 요인 구조를 학습하도록 설계함 \cite{andreini2020deep}. 특히 변동성이 큰 변수(총고정자본형성)에 대해서는 비선형 관계를 포착할 수 있는 DDFM의 장점이 기대됨. 한국 거시경제 데이터를 활용한 선행 연구에서도 딥러닝 모형(Mamba)이 DFM보다 우수한 성능을 보인 것으로 보고됨 \cite{kim2024deep}. \begin{center}[아직 실험 미진행]\end{center}
    \item \textbf{고빈도 데이터의 정보 함량}: 본 연구에서는 다양한 고빈도 지표(생산지수, 소비자 심리지수, 금융지표 등)를 활용하여 거시경제 변수 예측을 수행하도록 설계함. \begin{center}[아직 실험 미진행]\end{center}
    \item \textbf{nowcasting의 실용성}: 본 연구에서는 DFM과 DDFM을 활용한 nowcasting 프레임워크를 구축하여 공식 통계 발표 전에 현재 분기의 경제 상황을 추정할 수 있도록 설계함. \begin{center}[아직 실험 미진행]\end{center}
    \item \textbf{모형별 특성}: 목표 변수와 예측 기간에 따라 최적 모형이 달라질 것으로 예상되며, 단기 예측에서는 고빈도 데이터를 효과적으로 활용할 수 있는 모형이, 중기 예측에서는 장기 의존성을 학습할 수 있는 모형이 유리할 것으로 기대됨. \begin{center}[아직 실험 미진행]\end{center}
\end{itemize}

\subsection{연구의 기여도}

\begin{center}[아직 실험 미진행]\end{center}

\subsection{정책적 함의}



\begin{itemize}
    \item \textbf{데이터 품질}: 많은 변수들이 결측치를 포함하고 있어, 전처리 과정에서 정보 손실이 발생할 수 있는 상황임.
    \item \textbf{모형 해석가능성}: 딥러닝 기반 모형(DDFM)은 예측 성능이 우수할 것으로 기대되나, 예측 결과의 경제적 해석이 어려운 것으로 알려져 있음
    \item \textbf{외생 충격 고려}: 본 연구는 과거 데이터에 기반한 예측만을 다루며, 예상치 못한 외생 충격(예: 자연재해, 지정학적 사건)을 고려하지 않음.
    \item \textbf{한국 데이터에 국한}: 본 연구는 한국 데이터에만 적용되었으며, 다른 국가나 지역에 대한 일반화 가능성은 추가 검증이 필요한 상황임.
\end{itemize}

\subsubsection{향후 연구 방향}
본 연구의 프레임워크를 바탕으로 다음과 같은 향후 연구를 제안함:

\begin{itemize}
    \item \textbf{실시간 예측 시스템 구축}: 본 연구에서 설계한 모형을 실시간으로 업데이트하고 예측을 생성하는 시스템을 구축할 수 있을 것으로 기대됨.
    \item \textbf{비전통적 데이터 활용}: 소셜 미디어 데이터, 검색 트렌드, 위성 이미지 등 비전통적 데이터 소스를 활용한 예측 모형 개발.
    \item \textbf{해석 가능한 딥러닝 모형}: 딥러닝 모형의 예측 결과를 해석할 수 있는 방법론 개발 (예: attention weight 분석, 요인 해석).
    \item \textbf{앙상블 모형}: 여러 모형의 예측을 결합하는 앙상블 방법을 통해 예측 정확도를 향상시킬 수 있을 것으로 기대됨.
    \item \textbf{불확실성 정량화}: 예측뿐만 아니라 예측 불확실성도 함께 제공하는 확률적 예측 모형 개발.
    \item \textbf{다른 국가 데이터 적용}: 다른 국가나 지역의 데이터에 적용하여 일반화 가능성을 검증함.
\end{itemize}

본 연구는 고빈도 데이터를 활용한 거시경제 변수 예측에 대한 체계적인 분석 프레임워크를 구축하며, 향후 관련 연구의 기초 자료로 활용될 수 있을 것으로 기대됨. \begin{center}[아직 실험 미진행]\end{center}
