\section{투자 모형: KOEQUIPTE}

\subsection{대상 변수}

설비투자지수(Equipment Investment Index: KOEQUIPTE)는 투자 지표로 작용하며, 설비 및 기계에 대한 자본 지출을 측정한다. 이 지수는 고정자본형성의 핵심 구성요소이며 기업 투자 활동을 반영한다.

\subsection{데이터 구성}

투자 모형은 설비투자와 관련된 월간 및 분기 시계열 데이터를 활용한다. 데이터셋에는 제조업 및 건설업 고용, 자본재 수입, 생산자물가지수, 설비투자 지표, 건설 활동, 설비자금 대출 및 기업 대출금리와 같은 금융 지표와 관련된 변수들이 포함된다.

\subsection{모형 비교 결과}

KOEQUIPTE에 대해 네 가지 모형(ARIMA, VAR, DFM, DDFM)의 예측 성능을 세 가지 예측 수평선(1일, 7일, 28일)에서 비교한다. 성능 지표(표준화된 MSE, MAE, RMSE)는 표~\ref{tab:overall_metrics_by_target}에 제시되며 그림~\ref{fig:forecast_vs_actual_koequipte}에 시각화된다.

\subsection{예측 성능}

예측 대 실제 플롯(그림~\ref{fig:forecast_vs_actual_koequipte})은 평가 기간 동안의 역사적 시계열과 모형 예측을 보여준다. 예측 수평선별 상세 성능 지표는 표~\ref{tab:overall_metrics_by_horizon}에 제시된다. KOEQUIPTE에 대한 모든 모형-수평선 조합의 상세 지표는 표~\ref{tab:metrics_36_rows}에서 확인할 수 있다.

\begin{figure}[h]
\centering
\includegraphics[width=0.9\textwidth]{images/forecast_vs_actual_koequipte.png}
\caption{예측 대 실제: 설비투자지수 (KOEQUIPTE). 30개월의 역사적 데이터와 ARIMA, VAR, DFM, DDFM 모형의 30개월 예측을 보여준다.}
\label{fig:forecast_vs_actual_koequipte}
\end{figure}

\subsection{논의}

KOEQUIPTE에 대한 결과는 ARIMA가 모든 수평선에 걸쳐 보통 성능을 제공함을 보여준다. 표준화된 RMSE 값은 0.32(1일), 1.59(7일), 1.67(28일)이다. 모형은 예측 수평선이 증가함에 따라 상대적으로 안정적인 성능을 유지하지만, 오차는 다른 대상보다 높다. 이는 설비투자가 생산 또는 소비 지표보다 예측하기 어려울 수 있음을 시사한다.

VAR은 다시 1일 예측 정확도에서 우수한 성능을 보이지만(sMSE $\approx$ 3.7$\times$10$^{-9}$, sRMSE $\approx$ 6.0$\times$10$^{-5}$), 더 긴 수평선에서는 완전히 실패하며, 오차가 비현실적인 크기로 폭발한다(h=7일 경우 sRMSE $>$ 10$^{13}$, h=28일 경우 $>$ 10$^{60}$). 이 수치적 불안정성은 즉시 다음 기간을 넘어 투자 예측에 VAR을 부적합하게 만든다.

KOEQUIPTE에 대한 ARIMA 오차가 다른 대상에 비해 상대적으로 높은 것(평균 sRMSE = 1.19 vs. 다른 대상의 0.71-0.73)은 설비투자가 더 큰 변동성 또는 예측하기 어려운 패턴을 보임을 나타낸다. 이는 경기 순환 효과 및 정책 변화의 영향을 받는 자본 투자 결정의 덩어리 특성 때문일 수 있다.

DFM은 KOEQUIPTE에 대해 낮은 성능을 보이며, sRMSE 값이 4.21(1일) 및 6.11(7일)로 ARIMA보다 현저히 낮다. 이는 요인 모형이 설비투자의 변동성 및 불규칙한 패턴과 어려움을 겪고 있음을 시사한다. 그러나 DDFM은 1일 예측에서 탁월한 성능(sRMSE = 0.0103)을 보이며, ARIMA를 포함한 모든 다른 모형을 능가한다. 7일 예측의 경우, DDFM은 sRMSE = 1.91을 달성하며, 이는 ARIMA의 1.59보다 나쁘지만 여전히 합리적이다. 28일 수평선은 80/20 훈련-테스트 분할 후 테스트 데이터 부족으로 인해 DFM과 DDFM 모두에서 사용할 수 없다. DDFM의 우수한 단기 성능은 딥러닝 인코더가 전통적인 모형이 놓치는 투자 데이터의 복잡한 패턴을 효과적으로 포착함을 시사한다.
