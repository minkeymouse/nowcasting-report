\section{투자부문 모형}

\subsection{데이터 구성}

투자부문 데이터는 고용, 설비투자, 건설 등 주요 월간 지표와 3개 분기 지수를 포함하여 총 35개 데이터로 구성됨. 주요 변수는 고용/노동(취업자 수: 광공업, 건설업), 수출입(수입: 자본재), 물가(생산자물가지수), 설비투자(설비투자지수, 기계류, 운송장비), 건설(수주액, 인허가 면적, 착공 면적, 준공액), 산업생산(제조업 출하/재고: 자본재, 서비스업: 부동산·임대업, 사업시설·사업지원, 광공업생산지수, 생산: 건설업, 자본재, 내구재, 경기선행지수), 기업경기(BSI, FKI 지수), 소비자동향(CSI), 금융(설비자금대출, 기업대출금리) 등임.

\subsection{DFM 모형 추정}

4개 공통요인을 가정하고 DFM 모형을 추정하여, 이를 통해 월간 GDP 고정자본형성 지수를 산출함. 추정된 공통요인과 모수, 잔차항을 이용하여 발표된 분기 데이터로 추정함. 고정자본형성 뿐만 아니라 GDP 건설투자, 기계/장비투자 모두 월간지수로 산출함. 추정된 월간 총고정자본형성은 관측된 분기 성장률을 적절히 반영하면서, 설비투자 지수와 유사성을 보임(상관관계 0.67).

\subsection{분기 Nowcasting 성과}

월, 분기 데이터를 이용한 DFM 모형은 GDP 고정자본형성에 대한 nowcasting에서 양호한 성과를 보임. Nowcasting 예측오차는 시장 전문가(Bloomberg) 서베이 대비 우수하며, 발표 시점 평균 오차는 ±1.2\%p, 서베이 오차는 ±1.7\%p임.

\subsection{고빈도 DFM 모형}

추정된 월간 GDP 총고정자본형성을 사용하여, 고빈도 데이터를 포함하여 주, 월간 데이터로 구성된 고빈도 DFM 모형을 추정함. 고빈도 데이터는 주가, 금리, 환율, 원자재가격 등 금융시장 데이터와 뉴스 심리지수를 활용하였으며 KOSPI 건설 및 기계 섹터지수를 추가함. 데이터 개수는 월, 분기 데이터 35개에 고빈도 데이터 8개를 추가하여 43개이며, 요인 개수는 5개로 가정함.

고빈도 DFM 모형은 월간 총고정자본형성에 대해 양호한 nowcasting 성과를 보임. 평균 절대 예측오차는 4주전 2.4\%p, 1주전 1.8\%p 수준임.

\subsection{딥러닝 모형 비교}

동일한 데이터를 이용하여 딥러닝 모형으로 추정 시 nowcasting 성과가 개선됨. 1주전 nowcasting 오차는 1.2\%로 DFM 모형보다 작지만, 생산모형과 마찬가지로 월간 변동폭을 과소 추정하는 경향이 있음. AR(4) 모형도 대체로 유사한 예측력을 보이나 증감 방향성에서 lagging하는 경향이 있음.

따라서 고빈도 모형을 이용한 투자 모형에서도 DFM 모형값과 DNN 모형값의 평균을 사용함. 월간 증감폭의 방향성, 정도에 대해 월말에 근접할수록 양호한 예측력을 보이며, 실제 월간 데이터가 분기 GDP 발표 이후에 확보 가능하다는 점을 고려하면 1주전 높은 예측력을 보이는 것만으로도 의미있는 결과임.

