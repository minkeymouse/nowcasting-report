\section{Investment Model: KOEQUIPTE}

\subsection{Target Variable}

The Equipment Investment Index (KOEQUIPTE) serves as the investment indicator, measuring capital expenditure on equipment and machinery. This index is a key component of fixed capital formation and reflects business investment activity.

\subsection{Data Composition}

The investment model utilizes monthly and quarterly time series data relevant to equipment investment. The dataset includes variables related to employment in manufacturing and construction, capital goods imports, producer price indices, equipment investment indicators, construction activity, and financial indicators such as equipment financing loans and corporate lending rates.

\subsection{Model Comparison Results}

We compare the forecasting performance of four models (ARIMA, VAR, DFM, DDFM) on KOEQUIPTE across three forecast horizons (1, 7, and 28 days). Performance metrics (standardized MSE, MAE, and RMSE) will be presented in Table~\ref{tab:overall_metrics_by_target} and visualized in Figure~\ref{fig:forecast_vs_actual_koequipte}.

\subsection{Forecast Performance}

The forecast vs actual plot (Figure~\ref{fig:forecast_vs_actual_koequipte}) shows the historical series and model forecasts over the evaluation period. Detailed performance metrics by forecast horizon are presented in Table~\ref{tab:overall_metrics_by_horizon}. Detailed metrics for all model-horizon combinations for KOEQUIPTE are available in Table~\ref{tab:metrics_36_rows}.

\begin{figure}[h]
\centering
\includegraphics[width=0.9\textwidth]{images/forecast_vs_actual_koequipte.png}
\caption{Forecast vs Actual: Equipment Investment Index (KOEQUIPTE). Shows 30 months of historical data followed by 30 months of forecasts from ARIMA, VAR, DFM, and DDFM models.}
\label{fig:forecast_vs_actual_koequipte}
\end{figure}

\subsection{Discussion}

Results for KOEQUIPTE show that ARIMA provides moderate performance across all horizons, with standardized RMSE values of 0.32 (1-day), 1.59 (7-day), and 1.67 (28-day). The model maintains relatively stable performance as the forecast horizon increases, though errors are higher than for other targets. This suggests that equipment investment may be more difficult to forecast than production or consumption indicators.

VAR again demonstrates excellent 1-day forecast accuracy (sMSE ≈ 3.7×10⁻⁹, sRMSE ≈ 6.0×10⁻⁵) but completely fails for longer horizons, with errors exploding to impractical magnitudes (sRMSE > 10¹³ for h=7, > 10⁶⁰ for h=28). This numerical instability renders VAR unsuitable for investment forecasting beyond the immediate next period.

The relatively higher ARIMA errors for KOEQUIPTE compared to other targets (average sRMSE = 1.19 vs. 0.71-0.73 for others) indicate that equipment investment exhibits more volatility or less predictable patterns. This may be due to the lumpy nature of capital investment decisions, which are subject to business cycle effects and policy changes.

DFM shows poor performance for KOEQUIPTE, with sRMSE values of 4.21 (1-day) and 6.11 (7-day), significantly worse than ARIMA. This suggests that the factor model struggles with the volatility and irregular patterns in equipment investment. DDFM, however, demonstrates exceptional performance for 1-day forecasts (sRMSE = 0.0103), outperforming all other models including ARIMA. For 7-day forecasts, DDFM achieves sRMSE = 1.91, which is worse than ARIMA's 1.59 but still reasonable. The 28-day horizon is unavailable for both DFM and DDFM due to insufficient test data after the 80/20 train-test split. DDFM's superior short-term performance suggests that the deep learning encoder effectively captures complex patterns in investment data that traditional models miss.

