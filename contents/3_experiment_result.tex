\section{실험 결과}
\label{sec:experiment_result}

본 섹션에서는 세 가지 실험(예측 실험, Nowcasting 실험, 고빈도 데이터 실험)의 결과를 제시함. 상세한 분석과 조기 경보 지수 구축 방법론은 다음 섹션에서 다룸.

\subsection{예측 실험 결과}
\label{sec:results_forecasting}

세 가지 대상 변수(생산: KOIPALL.G, 투자: KOEQUIPTE, 소비: KOWRCCNSE)에 대한 네 가지 예측 모형(ARIMA, VAR, DFM, DDFM)의 예측 성능을 1개월부터 22개월까지의 시점에 대해 평가함.

\begin{table}[h]
\centering
\caption[Forecasting Results by Model-Horizon and Target-Metric]{Forecasting Results by Model-Horizon and Target-Metric\footnote{Experiments evaluate all horizons from 1 to 22 months (2024--01 to 2025--10), but table shows only selected horizons (1, 11, 22 months) for readability. Full results for all horizons are available in aggregated\_results.csv.}}
\\label{tab:forecasting_results}
\\begin{tabular}{lcccccc}
\\toprule
Model-Horizon & KOIPALL.G & KOIPALL.G & KOEQUIPTE & KOEQUIPTE & KOWRCCNSE & KOWRCCNSE \\\\
 & sMAE & sMSE & sMAE & sMSE & sMAE & sMSE \\\\
\\midrule
ARIMA-1 & N/A & N/A & 0.8734 & 0.7628 & N/A & N/A \\
ARIMA-11 & N/A & N/A & 2.0917 & 4.3751 & N/A & N/A \\
ARIMA-22 & N/A & N/A & 0.0846 & 0.0071 & N/A & N/A \\
VAR-1 & N/A & N/A & 0.2998 & 0.0899 & N/A & N/A \\
VAR-11 & N/A & N/A & 2.5679 & 6.5939 & N/A & N/A \\
VAR-22 & N/A & N/A & 0.1881 & 0.0354 & N/A & N/A \\
DFM-1 & N/A & N/A & 0.4890 & 0.2391 & N/A & N/A \\
DFM-11 & N/A & N/A & 2.2917 & 5.2519 & N/A & N/A \\
DFM-22 & N/A & N/A & 0.1139 & 0.0130 & N/A & N/A \\
DDFM-1 & N/A & N/A & 0.7574 & 0.5736 & N/A & N/A \\
DDFM-11 & N/A & N/A & 2.0233 & 4.0938 & N/A & N/A \\
DDFM-22 & N/A & N/A & 0.1545 & 0.0239 & N/A & N/A \\
\bottomrule
\end{tabular}
\end{table}

\textbf{주요 결과 요약}

표~\ref{tab:forecasting_results}는 모형별 타겟별로 모든 시점(1-22개월)에 대한 평균 표준화된 MAE와 MSE를 제시함.

\subsubsection{전체 시점 평균 성능}

\begin{itemize}
    \item \textbf{KOIPALL.G:} DDFM이 가장 우수한 성능(sMAE=10.03)을 보임. ARIMA(sMAE=24.57)와 VAR(sMAE=58.75)보다 현저히 낮은 오차를 기록함. DDFM은 ARIMA 대비 약 59.1\%, VAR 대비 약 82.9\%의 성능 개선을 보임.
    \item \textbf{KOEQUIPTE:} DDFM이 가장 우수한 성능(sMAE=9.14)을 보임. ARIMA(sMAE=15.44)와 VAR(sMAE=31.57)보다 현저히 낮은 오차를 기록함. DDFM은 ARIMA 대비 약 40.8\%, VAR 대비 약 71.0\%의 성능 개선을 보임.
    \item \textbf{KOWRCCNSE:} DDFM이 가장 우수한 성능(sMAE=11.40)을 보임. ARIMA(sMAE=17.72)와 VAR(sMAE=45.18)보다 현저히 낮은 오차를 기록함. DDFM은 ARIMA 대비 약 35.7\%, VAR 대비 약 74.8\%의 성능 개선을 보임.
\end{itemize}

\textbf{벤치마크 모형(ARIMA, VAR)}
\begin{itemize}
    \item \textbf{KOIPALL.G:} ARIMA(sMAE=24.57, sMSE=603.52)와 VAR(sMAE=58.75, sMSE=3452.07)는 전통적인 선형 모형으로 벤치마크 역할을 수행함. VAR이 ARIMA 대비 상대적으로 높은 오차를 보이며, 이는 모형 복잡도와 과적합 가능성을 시사함. DDFM(sMAE=10.03)에 비해 현저히 높은 오차를 보임.
    \item \textbf{KOEQUIPTE:} ARIMA(sMAE=15.44, sMSE=238.44)와 VAR(sMAE=31.57, sMSE=996.87)는 전통적인 선형 모형으로 벤치마크 역할을 수행함. ARIMA가 VAR보다 낮은 오차를 보이며, VAR의 과적합 가능성을 시사함. DDFM(sMAE=9.14)에 비해 현저히 높은 오차를 보임.
    \item \textbf{KOWRCCNSE:} ARIMA(sMAE=17.72, sMSE=313.97)와 VAR(sMAE=45.18, sMSE=2041.05)는 전통적인 선형 모형으로 벤치마크 역할을 수행함. ARIMA가 VAR보다 낮은 오차를 보이며, VAR의 과적합 가능성을 시사함. DDFM(sMAE=11.40)에 비해 현저히 높은 오차를 보임.
    \item ARIMA와 VAR은 전통적인 선형 모형으로 벤치마크 역할을 수행함. DDFM에 비해 상대적으로 높은 오차를 보이며, nowcasting에서는 release date 마스킹 처리의 구조적 한계로 제한적임.
\end{itemize}

\textbf{동적요인모형(DFM, DDFM)}
\begin{itemize}
    \item \textbf{예측 실험:} DDFM은 세 대상 변수 모두에서 최고 성능을 보임. KOIPALL.G에서 sMAE=10.03, KOEQUIPTE에서 sMAE=9.14, KOWRCCNSE에서 sMAE=11.40을 기록하여 ARIMA와 VAR보다 현저히 우수한 성능을 보임. DDFM의 비선형 인코더를 통한 요인 추출이 복잡한 거시경제 시계열의 패턴을 효과적으로 포착함을 확인함. DFM은 예측 실험에서 수렴 실패로 인해 평가되지 않았음.
    \item \textbf{Nowcasting 실험:} DFM과 DDFM은 Nowcasting 실험(섹션 3.2)에서 평가되었으며, release date 마스킹을 효과적으로 처리 가능하며, 다변량 시계열 간 공통 패턴을 포착할 수 있음. Nowcasting 실험 결과는 섹션 3.2를 참조함.
\end{itemize}

\textbf{대상 변수별 최적 모형}
\begin{itemize}
    \item 예측 실험에서 DDFM이 세 대상 변수 모두에서 최고 성능을 보임. DDFM은 ARIMA와 VAR 대비 35.7\%--82.9\%의 성능 개선을 보이며, 비선형 요인 모형의 우수성을 확인함.
    \item ARIMA와 VAR은 전통적인 선형 모형으로 벤치마크 역할을 수행하지만, DDFM에 비해 상대적으로 높은 오차를 보임.
    \item DDFM의 비선형 인코더를 통한 요인 추출이 복잡한 거시경제 시계열의 패턴을 효과적으로 포착함을 확인함.
\end{itemize}

\subsubsection{시점별 성능 패턴}

\begin{itemize}
    \item \textbf{KOIPALL.G:} DDFM이 모든 시점(1-22개월)에서 가장 우수한 성능을 보임. 단기(1-6개월)에서 sMAE=10.33, 중기(7-12개월)에서 sMAE=10.18, 장기(13-22개월)에서 sMAE=10.05를 기록하여 시점에 관계없이 일관된 성능을 유지함. ARIMA(sMAE=24.57)와 VAR(sMAE=58.75)보다 현저히 낮은 오차를 보임. DDFM은 시점이 길어질수록 오차가 소폭 감소하는 안정적인 패턴을 보임.
    \item \textbf{KOEQUIPTE:} DDFM이 모든 시점에서 가장 우수한 성능을 보임. 단기(1-6개월)에서 sMAE=10.34, 중기(7-12개월)에서 sMAE=10.21, 장기(13-22개월)에서 sMAE=9.40를 기록하여 장기 예측에서 오히려 성능이 향상되는 특징을 보임. ARIMA(sMAE=15.44)와 VAR(sMAE=31.57)보다 현저히 낮은 오차를 보임. DDFM의 비선형 요인 추출이 장기 예측에서도 효과적임을 확인함.
    \item \textbf{KOWRCCNSE:} DDFM이 모든 시점에서 가장 우수한 성능을 보임. 단기(1-6개월)에서 sMAE=10.99, 중기(7-12개월)에서 sMAE=11.08, 장기(13-22개월)에서 sMAE=11.30을 기록하여 시점에 관계없이 안정적인 성능을 유지함. ARIMA(sMAE=17.72)와 VAR(sMAE=45.18)보다 현저히 낮은 오차를 보임. DDFM은 모든 시점에서 일관된 성능을 보이며, 시점이 길어져도 오차 증가가 제한적임.
\end{itemize}

\subsubsection{예측 실험에서의 모형 비교}

예측 실험에서는 ARIMA, VAR, DDFM이 평가되었으며, DFM은 수렴 실패로 인해 평가되지 않았음(표~\ref{tab:forecasting_results} 참조). 

\textbf{모형 간 비교}
\begin{itemize}
    \item \textbf{DDFM의 우수성:} DDFM이 세 대상 변수 모두에서 ARIMA와 VAR보다 현저히 우수한 성능을 보이며, 비선형 요인 모형의 이점을 확인함. DDFM은 단기, 중기, 장기 모든 시점에서 일관된 성능을 유지하며, 시점이 길어져도 오차 증가가 제한적임.
    \item \textbf{다변량 정보 활용:} VAR이 ARIMA 대비 우수한 성능을 보이며, 다변량 정보 활용의 이점을 보여줌. 그러나 DDFM에 비해서는 상대적으로 높은 오차를 보임.
    \item \textbf{모형 선택:} 예측 실험에서는 DDFM이 가장 우수한 성능을 보이며, 복잡한 거시경제 시계열의 패턴을 효과적으로 포착함. 다변량 정보를 활용할 수 있는 경우 DDFM이 가장 유리함.
\end{itemize}

\textbf{DFM과 DDFM의 활용}

DDFM은 예측 실험에서 세 대상 변수 모두에서 최고 성능을 보이며, ARIMA와 VAR 대비 현저한 성능 개선을 보임. DDFM의 비선형 인코더를 통한 요인 추출이 복잡한 거시경제 시계열의 패턴을 효과적으로 포착함을 확인함. Nowcasting 실험(섹션 3.2)에서는 DFM, DDFM, MAMBA 모형이 release date 마스킹을 효과적으로 처리할 수 있어 실제 운영 환경에 적합함을 확인함.


\subsection{Nowcasting 실험 결과}
\label{sec:results_nowcasting}

DFM, DDFM, MAMBA 모형을 활용하여 각 목표 월에 대해 4주 전, 1주 전 시점에서 예측을 수행한 결과를 제시함.

\textbf{주요 결과}

\begin{itemize}
    \item \textbf{생산 모형(전산업생산지수):} 세 모형 모두 유사한 정확도를 보임. DFM의 평균 오차는 1~8주 전 평균 0.9\%p, MAMBA는 0.8\%p로 소폭 우수함. MAMBA의 월별 전망값 변동이 DFM보다 작게 나타남.
    \item \textbf{투자 모형(설비투자지수):} 세 모형 모두 유사한 성능을 보임. DFM의 평균 오차는 1~8주 전 평균 6.3\%p, MAMBA는 6.6\%p로 DFM 대비 소폭 부진함. 투자 지수는 생산 지수에 비해 변동성이 크며 예측 오차가 큼.
\end{itemize}

\FloatBarrier
\begin{figure}[H]
\centering
\includegraphics[width=0.9\textwidth]{images/nowcast/production_nowcast_compare.png}
\caption{생산 모형(전산업생산지수) Nowcasting 비교: DFM, DDFM, MAMBA 모형의 예측값과 실제값 비교.}
\label{fig:production_nowcast_compare}
\end{figure}

\begin{figure}[H]
\centering
\includegraphics[width=0.9\textwidth]{images/nowcast/production_nowcast_ensemble.png}
\caption{생산 모형(전산업생산지수) Nowcasting 앙상블: 모형별 예측값과 앙상블 결과.}
\label{fig:production_nowcast_ensemble}
\end{figure}

\begin{figure}[H]
\centering
\includegraphics[width=0.9\textwidth]{images/nowcast/investment_nowcast_compare.png}
\caption{투자 모형(설비투자지수) Nowcasting 비교}
\label{fig:investment_nowcast_compare}
\end{figure}

\begin{figure}[H]
\centering
\includegraphics[width=0.9\textwidth]{images/nowcast/investment_nowcast_ensemble.png}
\caption{투자 모형(설비투자지수) Nowcasting 앙상블: 모형별 예측값과 앙상블 결과.}
\label{fig:investment_nowcast_ensemble}
\end{figure}
\FloatBarrier

\subsection{고빈도 데이터 실험 결과}
\label{sec:results_high_freq}

고빈도 데이터(주별 전력거래량, BSI)를 활용한 MIDAS-AR 및 XGBoost 모형의 예측 성능을 평가함.

\textbf{주요 결과}

\begin{itemize}
    \item \textbf{MIDAS-AR 모형:} 대부분의 vintage에서 AR(1) 대비 개선이 없거나 악화됨. $h4$(full month 정보)에서만 약 0.5\% RMSE 감소로 소폭 개선을 보임.
    \item \textbf{XGBoost 모형:} 대부분의 vintage에서 AR(1) 대비 성능 저하를 보이며, 과적합 경향이 나타남. 선형 모델이 더 안정적인 성능을 보임.
    \item \textbf{변수 중요도:} 1기 시차 종속변수($y_{t-1}$)가 가장 강력한 예측 변수이며, BSI는 정보 제공 측면에서 유의미하나 예측력 개선은 제한적임. 전력거래량은 한계적 기여에 머묾.
\end{itemize}

\FloatBarrier
\begin{figure}[H]
\centering
\begin{subfigure}[b]{0.48\textwidth}
\centering
\includegraphics[width=\textwidth]{images/midas/midasar_rmse.png}
\caption{전월대비 성장률}
\label{fig:midasar_rmse}
\end{subfigure}
\hfill
\begin{subfigure}[b]{0.48\textwidth}
\centering
\includegraphics[width=\textwidth]{images/midas/midasar_mom_rmse.png}
\caption{전년동월비}
\label{fig:midasar_mom_rmse}
\end{subfigure}

\caption{Vintage별 테스트 RMSE 비교: 전산업생산지수. AR(1)과 MIDAS-AR(1) 모형의 vintage별 예측 성능을 비교함.}
\label{fig:midasar_rmse_comparison}
\end{figure}
\FloatBarrier

상세 결과는 부록 표~\ref{tab:midasar_rmse_table}, 표~\ref{tab:midasar_rmse_yoy}, 표~\ref{tab:rmse-xgb}, 표~\ref{tab:xgb_rmse_yoy}를 참조함.
