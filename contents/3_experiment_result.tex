\section{실험 결과}
\label{sec:experiment_result}

\subsection{예측 실험 결과}
\label{sec:results_forecasting}


세 가지 대상 변수(생산: KOIPALL.G, 투자: KOEQUIPTE, 소비: KOWRCCNSE)에 대한 네 가지 예측 모형(ARIMA, VAR, DFM, DDFM)의 예측 성능을 비교함.

\begin{itemize}
    \item \textbf{실험 범위:} 1개월부터 22개월까지의 시점에 대해 수행
    \item \textbf{결과 제시:} 표~\ref{tab:forecasting_results}에는 모든 시점(1-22개월)에 대한 평균값을 제시
\end{itemize} 


\textbf{결과 요약}
\begin{itemize}
    \item 표~\ref{tab:forecasting_results}는 모형별 타겟별로 모든 시점(1-22개월)에 대한 평균 표준화된 MAE와 MSE를 제시함
    \item 각 셀은 해당 모형-타겟 조합의 평균 지표값을 나타내며, 각 지표에서 최소값(최고 성능)은 굵은 글씨로 표시됨
    \item 상세한 시점별 결과는 부록에 제시됨
\end{itemize}

\begin{table}[h]
\centering
\caption[Forecasting Results by Model-Horizon and Target-Metric]{Forecasting Results by Model-Horizon and Target-Metric\footnote{Experiments evaluate all horizons from 1 to 22 months (2024--01 to 2025--10), but table shows only selected horizons (1, 11, 22 months) for readability. Full results for all horizons are available in aggregated\_results.csv.}}
\\label{tab:forecasting_results}
\\begin{tabular}{lcccccc}
\\toprule
Model-Horizon & KOIPALL.G & KOIPALL.G & KOEQUIPTE & KOEQUIPTE & KOWRCCNSE & KOWRCCNSE \\\\
 & sMAE & sMSE & sMAE & sMSE & sMAE & sMSE \\\\
\\midrule
ARIMA-1 & N/A & N/A & 0.8734 & 0.7628 & N/A & N/A \\
ARIMA-11 & N/A & N/A & 2.0917 & 4.3751 & N/A & N/A \\
ARIMA-22 & N/A & N/A & 0.0846 & 0.0071 & N/A & N/A \\
VAR-1 & N/A & N/A & 0.2998 & 0.0899 & N/A & N/A \\
VAR-11 & N/A & N/A & 2.5679 & 6.5939 & N/A & N/A \\
VAR-22 & N/A & N/A & 0.1881 & 0.0354 & N/A & N/A \\
DFM-1 & N/A & N/A & 0.4890 & 0.2391 & N/A & N/A \\
DFM-11 & N/A & N/A & 2.2917 & 5.2519 & N/A & N/A \\
DFM-22 & N/A & N/A & 0.1139 & 0.0130 & N/A & N/A \\
DDFM-1 & N/A & N/A & 0.7574 & 0.5736 & N/A & N/A \\
DDFM-11 & N/A & N/A & 2.0233 & 4.0938 & N/A & N/A \\
DDFM-22 & N/A & N/A & 0.1545 & 0.0239 & N/A & N/A \\
\bottomrule
\end{tabular}
\end{table}


\textbf{전체 시점 평균 성능 (표~\ref{tab:forecasting_results})}
\begin{itemize}
    \item \textbf{KOIPALL.G:} DDFM이 가장 낮은 sMAE(0.6865, 21개 시점 평균)와 sMSE(0.61)를 보여 우수한 성능을 보임. VAR(sMAE=0.94, sMSE=1.11)도 양호한 성능을 보이며, DFM(sMAE=14.9689, sMSE=225.30)은 매우 높은 오차를 보여 KOIPALL.G에 대해서는 부적합함.
    \item \textbf{KOEQUIPTE:} DFM과 DDFM이 거의 동일한 성능을 보이며(sMAE: DFM=1.1439, DDFM=1.1441, 평균 차이 0.000187, 21개 시점; sMSE: DFM=2.115, DDFM=2.115, 차이 0.0003), VAR(sMAE=1.37, sMSE=2.97)이 상대적으로 높은 오차를 보임.
    \item \textbf{KOWRCCNSE:} VAR이 가장 낮은 sMAE(0.32)와 sMSE(0.20)를 보여 우수한 성능을 보이며, DDFM(sMAE=0.4961, sMSE=0.49)도 양호한 성능을 보임. DFM(sMAE=2.7848, sMSE=8.21)은 상대적으로 높은 오차를 보임.
\end{itemize}

\textbf{시점별 성능 패턴}
\begin{itemize}
    \item \textbf{KOIPALL.G:} DDFM은 단기(1-6개월)에서 매우 우수한 성능을 보이며, 장기(13-21개월)에서도 안정적임. 반면 DFM은 모든 시점에서 극단적으로 높은 오차를 보임.
    \item \textbf{KOWRCCNSE:} VAR은 단기에서 우수하나 일부 시점에서 오차가 급증하며, DDFM은 대부분의 시점에서 안정적임.
    \item \textbf{KOEQUIPTE:} DFM과 DDFM은 모든 시점에서 거의 동일한 성능을 보임.
\end{itemize}

\subsection{Nowcasting 실험 결과}

\textbf{Nowcasting 결과}
\begin{itemize}
    \item \textbf{생산 모형(전산업생산지수):} DFM, DDFM, MAMBA 세 모형 모두 유사한 정확도를 보였음. DFM의 nowcasting 평균 오차는 발표 8주전 1.2\%p, 4주전 0.6\%p, 1주전 0.6\%p이며 1~8주 전 오차 평균값은 0.9\%p임. DDFM과 MAMBA도 유사한 성능을 보였으며, MAMBA의 nowcasting 평균 오차는 발표 8주전 0.9\%p, 4주전 0.8\%p, 1주전 0.8\%p이며 1~8주 전 오차 평균값은 0.8\%p임. MAMBA의 월별 전망값 변동이 DFM 모형보다 작게 나타남.
    \item \textbf{투자 모형(설비투자지수):} DFM, DDFM, MAMBA 세 모형 모두 유사한 성능을 보였음. DFM의 nowcasting 평균 오차는 발표 8주전 6.4\%p, 4주전 6.3\%p, 1주전 6.3\%p이며 1~8주 전 오차 평균값은 6.3\%p임. MAMBA 모형의 nowcasting 성과가 DFM 대비 소폭 부진하였으며, 평균 절대 예측오차는 8주전 6.7\%p, 4주전 6.6\%p, 1주전 6.6\%p이며 1~8주 평균값은 6.6\%p로 DFM 대비 0.3\%p 증가함.
\end{itemize}

\begin{figure}[h]
\centering
\includegraphics[width=0.9\textwidth]{images/nowcast/production_nowcast_compare.png}
\caption{생산 모형(전산업생산지수) Nowcasting 비교: DFM, DDFM, MAMBA 모형의 예측값과 실제값 비교.}
\label{fig:production_nowcast_compare}
\end{figure}

\begin{figure}[h]
\centering
\includegraphics[width=0.9\textwidth]{images/nowcast/production_nowcast_ensemble.png}
\caption{생산 모형(전산업생산지수) Nowcasting 앙상블: 모형별 예측값과 앙상블 결과.}
\label{fig:production_nowcast_ensemble}
\end{figure}

\begin{figure}[h]
\centering
\includegraphics[width=0.9\textwidth]{images/nowcast/investment_nowcast_compare.png}
\caption{투자 모형(설비투자지수) Nowcasting 비교}
\label{fig:investment_nowcast_compare}
\end{figure}

\begin{figure}[h]
\centering
\includegraphics[width=0.9\textwidth]{images/nowcast/investment_nowcast_ensemble.png}
\caption{투자 모형(설비투자지수) Nowcasting 앙상블: 모형별 예측값과 앙상블 결과.}
\label{fig:investment_nowcast_ensemble}
\end{figure}

\subsection{고빈도 데이터 실험 결과}

고빈도 데이터(주별 전력거래량, BSI)를 활용한 MIDAS-AR 및 XGBoost 모형의 예측 성능을 보고함. 다변량 고차원 모형(DFM, DDFM)과는 달리, 고빈도 지표의 단일변수 예측에서의 활용 가능성을 탐색하기 위한 실험임.

\textbf{실험 설계}
\begin{itemize}
    \item \textbf{종속변수:} 월별 전산업생산지수(계절조정)의 전월대비 성장률 및 전년동월비
    \item \textbf{설명변수:} 주별 전력거래량(로그--STL 계절조정 후 주간 성장률), 월별 BSI(수준 및 전년동월비)
    \item \textbf{표본 분할:} Train(2002--2020년), Validation(2021--2022년), Test(2023--2024년)
    \item \textbf{Vintage:} h0(전월 말), h1--h4(당월 1--4주)
\end{itemize}

\textbf{MIDAS-AR 모형 결과}
\begin{itemize}
    \item \textbf{전월대비 성장률:} ADF 검정 결과 정상성 가정 가능. $h0$--$h3$에서 MIDAS-AR(1)의 RMSE가 AR(1)보다 약간 열악하며, $h4$에서만 약 0.5\% RMSE 감소로 소폭 개선. AR(1)만으로도 단기 예측력이 높으며, 주별 전력거래량 추가는 full month 정보($h4$)에서만 약간의 개선을 보임.
    \item \textbf{전년동월비:} ADF 검정 결과 정상성 가정 가능. $h0$, $h3$에서 MIDAS-AR(1)의 RMSE가 AR(1)보다 약 0.7\% 악화, $h1$에서 약 7.4\% 악화, $h2$에서 약 1.3\% 개선(크기 작음), $h4$에서 두 모형 RMSE 동일. 대부분의 vintage에서 AR(1) 대비 개선이 없거나 악화됨.
    \item 상세 결과는 부록 표~\ref{tab:midasar_rmse_table}, 표~\ref{tab:midasar_rmse_yoy} 참조.
\end{itemize}

\begin{figure}[htbp]
\centering
\begin{subfigure}[b]{0.48\textwidth}
\centering
\includegraphics[width=\textwidth]{images/midas/midasar_rmse.png}
\caption{전월대비 성장률}
\label{fig:midasar_rmse}
\end{subfigure}
\hfill
\begin{subfigure}[b]{0.48\textwidth}
\centering
\includegraphics[width=\textwidth]{images/midas/midasar_mom_rmse.png}
\caption{전년동월비}
\label{fig:midasar_mom_rmse}
\end{subfigure}

\caption{Vintage별 테스트 RMSE 비교: 전산업생산지수. AR(1)과 MIDAS-AR(1) 모형의 vintage별 예측 성능을 비교함.}
\label{fig:midasar_rmse_comparison}
\end{figure}

\textbf{XGBoost 비선형 확장 결과}
\begin{itemize}
    \item \textbf{모형 구성:} (1) 선형 ARX: AR(1) + 고빈도 feature의 선형 효과, (2) AR(1)+XGB\_residual: AR(1) 잔차에 대한 XGBoost 보정, (3) XGB-direct: $(y_{t-1}, x_{t,h})$를 입력으로 $y_t$ 직접 예측
    \item \textbf{전월대비 성장률:} ARX는 대부분 vintage에서 AR(1) 대비 비슷하거나 약간 열악하며, $h4$에서만 약 1.4\% RMSE 감소. AR(1)+XGB\_residual는 모든 vintage에서 RMSE가 AR(1)보다 7--11\% 증가하여 과적합 경향. XGB-direct는 대부분 vintage에서 AR(1) 대비 2.6--4.5\% 성능 저하, $h3$, $h4$에서 각각 약 0.3\%, 0.2\% 개선(크기 매우 작음).
    \item \textbf{전년동월비:} ARX는 모든 vintage에서 AR(1) 대비 RMSE 감소율이 음수(약 $-1.6\%\sim -6.4\%$). AR(1)+XGB\_residual는 모든 vintage에서 감소율이 약 $-2.3\%\sim -6.7\%$. XGB-direct는 $h0$에서만 AR(1) 대비 약 4.4\% RMSE 감소, $h1$--$h4$에서는 모두 음수(약 $-2.3\%\sim -8.6\%$).
    \item 상세 결과는 부록 표~\ref{tab:rmse-xgb}, 표~\ref{tab:xgb_rmse_yoy}, 표~\ref{tab:arx_bsi} 참조.
\end{itemize}

\textbf{변수 중요도 및 요약}
\begin{itemize}
    \item \textbf{전력거래량:} 대부분의 모형에서 계수·Gain 모두 작고 비유의, AR(1) 대비 RMSE 개선 거의 없음. 다양한 변환·모형에도 불구하고 한계적 기여에 머묾.
    \item \textbf{BSI:} 선형 ARX와 XGBoost에서 전월·동월 BSI 변수들의 계수 및 Gain이 상대적으로 큼. 인샘플 적합과 feature importance 관점에서 의미 있는 정보 제공. 다만 테스트 RMSE 기준으로는 AR(1) 대비 뚜렷한·일관된 예측력 개선까지는 이어지지 않음.
    \item \textbf{핵심 변수:} 두 종속변수 모두에서 가장 일관된 설명력은 1기 시차 종속변수 $y_{t-1}$에서 나옴. Nowcasting에서 핵심 변수는 $y_{t-1}$과 BSI 계열 변수이며, 전력 변수는 부차적 설명 변수로 정리됨.
\end{itemize}

