\section{실험 결과}
\label{sec:experiment_result}

본 섹션에서는 세 가지 실험(예측 실험, Nowcasting 실험, 고빈도 데이터 실험)의 결과를 제시함. 상세한 분석과 조기 경보 지수 구축 방법론은 다음 섹션에서 다룸.

\subsection{예측 실험 결과}
\label{sec:results_forecasting}

세 가지 대상 변수(생산: KOIPALL.G, 투자: KOEQUIPTE, 소비: KOWRCCNSE)에 대한 네 가지 예측 모형(ARIMA, VAR, DFM, DDFM)의 예측 성능을 1개월부터 22개월까지의 시점에 대해 평가함.

\begin{table}[h]
\centering
\caption[Forecasting Results by Model-Horizon and Target-Metric]{Forecasting Results by Model-Horizon and Target-Metric\footnote{Experiments evaluate all horizons from 1 to 22 months (2024--01 to 2025--10), but table shows only selected horizons (1, 11, 22 months) for readability. Full results for all horizons are available in aggregated\_results.csv.}}
\\label{tab:forecasting_results}
\\begin{tabular}{lcccccc}
\\toprule
Model-Horizon & KOIPALL.G & KOIPALL.G & KOEQUIPTE & KOEQUIPTE & KOWRCCNSE & KOWRCCNSE \\\\
 & sMAE & sMSE & sMAE & sMSE & sMAE & sMSE \\\\
\\midrule
ARIMA-1 & N/A & N/A & 0.8734 & 0.7628 & N/A & N/A \\
ARIMA-11 & N/A & N/A & 2.0917 & 4.3751 & N/A & N/A \\
ARIMA-22 & N/A & N/A & 0.0846 & 0.0071 & N/A & N/A \\
VAR-1 & N/A & N/A & 0.2998 & 0.0899 & N/A & N/A \\
VAR-11 & N/A & N/A & 2.5679 & 6.5939 & N/A & N/A \\
VAR-22 & N/A & N/A & 0.1881 & 0.0354 & N/A & N/A \\
DFM-1 & N/A & N/A & 0.4890 & 0.2391 & N/A & N/A \\
DFM-11 & N/A & N/A & 2.2917 & 5.2519 & N/A & N/A \\
DFM-22 & N/A & N/A & 0.1139 & 0.0130 & N/A & N/A \\
DDFM-1 & N/A & N/A & 0.7574 & 0.5736 & N/A & N/A \\
DDFM-11 & N/A & N/A & 2.0233 & 4.0938 & N/A & N/A \\
DDFM-22 & N/A & N/A & 0.1545 & 0.0239 & N/A & N/A \\
\bottomrule
\end{tabular}
\end{table}

\textbf{주요 결과 요약}

표~\ref{tab:forecasting_results}는 모형별 타겟별로 모든 시점(1-22개월)에 대한 평균 표준화된 MAE와 MSE를 제시함.

\subsubsection{전체 시점 평균 성능}

\begin{itemize}
    \item \textbf{KOIPALL.G:} VAR이 가장 우수한 성능(sMAE=0.85)을 보임. ARIMA도 양호한 성능(sMAE=1.02)을 보임. DFM과 DDFM은 예측 실험에서 평가되지 않음(표~\ref{tab:forecasting_results} 참조).
    \item \textbf{KOEQUIPTE:} VAR이 가장 우수한 성능(sMAE=0.85)을 보임. ARIMA도 양호한 성능(sMAE=1.06)을 보임. DFM과 DDFM은 예측 실험에서 평가되지 않음.
    \item \textbf{KOWRCCNSE:} VAR이 가장 우수한 성능(sMAE=0.83)을 보임. ARIMA도 양호한 성능(sMAE=1.04)을 보임. DFM과 DDFM은 예측 실험에서 평가되지 않음.
\end{itemize}

\textbf{벤치마크 모형(ARIMA, VAR)}
\begin{itemize}
    \item \textbf{KOIPALL.G:} VAR이 가장 낮은 sMAE(0.85)와 sMSE(0.78)를 보여 우수한 성능을 보임. ARIMA(sMAE=1.02, sMSE=1.19)도 양호한 성능을 보이지만 VAR에 비해 상대적으로 높은 오차를 보임. VAR은 ARIMA 대비 약 16.7\%의 성능 개선을 보임.
    \item \textbf{KOEQUIPTE:} VAR이 가장 낮은 sMAE(0.85)와 sMSE(0.80)를 보여 우수한 성능을 보임. ARIMA(sMAE=1.06, sMSE=1.29)는 VAR에 비해 상대적으로 높은 오차를 보임. VAR은 ARIMA 대비 약 19.8\%의 성능 개선을 보임.
    \item \textbf{KOWRCCNSE:} VAR이 가장 낮은 sMAE(0.83)와 sMSE(0.81)를 보여 우수한 성능을 보임. ARIMA(sMAE=1.04, sMSE=1.25)는 VAR에 비해 상대적으로 높은 오차를 보임. VAR은 ARIMA 대비 약 20.2\%의 성능 개선을 보이며, 세 타겟 중 가장 큰 개선률을 보임.
    \item ARIMA와 VAR은 전통적인 선형 모형으로 벤치마크 역할을 수행함. 일부 대상 변수에서 양호한 성능을 보이지만, nowcasting에서는 release date 마스킹 처리의 구조적 한계로 제한적임.
\end{itemize}

\textbf{동적요인모형(DFM, DDFM)}
\begin{itemize}
    \item \textbf{예측 실험:} 표~\ref{tab:forecasting_results}에 따르면, DFM과 DDFM은 예측 실험에서 평가되지 않았음(N/A). 이는 주/월 혼합 주기 처리의 복잡성 또는 실험 설계상의 제약으로 인한 것으로 보임.
    \item \textbf{Nowcasting 실험:} DFM과 DDFM은 Nowcasting 실험(섹션 3.2)에서 평가되었으며, release date 마스킹을 효과적으로 처리 가능하며, 다변량 시계열 간 공통 패턴을 포착할 수 있음. Nowcasting 실험 결과는 섹션 3.2를 참조함.
\end{itemize}

\textbf{대상 변수별 최적 모형}
\begin{itemize}
    \item 예측 실험에서 평가된 모형(ARIMA, VAR) 중에서는 VAR이 세 대상 변수 모두에서 최고 성능을 보임.
    \item 각 모형은 대상 변수에 따라 매우 다른 성능 특성을 보이며, 단일 모형이 모든 대상 변수에서 최고 성능을 보이지는 않음.
    \item 대상 변수와 시계열 특성에 따라 적절한 모형을 선택하는 것이 중요함.
\end{itemize}

\subsubsection{시점별 성능 패턴}

\begin{itemize}
    \item \textbf{KOIPALL.G:} VAR은 모든 시점(1-22개월)에서 ARIMA보다 낮은 오차를 보이며, 특히 단기(1-6개월)에서 매우 우수한 성능을 보임(sRMSE=0.685). 중기(7-12개월)에서도 ARIMA(sRMSE=1.054)보다 낮은 오차를 보임(sRMSE=0.850). 장기(13-22개월)에서도 ARIMA(sRMSE=1.096)보다 낮은 오차를 보임(sRMSE=0.881). ARIMA는 전반적으로 안정적이지만 VAR에 비해 높은 오차를 보이며, 시점이 길어질수록 오차가 증가하는 경향을 보임.
    \item \textbf{KOEQUIPTE:} VAR은 모든 시점에서 ARIMA보다 낮은 오차를 보이며, 특히 단기(1-6개월)에서 우수한 성능을 보임(sRMSE=0.779). 중기(7-12개월)에서도 ARIMA(sRMSE=1.089)보다 낮은 오차를 보임(sRMSE=0.864). 장기(13-22개월)에서도 ARIMA(sRMSE=1.140)보다 낮은 오차를 보임(sRMSE=0.905). ARIMA는 장기로 갈수록 오차가 증가하는 경향을 보이며, VAR은 모든 시점에서 일관된 성능을 유지함.
    \item \textbf{KOWRCCNSE:} VAR은 모든 시점에서 ARIMA보다 낮은 오차를 보이며, 특히 단기(1-6개월)에서 매우 우수한 성능을 보임(sRMSE=0.634). 중기(7-12개월)에서도 ARIMA(sRMSE=1.061)보다 낮은 오차를 보임(sRMSE=0.842). 장기(13-22개월)에서도 ARIMA(sRMSE=1.106)보다 낮은 오차를 보임(sRMSE=0.891). ARIMA는 전반적으로 안정적이지만 VAR에 비해 높은 오차를 보이며, VAR은 특히 단기 예측에서 뛰어난 성능을 보임.
\end{itemize}

\subsubsection{예측 실험에서의 모형 비교}

예측 실험에서는 ARIMA와 VAR만 평가되었으며, DFM과 DDFM은 평가되지 않았음(표~\ref{tab:forecasting_results} 참조). 

\textbf{ARIMA vs VAR 비교}
\begin{itemize}
    \item \textbf{다변량 정보 활용:} VAR이 세 대상 변수 모두에서 ARIMA 대비 우수한 성능을 보이며, 이는 다변량 정보 활용의 이점을 보여줌.
    \item \textbf{시점별 성능:} VAR은 단기, 중기, 장기 모든 시점에서 ARIMA보다 낮은 오차를 보이며, 일관된 성능을 유지함.
    \item \textbf{모형 선택:} 예측 실험에서는 VAR이 ARIMA 대비 우수한 성능을 보이며, 다변량 정보를 활용할 수 있는 경우 VAR이 유리함.
\end{itemize}

\textbf{DFM과 DDFM의 활용}

DFM과 DDFM은 예측 실험에서는 평가되지 않았으나, Nowcasting 실험(섹션 3.2)에서 평가되었음. Nowcasting 실험에서는 DFM, DDFM, MAMBA 모형이 release date 마스킹을 효과적으로 처리할 수 있어 실제 운영 환경에 적합함을 확인함.

\begin{figure}[h]
\centering
\includegraphics[width=0.9\textwidth]{images/horizon_trend.png}
\caption{시점별 예측 성능 추이 (1-22개월)}
\label{fig:horizon_trend}
\end{figure}

모형별 타겟별 예측 성능 히트맵은 부록 B(그림~\ref{fig:appendix_accuracy_heatmap})를 참조함.

\subsection{Nowcasting 실험 결과}
\label{sec:results_nowcasting}

DFM, DDFM, MAMBA 모형을 활용하여 각 목표 월에 대해 4주 전, 1주 전 시점에서 예측을 수행한 결과를 제시함.

\textbf{주요 결과}

\begin{itemize}
    \item \textbf{생산 모형(전산업생산지수):} 세 모형 모두 유사한 정확도를 보임. DFM의 평균 오차는 1~8주 전 평균 0.9\%p, MAMBA는 0.8\%p로 소폭 우수함. MAMBA의 월별 전망값 변동이 DFM보다 작게 나타남.
    \item \textbf{투자 모형(설비투자지수):} 세 모형 모두 유사한 성능을 보임. DFM의 평균 오차는 1~8주 전 평균 6.3\%p, MAMBA는 6.6\%p로 DFM 대비 소폭 부진함. 투자 지수는 생산 지수에 비해 변동성이 크며 예측 오차가 큼.
\end{itemize}

\FloatBarrier
\begin{figure}[H]
\centering
\includegraphics[width=0.9\textwidth]{images/nowcast/production_nowcast_compare.png}
\caption{생산 모형(전산업생산지수) Nowcasting 비교: DFM, DDFM, MAMBA 모형의 예측값과 실제값 비교.}
\label{fig:production_nowcast_compare}
\end{figure}

\begin{figure}[H]
\centering
\includegraphics[width=0.9\textwidth]{images/nowcast/production_nowcast_ensemble.png}
\caption{생산 모형(전산업생산지수) Nowcasting 앙상블: 모형별 예측값과 앙상블 결과.}
\label{fig:production_nowcast_ensemble}
\end{figure}

\begin{figure}[H]
\centering
\includegraphics[width=0.9\textwidth]{images/nowcast/investment_nowcast_compare.png}
\caption{투자 모형(설비투자지수) Nowcasting 비교}
\label{fig:investment_nowcast_compare}
\end{figure}

\begin{figure}[H]
\centering
\includegraphics[width=0.9\textwidth]{images/nowcast/investment_nowcast_ensemble.png}
\caption{투자 모형(설비투자지수) Nowcasting 앙상블: 모형별 예측값과 앙상블 결과.}
\label{fig:investment_nowcast_ensemble}
\end{figure}
\FloatBarrier

\subsection{고빈도 데이터 실험 결과}
\label{sec:results_high_freq}

고빈도 데이터(주별 전력거래량, BSI)를 활용한 MIDAS-AR 및 XGBoost 모형의 예측 성능을 평가함.

\textbf{주요 결과}

\begin{itemize}
    \item \textbf{MIDAS-AR 모형:} 대부분의 vintage에서 AR(1) 대비 개선이 없거나 악화됨. $h4$(full month 정보)에서만 약 0.5\% RMSE 감소로 소폭 개선을 보임.
    \item \textbf{XGBoost 모형:} 대부분의 vintage에서 AR(1) 대비 성능 저하를 보이며, 과적합 경향이 나타남. 선형 모델이 더 안정적인 성능을 보임.
    \item \textbf{변수 중요도:} 1기 시차 종속변수($y_{t-1}$)가 가장 강력한 예측 변수이며, BSI는 정보 제공 측면에서 유의미하나 예측력 개선은 제한적임. 전력거래량은 한계적 기여에 머묾.
\end{itemize}

\FloatBarrier
\begin{figure}[H]
\centering
\begin{subfigure}[b]{0.48\textwidth}
\centering
\includegraphics[width=\textwidth]{images/midas/midasar_rmse.png}
\caption{전월대비 성장률}
\label{fig:midasar_rmse}
\end{subfigure}
\hfill
\begin{subfigure}[b]{0.48\textwidth}
\centering
\includegraphics[width=\textwidth]{images/midas/midasar_mom_rmse.png}
\caption{전년동월비}
\label{fig:midasar_mom_rmse}
\end{subfigure}

\caption{Vintage별 테스트 RMSE 비교: 전산업생산지수. AR(1)과 MIDAS-AR(1) 모형의 vintage별 예측 성능을 비교함.}
\label{fig:midasar_rmse_comparison}
\end{figure}
\FloatBarrier

상세 결과는 부록 표~\ref{tab:midasar_rmse_table}, 표~\ref{tab:midasar_rmse_yoy}, 표~\ref{tab:rmse-xgb}, 표~\ref{tab:xgb_rmse_yoy}를 참조함.
