\section{서론}

거시경제 변수의 정확한 예측은 정책 의사결정과 기업의 전략적 계획 수립에 중요함. 생산, 투자, 소비 지표는 경제 활동의 핵심을 나타내며, 실시간 평가가 필요함. 그러나 분기 GDP와 같은 주요 지표는 분기 종료 후 약 한 달이 지나야 공식 발표되므로, 실시간 경제 상황을 평가하고 시의적절한 정책 대응을 하기 어려움.

이에 따라 고빈도 데이터를 활용한 나우캐스팅 기법이 주목받고 있음 \cite{bok2017macroeconomic}. 나우캐스팅은 공식 통계가 발표되기 전에 다양한 고빈도 지표를 사용하여 현재의 거시경제 변수를 추정하는 기법임. 신속한 정책 대응이 필요한 위기 상황에서 그 중요성이 부각됨.

본 연구는 동적요인모형(Dynamic Factor Model, DFM)과 딥러닝 모형을 사용하여 세 가지 주요 한국 거시경제 지표에 대한 나우캐스팅을 수행함: 생산(전산업생산지수: KOIPALL.G), 투자(설비투자지수: KOEQUIPTE), 소비(도소매판매액: KOWRCCNSE). 네 가지 예측 모형(ARIMA, VAR, DFM, 심층 동적요인모형 DDFM)의 성능을 22개 예측 수평선(1개월부터 22개월까지)에서 비교하며, 각 수평선에 대한 지표를 평균하여 최종 성능 지표로 사용함. 표에는 가독성을 위해 1개월, 11개월, 22개월 수평선의 결과만 제시함. 

훈련 기간은 1985-2019년으로 설정하여 COVID-19 팬데믹 시기(2020-2023)를 제외하고, 예측 및 나우캐스팅 기간은 2024-2025년으로 설정하여 COVID-19 이후 경제 구조 변화 환경에서의 모형 성능을 평가함. 이는 데이터 누수를 방지하고 모형의 실제 예측 성능을 공정하게 평가하기 위한 설계임.

\subsection{선행연구 검토}

동적요인모형(DFM)은 많은 시계열에서 공통 요인을 추출하여 차원을 축소하고 혼합주기 데이터를 처리하는 기법임 \cite{stock2002forecasting}. 심층 동적요인모형(DDFM)은 오토인코더 기반 아키텍처를 사용하여 복잡한 요인 구조를 학습함 \cite{andreini2020deep, kim2024deep}. 혼합주기 데이터 처리를 위해 텐트 커널(tent kernel) 집계 방법이 사용되며, 집계 기간 중간에 가까운 관측값에 더 큰 가중치를 부여함 \cite{mariano2003new}.
