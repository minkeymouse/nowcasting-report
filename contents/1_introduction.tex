\section{서론}

\subsection{경제 시계열 예측의 중요성 부각}

\subsubsection{경제 정책 결정과 기업 경영 전략 수립에 있어 거시경제 변수의 정확한 예측은 핵심적인 역할을 수행하고 있음}
\begin{itemize}
    \item 특히 국내총생산(GDP), 소비, 투자와 같은 주요 거시경제 지표는 정부의 통화정책 및 재정정책 결정, 기업의 투자 의사결정, 그리고 금융기관의 리스크 관리에 직접적인 영향을 미치는 것으로 나타남 \cite{sims1986forecasting}
    \item 정책 결정자들은 경제 전망을 바탕으로 통화정책 금리, 재정지출 규모, 규제 정책 등을 결정하고 있으며, 이러한 결정들은 수백만 명의 경제 주체에 영향을 미치고 있음
    \item 따라서 예측의 정확도는 단순한 통계적 문제를 넘어 사회경제적 파급효과를 가진 중요한 이슈로 평가됨
    \item 또한 기업들은 경제 전망을 바탕으로 투자 계획, 고용 계획, 재고 관리 등을 수립하고 있으며, 금융기관들은 리스크 관리와 자산 배분 결정에 경제 전망을 활용하고 있음
\end{itemize}

\subsubsection{전통적으로 이러한 거시경제 변수들은 공식 통계가 발표되기까지 상당한 시차가 존재함}
\begin{itemize}
    \item 특히 분기별 GDP의 경우 해당 분기가 종료된 후 약 한 달 이상이 지나야 공식 발표가 이루어지는 상황임
    \item 이러한 정보의 지연은 실시간 경제 상황 파악과 신속한 정책 대응을 어렵게 만드는 것으로 나타남
    \item 예를 들어, 2020년 COVID-19 팬데믹 초기에는 경제 상황이 급격히 변화하였으나, 공식 통계 발표까지의 시차로 인해 정책 대응이 지연될 수 있었음
\end{itemize}

\subsubsection{이에 따라 최근 관측 가능한 고빈도 데이터를 활용하여 현재 시점의 경제 상황을 추정하는 나우캐스팅(nowcasting) 기법이 주목받고 있음 \cite{bok2017macroeconomic}}
\begin{itemize}
    \item 나우캐스팅은 공식 통계가 발표되기 전에 다양한 고빈도 지표(월간 지표, 주간 지표, 일간 지표 등)를 활용하여 현재 시점의 거시경제 변수를 추정하는 기법임
    \item 이러한 접근법은 특히 경제 위기 상황에서 신속한 정책 대응이 필요한 경우 그 중요성이 더욱 부각되고 있음
    \item 나우캐스팅을 통해 정책 결정자들은 공식 통계 발표 전에 경제 상황을 파악하고, 필요시 선제적인 정책 조치를 취할 수 있음
\end{itemize}

\subsection{COVID-19 이후 DFM 예측력 감소와 활용 가능한 데이터 증가}

\subsubsection{2020년 COVID-19 팬데믹 이후 전 세계 경제는 전례 없는 불확실성과 급격한 변화를 경험한 것으로 나타남}
\begin{itemize}
    \item 이러한 구조적 변화는 기존 예측 모델들의 성능에 심각한 도전을 제기한 것으로 평가됨
    \item 특히 동적 요인 모델(Dynamic Factor Model, DFM)과 같은 전통적인 시계열 예측 모델들은 안정적인 경제 환경에서 개발되었기 때문에, 팬데믹으로 인한 급격한 구조 변화를 포착하는 데 한계를 보인 것으로 나타남 \cite{huber2020nowcasting, schorfheide2020nowcasting}
    \item Schorfheide와 Song (2020)의 연구에서도 혼합 빈도 VAR을 활용한 나우캐스팅에서 팬데믹 기간 동안 예측 성능 저하가 관찰되었으며, 비모수적 접근법이 구조적 변화 시기에 더 효과적일 수 있음을 제시함 \cite{schorfheide2020nowcasting}
\end{itemize}

\subsubsection{반면, COVID-19 이후 디지털화의 가속화와 함께 활용 가능한 고빈도 데이터의 양과 다양성은 크게 증가한 것으로 나타남}
\begin{itemize}
    \item 실시간 거래 데이터, 모바일 위치 데이터, 온라인 검색 트렌드, 소셜 미디어 데이터 등 다양한 비전통적 데이터 소스가 등장하였으며, 이러한 데이터들은 경제 활동의 실시간 변화를 반영할 수 있는 잠재력을 보유하고 있는 것으로 평가됨 \cite{lewis2020measuring}
\end{itemize}

\subsubsection{이러한 환경 변화는 두 가지 중요한 연구 질문을 제기하고 있음}
\begin{itemize}
    \item 첫째, 전통적인 DFM 모델의 예측 성능을 개선하기 위한 방법은 무엇인가?
    \item 둘째, 새로 등장한 고빈도 데이터를 효과적으로 활용하여 예측 정확도를 향상시킬 수 있는가?
\end{itemize}

\subsection{다양한 시계열 모형의 활용 현황}

\subsubsection{거시경제 변수 예측을 위한 다양한 시계열 모형들이 개발되어 왔음}
\begin{itemize}
    \item 전통적인 통계 모형으로는 자기회귀 통합 이동평균 모형(ARIMA), 벡터 자기회귀 모형(VAR), 벡터 오차수정 모형(VECM) 등이 있음 \cite{johansen1988statistical}
    \item 이러한 모형들은 선형 관계와 정상성 가정에 기반하고 있어, 복잡한 비선형 관계를 포착하는 데 한계가 있는 것으로 나타남
\end{itemize}

\subsubsection{최근 머신러닝 기법의 발전과 함께 트리 기반 앙상블 모형이 시계열 예측에 널리 활용되고 있음}
\begin{itemize}
    \item 트리 기반 앙상블 모형인 XGBoost와 LightGBM은 비선형 관계를 효과적으로 학습할 수 있으며, 특징 공학을 통해 다양한 변수 간 상호작용을 포착할 수 있는 것으로 평가됨
\end{itemize}

\subsubsection{딥러닝 기반 시계열 예측 모형으로는 DeepAR과 Temporal Fusion Transformer (TFT)가 대표적임}
\begin{itemize}
    \item DeepAR은 자기회귀 순환 신경망을 활용한 확률적 예측 모형으로, 시계열의 장기 의존성을 효과적으로 모델링할 수 있는 것으로 나타남 \cite{salinas2020deepar, lim2021temporal}
    \item TFT는 어텐션 메커니즘을 활용하여 시계열의 시간적 패턴과 외생 변수 간의 관계를 동시에 학습하는 모형임
    \item 또한 상태 공간 모형을 딥러닝에 접목한 Deep State Space Models도 시계열 예측에 활용되고 있음 \cite{rangapuram2018deep}
\end{itemize}

\subsubsection{동적 요인 모형(DFM)은 많은 시계열에서 공통 요인을 추출하여 차원을 축소하고, 혼합 빈도 데이터를 효과적으로 처리할 수 있는 장점을 가지고 있음}
\begin{itemize}
    \item Stock과 Watson (2002)은 주성분 분석을 활용한 요인 추출 방법이 고차원 시계열 데이터의 차원 축소에 효과적임을 보여주었으며, 이를 통해 거시경제 변수 예측의 정확도를 향상시킬 수 있음을 입증함 \cite{stock2002forecasting}
    \item 최근에는 딥러닝 기법을 DFM에 접목한 심층 동적 요인 모형(Deep Dynamic Factor Model, DDFM)이 제안되었으며, 변분 자기인코더를 활용하여 비선형 요인 구조를 학습할 수 있음 \cite{andreini2020deep}
    \item 한국 거시경제 데이터를 활용한 최근 연구에서도 DDFM이 DFM보다 우수한 성능을 보였으며, 특히 변동성이 큰 거시경제 변수에 대해서는 비선형 관계를 포착할 수 있는 DDFM의 장점이 두드러짐 \cite{kim2024deep}
\end{itemize}

\subsection{연구 목적 및 기여도}

\subsubsection{본 연구는 고빈도 데이터를 활용하여 한국의 주요 거시경제 변수(GDP, 소비, 투자)를 예측하는 다양한 모형들의 성능을 체계적으로 비교 분석하는 것을 목적으로 함}
\begin{itemize}
    \item 9개의 다양한 예측 모형(ARIMA, VAR, VECM, DeepAR, TFT, XGBoost, LightGBM, DFM, DDFM)을 동일한 데이터셋과 평가 기준으로 비교하여 각 모형의 장단점을 분석함
    \item 1일, 7일, 28일의 다양한 예측 기간에 대한 표준화된 성능 지표(MSE, MAE, RMSE)를 계산하여 단기 및 중기 예측 성능을 평가함
    \item 마스킹된 데이터를 활용한 백테스팅을 통해 DFM과 DDFM의 나우캐스팅 성능을 비교함
    \item DFM과 DDFM에 대한 Ablation study를 수행하여 각 하이퍼파라미터가 모형 성능에 미치는 영향을 분석함
\end{itemize}

\subsubsection{본 연구의 기여도는 다음과 같음}
\begin{itemize}
    \item 첫째, 다양한 예측 모형을 체계적으로 비교함으로써 한국 거시경제 데이터에 대한 각 모형의 상대적 성능을 제시함
    \item 둘째, 고빈도 데이터를 활용한 나우캐스팅 프레임워크를 구축하여 실무에 활용 가능한 예측 시스템을 제안함
    \item 셋째, 딥러닝 기반 DDFM의 효과성을 검증하고, 기존 DFM 대비 개선된 성능을 보여줌
    \item 이러한 연구 결과는 정책 결정자와 실무진에게 실용적인 예측 도구를 제공할 수 있을 것으로 기대됨
\end{itemize}
