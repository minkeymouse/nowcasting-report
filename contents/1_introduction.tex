\section{서론}

거시경제 변수의 정확한 예측은 정책 의사결정과 기업의 전략적 계획 수립에 중요함. 
\begin{itemize}
    \item 생산, 투자, 소비 지표는 경제 활동의 핵심을 나타냄
    \item 분기 GDP와 같은 주요 지표는 분기 종료 후 약 한 달이 지나야 공식 발표됨
    \item 실시간 경제 상황 평가와 시의적절한 정책 대응의 어려움
\end{itemize}

고빈도 데이터를 활용한 나우캐스팅 기법이 주목받고 있음 \cite{bok2017macroeconomic}:
\begin{itemize}
    \item 공식 통계 발표 전 다양한 고빈도 지표를 사용하여 현재 거시경제 변수 추정
    \item 신속한 정책 대응이 필요한 위기 상황에서 중요성 부각
\end{itemize}

본 연구의 목적:
\begin{itemize}
    \item \textbf{대상 변수:} 생산(KOIPALL.G), 투자(KOEQUIPTE), 소비(KOWRCCNSE)
    \item \textbf{모형:} ARIMA, VAR, DFM, DDFM (4개 모형 비교)
    \item \textbf{평가:} 22개 예측 수평선(1--22개월), 표에는 1, 11, 22개월만 제시
    \item \textbf{훈련 기간:} 1985--2019년 (COVID-19 시기 제외)
    \item \textbf{예측 기간:} 2024--2025년 (COVID-19 이후 구조 변화 환경 평가)
\end{itemize}

\subsection{선행연구 검토}

동적요인모형(DFM)은 많은 시계열에서 공통 요인을 추출하여 차원을 축소하고 혼합주기 데이터를 처리하는 기법임 \cite{stock2002forecasting}. DFM은 state-space 형태로 표현되며, Kalman filter를 사용하여 추론할 수 있어 혼합주기(mixed frequency) 데이터와 비동기적 데이터 발표를 자연스럽게 처리할 수 있음. 이는 nowcasting에 특히 적합한 특성으로, 실시간 데이터 흐름의 불규칙성(jagged edges)을 효과적으로 다룰 수 있음 \cite{banbura2012nowcasting}. DFM의 구조적 특성으로 인해 release date 기반 데이터 마스킹을 효과적으로 처리할 수 있어, 실제 운영 환경에서 특정 시점에 사용 가능한 데이터만을 사용하여 예측하는 nowcasting 시나리오에 적합함.

심층 동적요인모형(DDFM)은 오토인코더 기반 아키텍처를 사용하여 복잡한 비선형 요인 구조를 학습함 \cite{andreini2020deep}. DDFM은 전통적인 DFM의 선형 가정을 완화하고, 딥러닝을 통해 대규모 데이터셋에서 더 유연한 요인 추출이 가능함. 특히 수백 개의 거시경제 변수를 포함한 혼합주기 데이터에서도 효과적으로 작동함.
