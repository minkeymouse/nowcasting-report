\section{서론}

거시경제 변수의 정확한 예측은 정책 의사결정과 기업의 전략적 계획 수립에 중요함. 생산, 투자, 소비 지표는 경제 활동의 핵심을 나타내며, 실시간 평가가 필요함. 그러나 분기 GDP와 같은 주요 지표는 분기 종료 후 약 한 달이 지나야 공식 발표되므로, 실시간 경제 상황을 평가하고 시의적절한 정책 대응을 하기 어려움.

이에 따라 고빈도 데이터를 활용한 나우캐스팅 기법이 주목받고 있음 \cite{bok2017macroeconomic}. 나우캐스팅은 공식 통계가 발표되기 전에 다양한 고빈도 지표를 사용하여 현재의 거시경제 변수를 추정하는 기법임. 신속한 정책 대응이 필요한 위기 상황에서 그 중요성이 부각됨.

본 연구는 동적요인모형(Dynamic Factor Model, DFM)과 딥러닝 모형을 사용하여 세 가지 주요 한국 거시경제 지표에 대한 나우캐스팅을 수행함: 생산(전산업생산지수: KOIPALL.G), 투자(설비투자지수: KOEQUIPTE), 소비(도소매판매액: KOWRCCNSE). 네 가지 예측 모형(ARIMA, VAR, DFM, 심층 동적요인모형 DDFM)의 성능을 3개 예측 수평선(1일, 7일, 28일)에서 비교하며, 각 수평선에 대한 지표를 평균하여 최종 성능 지표로 사용함.

\subsection{선행연구 검토}

\subsubsection{동적요인모형 (Dynamic Factor Model)}

동적요인모형(DFM)은 많은 시계열에서 공통 요인을 추출하여 차원을 축소하고 혼합주기 데이터를 처리하는 기법임 \cite{stock2002forecasting}. Stock and Watson (2002)은 주성분 분석을 활용한 대규모 예측 변수 집합에서의 예측 방법을 다룸. DFM은 관측된 시계열을 소수의 공통 요인과 특이 요인으로 분해하여, 고차원 시계열 데이터의 차원 축소와 예측 성능 향상을 목표로 할 수 있음.

DFM의 핵심 아이디어는 많은 거시경제 변수들이 소수의 공통 요인(common factors)에 의해 주도된다는 관찰에 기반함. 이러한 공통 요인은 경기 순환, 인플레이션 압력, 금융 조건 등과 같은 거시경제의 주요 동력을 포착할 수 있음. DFM은 이러한 요인을 추출하고, 요인의 동적 진화를 모델링하여 미래 값을 예측함.

\subsubsection{심층 동적요인모형 (Deep Dynamic Factor Model)}

심층 동적요인모형(DDFM)은 오토인코더 기반 아키텍처를 사용하여 복잡한 요인 구조를 학습하는 딥러닝 기법임 \cite{andreini2020deep}. Andreini et al. (2020)은 비선형 인코더를 사용하는 DDFM을 다룸. DDFM은 다음과 같은 특징을 가질 수 있음:

\begin{itemize}
    \item \textbf{비선형 인코더:} 전통적인 DFM의 선형 요인 적재 행렬을 비선형 신경망 인코더로 대체하여 복잡한 비선형 관계를 학습할 수 있음.
    \item \textbf{오토인코더 구조:} 인코더-디코더 아키텍처를 사용하여 입력 시계열을 저차원 요인 공간으로 매핑하고, 이를 다시 재구성함으로써 요인을 학습함.
    \item \textbf{동적 요인 모델링:} 추출된 요인은 여전히 동적 상태 공간 모델을 따르며, 칼만 필터를 통해 추정됨.
\end{itemize}

한국 거시경제 변수에 대한 DDFM 적용 연구도 최근에 수행됨 \cite{kim2024deep}.

\subsubsection{혼합주기 데이터 처리: 텐트 커널 방법}

혼합주기 데이터 처리를 위해 텐트 커널(tent kernel) 집계 방법이 사용됨 \cite{mariano2003new}. Mariano and Murasawa (2003)는 집계 기간 중간에 가까운 관측값에 더 큰 가중치를 부여하는 텐트 커널 접근법을 다룸. 이 방법은 분기 데이터와 월간 데이터를 결합할 때 유용함.

텐트 커널 방법의 핵심은 다음과 같음: 분기 데이터를 월간 요인에 매핑할 때, 분기 중간 시점에 가까운 월간 관측값에 더 큰 가중치를 부여함. 예를 들어, 1분기(Q1) 데이터를 월간 요인에 매핑할 때, 2월 관측값에 가장 큰 가중치를 부여하고, 1월과 3월 관측값에는 상대적으로 작은 가중치를 부여함. 이는 분기 데이터가 분기 전체 기간에 걸쳐 균등하게 분포되어 있다는 가정보다 더 현실적임.

텐트 커널의 수학적 표현은 다음과 같음:
\begin{align}
w_t = \begin{cases}
1 - \frac{|t - t_{mid}|}{h} & \text{if } |t - t_{mid}| \leq h \\
0 & \text{otherwise}
\end{cases}
\end{align}

여기서 $t_{mid}$는 집계 기간의 중간 시점이고, $h$는 반폭(half-width)임. 이 방법은 더 느린 주기의 시계열을 더 빠른 주기의 잠재 요인에 연결할 때 유용함.
