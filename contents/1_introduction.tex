\section{Introduction}

Accurate forecasting of macroeconomic variables is crucial for policy decision-making and corporate strategic planning. In particular, production, investment, and consumption indicators represent the core of economic activity, and real-time assessment is essential. However, key indicators such as quarterly GDP are officially released only after approximately one month following the end of the quarter, making it difficult to assess the real-time economic situation and respond with timely policy measures.

Accordingly, nowcasting techniques utilizing high-frequency data have gained attention \cite{bok2017macroeconomic}. Nowcasting is a technique that estimates current macroeconomic variables using various high-frequency indicators before official statistics are released. Its importance is particularly highlighted in crisis situations where rapid policy response is needed.

This study constructs a nowcasting system using Dynamic Factor Models (DFM) and deep learning models to forecast three key Korean macroeconomic indicators: production (Industrial Production Index, All Industries: KOIPALL.G), investment (Equipment Investment Index: KOEQUIPTE), and consumption (Wholesale and Retail Trade Sales: KOWRCCNSE). We compare the performance of four forecasting models: ARIMA, VAR, DFM, and Deep Dynamic Factor Model (DDFM) across three forecast horizons (1, 7, and 28 days).
