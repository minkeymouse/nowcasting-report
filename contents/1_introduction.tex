\section{서론}

거시경제 변수의 정확한 예측은 정책 의사결정과 기업의 전략적 계획 수립에 중요함.

본 연구의 목적:
\begin{itemize}
    \item 대상 변수: 생산(KOIPALL.G), 투자(KOEQUIPTE), 소비(KOWRCCNSE)
    \item 모형: ARIMA, VAR, DFM, DDFM (4개 모형 비교)
    \item 평가: 22개 예측 시점(1--22개월), 표에는 1, 11, 22개월만 제시
    \item 훈련 기간: 1985--2019년 (COVID-19 시기 제외)
    \item 예측 기간: 2024--2025년 (COVID-19 이후 구조 변화 환경 평가)
\end{itemize}

\subsection{선행연구 검토}

동적요인모형(DFM)은 많은 시계열에서 공통 요인을 추출해 소수의 동태적 요인으로 설명하는 대표적 차원축소 기법으로, 관측식과 상태식을 갖는 state-space 형태를 취함 \cite{stock2002forecasting}. 대규모 이질적 거시 지표 간의 공분산 구조를 소수 요인으로 집약해 수십~수백 개 변수의 동시 예측이 가능하며, Kalman filter를 통해 누락·비동기 데이터(혼합주기, jagged edges)를 자연스럽게 처리할 수 있다는 점에서 나우캐스팅에 핵심적으로 활용됨 \cite{banbura2012nowcasting, bok2019frbny}. 뉴욕 연준 Nowcast 플랫폼 등 실무 시스템 역시 DFM 기반으로 실시간 발표 흐름에 맞춰 주별 업데이트를 수행해 예측치를 개선하는 방식을 채택한다 \cite{bok2019frbny}.

DFM 확장은 고빈도 보조지표를 포함하는 혼합주기 모델, 관측시점별 마스킹을 반영하는 실시간 필터링 기법, 그리고 비선형성을 부분적으로 허용하는 변형 등으로 발전해 왔다 \cite{banbura2012nowcasting, huber2020nowcasting}. 이러한 확장들은 공통 요인 추출의 안정성과 수치적 강건성을 유지하면서도 발표시차와 결측이 많은 실사례에서 예측력을 높이는 데 초점을 둔다.

심층 동적요인모형(DDFM)은 오토인코더 기반 비선형 인코더를 사용해 요인 구조를 학습함으로써 전통적 DFM의 선형 가정을 완화한다 \cite{andreini2020deep}. 비선형 인코더는 고차원 거시 데이터의 복잡한 상호작용을 더 적은 요인으로 포착하면서도, 요인층 뒤에는 여전히 선형 state-space(예: VAR(1))를 두어 필터링·스무딩 안정성을 유지한다. 결과적으로 DDFM은 (1) 대규모/고빈도 데이터에서도 표현력을 확보하고, (2) Kalman 필터를 통한 실시간 업데이트와 관측 마스킹 처리가 가능하며, (3) 선형 DFM 대비 중·단기 구간에서 비선형 패턴을 더 잘 포착할 잠재력을 갖는다 \cite{andreini2020deep}. 최근 연구들은 고빈도 외생 변수와 결합하거나 비선형 활성화·정규화 기법을 도입해 예측 성능을 개선하고 있으며, 본 연구도 이러한 딥 요인 접근을 DFM과 병행 비교하여 실증적으로 평가한다.
