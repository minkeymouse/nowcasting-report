\section{서론}

거시경제 변수의 정확한 예측은 정책 의사결정과 기업의 전략적 계획 수립에 핵심적임. 특히 생산, 투자, 소비 지표는 경제 활동의 핵심을 나타내며, 실시간 평가가 필수적임. 그러나 분기 GDP와 같은 주요 지표는 분기 종료 후 약 한 달이 지나야 공식 발표되므로, 실시간 경제 상황을 평가하고 시의적절한 정책 대응을 하기 어려움.

이에 따라 고빈도 데이터를 활용한 나우캐스팅 기법이 주목받고 있음 \cite{bok2017macroeconomic}. 나우캐스팅은 공식 통계가 발표되기 전에 다양한 고빈도 지표를 사용하여 현재의 거시경제 변수를 추정하는 기법임. 특히 신속한 정책 대응이 필요한 위기 상황에서 그 중요성이 부각됨.

본 연구는 동적요인모형(Dynamic Factor Model, DFM)과 딥러닝 모형을 사용하여 세 가지 주요 한국 거시경제 지표를 예측하는 나우캐스팅 시스템을 구축함: 생산(전산업생산지수: KOIPALL.G), 투자(설비투자지수: KOEQUIPTE), 소비(도소매판매액: KOWRCCNSE). 네 가지 예측 모형(ARIMA, VAR, DFM, 심층 동적요인모형 DDFM)의 성능을 30개 예측 수평선(1일부터 30일까지)에서 비교하며, 각 수평선에 대한 지표를 평균하여 최종 성능 지표로 사용함.

\subsection{선행연구 검토}

동적요인모형(DFM)은 많은 시계열에서 공통 요인을 추출하여 차원을 축소하고 혼합주기 데이터를 효과적으로 처리하는 기법임 \cite{stock2002forecasting}. Stock and Watson (2002)은 주성분 분석을 활용한 대규모 예측 변수 집합에서의 예측 방법을 제시하였으며, 이는 DFM의 이론적 기반을 제공함.

심층 동적요인모형(DDFM)은 오토인코더 기반 아키텍처를 사용하여 복잡한 요인 구조를 학습하는 딥러닝 기법임 \cite{andreini2020deep}. Andreini et al. (2020)은 비선형 인코더를 통해 전통적인 선형 DFM의 한계를 극복하고 복잡한 요인 관계를 학습할 수 있음을 보여줌.

혼합주기 데이터 처리를 위해 텐트 커널(tent kernel) 집계 방법이 사용됨 \cite{mariano2003new}. Mariano and Murasawa (2003)는 집계 기간 중간에 가까운 관측값에 더 큰 가중치를 부여하는 텐트 커널 접근법을 제안하였으며, 이를 통해 서로 다른 주기의 데이터를 효과적으로 결합할 수 있게 함. 이 방법은 분기 데이터를 월간 요인에 매핑하거나, 더 느린 주기의 시계열을 더 빠른 주기의 잠재 요인에 연결할 때 특히 유용함.
