\section{이슈 분석}

관찰된 문제점 및 제한사항은 다음과 같음:

\subsection{모형별 기술적 제한사항}

\subsubsection{VAR의 긴 수평선에서의 불안정성}

VAR은 1개월 예측을 넘어서는 수평선에서 큰 오차를 보이며, 이는 다단계 예측에 VAR 사용을 제한함. 정규화 기법이나 베이지안 VAR(BVAR) 등의 대안을 고려할 수 있음.

\subsubsection{DFM의 수치적 불안정성 및 해결}

DFM은 EM 알고리즘 수렴 중 일부 대상 변수에서 수치적 불안정성 문제를 보였으나, \textbf{EZZ의 NaN/Inf 제거} 해결책 적용 후 모든 타겟에서 성공적으로 훈련됨.

\paragraph{문제 발생 원인}

DFM 수치적 불안정성 발생 과정:
\begin{itemize}
    \item \textbf{Forward pass:} $V_u = V - VCF @ VC^T$ 계산 시 cancellation error 발생
    \item \textbf{오차 누적:} $V = A @ V_u @ A^T + Q$ 계산에서 오차 누적 $\to$ NaN/Inf 발산
    \item \textbf{Backward pass:} 손상된 V, Vu 전파 $\to$ smoothed factors ($EZ$, $EZZ$) 손상
    \item \textbf{Eigendecomposition 실패:} 손상된 $EZZ$가 $sum\_EZZ$에 포함 $\to$ C matrix 업데이트 실패
    \item \textbf{시리즈 개수 영향:} $n_{\text{obs}} \geq 33$일 때 수치적 오차 더 빠르게 누적
\end{itemize}

\paragraph{해결 방법}

\textbf{EZZ의 NaN/Inf 제거} 적용:
\begin{itemize}
    \item sum\_EZZ 계산 전 EZZ의 NaN/Inf를 0으로 대체
    \item 손상된 시간 단계 제외, 정상 시간 단계만 사용
    \item 이론적 근거: robust statistics \cite{huber1981robust}
\end{itemize}

\textbf{추가 수치 안정화 기법:}
\begin{itemize}
    \item F Matrix Regularization 강화
    \item Pseudo-inverse 사용 (solve 실패 시 fallback)
    \item 사전정규화, R 행렬 최소값 $10^{-4}$ 강제, 대칭성 강제
\end{itemize}

\paragraph{해결 결과}

모든 타겟에서 DFM 모델 정상 훈련:
\begin{itemize}
    \item \textbf{KOIPALL.G (33 시리즈):} C matrix NaN 100\% $\to$ 0\%, iterations 4 $\to$ 40, log-likelihood -3102.04 $\to$ 356.35 (개선: +1531.94)
    \item \textbf{KOWRCCNSE (39 시리즈):} C matrix NaN 94.9\% $\to$ 0\%, iterations 4 $\to$ 68, log-likelihood -2671.57 $\to$ 817.54 (개선: +6754.92)
    \item \textbf{KOEQUIPTE (32 시리즈):} 이전부터 안정적, 해결 후 정상 훈련 (51 iterations, log-likelihood -814.39)
\end{itemize}

\paragraph{해결책의 의미와 한계}

\textbf{의미:}
\begin{itemize}
    \item 실용적 접근법: 근본 원인 직접 해결 대신 손상된 결과 제외로 문제 우회
    \item 이론적 근거: 부동소수점 연산의 근본적 한계 \cite{higham2002computing}
    \item 효과: 정상 시간 단계만 사용하여 EM algorithm 정상 수렴 가능
\end{itemize}

\textbf{한계:}
\begin{itemize}
    \item 손상된 시간 단계가 상당 부분 차지 시 유효 표본 크기 감소 $\to$ 추정 정확도 저하
    \item 근본 원인 미해결: 더 극단적 상황에서 문제 재발 가능
\end{itemize}

\textbf{향후 개선 방향:}
\begin{itemize}
    \item Square-root Kalman filter, UD factorization
    \item 고정밀도 부동소수점 연산
    \item 제한: 계산 비용 증가 또는 구현 복잡도 증가
\end{itemize}

\subsection{실험 설계의 제한사항}

\begin{itemize}
    \item \textbf{훈련-예측 간격:} 4년 간격으로 COVID-19 제외 및 데이터 누수 방지, 그러나 최신 경제 패턴 반영 제한
    \item \textbf{테스트 데이터 부족:} 80/20 분할 후 각 수평선당 단일 테스트 포인트 $\to$ 통계적 신뢰성 제한
    \item \textbf{Nowcasting 제한:} Release date 정보 정확성, ARIMA/VAR 마스킹 구현 제한
\end{itemize}

\subsection{향후 연구 방향}

\begin{itemize}
    \item \textbf{모형 개선:} Robust Kalman filter, adaptive state space dimension
    \item \textbf{실험 설계 개선:} 롤링 윈도우 평가, 교차 검증
    \item \textbf{Release date 마스킹 개선}
    \item \textbf{추가 모형 비교}
\end{itemize}
