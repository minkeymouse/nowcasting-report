\section{이슈 분석}

관찰된 문제점 및 제한사항은 다음과 같음:

\subsection{모형별 기술적 제한사항}

\subsubsection{VAR의 긴 수평선에서의 불안정성}

VAR은 벤치마크 모형으로 포함되었으며, 긴 수평선(>7개월)에서 수치적 불안정성을 보임. 이는 다단계 예측에 VAR 사용을 제한하며, 정규화 기법이나 베이지안 VAR(BVAR) 등의 대안을 고려할 수 있음.

\subsubsection{DDFM의 성능 특성}

DDFM은 단기(1개월)와 중기(11개월) 수평선에서 우수한 성능을 보이며, 전체 수평선 평균에서도 ARIMA에 근접한 성능을 보임. 그러나 장기(22개월) 수평선에서는 상대적으로 높은 오차를 보임. 이러한 성능 특성은 다음과 같이 해석됨:

\textbf{비교의 공정성:} 모든 모델은 동일한 데이터(80/20 분할, 월별 리샘플링), 동일한 평가 지표, 동일한 예측 수평선을 사용함. 하이퍼파라미터 튜닝은 모든 모델에 대해 수행하지 않았음. 따라서 비교는 공정함.

\textbf{성능 특성 분석:}
\begin{itemize}
    \item \textbf{단중기 예측 우수성:} DDFM은 단기(1개월, sMAE=0.47)와 중기(11개월, sMAE=0.54) 수평선에서 가장 낮은 오차를 보여, 비선형 인코더가 중단기 예측에서 효과적임을 보여줌
    \item \textbf{전체 평균 성능:} 전체 수평선 평균에서 sMAE=0.78, sMSE=1.08로 ARIMA(sMAE=0.77, sMSE=1.05)에 근접한 성능을 보여, 전반적으로 경쟁력 있는 성능을 보임
    \item \textbf{장기 예측 제한:} 장기(22개월) 수평선에서 상대적으로 높은 오차(sMAE=1.60)를 보이나, 이는 모든 모델에서 장기 예측의 어려움을 반영하며, 특히 비선형 모델의 일반화 한계와 관련될 수 있음
    \item \textbf{모델 복잡도:} 현재 구현은 [64, 32] encoder 구조를 사용하여 원본 DDFM의 기본 구조(16, 4)보다 많은 파라미터를 가짐. 더 큰 모델은 이론적으로 더 많은 용량을 제공하나, 훈련 데이터(1985-2019, 월별, 약 336개 관측치)에 비해 과도하게 복잡할 경우 과적합 가능성이 있음
    \item \textbf{하이퍼파라미터 최적화:} 현재 설정은 원본 구현과 동일하나, 모델 복잡도가 다르므로 추가 최적화를 통해 성능 개선 가능
\end{itemize}

\subsubsection{DFM의 수치적 불안정성 및 해결}

DFM은 EM 알고리즘 수렴 중 일부 대상 변수에서 수치적 불안정성 문제를 보였으나, 수치 안정화 기법 적용 후 모든 타겟에서 성공적으로 훈련됨. 

\textbf{문제 원인:}
\begin{itemize}
    \item Kalman filter의 재귀적 공분산 업데이트 과정에서 부동소수점 오차 누적
    \item 관측 차원이 증가할수록 공분산 행렬의 condition number 증가로 수치적 불안정성 가속화
    \item EM algorithm의 M-step에서 ill-conditioned 행렬로 인한 수렴 실패
\end{itemize}

\textbf{해결 방법:}
\begin{itemize}
    \item Robust statistics 접근법: 손상된 시간 단계의 EZZ 제외
    \item 수치 선형대수학 기법: 사전정규화, 공분산 행렬 대칭성 강제, R 행렬 최소값 설정 \cite{golub2013matrix, higham2002computing}
\end{itemize}

\textbf{결과:} 모든 타겟에서 DFM 모델이 정상적으로 훈련됨. 이는 DFM의 실용적 한계를 보여주며, DDFM과 같은 대안적 접근법의 필요성을 시사함.

\subsubsection{ARIMA/VAR 모형의 Nowcasting 제한사항}

ARIMA와 VAR 모형은 forecasting 평가에서는 성공적으로 결과를 생성했으나, nowcasting 평가에서는 구조적 한계로 인해 제외됨:

\textbf{Nowcasting에서의 구조적 한계:}
\begin{itemize}
    \item Release date 마스킹 처리의 구조적 한계: ARIMA/VAR은 단변량/다변량 시계열 모형으로, release date 기반 데이터 마스킹을 DFM/DDFM처럼 직접적으로 처리하기 어려움
    \item 요인 모형(DFM/DDFM)은 요인 공간에서 마스킹을 처리한 후 관측 공간으로 변환할 수 있으나, ARIMA/VAR은 이러한 구조적 유연성이 없음
    \item 본 연구에서는 nowcasting 실험을 DFM과 DDFM 모형에 대해서만 수행함
\end{itemize}

\subsection{실험 설계의 제한사항}

\begin{itemize}
    \item \textbf{훈련-예측 간격:} 4년 간격으로 COVID-19 제외 및 데이터 누수 방지, 그러나 최신 경제 패턴 반영 제한
    \item \textbf{테스트 데이터 부족:} 80/20 분할 후 각 수평선당 단일 테스트 포인트 $\to$ 통계적 신뢰성 제한
    \item \textbf{Nowcasting 제한:} Release date 정보 정확성, ARIMA/VAR은 구조적 한계로 인해 nowcasting 실험에서 제외
\end{itemize}

\subsection{결과 검증 및 재현성}

본 연구의 모든 예측 결과는 다음과 같은 검증 과정을 거쳤음:

\begin{itemize}
    \item \textbf{수평선 계산 검증:} 예측 수평선 계산 로직을 검증하여 예측값이 올바른 테스트 시점과 비교되도록 보장함. 초기 구현에서 발견된 1개월 오프셋 버그를 수정하여, 예측값이 정확히 해당 수평선의 실제값과 비교되도록 함
    \item \textbf{모델 예측 검증:} 모든 모델(ARIMA, VAR, DFM, DDFM)이 정상적으로 예측을 생성하는지 검증함. DFM과 DDFM 모델의 경우 초기 구현에서 발견된 import 오류를 수정하여 모든 수평선에서 유효한 예측값을 생성하도록 함
    \item \textbf{결과 완전성:} 3개 대상 변수 × 4개 모델 × 22개 수평선 = 264개 결과 포인트 중 97\% 이상이 유효한 값을 가지며, 특히 단기(1개월)와 중기(11개월) 수평선에서는 모든 모델이 100\% 유효한 결과를 생성함
    \item \textbf{재현성:} 모든 실험은 저장된 체크포인트를 사용하여 재현 가능하며, 동일한 설정으로 실행 시 동일한 결과를 보장함
\end{itemize}

\subsection{향후 연구 방향}

\begin{itemize}
    \item \textbf{모형 개선:} Robust Kalman filter, adaptive state space dimension
    \item \textbf{실험 설계 개선:} 롤링 윈도우 평가, 교차 검증
    \item \textbf{Release date 마스킹 개선}
    \item \textbf{추가 모형 비교}
\end{itemize}
