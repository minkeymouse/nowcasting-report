\section{이슈 분석}

관찰된 문제점 및 제한사항은 다음과 같음:

\subsection{모형별 기술적 제한사항}

\subsubsection{VAR의 긴 수평선에서의 불안정성}

VAR은 1개월 예측을 넘어서는 수평선에서 큰 오차를 보이며, 이는 다단계 예측에 VAR 사용을 제한함. 정규화 기법이나 베이지안 VAR(BVAR) 등의 대안을 고려할 수 있음.

\subsubsection{DFM의 수치적 불안정성}

DFM은 EM 알고리즘 수렴 중 일부 대상 변수에서 수치적 불안정성 문제를 보임. 모델별 성공/실패 현황은 표~\ref{tab:model_success_failure}에 요약되어 있음.

\begin{table}[h]
\centering
\caption{모델별 성공/실패 현황}
\label{tab:model_success_failure}
\begin{tabular}{lccc}
\toprule
모델 & KOEQUIPTE & KOIPALL.G & KOWRCCNSE \\
\midrule
DFM & 성공 (32 시리즈) & 실패 (33 시리즈) & 실패 (39 시리즈) \\
DDFM & 성공 (32 시리즈) & 성공 (33 시리즈) & 성공 (39 시리즈) \\
ARIMA & 성공 (41 시리즈) & 성공 (40 시리즈) & 성공 (47 시리즈) \\
VAR & 성공 (41 시리즈) & 성공 (40 시리즈) & 성공 (47 시리즈) \\
\bottomrule
\end{tabular}
\end{table}

DFM의 수치적 불안정성 원인은 다음과 같이 분석될 수 있음:

\textbf{시리즈 개수 임계값}: 관찰된 패턴에 따르면, 시리즈 개수가 32개 이하일 때는 안정적이지만, 33개 이상일 때는 불안정함. KOEQUIPTE (32개 시리즈)는 성공했으나, KOIPALL.G (33개 시리즈)와 KOWRCCNSE (39개 시리즈)는 실패함.

\textbf{시리즈/state space 비율}: State space 차원은 15차원(요인 3개 × tent kernel 파라미터 5개)이며, 시리즈 개수와의 비율이 중요함. KOEQUIPTE는 32/15 = 2.13으로 안정적이지만, KOIPALL.G (33/15 = 2.20)와 KOWRCCNSE (39/15 = 2.60)는 불안정함. 혼합주기 데이터 처리를 위해 tent kernel을 적용하는 시계열이 많을수록 수렴이 어려워질 수 있음 \cite{bok2019frbny}.

\textbf{Kalman Filter의 F Matrix 불안정성}: Innovation covariance 행렬 $F = C_t V C_t^T + R_t$의 크기는 $n_{\text{obs}} \times n_{\text{obs}}$이며, $n_{\text{obs}}$가 클수록 역행렬 계산 시 수치적 불안정성이 증가할 수 있음. $n_{\text{obs}} \geq 33$일 때 F matrix inversion이 불안정하며, KOIPALL.G ($n_{\text{obs}}=33$)와 KOWRCCNSE ($n_{\text{obs}}=39$)에서 F matrix inversion 실패로 인한 연쇄 반응이 관찰됨.

\textbf{실패 메커니즘 (8단계 연쇄 반응)}: (1) F matrix inversion 불안정 ($n_{\text{obs}} \geq 33$); (2) Kalman gain (VCF) 손상; (3) Posterior covariance (Vu) 손상; (4) Prior covariance (V) 손상; (5) 다음 시간의 F matrix 손상; (6) Smoothed factors (EZ, EZZ) 손상; (7) sum\_EZZ ill-conditioning; (8) C matrix 업데이트 실패.

\textbf{적용된 개선 사항}: 다음과 같은 수치 안정화 기법을 적용함: (1) F Matrix Regularization 강화 (n\_obs 기반, $n_{\text{obs}} \geq 30$일 때 자동 강화); (2) Pseudo-inverse 사용 (solve 실패 시 fallback, 원본 MATLAB 방식); (3) 사전정규화, R 행렬 최소값 $10^{-4}$ 강제, 대칭성 강제 등. 이러한 개선으로 KOIPALL.G의 C matrix NaN 비율이 100\%에서 97.0\%로, KOWRCCNSE는 94.9\%에서 66.7\%로 감소했으나, 여전히 실패함.

\textbf{해결 방안}: (1) 시리즈 개수를 32개 이하로 제한하거나 DDFM 사용; (2) 향후 연구에서 더 강한 사전 regularization (sum\_EZZ), robust Kalman filter, adaptive state space dimension, series selection 등 고려 가능.

\subsection{실험 설계의 제한사항}

\subsubsection{훈련 기간과 예측 기간 사이의 간격}

본 연구에서는 훈련 기간(1985-2019)과 예측 기간(2024-2025) 사이에 약 4년의 간격이 존재함. 이는 COVID-19 시기(2020-2023)를 제외하고 데이터 누수를 방지하기 위한 의도적인 설계임. 그러나 이 간격으로 인해 COVID-19 이후 경제 구조 변화가 모형의 예측 성능에 미치는 영향을 평가할 수 있지만, 훈련 데이터와 예측 기간 사이의 시간적 거리로 인해 모형이 최신 경제 패턴을 반영하지 못할 수 있음.

\subsubsection{테스트 데이터 부족}

80/20 훈련-테스트 분할 후 테스트 데이터가 부족하여 일부 수평선 평가가 제한됨. 특히 22개 월별 수평선(1개월부터 22개월까지)에 대해 평가를 수행했으나, 테스트 데이터 부족으로 인해 일부 수평선에서 평가가 제한될 수 있음. 평가 기간 확장이나 롤링 윈도우 평가를 고려할 수 있음.

\subsubsection{통계적 신뢰성 제한}

각 수평선당 단일 테스트 포인트를 사용하여 통계적 신뢰성이 제한됨. 더 긴 평가 기간이나 교차 검증을 고려할 수 있음.

\subsection{Nowcasting 실험의 제한사항}

\subsubsection{Release date 정보의 정확성}

Release date 정보가 부정확하거나 누락된 경우, 마스킹이 정확하게 수행되지 않을 수 있음. 실제 운영 환경에서는 정기적인 업데이트가 필요함.

\subsubsection{ARIMA/VAR 모형의 Release date 마스킹 구현}

ARIMA와 VAR 모형의 경우 release date 기반 마스킹이 완전히 구현되지 않아 근사화된 방법을 사용하므로, DFM과 DDFM 모형과의 공정한 비교에 제한이 있을 수 있음.

\subsection{향후 연구 방향}

모형 개선(robust Kalman filter, adaptive state space dimension), 실험 설계 개선(롤링 윈도우 평가, 교차 검증), Release date 마스킹 개선, 추가 모형 비교 등을 고려할 수 있음.
