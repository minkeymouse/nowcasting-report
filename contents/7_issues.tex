\section{이슈 분석}

관찰된 문제점 및 제한사항은 다음과 같음:

\subsection{모형별 기술적 제한사항}

\subsubsection{VAR의 긴 수평선에서의 불안정성}

VAR은 1개월 예측을 넘어서는 수평선에서 큰 오차를 보이며, 이는 다단계 예측에 VAR 사용을 제한함. 정규화 기법이나 베이지안 VAR(BVAR) 등의 대안을 고려할 수 있음.

\subsubsection{DFM의 수치적 불안정성 및 해결}

DFM은 EM 알고리즘 수렴 중 일부 대상 변수에서 수치적 불안정성 문제를 보였으나, \textbf{EZZ의 NaN/Inf 제거} 해결책 적용 후 모든 타겟에서 성공적으로 훈련됨. 모델별 성공/실패 현황은 표~\ref{tab:model_success_failure}에 요약되어 있음.

\begin{table}[h]
\centering
\caption{모델별 성공/실패 현황}
\label{tab:model_success_failure}
\begin{tabular}{lccc}
\toprule
모델 & KOEQUIPTE & KOIPALL.G & KOWRCCNSE \\
\midrule
DFM & 성공 (32 시리즈) & 성공 (33 시리즈) & 성공 (39 시리즈) \\
DDFM & 성공 (32 시리즈) & 성공 (33 시리즈) & 성공 (39 시리즈) \\
ARIMA & 성공 (41 시리즈) & 성공 (40 시리즈) & 성공 (47 시리즈) \\
VAR & 성공 (41 시리즈) & 성공 (40 시리즈) & 성공 (47 시리즈) \\
\bottomrule
\end{tabular}
\end{table}

\paragraph{문제 발생 원인}

DFM의 수치적 불안정성은 다음과 같은 연쇄 반응으로 발생함:

\textbf{1. Kalman Filter Forward Pass에서 수치적 오차 누적}: F matrix inversion 자체는 문제가 아님 (condition number가 정상 범위 내). 실제 문제는 forward pass에서 V, Vu 업데이트 과정에서 수치적 오차가 누적되는 것임. 특히 $n_{\text{obs}} \geq 33$일 때 작은 수치 오차도 증폭되어 여러 시간 단계를 거치면서 NaN/Inf가 발생함. 수학적으로는 $V_u = V - VCF @ VC^T$ (posterior covariance 업데이트)와 $V = A @ V_u @ A^T + Q$ (다음 시간의 prior covariance) 계산 과정에서 오차가 누적됨.

\textbf{2. Kalman Filter Backward Pass (Smoother)에서 NaN/Inf 전파}: Forward pass에서 손상된 V, Vu가 backward pass로 전파됨. Backward pass는 forward pass 결과에 의존하므로 손상이 전파되어 smoothed factors ($EZ$, $EZZ$)가 손상됨. 로그에서 확인된 바와 같이, `VmT` recursion에서 `VmU` (forward pass 결과)가 이미 NaN/Inf를 포함하고 있어 backward pass에서 손상이 전파됨.

\textbf{3. 손상된 EZZ로 인한 sum\_EZZ ill-conditioning}: 손상된 Kalman filter smoother로 인해 smoothed factors의 공분산 행렬 $EZZ = E[Z_t Z_t^T]$가 손상됨. 손상된 $EZZ$가 $sum\_EZZ = \sum_t EZZ$ 계산에 포함되면, sum\_EZZ가 ill-conditioned가 되어 eigendecomposition이 실패함 (error code 49: too many repeated eigenvalues, 또는 error code 55: ill-conditioned matrix).

\textbf{4. C Matrix 업데이트 실패}: sum\_EZZ eigendecomposition 실패로 인해 condition number 계산이 불가능하며, adaptive regularization (최대 $10^{-3}$)도 충분하지 않아 $C_{\text{new}} = \text{solve}(sum\_EZZ_{\text{reg}}^T, sum\_yEZ^T)^T$ 계산이 실패함. 결과적으로 C matrix가 NaN으로 채워지고, EM algorithm이 조기 수렴함 (4 iterations).

\textbf{시리즈 개수와의 관계}: 시리즈 개수가 33개 이상일 때 ($n_{\text{obs}} \geq 33$) 수치적 오차가 더 빠르게 누적되어 NaN/Inf가 발생함. KOEQUIPTE (32개 시리즈, $n_{\text{obs}} \leq 32$)는 안정적이었으나, KOIPALL.G (33개 시리즈)와 KOWRCCNSE (39개 시리즈)에서는 문제가 발생함. State space 차원은 15차원(요인 3개 × tent kernel 파라미터 5개)이며, 시리즈/state space 비율이 클수록 (KOIPALL.G: 2.20, KOWRCCNSE: 2.60) 불안정성이 증가함.

\paragraph{해결 방법}

\textbf{EZZ의 NaN/Inf 제거} 방법을 적용하여 문제를 해결함. 이 방법은 sum\_EZZ 계산 전에 EZZ의 NaN/Inf를 0으로 대체하여 손상된 시간 단계를 sum\_EZZ 계산에서 제외함. 수학적으로는 다음과 같이 표현됨:

\begin{align}
EZZ_{\text{clean}}[t] &= \begin{cases}
0 & \text{if } EZZ[t] \text{ contains NaN/Inf} \\
EZZ[t] & \text{otherwise}
\end{cases} \\
sum\_EZZ &= \sum_t EZZ_{\text{clean}}[t]
\end{align}

이렇게 하면 손상된 시간 단계가 sum\_EZZ 계산에서 제외되어, 정상적인 시간 단계만 사용하여 sum\_EZZ가 계산됨. 결과적으로 sum\_EZZ eigendecomposition이 성공하고, C matrix 업데이트가 정상적으로 수행됨.

\textbf{적용된 추가 수치 안정화 기법}: (1) F Matrix Regularization 강화 (n\_obs 기반, $n_{\text{obs}} \geq 30$일 때 자동 강화); (2) Pseudo-inverse 사용 (solve 실패 시 fallback, 원본 MATLAB 방식); (3) 사전정규화, R 행렬 최소값 $10^{-4}$ 강제, 대칭성 강제 등. 이러한 기법들은 EZZ NaN/Inf 제거와 함께 작동하여 수치적 안정성을 향상시킴.

\paragraph{해결 결과}

해결책 적용 후 모든 타겟에서 DFM 모델이 정상적으로 훈련됨:

\textbf{KOIPALL.G (33 시리즈)}: 이전에는 C matrix 100\% NaN, 4 iterations, log-likelihood -3102.04로 실패했으나, 해결 후 C matrix NaN 없음, 40 iterations, log-likelihood 356.35로 성공함. Log-likelihood가 1531.94만큼 개선됨.

\textbf{KOWRCCNSE (39 시리즈)}: 이전에는 C matrix 94.9\% NaN, 4 iterations, log-likelihood -2671.57로 실패했으나, 해결 후 C matrix NaN 없음, 68 iterations, log-likelihood 817.54로 성공함. Log-likelihood가 6754.92만큼 개선됨.

\textbf{KOEQUIPTE (32 시리즈)}: 이전부터 안정적이었으며, 해결 후에도 정상적으로 훈련됨 (51 iterations, log-likelihood -814.39, 모델 저장 완료).

\paragraph{해결책의 의미}

이 해결책은 근본 원인(Kalman filter의 forward/backward pass에서 수치적 오차 누적 및 NaN/Inf 전파)을 직접 해결하는 것이 아니라, 손상된 결과를 제외함으로써 문제를 우회하는 실용적 접근법임. 이는 다음과 같은 의미를 가짐:

\textbf{1. 실용적 해결책}: Kalman filter의 forward/backward pass에서 수치적 오차를 완전히 제거하는 것은 매우 어려운 문제임. 대신, 손상된 시간 단계를 제외하고 정상적인 시간 단계만 사용하여 요인 적재 행렬 C를 업데이트함으로써, EM algorithm이 정상적으로 수렴할 수 있게 함.

\textbf{2. 통계적 타당성}: 손상된 시간 단계는 전체 데이터의 일부에 불과하므로, 나머지 정상적인 시간 단계만으로도 요인 적재 행렬을 충분히 추정할 수 있음. 이는 robust statistics의 관점에서 outlier를 제외하는 것과 유사한 접근법임.

\textbf{3. 근본 원인 해결의 어려움}: Forward/backward pass에서 수치적 오차를 완전히 제거하려면 더 정밀한 수치 연산 방법이나 Kalman filter 알고리즘 자체의 개선이 필요함. 이는 향후 연구 과제로 남아있음.

\textbf{4. 확장 가능성}: 이 해결책은 시리즈 개수에 관계없이 적용 가능하며, 더 많은 시리즈를 사용하는 경우에도 안정적으로 작동할 것으로 예상됨. 다만, 손상된 시간 단계가 너무 많으면 유효 표본 크기가 감소하여 추정 정확도가 저하될 수 있음.

\subsection{실험 설계의 제한사항}

\subsubsection{훈련 기간과 예측 기간 사이의 간격}

본 연구에서는 훈련 기간(1985-2019)과 예측 기간(2024-2025) 사이에 약 4년의 간격이 존재함. 이는 COVID-19 시기(2020-2023)를 제외하고 데이터 누수를 방지하기 위한 의도적인 설계임. 그러나 이 간격으로 인해 COVID-19 이후 경제 구조 변화가 모형의 예측 성능에 미치는 영향을 평가할 수 있지만, 훈련 데이터와 예측 기간 사이의 시간적 거리로 인해 모형이 최신 경제 패턴을 반영하지 못할 수 있음.

\subsubsection{테스트 데이터 부족}

80/20 훈련-테스트 분할 후 테스트 데이터가 부족하여 일부 수평선 평가가 제한됨. 특히 22개 월별 수평선(1개월부터 22개월까지)에 대해 평가를 수행했으나, 테스트 데이터 부족으로 인해 일부 수평선에서 평가가 제한될 수 있음. 평가 기간 확장이나 롤링 윈도우 평가를 고려할 수 있음.

\subsubsection{통계적 신뢰성 제한}

각 수평선당 단일 테스트 포인트를 사용하여 통계적 신뢰성이 제한됨. 더 긴 평가 기간이나 교차 검증을 고려할 수 있음.

\subsection{Nowcasting 실험의 제한사항}

\subsubsection{Release date 정보의 정확성}

Release date 정보가 부정확하거나 누락된 경우, 마스킹이 정확하게 수행되지 않을 수 있음. 실제 운영 환경에서는 정기적인 업데이트가 필요함.

\subsubsection{ARIMA/VAR 모형의 Release date 마스킹 구현}

ARIMA와 VAR 모형의 경우 release date 기반 마스킹이 완전히 구현되지 않아 근사화된 방법을 사용하므로, DFM과 DDFM 모형과의 공정한 비교에 제한이 있을 수 있음.

\subsection{향후 연구 방향}

모형 개선(robust Kalman filter, adaptive state space dimension), 실험 설계 개선(롤링 윈도우 평가, 교차 검증), Release date 마스킹 개선, 추가 모형 비교 등을 고려할 수 있음.
