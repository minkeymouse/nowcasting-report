\section{이슈 분석}

관찰된 문제점 및 제한사항은 다음과 같음:

\subsection{모형별 기술적 제한사항}

\subsubsection{VAR의 긴 수평선에서의 불안정성}

VAR은 1개월 예측을 넘어서는 수평선에서 큰 오차를 보이며, 이는 다단계 예측에 VAR 사용을 제한함. 정규화 기법이나 베이지안 VAR(BVAR) 등의 대안을 고려할 수 있음.

\subsubsection{DFM의 수치적 불안정성 및 해결}

DFM은 EM 알고리즘 수렴 중 일부 대상 변수에서 수치적 불안정성 문제를 보였으나, \textbf{EZZ의 NaN/Inf 제거} 해결책 적용 후 모든 타겟에서 성공적으로 훈련됨. 모델별 성공/실패 현황은 표~\ref{tab:model_success_failure}에 요약되어 있음.

\begin{table}[h]
\centering
\caption{모델별 성공/실패 현황}
\label{tab:model_success_failure}
\begin{tabular}{lccc}
\toprule
모델 & KOEQUIPTE & KOIPALL.G & KOWRCCNSE \\
\midrule
DFM & 성공 (32 시리즈) & 성공 (33 시리즈) & 성공 (39 시리즈) \\
DDFM & 성공 (32 시리즈) & 성공 (33 시리즈) & 성공 (39 시리즈) \\
ARIMA & 성공 (41 시리즈) & 성공 (40 시리즈) & 성공 (47 시리즈) \\
VAR & 성공 (41 시리즈) & 성공 (40 시리즈) & 성공 (47 시리즈) \\
\bottomrule
\end{tabular}
\end{table}

\paragraph{문제 발생 원인}

DFM의 수치적 불안정성은 Kalman filter의 forward/backward pass에서 수치적 오차가 누적되어 발생함. Forward pass에서 $V_u = V - VCF @ VC^T$ 계산 시 cancellation error가 발생하고, $V = A @ V_u @ A^T + Q$ 계산에서 오차가 누적되어 NaN/Inf로 발산함. Backward pass에서 손상된 V, Vu가 전파되어 smoothed factors ($EZ$, $EZZ$)가 손상되고, 손상된 $EZZ$가 $sum\_EZZ$ 계산에 포함되면 eigendecomposition이 실패하여 C matrix 업데이트가 실패함. 시리즈 개수가 33개 이상일 때 ($n_{\text{obs}} \geq 33$) 수치적 오차가 더 빠르게 누적됨.

\paragraph{해결 방법}

\textbf{EZZ의 NaN/Inf 제거} 방법을 적용하여 문제를 해결함. sum\_EZZ 계산 전에 EZZ의 NaN/Inf를 0으로 대체하여 손상된 시간 단계를 제외함. 이는 robust statistics의 관점에서 타당하며 \cite{huber1981robust}, 손상된 시간 단계를 제외하더라도 나머지 정상적인 시간 단계만으로도 요인 적재 행렬 $C$를 충분히 추정할 수 있음. 추가로 F Matrix Regularization 강화, Pseudo-inverse 사용, 사전정규화, R 행렬 최소값 $10^{-4}$ 강제, 대칭성 강제 등의 수치 안정화 기법을 적용함.

\paragraph{해결 결과}

해결책 적용 후 모든 타겟에서 DFM 모델이 정상적으로 훈련됨:

\textbf{KOIPALL.G (33 시리즈)}: 이전에는 C matrix 100\% NaN, 4 iterations, log-likelihood -3102.04로 실패했으나, 해결 후 C matrix NaN 없음, 40 iterations, log-likelihood 356.35로 성공함. Log-likelihood가 1531.94만큼 개선됨.

\textbf{KOWRCCNSE (39 시리즈)}: 이전에는 C matrix 94.9\% NaN, 4 iterations, log-likelihood -2671.57로 실패했으나, 해결 후 C matrix NaN 없음, 68 iterations, log-likelihood 817.54로 성공함. Log-likelihood가 6754.92만큼 개선됨.

\textbf{KOEQUIPTE (32 시리즈)}: 이전부터 안정적이었으며, 해결 후에도 정상적으로 훈련됨 (51 iterations, log-likelihood -814.39, 모델 저장 완료).

\paragraph{해결책의 의미와 한계}

이 해결책은 근본 원인을 직접 해결하는 것이 아니라, 손상된 결과를 제외함으로써 문제를 우회하는 실용적 접근법임. 부동소수점 연산의 근본적인 한계로 인해 완벽한 수치적 정확도는 불가능하므로 \cite{higham2002computing}, 손상된 시간 단계를 제외하고 정상적인 시간 단계만 사용하여 EM algorithm이 정상적으로 수렴할 수 있게 함. 다만, 손상된 시간 단계가 전체 데이터의 상당 부분을 차지하는 경우 유효 표본 크기가 감소하여 추정 정확도가 저하될 수 있음. 근본 원인을 해결하기 위해서는 square-root Kalman filter, UD factorization, 고정밀도 부동소수점 연산 등의 방법을 고려할 수 있으나, 계산 비용이 증가하거나 구현이 복잡함.

\subsection{실험 설계의 제한사항}

훈련 기간(1985-2019)과 예측 기간(2024-2025) 사이의 4년 간격은 COVID-19 시기를 제외하고 데이터 누수를 방지하기 위한 설계이나, 모형이 최신 경제 패턴을 반영하지 못할 수 있음. 80/20 훈련-테스트 분할 후 테스트 데이터 부족으로 인해 각 수평선당 단일 테스트 포인트를 사용하여 통계적 신뢰성이 제한됨. Nowcasting 실험에서는 release date 정보의 정확성과 ARIMA/VAR 모형의 release date 마스킹 구현 제한이 있음.

\subsection{향후 연구 방향}

모형 개선(robust Kalman filter, adaptive state space dimension), 실험 설계 개선(롤링 윈도우 평가, 교차 검증), Release date 마스킹 개선, 추가 모형 비교 등을 고려할 수 있음.
