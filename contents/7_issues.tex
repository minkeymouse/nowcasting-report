\section{이슈 분석}

관찰된 문제점 및 제한사항은 다음과 같음:

\subsection{모형별 기술적 제한사항}

\subsubsection{VAR의 긴 수평선에서의 불안정성}

VAR은 1개월 예측을 넘어서는 수평선에서 큰 오차를 보이며, 이는 다단계 예측에 VAR 사용을 제한함. 정규화 기법이나 베이지안 VAR(BVAR) 등의 대안을 고려할 수 있음.

\subsubsection{DDFM의 성능 제한사항}

DDFM은 중간 수평선(11개월)에서 우수한 성능을 보이나, 단기(1개월)와 장기(22개월) 수평선에서는 VAR, DFM, ARIMA보다 높은 오차를 보임. 원본 DDFM 구현 \cite{andreini2020deep}과 비교 분석 결과, 다음과 같은 원인을 확인함:

\textbf{비교의 공정성:} 모든 모델은 동일한 데이터(80/20 분할, 월별 리샘플링), 동일한 평가 지표, 동일한 예측 수평선을 사용함. 하이퍼파라미터 튜닝은 모든 모델에 대해 수행하지 않았음. 따라서 비교는 공정함.

\textbf{성능 제한 원인:}
\begin{itemize}
    \item \textbf{모델 복잡도 과다:} 원본 DDFM의 기본 encoder 구조는 (16, 4)인데, 현재 구현은 [64, 32]를 사용하여 약 7배 많은 파라미터를 가짐. 더 큰 모델은 이론적으로 더 많은 용량을 제공하나, 훈련 데이터(1985-2019, 월별, 약 336개 관측치)에 비해 과도하게 복잡하여 과적합 발생. 원본 구현은 파라미터당 10-20개 샘플을 권장하나, 현재 설정은 파라미터당 2.5개 샘플로 이 기준을 충족하지 못함
    \item \textbf{factor\_order 차이:} 원본 DDFM의 기본값은 factor\_order=2이나, 현재는 factor\_order=1을 사용하여 더 단순한 동역학 모델을 가정함. 이는 잠재적 불리함일 수 있음
    \item \textbf{선형 관계 가정:} 시계열이 선형적일 경우 비선형 인코더가 불필요한 복잡성을 도입하여 일반화 성능 저하 가능
    \item \textbf{하이퍼파라미터:} 원본과 동일한 설정(epochs=100, learning\_rate=0.005, batch\_size=100)을 사용하나, 모델 복잡도가 다르므로 최적화 필요
\end{itemize}

\subsubsection{DFM의 수치적 불안정성 및 해결}

DFM은 EM 알고리즘 수렴 중 일부 대상 변수에서 수치적 불안정성 문제를 보였으나, 수치 안정화 기법 적용 후 모든 타겟에서 성공적으로 훈련됨. 

\textbf{문제 원인:}
\begin{itemize}
    \item Kalman filter의 재귀적 공분산 업데이트 과정에서 부동소수점 오차 누적
    \item 관측 차원이 증가할수록 공분산 행렬의 condition number 증가로 수치적 불안정성 가속화
    \item EM algorithm의 M-step에서 ill-conditioned 행렬로 인한 수렴 실패
\end{itemize}

\textbf{해결 방법:}
\begin{itemize}
    \item Robust statistics 접근법: 손상된 시간 단계의 EZZ 제외
    \item 수치 선형대수학 기법: 사전정규화, 공분산 행렬 대칭성 강제, R 행렬 최소값 설정 \cite{golub2013matrix, higham2002computing}
\end{itemize}

\textbf{결과:} 모든 타겟에서 DFM 모델이 정상적으로 훈련됨. 이는 DFM의 실용적 한계를 보여주며, DDFM과 같은 대안적 접근법의 필요성을 시사함.

\subsubsection{ARIMA/VAR 모형의 Nowcasting 제한사항}

ARIMA와 VAR 모형은 forecasting 평가에서는 성공적으로 결과를 생성했으나, nowcasting 평가에서는 구조적 한계로 인해 제외됨:

\textbf{Nowcasting에서의 구조적 한계:}
\begin{itemize}
    \item Release date 마스킹 처리의 구조적 한계: ARIMA/VAR은 단변량/다변량 시계열 모형으로, release date 기반 데이터 마스킹을 DFM/DDFM처럼 직접적으로 처리하기 어려움
    \item 요인 모형(DFM/DDFM)은 요인 공간에서 마스킹을 처리한 후 관측 공간으로 변환할 수 있으나, ARIMA/VAR은 이러한 구조적 유연성이 없음
    \item 본 연구에서는 nowcasting 실험을 DFM과 DDFM 모형에 대해서만 수행함
\end{itemize}

\subsection{실험 설계의 제한사항}

\begin{itemize}
    \item \textbf{훈련-예측 간격:} 4년 간격으로 COVID-19 제외 및 데이터 누수 방지, 그러나 최신 경제 패턴 반영 제한
    \item \textbf{테스트 데이터 부족:} 80/20 분할 후 각 수평선당 단일 테스트 포인트 $\to$ 통계적 신뢰성 제한
    \item \textbf{Nowcasting 제한:} Release date 정보 정확성, ARIMA/VAR은 구조적 한계로 인해 nowcasting 실험에서 제외
\end{itemize}

\subsection{향후 연구 방향}

\begin{itemize}
    \item \textbf{모형 개선:} Robust Kalman filter, adaptive state space dimension
    \item \textbf{실험 설계 개선:} 롤링 윈도우 평가, 교차 검증
    \item \textbf{Release date 마스킹 개선}
    \item \textbf{추가 모형 비교}
\end{itemize}
