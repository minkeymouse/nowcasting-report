\subsubsection{Performance}

모형별 훈련 시간: VAR은 빠르게 훈련되며(수 초~수십 초), DFM과 DDFM은 상대적으로 오래 걸림(수 분~수십 분). ARIMA는 평가 결과가 없어 훈련 시간을 확인할 수 없음. 예측 시간은 VAR이 빠르고(밀리초 단위), DFM/DDFM은 상대적으로 느림(초 단위). 이는 요인 모형의 구조적 복잡성과 Kalman filter의 재귀적 계산 때문임.

그림~\ref{fig:accuracy_heatmap}은 모형별 대상 변수별 표준화된 RMSE를 히트맵으로 시각화함. 낮은 값(어두운 색상)은 더 나은 성능을 나타냄.

\begin{figure}[h]
\centering
\includegraphics[width=0.8\textwidth]{images/accuracy_heatmap.png}
\caption{정확도 히트맵: 모형 및 대상 변수별 표준화된 RMSE. 낮은 값(어두운 색상)은 더 나은 성능을 나타냄.}
\label{fig:accuracy_heatmap}
\end{figure}

그림~\ref{fig:horizon_performance_trend}은 시점별 성능 추세를 시각화함.

\begin{figure}[h]
\centering
\includegraphics[width=0.8\textwidth]{images/horizon_trend.png}
\caption{시점별 성능 추세: 각 모형에 대한 예측 시점(1개월부터 22개월까지)에 걸친 표준화된 MSE.}
\label{fig:horizon_performance_trend}
\end{figure}

\textbf{시점별 성능 패턴 분석:} 시점별 성능 추세를 분석하면 모형의 장기 예측 능력과 안정성을 평가할 수 있음. 

\textbf{KOIPALL.G:} DDFM은 대부분의 시점에서 안정적이고 낮은 오차를 보임. 특히 단기(1-6개월)에서 매우 우수한 성능을 보이며, 구체적으로 horizon 1에서 sMAE=0.12, horizon 6에서 sMAE=0.15로 매우 낮은 오차를 보임. 중기(7-12개월)에서는 sMAE가 0.39-0.95 범위로 증가하지만 여전히 양호한 수준이며, 장기(13-21개월)에서는 sMAE가 0.54-1.33 범위로 일부 증가하나 전반적으로 안정적임. 반면 DFM은 모든 시점에서 극단적으로 높은 오차를 보이며, 최소값이 horizon 3에서 sMAE=12.47, 최대값이 horizon 18에서 sMAE=16.78임. 이는 월별 시계열에 분기별 집계 가정이 부적합하기 때문임. VAR은 중간 수준의 성능을 보이며, 시점에 따라 변동이 있음(sMAE 0.34-1.96).

\textbf{KOWRCCNSE:} VAR은 단기(1-3개월)에서 우수한 성능을 보이지만(sMAE: 0.24-0.38), horizon 2에서 sMAE=0.98로 급증한 후 다시 감소하는 패턴을 보임. 중기(4-12개월)에서는 대부분 sMAE < 0.22를 보이지만, horizon 14(sMAE=0.59), horizon 19(sMAE=0.90), horizon 22(sMAE=1.14)에서 오차가 급증함. 이는 VAR의 다단계 예측에서 오차 누적과 공분산 행렬의 수치적 불안정성 때문임. DDFM은 대부분의 시점에서 안정적이고 낮은 오차를 보이며, 특히 horizon 1에서 sMAE=0.09, horizon 4-8에서 sMAE < 0.14를 보임. 중기(9-16개월)에서도 양호한 성능(sMAE 0.23-0.46)을 유지하나, 일부 시점(horizon 14: sMAE=0.82, horizon 19-20: sMAE 1.12-1.26)에서 오차가 증가함. DFM은 중간 수준의 성능을 보이지만 시점에 따라 변동이 크며, 특히 초기 시점(horizon 1: sMAE=2.38, horizon 2: sMAE=3.98)에서 높은 오차를 보임.

\textbf{KOEQUIPTE:} DFM과 DDFM이 모든 시점에서 거의 동일한 성능을 보이며, 이는 두 모형이 유사한 선형 요인 구조를 학습했음을 시사함. 구체적으로, 두 모형은 21개 시점 모두에서 최대 sMAE 차이가 0.002 수준으로 거의 동일함. 두 모형 모두 단기(1-3개월)에서 중간 수준의 성능(sMAE 1.03-1.07)을 보이며, 중기(4-12개월)에서도 유사한 성능을 유지함. 일부 시점(horizon 7-8: sMAE 2.33-2.33, horizon 13-14: sMAE 3.21-3.28)에서 오차가 증가하지만, 두 모형이 동일한 패턴을 보임. VAR은 시점에 따라 변동이 크며, 일부 시점(horizon 7-8: sMAE 2.68-2.02, horizon 13-14: sMAE 3.88-3.82, horizon 21-22: sMAE 2.35)에서 매우 높은 오차를 보임.

\textbf{장기 예측 안정성:} DDFM은 KOIPALL.G와 KOWRCCNSE에서 장기 예측(13-21개월)에서도 상대적으로 안정적인 성능을 보이며, 이는 비선형 인코더가 장기 패턴을 효과적으로 포착할 수 있음을 시사함. 반면 VAR은 긴 시점에서 수치적 불안정성을 보이며, DFM은 KOIPALL.G에서 모든 시점에 걸쳐 불안정함. KOEQUIPTE에서는 DFM과 DDFM이 모두 유사한 안정성을 보이며, 이는 해당 시계열이 선형 관계가 강하거나 비선형 인코더가 추가적인 이점을 제공하지 못함을 의미함.

\textbf{시점 가중 성능 평가:} 실용적 예측 관점에서 단기 예측이 장기 예측보다 중요하다는 점을 반영하여, 시점 가중 메트릭을 통해 모델 성능을 평가함. 시점 가중 메트릭은 단기 예측(1-6개월, 가중치 2.0), 중기 예측(7-12개월, 가중치 1.0), 장기 예측(13-22개월, 가중치 0.5)으로 분류하여 가중 평균을 계산함. 이를 통해 단기 예측 성능을 더 강조한 평가가 가능하며, DDFM 예측 품질 분석에 통합되어 시점별 가중 개선 비율을 제공함. 예를 들어, KOEQUIPTE에서 DDFM의 시점 가중 sMAE는 단기 예측에 더 높은 가중치를 부여하여 계산되며, 이를 통해 실용적 예측 관점에서의 성능을 더 정확히 평가할 수 있음.

\textbf{강건 통계 기반 성능 평가:} 일부 시점(예: KOEQUIPTE의 horizon 7-8, 13-14)에서 극단적 오차가 발생하는 경우, 평균 기반 메트릭이 왜곡될 수 있음. 이러한 문제를 해결하기 위해 중앙값(median)과 사분위수 범위(IQR)를 사용한 강건한 대안 메트릭을 활용함. 강건 통계 기반 메트릭은 특정 시점에서의 수치적 불안정성이나 극단적 오차로 인한 왜곡을 줄이며, DDFM 성능 평가의 신뢰성을 향상시킴. 예를 들어, KOEQUIPTE의 horizon 7-8과 13-14에서 sMAE가 2.33-3.28로 급증하는 경우, 평균 기반 메트릭은 이러한 극단값에 의해 왜곡될 수 있으나, 중앙값 기반 메트릭은 더 안정적인 성능 평가를 제공함. 

부트스트랩 신뢰구간은 메트릭의 불확실성을 정량화하여 통계적 신뢰성을 향상시키며, 기본적으로 1000회 재표본 추출을 수행하여 95\% 신뢰구간을 제공함. 이를 통해 DDFM과 DFM 간의 성능 차이를 통계적으로 검증할 수 있으며, 특히 표본 크기가 작거나 특정 시점에서의 오차 변동성이 큰 경우 유용함. 

분위수 기반 오차 메트릭은 여러 분위수(0.1, 0.25, 0.5, 0.75, 0.9)에서 sMAE와 sMSE를 계산하여 오차 분포의 전체적인 모양을 파악함. 중앙값(0.5 분위수)은 평균보다 이상치에 강건한 성능 지표를 제공하며, IQR sMAE는 오차 분포의 퍼짐 정도를 측정함. 꼬리 비율(90번째/10번째 분위수)은 오차 분포의 꼬리 두께를 측정하여 가끔 극단적인 예측 오차가 발생하는지 평가함. 이러한 분위수 기반 메트릭은 변동성이 큰 시점이나 왜도가 있는 오차 분포에서 더 신뢰할 수 있는 성능 평가를 제공함.

