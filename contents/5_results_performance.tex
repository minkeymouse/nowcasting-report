\subsubsection{Performance}

모형별 훈련 시간: VAR은 빠르게 훈련되며(수 초~수십 초), DFM과 DDFM은 상대적으로 오래 걸림(수 분~수십 분). ARIMA는 평가 결과가 없어 훈련 시간을 확인할 수 없음. 예측 시간은 VAR이 빠르고(밀리초 단위), DFM/DDFM은 상대적으로 느림(초 단위). 이는 요인 모형의 구조적 복잡성과 Kalman filter의 재귀적 계산 때문임.

그림~\ref{fig:accuracy_heatmap}은 모형별 대상 변수별 표준화된 RMSE를 히트맵으로 시각화함. 낮은 값(어두운 색상)은 더 나은 성능을 나타냄.

\begin{figure}[h]
\centering
\includegraphics[width=0.8\textwidth]{images/accuracy_heatmap.png}
\caption{정확도 히트맵: 모형 및 대상 변수별 표준화된 RMSE. 낮은 값(어두운 색상)은 더 나은 성능을 나타냄.}
\label{fig:accuracy_heatmap}
\end{figure}

그림~\ref{fig:horizon_performance_trend}은 시점별 성능 추세를 시각화함.

\begin{figure}[h]
\centering
\includegraphics[width=0.8\textwidth]{images/horizon_trend.png}
\caption{시점별 성능 추세: 각 모형에 대한 예측 시점(1개월부터 22개월까지)에 걸친 표준화된 MSE.}
\label{fig:horizon_performance_trend}
\end{figure}

\textbf{시점별 성능 패턴:} KOIPALL.G에서 DDFM은 단기에서 매우 우수하며 장기에서도 안정적임. 반면 DFM은 모든 시점에서 극단적으로 높은 오차를 보임. KOWRCCNSE에서 VAR은 단기에서 우수하나 일부 시점에서 오차가 급증하며, DDFM은 대부분의 시점에서 안정적임. KOEQUIPTE에서 DFM과 DDFM은 모든 시점에서 거의 동일한 성능을 보임. DDFM은 KOIPALL.G와 KOWRCCNSE에서 장기 예측에서도 상대적으로 안정적인 성능을 보이며, 이는 비선형 인코더가 장기 패턴을 효과적으로 포착할 수 있음을 시사함.

