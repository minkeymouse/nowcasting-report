\section{연구 목적 및 평가 기준}

본 보고서는 한국 경제의 \textbf{산업생산지수(월별)}를 nowcasting하기 위한 고빈도\slash 미시 데이터 후보들을 검토하고,
각 데이터의 \textbf{접근성(공개 여부, 유료 여부, 법적 제약)}, \textbf{발표 시차}, \textbf{빈도(주별 이상 여부)}를 기준으로
실질적으로 활용 가능한지 평가하는 것을 목적으로 한다.

특히 다음 두 가지 조건을 핵심 평가 기준으로 삼는다.

\begin{enumerate}
  \item \textbf{빈도 기준}
        \begin{itemize}
          \item 원칙적으로 \textbf{주간(weekly) 이상 고빈도} 데이터를 선호한다.
          \item 일\slash 5분\slash 실시간 데이터는 주간으로 집계하여 활용 가능하다.
        \end{itemize}

  \item \textbf{발표 시차 기준}
        \begin{itemize}
          \item 종속변수인 산업생산지수(산업활동동향)는 참조월의 \textbf{다음 달 말}에 발표된다.
          \item 같은 참조월에 대해 그보다 먼저 발표되는 월별 데이터는 유효한 predictor로 활용 가능하다.
          \item 한국은행의 BSI/ESI/CSI 등 심리지수는 해당월 말 06:00에 발표되어,
                산업생산 및 설비투자 지수보다 \textbf{약 3--5주 선행}한다.
        \end{itemize}
\end{enumerate}

아울러, 일반 국민과 연구기관이 \textbf{법\slash 제도상 접근할 수 없는 데이터, 유료 상용 서비스, 비정기\slash 1회성 데이터}는
모형 구축의 지속가능성이 낮다고 보고 별도로 구분한다.

\section{데이터 유형별 검토}

\subsection{기업 실적 데이터}

\subsubsection{상장사 재무데이터 (NICE V-se, KIS-Value, Value Search 등)}

\begin{itemize}
  \item \textbf{내용}: 상장사 매출, 재고, 생산량 등 재무제표 기반 분기\slash 연간 데이터.
  \item \textbf{빈도}: 분기 및 연간.
  \item \textbf{접근성}:
        \begin{itemize}
          \item 구 KIS-Value, 현 Value Search 등 상용 서비스로 \textbf{유료 구독}이 필요하다.
          \item 일반 국민과 공공 연구기관이 무료 또는 실시간으로 이용하기 어렵다.
        \end{itemize}
  \item \textbf{평가}:
        \begin{itemize}
          \item 분기 빈도이므로 월별 산업생산지수 nowcasting에는 직접 활용이 곤란하다.
          \item 또한 유료 상용 서비스로, 정책 연구용 공개 모형에서 재현성과 접근성 확보가 어렵다.
        \end{itemize}
\end{itemize}

\noindent\textbf{결론}: 산업생산지수 nowcasting용 핵심 후보에서는 제외한다. 다만 장기 구조모형에서
기업 재무건전성과 투자행태 분석용 보조 변수로는 활용 가능성이 있다.

\subsection{설비가동률 및 전력 관련 데이터}

\subsubsection{한국전력공사(KEPCO) 전력판매량}

\begin{itemize}
  \item \textbf{내용}: 시군구, 용도업종, 계약종별 전력판매량(\texttt{kWh}).
  \item \textbf{기간 및 형식}: 2004--2025년 월별 데이터, 엑셀(\texttt{xlsx}) 형식.
  \item \textbf{업데이트 주기}: ``대략 3개월에 한 번'' 수준으로 \textbf{정기성이 떨어짐}.
  \item \textbf{접근성}: 한국전력공사 홈페이지에서 무료 다운로드 가능.
  \item \textbf{평가}:
        \begin{itemize}
          \item 공간 및 업종 분해 수준은 우수하나, \textbf{월별}이며 업데이트가 불규칙적이다.
          \item 참조월 종료 직후 시점에서 신속히 확보하기 어렵고,
                산업생산지수 발표 시점(다음 달 말)에 비해 명확한 선행성을 확보하기 어렵다.
        \end{itemize}
\end{itemize}

\noindent\textbf{결론}: 모형 추정과 검증용 후행 자료로는 의미가 있으나,
실시간 nowcasting용 선행지표로서의 활용성은 낮다.

\subsubsection{한국전력거래소(KPX) 시계열 데이터}

\paragraph{(1) 시도별 시간대별 전력 계량 데이터(2013--2023, 1회성)}

\begin{itemize}
  \item \textbf{내용}: 2013--2023년 시도별\slash 시간대별 전력데이터(GWh).
  \item \textbf{특징}: 공공데이터포털에 1회성 데이터로 업로드되었으며, 차기 등록 예정은 미정이다.
  \item \textbf{접근성}: CSV 형식으로 무상 제공(일반인 다운로드 가능).
  \item \textbf{평가}:
        \begin{itemize}
          \item 시간 단위 고빈도이므로 주\slash 월 단위로 집계해 모형 추정에 활용 가능하다.
          \item 그러나 \textbf{실시간 업데이트 계획이 없어}, 과거 구간에서의 모형 학습과 백테스트 용도에 한정된다.
        \end{itemize}
\end{itemize}

\paragraph{(2) 시도별 시간대별 전력거래량(2001--2024)}

\begin{itemize}
  \item \textbf{내용}: 2001--2024년 시도별\slash 시간대별 전력거래량(MWh), CSV 형식.
  \item \textbf{업데이트 계획}: 공공데이터포털에 ``차기 등록 예정일 2026-06-30''으로 명시.
  \item \textbf{법적 제약}:
        \begin{itemize}
          \item 개별 발전기 단위 전력거래량 및 주소 등은 \textbf{사업자의 영업정보}로,
                정보공개법 제9조에 따라 제3자 제공이 불가하다.
        \end{itemize}
  \item \textbf{평가}:
        \begin{itemize}
          \item 시도 단위 집계는 무료로 제공되어 접근성이 양호하다.
          \item 일별 시간대별 전력거래량이므로 산업 생산 활동이 보다 활발한 주중, 주간 시간대와 주택 전기 사용 비중이 큰 주말 및 야간 시간대로 구분한다면 보다 예측력을 높일 수 있을 것으로 보인다.
          \item 그러나 대규모 일괄 갱신 방식이라, 월별 nowcasting에 필요한
                ``전월 데이터의 신속한 제공''과는 거리가 있다.
        \end{itemize}
\end{itemize}

\paragraph{(3) KPX 전력수급현황 실시간 API(5분 단위)}

\begin{itemize}
  \item \textbf{내용}: 전국 단위 aggregate 전력수급현황(5분 단위), 짧은 시계열, 오픈 API 제공.
  \item \textbf{포맷}: XML 및 JSON(\LaTeX\ 코드가 아닌 프로그래밍 환경에서 처리 필요).
  \item \textbf{접근성}:
        \begin{itemize}
          \item API 키 발급 후 무료 사용 가능하다.
        \end{itemize}
  \item \textbf{평가}:
        \begin{itemize}
          \item 5분 빈도 자료는 주간 또는 월간 지표로 쉽게 집계할 수 있다.
          \item KPX 전력거래량 데이터는 nowcasting 모형 추정에 사용하고 실제 nowcasting에는 KPX 전력수급현황 실시간 데이터를 사용하는 방안이 있다.
          \item 발표 시차가 사실상 실시간에 가까워, 산업생산지수에 대해 \textbf{강한 선행성}을 가진다.
          \item 전력수요는 제조업 가동과 밀접히 연동되어 있어 산업생산의 선행 또는 동행지표로 활용 여지가 크다.
        \end{itemize}
\end{itemize}

\noindent\textbf{결론}: 실시간 nowcasting을 위한 \textbf{핵심 고빈도 후보}로 평가된다.
다만 aggregate 수준이므로 업종별 및 용도별 식별은 불가능하다. 전력거래량 데이터와 마찬가지로 산업 생산 활동이 활발한 시간대 별로 구분하여 사용할 필요가 있다.

\subsection{ESG/탄소 및 환경 센서 데이터}

\subsubsection{굴뚝자동측정기기(TMS) 데이터 – 한국환경공단}

\begin{itemize}
  \item \textbf{내용}: 굴뚝 원격감시체계(CleanSYS)에 연결된 대형 배출사업장(제철·제강, 민간발전, 석유화학 등)의
        굴뚝 배출가스 농도 측정값을 실시간으로 제공한다. 측정 대상 오염물질은 총 7종으로, 먼지, 황산화물(SOx), 
        질소산화물(NOx), 염화수소, 불화수소, 암모니아, 일산화탄소를 포함한다.
  \item \textbf{데이터 빈도 및 업데이트 주기}
        \begin{itemize}
          \item OpenAPI(\texttt{rltmMesureResult})는 배출구별 측정시간(\texttt{mesure\_dt})이 포함된
                \textbf{시각별 실시간 측정값}을 반환한다. 예시 응답은 \verb|2020-12-10 14:00| 등
                1시간 단위 타임스탬프를 사용한다.
            \item 정확한 빈도를 알기 위해서는 실제로 OpenAPI를 사용해볼 필요가 있다.
          \item 공공데이터포털에는 ``굴뚝자동측정기기 측정결과공개시스템\_실시간공개'' 등의 CSV 파일이 별도로 제공되나, 과거 데이터의 시계열 길이를 정확히 파악하기 어렵다.
        \end{itemize}
  \item \textbf{접근성}:
        \begin{itemize}
          \item OpenAPI는 무료이며, 인증키 발급 후 REST 방식(JSON, XML)으로 호출 가능하다.
          \item 다만 실시간값을 장기간 축적하려면 자체적으로 크롤러 및 DB를 구축하여 저장해야 하며, 과거 데이터의 존재 여부가 불확실하여 nowcasting 모형 추정에 한계가 있음.
        \end{itemize}
  \item \textbf{nowcasting 활용성 평가}
        \begin{itemize}
          \item \textbf{장점}: 대형 발전·제조 사업장의 굴뚝 배출량은 설비 가동률과 밀접하게 연동되므로,
                산업생산지수 중 특히 \emph{중화학·에너지 집약 산업}에 대한 정보를 포함할 가능성이 높음.
                시간 단위 데이터를 주간 또는 월간 합·평균·분산 등으로 집계하여 nowcasting에 활용할 수 있음.
          \item \textbf{한계}:
                \begin{itemize}
                  \item TMS 부착 사업장은 전체 사업장의 일부에 불과하여 \textbf{표본 편의} 문제가 있음.
                  \item 일부 사업장은 비정기적 가동을 하거나, 환경 규제 강화 및 방지시설 운전 등
                        \textbf{정책·규제 요인}에 의해 배출량이 크게 변동할 수 있어,
                        순수한 생산량 신호로만 해석하기 어려움.
                  \item 실시간 API를 이용한 \emph{자체적인 데이터 축적} 없이는 과거 장기 시계열을 완전하게 확보하기 어려움.
                \end{itemize}
        \end{itemize}
\end{itemize}

\noindent\textbf{결론}: 고빈도·사업장 패널 구조를 가진다는 점에서 \textbf{가치 있는 보조 지표}이며,
전력수급 데이터와 결합하여 에너지 집약 산업의 가동률을 추정하는 데 특히 유용할 수 있음.
다만 접근·구축 비용과 표본 편의 문제를 고려하면, 산업생산지수 nowcasting 모형의
\emph{핵심 단일 지표}라기보다는 \emph{보완 변수}로 활용하는 것이 적절함. 무엇보다, nowcasting 모형 추정을 위해서는 과거 데이터의 확보가 필수적이나, 현재로서는 확인되지 않아 본 보고서에서의 활용은 제한적임.

\subsubsection{에어코리아 대기오염정보}

\begin{itemize}
  \item \textbf{내용}: 전국 대기오염 측정소에서 관측하는 주요 오염물질은 미세먼지(PM10), 초미세먼지(PM2.5), 
        오존(O$_3$), 이산화질소(NO$_2$), 일산화탄소(CO), 아황산가스(SO$_2$) 등 6가지이며, 
        각 항목의 농도값과 함께 통합대기환경지수(CAI) 및 등급 정보가 제공된다.
  \item \textbf{데이터 빈도 및 업데이트 주기}
        \begin{itemize}
          \item 실시간 대기정보는 \textbf{해당 시각을 기준으로 직전 1시간 동안 측정한 값의 평균}을
                매시(hourly) 단위로 제공한다.
          \item OpenAPI ``대기오염정보'' 서비스는 측정소별\slash 시도별 실시간 측정정보 조회, 통합대기환경지수 
                나쁨 이상 측정소 목록조회, 대기질 예보 및 초미세먼지 주간예보 조회 등의 기능으로 구성되며, 
                실시간 측정값은 \textbf{1시간 간격으로 지속적으로 갱신}된다.
          \item 일부 통계·요약 파일 데이터(XLSX)는 ``수시(1회성 업로드)'' 형식으로 제공되지만,
                nowcasting에는 주로 \textbf{실시간 OpenAPI}를 사용하게 된다.
        \end{itemize}
  \item \textbf{접근성}:
        \begin{itemize}
          \item 공공데이터포털에서 API 인증키를 무료로 발급받아 사용할 수 있으며, 
                별도의 이용료는 발생하지 않는다.
          \item 제공되는 기술 가이드와 샘플 코드가 상세하게 작성되어 있어, 
                프로그래밍 기초 지식만 있으면 측정소별 시계열 패널 데이터를 
                자동으로 수집하는 작업을 수행할 수 있다.
        \end{itemize}
  \item \textbf{nowcasting 활용성 평가}
        \begin{itemize}
          \item \textbf{장점}:
                \begin{itemize}
                  \item 전국 수백 개 측정소의 \textbf{고빈도 패널} 구조를 활용하여,
                        시·도별 또는 산업단지 주변의 평균, 분산, 극값 등을 추출하여
                        주간·월간 환경 스트레스 지표를 구성할 수 있음.
                  \item 교통량, 난방 수요, 일부 산업 활동 등과 관련된 신호를 일정 부분 포함하고 있어
                        \emph{도시 서비스업·교통 관련 활동}과 연관된 경기 상황을 보조적으로 파악할 수 있음.
                \end{itemize}
          \item \textbf{한계}:
                \begin{itemize}
                  \item 대기오염 수준은 기상 조건(풍향, 강수, 기온 역전 등), 국외 유입 등
                        \textbf{비경제적 요인}의 영향을 크게 받는다.
                  \item 공장 가동, 난방, 교통량 등 다양한 요인이 혼합되어 있어,
                        제조업 \emph{산업생산지수}와의 직접적인 연관성은 상대적으로 약할 수 있다.
                \end{itemize}
        \end{itemize}
\end{itemize}

\noindent\textbf{결론}: 에어코리아 데이터는 시계열·패널 구조와 업데이트 빈도가 우수하여
기술적으로는 nowcasting에 활용이 용이하다. 다만 신호 해석 시 기상 변수와의 동시 고려가 필수이며,
\textbf{전국 제조업 생산의 직접적인 대용변수라기보다는, 도시 활동 수준 및 환경 스트레스의 보조 지표}로
위치시키는 것이 적절하다.

\subsubsection{해양수질자동측정망 관측정보(해양수산부·해양환경공단)}

\begin{itemize}
  \item \textbf{내용}: 연안 오염우심해역(시화·마산·광양만, 부산 수영만, 새만금, 울산, 하구역 등)에 설치된
        해양수질자동측정소에서 수온, 염분, 용존산소, pH, 탁도, 전기전도도 등 수질 항목과
        일부 기상 항목을 \textbf{상시(continuous)} 측정하는 관측망이다.
  \item \textbf{데이터 빈도 및 업데이트 주기}
        \begin{itemize}
          \item 해양환경정보포털에 따르면,
                수질측정장치 자료는 \textbf{1시간 간격}, 수질·기상측정센서 자료는 \textbf{5분 간격}으로 생성됨.
          \item 관측소 위치·운영현황을 담은 ``해양수질자동측정소 정보'' 메타데이터는
                \textbf{연간 주기로 업데이트}되며, 관측 자료 자체는 포털에서 실시간 시계열로 조회 가능함.
          \item 공공데이터포털 OpenAPI(예: 하구 및 만 정점조회, 관측서비스)를 통해 정점별 코드·위치 및
                관측값을 JSON/XML 형식으로 조회할 수 있음.
        \end{itemize}
  \item \textbf{접근성}:
        \begin{itemize}
          \item API 및 파일 데이터 모두 무료 공개이며, 인증키 발급 후 사용 가능하다.
          \item 다만 해역·정점 코드 체계를 이해해야 하고, 원하는 기간·정점을 반복 호출하여
                자체적으로 시계열을 구축해야 하므로, 실질 활용에는 일정 수준의 데이터 처리 역량이 필요하다.
        \end{itemize}
  \item \textbf{nowcasting 활용성 평가}
        \begin{itemize}
          \item \textbf{장점}:
                \begin{itemize}
                  \item 시화·마산·광양만, 울산·부산 등 \textbf{연안 산업단지 인근 해역}의 수질 변화를 고빈도로 관측하므로,
                        석유화학, 제철, 조선 등 수출·중화학 공업의 활동과 관련된 방류 패턴을
                        간접적으로 파악할 수 있음.
                  \item 시간해상도가 5분~1시간 수준이므로, 주간·월간 집계뿐 아니라
                        변동성, 극값, 첨두 빈도 등 다양한 통계량을 구성할 수 있음.
                \end{itemize}
          \item \textbf{한계}:
                \begin{itemize}
                  \item 특정 하구·만 해역에 공간적으로 집중되어 있어,
                        전국 산업생산지수 전체를 대표하기에는 \textbf{공간 범위가 제한적}이다.
                  \item 강우·하천유입, 조석, 해류 등 \textbf{자연환경 요인}의 영향이 매우 커서,
                        순수한 공장 가동 신호를 분리하기 위해서는 기상·수문 자료와의 통합 분석이 필요하다.
                \end{itemize}
        \end{itemize}
\end{itemize}

\noindent\textbf{결론}: 해양수질자동측정망 데이터는 특정 연안 산업단지(예: 시화·광양·울산 등)에 대한
\textbf{지역별 산업활동 nowcasting}에서 유용한 보조 지표로 활용될 수 있다.
다만 전국 산업생산지수 nowcasting에서 핵심 설명변수로 사용하기에는 공간 커버리지와 자연환경 영향이 크므로,
전력수급·심리지표와 결합한 \emph{보완적 지표}로 활용하는 방향이 바람직하다.



\subsection{텍스트\slash 뉴스 및 심리지수}

\subsubsection{한국은행 뉴스심리지수}

\begin{itemize}
  \item \textbf{내용}: 인터넷 포털의 경제뉴스 텍스트를 바탕으로 구축한 뉴스심리지수.
  \item \textbf{기간 및 빈도}: 2005--2025년 일별 시계열 데이터.
  \item \textbf{접근성}: 한국은행 ECOS 및 대시보드를 통해 공개(무료).
  \item \textbf{평가}:
        \begin{itemize}
          \item 일별 자료이므로 주 및 월 단위로 집계가 가능하다.
          \item 발표 시차가 거의 없고, 뉴스 자체가 경제 이벤트에 선행 또는 동행하므로
                산업생산 nowcasting에 매우 적합하다.
          \item 공공 지표이므로 재현성과 접근성 측면에서도 우수하다.
        \end{itemize}
\end{itemize}

\noindent\textbf{결론}: 산업생산지수 nowcasting 모형의 \textbf{핵심 텍스트\slash 심리 지표}로 활용을 권장한다.

\subsubsection{한국은행 BSI\slash ESI\slash CSI 등 월간 심리지수}

\begin{itemize}
  \item \textbf{내용}: 경제심리지수(ESI), 소비자심리지수(CCSI), 현재 및 전망 CSI,
        기업심리지수(CBSI) 등 여러 월간 설문지표.
  \item \textbf{발표 시점}:
        \begin{itemize}
          \item BSI, ESI, CSI는 참조월 말일 전후 06:00에 발표된다.
          \item 산업활동동향(산업생산 및 설비투자)은 참조월이 끝난 후 다음 달 말경에 공표된다.
          \item 따라서 같은 참조월에 대해서는 심리지수가 산업생산지수보다 대략 3주에서 5주 정도 먼저 발표된다.
        \end{itemize}
  \item \textbf{접근성}: 한국은행 ECOS 시스템과 보도자료를 통해 무상으로 공개된다.
  \item \textbf{평가}:
        \begin{itemize}
          \item 월간 빈도이지만 산업생산지수보다 앞서 발표되기 때문에 nowcasting 모형에 활용할 수 있다.
          \item 기업과 가계의 경기 인식 및 전망을 반영하므로 산업생산의 변화 방향을 예측하는 데 도움이 된다.
        \end{itemize}
\end{itemize}

\noindent\textbf{결론}: 월간 빈도이지만 산업생산지수보다 선행 발표되는 대표적인 선행지표로서,
산업생산지수 nowcasting 모형에 포함하는 것을 권장한다.


\subsubsection{BIGKinds(뉴스 빅데이터, 한국언론진흥재단)}

\begin{itemize}
  \item \textbf{내용}: BIGKinds는 한국언론진흥재단에서 개발·운영하는 뉴스 빅데이터 분석 플랫폼이다.
        주요 일간지와 방송사 등 약 100개 언론 매체의 기사를 수집하여 검색 및 분석 서비스를 제공한다.
        사용자가 입력한 검색어를 바탕으로 관련 기사 목록을 제공하고, 뉴스 클러스터링을 통한 ``오늘의 이슈'', 
        개체명 인식을 통한 ``오늘의 키워드'' 등의 분석 결과를 시각화하여 제시한다.
  \item \textbf{데이터 빈도 및 업데이트 주기}
        \begin{itemize}
          \item \emph{오늘의 이슈}: 일일 수집 뉴스에 대해 클러스터링 알고리즘을 적용하여
                상위 10개 이슈를 선정하며, 하루에 두 차례(예: 오전 8시, 오후 5시) 정기적으로 분석을 수행한다.
          \item \emph{오늘의 키워드}: 일일 수집 뉴스에서 인물명, 기관명, 지명 등의 개체명을 추출하여
                각 키워드별 뉴스 건수를 기준으로 순위를 매겨 시각화한다.
          \item \emph{키워드 트렌드}: 사용자가 특정 키워드를 입력하면, 해당 키워드가 등장한 뉴스 건수를 
                \textbf{일/주/월/년} 단위로 집계하여 그래프로 표시한다.
                사용자는 조회 기간(예: 최근 1개월, 3개월, 1년)과 집계 주기(일/주/월/년)를 선택할 수 있어,
                특정 키워드의 \textbf{주간 뉴스 등장 빈도 변화}를 추적할 수 있다.
          \item 분석 결과 페이지에서는 키워드 트렌드, 연관어 분석, 관계도 분석 결과를 함께 제공하며,
                ``데이터 다운로드'' 메뉴를 통해 주간 기사 건수 시계열을 엑셀 파일 등으로 
                내려받아 별도의 nowcasting 지표로 변환할 수 있다.
        \end{itemize}
  \item \textbf{접근성 및 법적 제약}
        \begin{itemize}
          \item BIGKinds는 무료 회원가입 후 이용할 수 있는 \textbf{웹 기반 서비스}이며,
                키워드 검색, 키워드 트렌드 그래프, 이슈 리포트 등을 웹 브라우저에서 조회하고
                일부 분석 결과를 엑셀 파일이나 이미지 파일로 저장할 수 있다.
          \item 다만 각 기사 원문은 저작권법에 의해 보호되며, 사이트 이용약관에 따라
                무단 전재, 복제, 대량 수집 행위가 금지된다. 따라서 \textbf{기사 원문 전체를
                자동 수집하여 자체 데이터베이스를 구축하는 방식}은 법적·약관상 제약이 있다.
          \item 한국언론진흥재단은 BIGKinds에서 생성된 분석 정보 중 일부를 공공데이터포털을 통해 공개하고 있다.
                ``오늘의 이슈 Top 10''과 같은 요약 통계는 파일 다운로드 및 Open API 형태로
                제공되며, 인증키를 발급받으면 REST API 방식으로 호출할 수 있다. 다만 이는 전체 기사 원문이
                아니라, \textbf{분석 및 요약 과정을 거친 통계 지표}라는 점에서 활용 범위가 제한적이다.
          \item BIGKinds의 검색 및 분석 기능을 외부에서 호출하기 위한 Open API는
                주로 공공기관 및 언론사와의 협약을 통해 제공되는 것으로 알려져 있으며,
                일반 연구자나 개인에게는 일괄적으로 개방되지 않는다.
        \end{itemize}
  \item \textbf{키워드별 주간 기사 건수 시계열 구축 가능성}
        \begin{itemize}
          \item BIGKinds 검색창에 특정 키워드를 입력하고 조회 기간을 지정하면,
                해당 키워드가 포함된 뉴스 기사 건수를 \textbf{일 단위 또는 주 단위}로 집계한
                트렌드 그래프를 조회할 수 있다.
          \item 집계 주기를 ``주간''으로 설정하면, 지정한 기간 동안의 \textbf{주별 기사 건수}가
                막대그래프 또는 선그래프로 표시되며, 이 그래프에 해당하는 데이터는
                분석 결과 다운로드 메뉴를 통해 엑셀 파일로 내려받을 수 있다.
          \item ``산업생산'', ``설비투자'', 특정 업종명(자동차, 반도체 등) 또는
                특정 기업명을 검색 키워드로 지정하고 주간 기사 건수 시계열을 다운로드하면,
                해당 주의 \textbf{뉴스 언급 빈도}를 측정하는 주간 텍스트 기반 지표를
                생성할 수 있다.
        \end{itemize}
  \item \textbf{산업생산지수 nowcasting 관점 평가}
        \begin{itemize}
          \item \textbf{빈도 측면}에서, BIGKinds의 키워드 트렌드 기능은 일 단위부터
                주 단위까지 선택 가능한 기사 건수 시계열을 제공하므로, 월별 산업생산지수를
                nowcasting하는 데 필요한 시간 해상도를 충족한다.
          \item \textbf{접근성 측면}에서는, 웹 인터페이스와 공공데이터포털을 통해
                일부 분석 지표를 무료로 이용할 수 있으나, 전체 기사 원문 및
                고빈도 메타데이터에 대한 API 접근은 재단과의 협약 또는 별도 계약이
                필요할 수 있다.
          \item \textbf{법적·제도적 측면}에서, 정책 연구 보고서에서는
                포털 및 언론사 웹페이지를 비공식적으로 크롤링하는 방식보다는,
                BIGKinds 웹 서비스와 공공데이터포털이 제공하는 \textbf{공식 분석 지표}
                (예: 키워드별 주간 기사 건수, 오늘의 이슈 Top 10 등)를 활용하여
                텍스트 기반 선행지수를 구축하는 방향이 바람직하다.
          \item 실제 모형 구축 시에는, BIGKinds에서 추출한 \textbf{키워드별 주간 기사 건수 지수}를
                산업생산지수의 선행변수로 포함하고, 한국은행 뉴스심리지수와 결합하여
                뉴스의 양(Volume)과 감성(Sentiment)을 동시에 반영하는 텍스트 기반
                nowcasting 변수를 구성하는 접근이 유효하다.
        \end{itemize}
\end{itemize}

\noindent\textbf{결론}: BIGKinds는 뉴스 데이터를 기반으로 한 \textbf{고빈도 텍스트 지표의 원천}이며,
키워드별 주간 기사 건수 시계열을 비교적 용이하게 구축할 수 있다는 점에서 산업생산지수
nowcasting에 유의미한 후보 데이터이다. 다만 기사 원문 자체의 대량 수집에는 저작권 및
이용약관 제약이 있으므로, 공공데이터포털 및 BIGKinds가 공식적으로 제공하는
분석 지표와 다운로드 기능을 활용하여 \emph{키워드별 주간 뉴스량 지수}를 구성하는 방향이
현실적이다.


\subsection{운송 및 물류 관련 데이터}

\subsubsection{항만 물동량 통계 (국가물류통합정보센터, PORT\textendash MIS)}

국가물류통합정보센터(NLIC)는 통합 PORT\textendash MIS 데이터를 기반으로 항만별 월간 물동량 통계를 산출하여 공개한다. 항만명(부산, 광양 등)과 수출입/환적/연안, 입출항 구분별 화물 처리실적(톤)을 월별로 조회할 수 있으며, 기준년월 선택 메뉴를 통해 \textbf{2010년 1월 부터 2025년 10월까지}의 자료가 제공되고 있어 \textbf{\emph{월 빈도의 시계열}} 확보가 가능하다.\footnote{국가물류통합정보센터 ``항만별 물동량 통계'' 화면에서 기준년월 선택 및 항만별 월간/누계 물동량을 제공하고 있으며, 메타정보에 업데이트 주기 ``매월''로 명시되어 있음.} 자료 설명에 따르면 해당 통계는 통합 PORT\textendash MIS의 화물처리실적(화물 반출입 신고정보)을 기초로 작성된다.

메타정보상 업데이트 주기는 \emph{매월}로 표시되어 있고, 기준년월이 2025년 10월까지 제공되는 점을 고려할 때, 전월 자료가 비교적 짧은 시차로 집계·공표되고 있음을 알 수 있다.\footnote{동일 화면의 자료설명 표에서 ``업데이트 주기 : 매월''로 기재.} PORT\textendash MIS 원자료에 대한 해양수산 관련 보고서에서는 ``전월 물동량 통계가 매월 22일경 확정·공표된다''고 언급하고 있어, 실무적으로는 \emph{전월 자료가 익월 하순(22일 전후)에 확정되는 것으로 추정}된다.\footnote{한국해양수산개발원 KMI의 해양수산 지표 보고서에서 Port\textendash MIS 물동량 통계에 대해 ``매월 22일에 전월 통계가 확정 공표되므로 2019년 10월 통계까지 포함''이라는 설명이 제시됨. }

한편, 산업생산지수는 광업·제조업동향조사 결과를 포함한 ``산업활동동향'' 공표를 통해 전월 통계가 
\emph{매월 30일경} 발표된다.\footnote{광업·제조업동향조사 메타정보에 ``매월 30일경 생산·소비·투자·경기부문을 종합하여 산업활동동향으로 공표''라고 명시.} 
최근 공표일을 확인하면 2025년 10월치 산업활동동향이 11월 28일, 9월치는 10월 31일에 발표되는 등, 
일반적으로 \emph{익월 말}에 공표되고 있다.\footnote{산업활동동향 보도자료 목록에서 2025년 4\textasciitilde 10월 자료의 공표일이 4월 30일, 5월 30일, 7월 1일, 7월 31일, 8월 29일, 10월 27일, 10월 31일, 11월 28일로 나타남.}

이를 종합하면, 항만 물동량 통계는 산업생산지수에 비해 \emph{약 1주일 정도 이른 시점(익월 22일 전후)}에 전월 값이 확정·제공되는 것으로 보이며, 월별 산업생산지수를 nowcasting하는 데 있어 시차 측면의 이점을 갖는다. 다만,
\begin{itemize}
  \item 현재 시점에서는 공식적인 ``정식 공표일''이 별도의 공표계획표로 제공되지는 않고 있으며,
  \item NLIC 포털 화면에서 과거 자료를 일괄 다운로드하는 과정에서 일부 개편·지연 가능성이 존재하므로
\end{itemize}
정책용 nowcasting 모델에 활용하기 위해서는 PORT\textendash MIS 및 NLIC 운영기관과의 추가적인 공표일 확인이 필요하다.

접근성 측면에서는, 항만 물동량 통계는 NLIC 누리집에서 \emph{무료·공개}로 제공되며, 엑셀 다운로드 기능을 통해 별도의 승인절차 없이 이용 가능하다. 따라서 ``일반인 접근이 어려운 데이터''에는 해당하지 않으며, 실무 연구자가 비교적 손쉽게 확보 가능한 고빈도 물류변수로 평가된다.

\subsubsection{해상운임 지수 (국내·국외)}

국가물류통합정보센터는 국내 및 국외 해상운임 지수를 별도 통계로 제공한다. 국내 해상운임 지수는 국내 항로 운임을 집계한 월별 지수이며, 자료설명에서 업데이트 주기가 ``매월''로 명시되어 있다.\footnote{NLIC ``국내 해상운임 지수'' 메타정보에서 업데이트 주기를 매월로 제시.} 2019년 1월 부터 최근 2025 10월까지 제공되고 있다. 또한 수출 수입 데이터가 별개로 존재하며 대상 국가별로 나누어져있다 (미국 동서부, 일본, EU, 중국 등)

국외 해상운임 지수는 상하이해운거래소(Shanghai Shipping Exchange) 등 해외 기관의 컨테이너·벌크 운임지수를 취합한 것으로, NLIC 메타정보상 \emph{업데이트 주기가 ``매주''}로 표시되어 있어 주간 단위의 시계열 데이터를 제공한다.\footnote{NLIC ``국외 해상운임 지수'' 자료설명 표에 업데이트 주기 ``매주''로 명시.} 실제 화면상 관측일자는 주 단위(예: 2025년 4월 18일, 4월 25일 등)로 구성되어 있어, 우리가 요구하는 \emph{최소 주별 빈도} 조건을 충족한다. 2014년부터 제공하고 있다.

민간 부문의 글로벌 컨테이너 항만 물동량·운임 지수(Drewry Port Throughput Index 등)는 국제 해운시장 분석을 위해 널리 활용되지만, 대부분 \emph{유료 구독형 서비스}로 제공된다.\footnote{KMI의 글로벌 컨테이너 항만물동량 분석 보고서는 Drewry Port Throughput Index 등 민간지수를 인용하며, 해당 지수는 민간기관이 매월 발표하는 상용 통계임을 언급.} nowcasting 모형에서 사용하기 위해서는 별도의 구독·이용계약이 필요하다.

반면, NLIC가 제공하는 국내·국외 해상운임 지수는 공공 포털을 통해 무료로 제공되므로 접근성 측면에서 유리하다. 산업생산지수 nowcasting 측면에서는,
\begin{itemize}
  \item 국내 제조업·수출입 구조가 해상운송에 크게 의존하고 있고,
  \item 국외 해상운임 지수가 글로벌 물동량 및 교역 경기의 선행지표로 작용할 수 있다는 점에서,
\end{itemize}
주간·월간 운임 지수를 설명변수로 포함하는 것이 유의미하다. 특히 국외 해상운임 지수는 \emph{주별 갱신}이 이루어지므로, 월별 산업생산 공표 이전에 상대적으로 풍부한 정보 집합을 제공할 수 있다는 장점이 있다. 다만, 운임지수는 공급측 요인(선복량 조정 등)과 금융적 요인도 반영하므로, 단순 수준값보다는 변화율·이동평균 등으로 변환하여 사용하고, 모형 내에서 경제적 해석 가능성을 별도로 검토하는 것이 바람직하다.

\subsubsection{항공화물 및 수출입 화물 물동량 통계}

NLIC의 항공화물통계 메뉴에서는 ``공항별 물동량 통계''와 ``수출입화물 물동량 통계''를 제공한다. 
공항별 물동량 통계는 인천공항, 김포공항 등 주요 국내 공항에서 처리된 화물량을 월 단위로 집계하여 제공하며, 
메타정보상 업데이트 주기는 \emph{매월}로 표시되어 있다.\footnote{국가물류통합정보센터 ``공항별 물동량 통계'' 자료설명에서 업데이트 주기를 매월로 명시.} 
수출입화물 물동량 통계는 항공 수출입 화물을 대상으로 국가·노선별 물동량을 월별로 집계하며, 
메타정보에 ``업데이트 주기 매월''로 명시되어 있다.\footnote{NLIC ``수출입화물 물동량 통계'' 자료설명에서 업데이트 주기 매월로 제시.} 
다만 정확한 공표일은 확인하지 못했다.

이들 통계는 일반적으로 관세청 통관자료 및 항공사·공항운영기관의 화물 처리실적을 기초로 작성되며, 전월 자료가 익월 중에 업데이트되는 구조임. 공식적인 공표일(예: 매월 몇 일)이 산업활동동향처럼 명시되어 있지는 않지만, 현재(12월 초) 2013년 1월부터 2025년 10월까지 제공되는 점을 고려하면, 산업생산지수(익월 말 공표)와 유사하거나 다소 이른 시점에 접근 가능할 것으로 판단됨. 즉,
\begin{itemize}
  \item 월빈도이지만,
  \item 업데이트가 산업생산지수 공표 이전에 이루어질 가능성이 있어,
\end{itemize}
제조업 중 수출지향 업종(전자, 기계, 운송장비 등)의 생산을 설명하는 보조지표로 nowcasting 모형에 포함할 여지가 있음.

데이터 접근성은 항만 물동량 통계와 마찬가지로 \emph{무료·공개}이며, 엑셀 다운로드 기능을 통한 일괄 수집이 가능함. 다만, 국가·노선별 세부 분류까지 모두 활용할 경우 차원이 급격히 증가하므로, nowcasting 모형에서는 인천공항 총 수출입 화물량, 주요 교역국(미국, 중국, EU 등) 상대 화물량, 또는 적절한 가중평균 지수 등으로 축약하여 사용하는 것이 현실적임.

\subsubsection{내륙화물(도로) 수송실적 및 생활물류(택배) 통계}

내륙 운송과 관련하여, 국토교통부는 도로운송 수송실적을 국가데이터포털을 통해 제공한다. 
이 통계는 전국 시·도 간 기점-종점(O/D)별 화물 운송량, 품목별 수송실적, 그리고 품목별 O/D 물동량을 
연도 단위로 집계한 자료로, 화물량은 톤 단위로 측정된다.\footnote{공공데이터포털 ``국토교통부\_도로운송수송실적'' 설명에 따르면, 도로운송수송실적은 시도 간 O/D별 및 품목별 수송실적을 제공하는 연도별 데이터이며, 데이터 기준연도는 2022년, 최종 취합 및 등록은 2025년에 이루어짐.} 
메타정보상 업데이트 주기는 ``수시(1회성 데이터)''로 표기되어 있으며, 데이터 기준일은 2022년, 
공개 시점은 2025년으로 명시되어 있음.\footnote{같은 메타정보에서 업데이트 주기 ``수시(1회성 데이터)'', 데이터 기준일 2025년, 내용은 2022년 실적이라는 설명이 제시됨.}

이는 도로운송 수송실적 통계가 \emph{행정자료를 활용한 사후 집계 통계}로서, 연간 단위로 상당한 시차를 두고 공표되는 성격임을 의미함. 따라서,
\begin{itemize}
  \item 빈도가 연간(연도별) 수준에 머물고,
  \item 공표 시차도 산업생산지수에 비해 훨씬 길기 때문에,
\end{itemize}
월별 산업생산지수 nowcasting에는 적합하지 않음. 이 통계는 중장기 물류 인프라 계획 수립, 지역 간 물류흐름 분석 등 구조적 분석에 적합한 데이터로 판단됨.

생활물류(택배)와 관련해서는 NLIC의 생활물류통계 메뉴에서 연도별·월별 택배 물동량 및 매출 통계를 제공하고 있으나, 자료설명에 따르면 연간 업데이트 주기를 갖는 연간 통계가 중심이다.\footnote{NLIC 생활물류통계의 ``연도별 생활물류실적'' 자료설명에서 업데이트 주기를 연간으로 명시하고, 택배 물동량·매출 통계를 연도별로 제공. :contentReference[oaicite:12]{index=12}} 월별 통계표도 제공되지만, 실제로는 과거 연도의 실적을 사후 집계한 형태로, 통상 전년도 자료가 다음 해에 일괄 업데이트되는 구조로 보인다. 이 경우,
\begin{itemize}
  \item 택배 물동량이 소비·전자상거래 관련 경기지표로서 의미는 있으나,
  \item 월별 산업생산지수를 \emph{실시간에 가깝게 Nowcasting}하는 용도에는 시차가 너무 길어 실용성이 제한된다.
\end{itemize}

요약하면, 운송 및 물류 영역에서 nowcasting에 실질적으로 활용 가능한 공공 데이터는
\begin{enumerate}
  \item \textbf{항만별 월간 물동량 통계(PORT\textendash MIS 기반)}: 월빈도, 익월 22일경 전월 값 확정, 산업생산지수 공표(익월 말)보다 약간 빠른 공표.
  \item \textbf{국내 해상운임 지수}: 월빈도, 매월 업데이트.
  \item \textbf{국외 해상운임 지수}: \emph{주빈도}, 매주 업데이트.
  \item \textbf{항공화물(공항별·수출입 화물 물동량)}: 월빈도, 매월 업데이트.
\end{enumerate}
정도로 정리할 수 있다. 이들 지표는 모두 무료·공개 데이터이며, 산업생산지수보다 짧거나 유사한 시차로 공표되므로, 산업생산지수 nowcasting 모형의 설명변수 후보로 유의미하다. 반면,
\begin{itemize}
  \item 도로운송 수송실적(O/D별 연도별 화물 물동량),
  \item 연도별 생활물류(택배) 통계,
  \item 민간 유료 서비스(글로벌 컨테이너 항만 물동량 지수 등)
\end{itemize}
은 공표 시차·빈도 또는 비용 측면에서 nowcasting용 실시간 지표로 활용하기에는 한계가 있어, 구조적 분석 또는 보조적 참고자료로 활용하는 것이 적절하다.


\section{종합 평가 및 권고}

\subsection{요약 표}

아래 표는 관측 빈도, 발표 시차, 접근성을 기준으로 주요 후보들을 요약한 것이다.

\begin{table}[htbp]
\centering
\small
\setlength{\tabcolsep}{4pt}      % 열 간격 약간 줄이기
\renewcommand{\arraystretch}{1.0}% 행 높이 기본값

\begin{tabular}{p{2.5cm} p{3cm} p{3.5cm} p{3cm} p{3cm}}
\toprule
유형 & 데이터 & 빈도/발표시차 & 접근성/비용 & 비고 \\
\midrule
기업 실적 &
상장사 재무데이터 &
분기/연간, 공시 후 지연 &
유료 상용 &
nowcasting 부적합 \\
\midrule
전력 &
KEPCO 전력판매량 &
월별, 업데이트 불규칙 &
무료 xlsx &
선행지표로 한계 \\
\midrule
전력 &
KPX 전력계량(1회성) &
시간별(과거 2013--23) &
무료 csv &
모형 학습용 \\
\midrule
전력 &
KPX 전력거래량 &
시간별, 2001--24,
차기 업로드 2026 예정 &
무료 csv,
발전기 단위 비공개 &
실시간 갱신 부재 \\
\midrule
전력 &
KPX 전력수급현황 API &
5분 실시간 &
무료, 코딩 필요 &
핵심 고빈도 후보 \\
\midrule
ESG/환경 &
굴뚝 TMS &
실시간, 사업장 단위 &
무료 API &
보조 변수로 유망 \\
\midrule
ESG/환경 &
에어코리아 &
실시간, 측정소 단위 &
무료 API &
주요 변수 비권장 \\
\midrule
ESG/환경 &
해양수질자동측정망 &
실시간, 정점 단위 &
무료 API &
특정 연안 산업에 한정 \\
\midrule
텍스트 &
뉴스심리지수 &
일별, 거의 동시 공표 &
무료, 공개 &
핵심 텍스트 지표 \\
\midrule
심리지수 &
BSI/ESI/CSI/CBSI &
월별, 참조월 말 발표 &
무료 &
산업생산보다 3--5주 선행 \\
\midrule
물류 &
항만 물동량 &
월별, 익월 22일 전후 &
무료, 엑셀 &
공표 일정 확인 필요 \\
\midrule
물류 &
화물 운송량 &
연간, 공표 시차 큼 &
무료 &
nowcasting 부적합 \\
\bottomrule
\end{tabular}

\caption{관측 빈도, 발표 시차, 접근성 기준 주요 데이터 요약 (1)}
\end{table}

\begin{table}[htbp]
\centering
\small
\setlength{\tabcolsep}{4pt}
\renewcommand{\arraystretch}{1.0}

\begin{tabular}{p{2.5cm} p{3cm} p{3.5cm} p{3cm} p{3cm}}
\toprule
유형 & 데이터 & 빈도/발표시차 & 접근성/비용 & 비고 \\
\midrule
텍스트/뉴스 &
BIGKinds
(뉴스 빅데이터) &
기사 실시간 수집,
키워드 트렌드 일/주/월 집계 &
웹 서비스 무료,
일부 공공 API,
원문 크롤링 제약 &
키워드별 주간 기사수 지수,
뉴스량 선행지표 \\
\midrule
물류 &
국내 해상운임 지수 &
월별, 매월 업데이트 &
무료·공개 (NLIC) &
국내 해상운송 비용 구조 반영,
월간 경기 보조지표 \\
\midrule
물류 &
국외 해상운임 지수 &
주별, 매주 업데이트,
IPI보다 크게 선행 &
무료·공개 (NLIC) &
글로벌 교역·물동량 선행지표,
주간 nowcasting 후보 \\
\midrule
물류 &
항공화물 물동량
(공항별·수출입) &
월별, 전월 자료가
익월 중 공표 &
무료·공개 (NLIC),
엑셀 다운로드 &
수출지향 제조업 활동 보조지표,
IPI보다 비슷하거나 약간 선행 \\
\bottomrule
\end{tabular}

\caption{관측 빈도, 발표 시차, 접근성 기준 주요 데이터 요약 (2)}
\end{table}


\subsection{우선 활용 권고}

\begin{enumerate}
  \item \textbf{최우선 고빈도 후보(주간 이상 집계 가능)}
        \begin{itemize}
          \item 한국전력거래소 전력수급현황 API(5분).
          \item 한국은행 뉴스심리지수(일별).
          \item KLIC 주별 해상운임지수 데이터
        \end{itemize}

        위 세 축을 이용해 전력, 해상운임임, 뉴스심리에 기반한 \textbf{주간 합성 경기지수}를 구성하고,
        이를 산업생산지수 nowcasting 모형(DFM, MIDAS, Bayesian VAR 등)에 투입하는 것이 바람직하다.

  \item \textbf{월간 선행지표로서의 활용}
        \begin{itemize}
          \item 한국은행 BSI, ESI, CSI, CBSI: 산업생산지수보다 3--5주 먼저 발표되므로
                월말 기준으로 산업생산에 대한 선행 정보를 제공함.
          \item 항만 물동량 통계: 공표 시점이 산업활동동향보다 빠를 경우
                수출 주도 제조업의 실질생산을 반영하는 보완 지표로 활용 가능함.
        \end{itemize}

  \item \textbf{제외 또는 제한적 활용 대상}
        \begin{itemize}
          \item 상장사 재무데이터(유료, 분기/연간 빈도).
          \item KEPCO 전력판매량(월별, 불규칙 업데이트).
          \item 화물 운송량(연간).
          \item 에어코리아 대기오염(비경제 요인 영향이 큼).
        \end{itemize}
\end{enumerate}

\section{향후 과제 및 결론}

\subsection{향후 과제}

\begin{enumerate}
  \item \textbf{공개 데이터 공표 일정의 체계적 정리}:
        항만 물동량, KEPCO 전력판매량 등 월별 지표의 실제 공표일을 추가로 수집하여
        산업생산지수 대비 선행성 여부를 정량적으로 확인할 필요가 있음.

  \item \textbf{API 기반 데이터 수집 인프라 구축}:
        KPX, 환경공단, 해양수산부, 한국은행 등 각 기관 API에서
        일\slash 주간 단위의 패널을 자동으로 축적하는 수집\slash 정제 파이프라인을
        KDI 혹은 기획재정부 내부 시스템에 구축하는 것이 중요하다.

  \item \textbf{예측력 검증 및 변수 선정}:
        위 후보 데이터로 산업생산지수를 nowcasting하는 파일럿 모형을 구축한 뒤,
        DFM, MIDAS 회귀, 머신러닝(랜덤포레스트, LASSO 등)을 활용한
        변수 중요도 평가를 통해 실제 예측력이 높은 지표를 선별해야 한다.
\end{enumerate}

\subsection{결론}

본 조사를 통해, 일반 국민 및 공공 연구기관이 비용과 법적 제약 없이 활용할 수 있으면서
주간 이상 고빈도 또는 산업생산지수보다 선행 발표되는 데이터로는
\begin{enumerate}
  \item 한국전력거래소 전력수급현황 실시간 데이터,
  \item 한국은행 뉴스심리지수,
  \item 한국은행 BSI, ESI, CSI, CBSI,
  \item (보조적으로) 국가물류통합정보센터 해상운임지수
\end{enumerate}
가 유효 후보로 도출되었다.

이들 데이터를 기반으로 산업생산지수의 nowcasting 모형을 설계하고,
필요 시 추가 데이터(텍스트 뉴스 원자료, 지역별 세부 전력 및 환경 지표 등)를 단계적으로 확장하는 전략이
KDI 및 기획재정부 차원의 실시간 경기 진단 체계 구축에 가장 현실적이고 효율적인 접근으로 판단된다.
