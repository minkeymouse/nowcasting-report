\subsection*{D. 혼합주기 예측 실험 세부사항}

\subsubsection*{Vintage별 테스트 RMSE: AR(1) vs MIDAS-AR(1)}

\begin{table}[htbp]
\centering
\small
\begin{tabular}{lcc}
\toprule
Vintage & AR(1) & MIDAS-AR(1) \\
\midrule
h0 & 0.950 (0.0) & 0.952 (-0.2) \\
h1 & 0.950 (0.0) & 0.951 (-0.1) \\
h2 & 0.950 (0.0) & 0.952 (-0.2) \\
h3 & 0.950 (0.0) & 0.951 (-0.1) \\
h4 & 0.950 (0.0) & 0.945 (0.5)  \\
\bottomrule
\end{tabular}
\caption{Vintage별 테스트 RMSE 및 AR(1) 대비 RMSE 감소율: AR(1) vs MIDAS-AR(1) (2023--2024). 종속변수: 전산업생산지수 성장률. 괄호 안 숫자는 AR(1) 대비 RMSE 감소율(\%)임.}
\label{tab:midasar_rmse_table}
\end{table}

\subsubsection*{MIDAS-AR 모형 적합 결과: 전산업생산지수 성장률 (전월대비)}

\begin{figure}[htbp]
\centering
\includegraphics[width=0.7\textwidth]{images/midas/midasar_sample_fit.png}
\caption{MIDAS-AR 모형 인샘플 적합 결과: 전산업생산지수 성장률 (전월대비). 훈련 기간(2002--2022년)에서의 모형 적합도를 보여줌.}
\label{fig:midasar_sample_fit}
\end{figure}

\begin{figure}[htbp]
\centering
\includegraphics[width=0.7\textwidth]{images/midas/midasar_test_fit.png}
\caption{MIDAS-AR 모형 테스트 적합 결과: 전산업생산지수 성장률 (전월대비). 테스트 기간(2023--2024년)에서의 예측값과 실제값 비교.}
\label{fig:midasar_test_fit}
\end{figure}

\begin{figure}[htbp]
\centering
\includegraphics[width=0.7\textwidth]{images/midas/midasar_weight.png}
\caption{MIDAS-AR 모형 exp-Almon 가중치: 전산업생산지수 성장률 (전월대비). Vintage별로 선택된 고빈도 래그에 대한 가중치 분포를 보여줌.}
\label{fig:midasar_weight}
\end{figure}

\begin{table}[htbp]
\centering
\small
\begin{tabular}{lcc}
\toprule
Vintage & AR(1) & MIDAS-AR(1) \\
\midrule
h0 & 1.49 (0.0) & 1.50 (-0.7) \\
h1 & 1.49 (0.0) & 1.60 (-7.4) \\
h2 & 1.49 (0.0) & 1.47 (1.3)  \\
h3 & 1.49 (0.0) & 1.50 (-0.7) \\
h4 & 1.49 (0.0) & 1.49 (0.0)  \\
\bottomrule
\end{tabular}
\caption{Vintage별 테스트 RMSE 및 AR(1) 대비 RMSE 감소율: 전년동월비 (2023--2024). 괄호 안 숫자는 AR(1) 대비 RMSE 감소율(\%)임.}
\label{tab:midasar_rmse_yoy}
\end{table}

\subsubsection*{MIDAS-AR 모형 적합 결과: 전산업생산지수 전년동월비}

\begin{figure}[htbp]
\centering
\includegraphics[width=0.7\textwidth]{images/midas/midasar_mom_sample_fit.png}
\caption{MIDAS-AR 모형 인샘플 적합 결과: 전산업생산지수 전년동월비. 훈련 기간(2002--2022년)에서의 모형 적합도를 보여줌.}
\label{fig:midasar_mom_sample_fit}
\end{figure}

\begin{figure}[htbp]
\centering
\includegraphics[width=0.7\textwidth]{images/midas/midasar_mom_test_fit.png}
\caption{MIDAS-AR 모형 테스트 적합 결과: 전산업생산지수 전년동월비. 테스트 기간(2023--2024년)에서의 예측값과 실제값 비교.}
\label{fig:midasar_mom_test_fit}
\end{figure}

\begin{figure}[htbp]
\centering
\includegraphics[width=0.7\textwidth]{images/midas/midasar_mom_weight.png}
\caption{MIDAS-AR 모형 exp-Almon 가중치: 전산업생산지수 전년동월비. Vintage별로 선택된 고빈도 래그에 대한 가중치 분포를 보여줌.}
\label{fig:midasar_mom_weight}
\end{figure}

\subsubsection*{Vintage별 테스트 RMSE: XGBoost 모형 비교}

\begin{table}[htbp]
\centering
\small
\begin{tabular}{lcccc}
\toprule
Vintage & AR(1) & ARX (linear) & AR(1)+XGB\_residual & XGB-direct \\
\midrule
h0 & 0.952 (0.0)         & \textbf{0.950} (0.2) & 1.110 (-10.3) & 1.030 (-4.5) \\
h1 & \textbf{0.953} (0.0)& 0.964 (-1.2)        & 1.040 (-11.2) & 0.979 (-2.6) \\
h2 & \textbf{0.953} (0.0)& 0.964 (-1.2)        & 1.040 (-10.2) & 1.000 (-4.4) \\
h3 & \textbf{0.953} (0.0)& 0.964 (-1.2)        & 1.000 (-7.0)  & 1.000 (-7.0)  \\
h4 & 0.953 (0.0)         & \textbf{0.940} (1.4)& 1.040 (-7.0)  & 0.951 (0.2)  \\
\bottomrule
\end{tabular}
\caption{Vintage별 테스트 RMSE 및 AR(1) 대비 RMSE 감소율 (2023--2024). 종속변수: 전산업생산지수 성장률. 각 셀은 2023--2024년 테스트 구간에서의 RMSE와, 괄호 안의 AR(1) 대비 RMSE 감소율(\%)을 함께 보고함. 감소율은 $100 \times (1 - \text{RMSE}_{m,h} / \text{RMSE}_{\text{AR(1)},h})$로 정의되며, 양수 값은 동일한 vintage에서 AR(1) 모형보다 예측 오차가 작다는 것을 의미함.}
\label{tab:rmse-xgb}
\end{table}

\begin{figure}[htbp]
\centering
\includegraphics[width=0.8\textwidth]{images/midas/xgboost_test_plots.png}
\caption{XGBoost 모형 테스트 결과: 전산업생산지수 성장률 (전월대비). AR(1), ARX, AR(1)+XGB\_residual, XGB-direct 모형의 vintage별 예측값과 실제값 비교.}
\label{fig:xgboost_test_plots}
\end{figure}

\begin{table}[htbp]
\centering
\small
\begin{tabular}{lcccc}
\toprule
Vintage & AR(1) & ARX (linear) & AR(1)+XGB\_residual & XGB-direct \\
\midrule
h0 & 1.49 (0.0)         & 1.51 (-1.6)        & 1.52 (-2.3)        & \textbf{1.42} (4.4) \\
h1 & \textbf{1.48} (0.0)& 1.58 (-6.4)        & 1.55 (-4.3)        & 1.61 (-8.6)         \\
h2 & \textbf{1.48} (0.0)& 1.58 (-6.4)        & 1.58 (-6.7)        & 1.55 (-4.3)         \\
h3 & \textbf{1.48} (0.0)& 1.58 (-6.4)        & 1.55 (-4.3)        & 1.58 (-6.7)         \\
h4 & \textbf{1.48} (0.0)& 1.53 (-2.9)        & 1.53 (-2.9)        & 1.52 (-2.3)         \\
\bottomrule
\end{tabular}
\caption{Vintage별 테스트 RMSE 및 AR(1) 대비 RMSE 감소율: 전년동월비 (2023--2024). 각 셀은 2023--2024년 테스트 구간에서의 RMSE와, 괄호 안의 AR(1) 대비 RMSE 감소율(\%)을 함께 보고함. 감소율은 $100 \times (1 - \mathrm{RMSE}_{m,h}/\mathrm{RMSE}_{\mathrm{AR(1)},h})$로 정의되며, 양수 값은 동일한 vintage에서 AR(1) 모형보다 예측 오차가 작다는 것을 의미함.}
\label{tab:xgb_rmse_yoy}
\end{table}

\subsubsection*{변수 중요도 히트맵}

\begin{figure}[htbp]
\centering
\includegraphics[width=0.7\textwidth]{images/midas/heatmap_iip.png}
\caption{변수 중요도 히트맵: 전산업생산지수 성장률 (전월대비). 각 변수의 vintage별 중요도를 시각화함.}
\label{fig:heatmap_iip}
\end{figure}

\begin{figure}[htbp]
\centering
\includegraphics[width=0.7\textwidth]{images/midas/heatmap_mom.png}
\caption{변수 중요도 히트맵: 전산업생산지수 전년동월비. 각 변수의 vintage별 중요도를 시각화함.}
\label{fig:heatmap_mom}
\end{figure}

\subsubsection*{ARX 모형 추정 결과}

\begin{table}[htbp]
\centering
\small
\begin{tabular}{lrrr}
\toprule
변수 & 계수 추정치 & 표준오차 & t값 \\
\midrule
상수항          & -1.699        & 1.906 & -0.89 \\
$y_{t-1}$       & -0.403$^{***}$& 0.071 & -5.67 \\
$\text{pw}_{t-1,w1}$     &  0.000        & 0.000 &  0.21 \\
$\text{BSI}_{t-1}$       & -0.076$^{*}$  & 0.035 & -2.20 \\
$\text{BSI}_{t-1}^{\text{YoY}}$ & -0.051$^{*}$  & 0.022 & -2.33 \\
$\text{pw}_{t,w1}$       &  0.000        & 0.000 & -0.21 \\
$\text{pw}_{t,w1}^{\text{YoY}}$ & -0.000       & 0.005 & -0.09 \\
$\text{BSI}_{t}$         &  0.096$^{**}$ & 0.035 &  2.76 \\
$\text{BSI}_{t}^{\text{YoY}}$   &  0.053$^{*}$  & 0.022 &  2.47 \\
\midrule
$R^{2}$       & \multicolumn{3}{r}{0.239} \\
조정 $R^{2}$  & \multicolumn{3}{r}{0.205} \\
관측치 수     & \multicolumn{3}{r}{188} \\
\bottomrule
\end{tabular}
\caption{ARX 모형 추정 결과: 월별 IIP 성장률에 대한 BSI 및 전력거래량의 영향. 종속변수는 전산업생산지수 월별 성장률($y_t$)이며, $y_{t-1}$은 1기 시차, $\text{pw}$는 월별(또는 주별) 전력거래량 관련 변수, $\text{BSI}$는 기업경기실사지수, ``YoY''는 전년동월 대비 변화를 의미함. 두 번째 열은 계수 추정치, 세 번째 열은 표준오차, 네 번째 열은 t값을 나타냄. $^{***}$ $p<0.01$, $^{**}$ $p<0.05$, $^{*}$ $p<0.10$.}
\label{tab:arx_bsi}
\end{table}

\begin{table}[htbp]
\centering
\small
\begin{tabular}{lrrr}
\toprule
변수 & 계수 추정치 & 표준오차 & t값 \\
\midrule
상수항                    & -11.587$^{***}$ &  2.754 & -4.21 \\
$y_{t-1}$                 &   0.398$^{***}$ &  0.066 &  6.07 \\
$\text{pw}_{t-1,w1}$      &  -0.000         &  0.000 & -0.45 \\
$\text{BSI}_{t-1}$        &   0.113$^{*}$   &  0.046 &  2.48 \\
$\text{BSI}_{t-1}^{\text{YoY}}$ &  -0.076$^{**}$  &  0.029 & -2.64 \\
$\text{pw}_{t,w1}$        &   0.000         &  0.000 &  1.58 \\
$\text{pw}_{t,w1}^{\text{YoY}}$ &  -0.008         &  0.006 & -1.28 \\
$\text{BSI}_{t}$          &   0.004         &  0.045 &  0.08 \\
$\text{BSI}_{t}^{\text{YoY}}$   &   0.123$^{***}$ &  0.028 &  4.40 \\
\midrule
$R^{2}$       & \multicolumn{3}{r}{0.788} \\
조정 $R^{2}$  & \multicolumn{3}{r}{0.779} \\
관측치 수     & \multicolumn{3}{r}{188}   \\
\bottomrule
\end{tabular}
\caption{ARX 모형 추정 결과: 전년동월비에 대한 BSI 및 전력거래량의 영향. 종속변수는 전산업생산지수 월별 전년동월비($y_t$)이며, $y_{t-1}$은 1기 시차, $\text{pw}$는 전력거래량 관련 변수, $\text{BSI}$는 기업경기실사지수, ``YoY''는 전년동월 대비 변화를 의미함. 두 번째 열은 계수 추정치, 세 번째 열은 표준오차, 네 번째 열은 t값을 나타냄. $^{***}$ $p<0.01$, $^{**}$ $p<0.05$, $^{*}$ $p<0.10$.}
\label{tab:arx_mom}
\end{table}

