\subsection*{C. 나우캐스팅 실험 세부사항}

본 부록에서는 나우캐스팅 실험의 세부사항을 제시함.

\subsubsection*{Big 데이터 DFM 모형 구성}

\FloatBarrier
\begin{table}[H]
\centering
\small
\begin{tabular}{lcc}
\toprule
구분 & 변수 & 개수 \\
\midrule
거시 & 실질 GDP, 소비, 민간총투자, 건설투자, 설비투자, 정부지출 & 8 \\
투자 & 설비투자, 기계수주액, 건설기성액 & 7 \\
산업생산 & 산업생산(총계, 주요 산업, 서비스업), 가동률, 출하/재고지수 & 23 \\
소비 & 소매판매, 신용카드매출액 & 6 \\
수출입 & 수출금액, 수입금액, 품목별 수출입 금액 & 7 \\
노동 & 실업률, 고용자수, 근로시간 & 7 \\
물가 & 소비자 물가, 생산자 물가, 수출입 물가, 원자재 가격 & 8 \\
서베이 & 기업경기조사, 소비자동향조사, 기업경기동향조사(한경협) & 35 \\
금융 & 가계/기업대출 잔액, 금리, 단기금리, 신용스프레드, 주가, 환율 & 9 \\
\midrule
합계 & & 110 \\
\bottomrule
\end{tabular}
\caption{Big 데이터 DFM 모형 변수 구성 (110개)}
\label{tab:big_data_variables}
\end{table}
\FloatBarrier

\subsubsection*{생산 부문 모형 변수 구성}

\FloatBarrier
\begin{table}[htbp]
\centering
\small
\begin{tabular}{lllll}
\toprule
분류 & 데이터 이름 & 주기 & 변환 & 시차 \\
\midrule
금융 & 주가지수 & 주 & 전월차 & 1 \\
금융 & 국채금리 & 주 & 전월차 & 1 \\
금융 & 회사채금리 & 주 & 전월차 & 1 \\
금융 & 원달러환율 & 주 & 전월차 & 1 \\
기업경기 & 뉴스심리지수 & 주 & 전월차 & 7 \\
기업경기 & 미국 경제정책 불확실성 지수 & 주 & 전월차 & 5 \\
금융 & 코스피 전기·전자업 섹터 지수 & 월 & 전월차 & 1 \\
고용/노동 & 실업률 & 월 & 전월차 & 11 \\
고용/노동 & 취업자 수 & 월 & 전월차 & 11 \\
고용/노동 & 경제활동인구 & 월 & 전월차 & 11 \\
고용/노동 & 근로자 주당 평균 노동시간 & 월 & 전월차 & 11 \\
수출입 & 수출(FOB, 달러) & 월 & 전월차 & 1 \\
수출입 & 대중국 수출(달러) & 월 & 전월차 & 1 \\
수출입 & 수출 물량 : 반도체 & 월 & 전월차 & 28 \\
수출입 & 수출 물량 : 자동차 & 월 & 전월차 & 28 \\
수출입 & 수입(CIF, 관세기준. 달러) & 월 & 전월차 & 1 \\
수출입 & 순상품교역조건 & 월 & 전월차 & 14 \\
소비/지출 & 소매판매액지수(계절조정) & 월 & 전월차 & 28 \\
물가 & 소비자물가지수 & 월 & 전월차 & 3 \\
물가 & 생산자물가지수 & 월 & 전월차 & 20 \\
물가 & 소비자물가 : 식료품·에너지 제외 & 월 & 전월차 & 3 \\
설비투자 & 설비투자지수 & 월 & 전월차 & 30 \\
산업생산 & 제조업 출하지수 & 월 & 전월차 & 30 \\
산업생산 & 제조업 재고지수 & 월 & 전월차 & 30 \\
산업생산 & 서비스업 활동지수 & 월 & 전월차 & 30 \\
산업생산 & 전산업생산지수 & 월 & 전월차 & 30 \\
산업생산 & 광공업생산지수 & 월 & 전월차 & 30 \\
산업생산 & 생산 : 화학제품·의약·재외 & 월 & 전월차 & 30 \\
산업생산 & 생산 : 전자부품·컴퓨터·영상·통신 & 월 & 전월차 & 30 \\
산업생산 & 생산 : 자동차 및 트레일러 & 월 & 전월차 & 30 \\
산업생산 & 생산 : 기타 운송장비·조선 & 월 & 전월차 & 30 \\
산업생산 & 생산 : 건설업 & 월 & 전월차 & 30 \\
산업생산 & 경기선행지수 & 월 & 전월차 & 30 \\
산업생산 & 경기동행지수 & 월 & 전월차 & 30 \\
기업경기 & 기업경기실사지수(BSI) 종합 & 월 & 전월차 & -5 \\
기업경기 & 기업경기실사지수(BSI) 기업경기전망 & 월 & 전월차 & -5 \\
기업경기 & 경기실적(전산업) & 월 & 전월차 & -5 \\
기업경기 & 경기전망(전산업) & 월 & 전월차 & -35 \\
기업경기 & FKI 기업경기지수(전산업, 계절조정) & 월 & 전월차 & -5 \\
기업경기 & 내수 실적(전산업) & 월 & 전월차 & -5 \\
기업경기 & 경기전망(전산업, 계절조정) & 월 & 전월차 & -35 \\
기업경기 & 제조업 PMI 지수 & 월 & 전월차 & 3 \\
기업경기 & 제조업 PMI 생산 & 월 & 전월차 & 3 \\
소비자동향 & 소비자심리지수(종합) & 월 & 전월차 & -5 \\
소비자동향 & 향후 경기전망 & 월 & 전월차 & -5 \\
\midrule
합계 & & & & 41 \\
\bottomrule
\end{tabular}
\caption{생산 부문 모형 변수 구성 (41개)}
\label{tab:production_variables}
\end{table}


\FloatBarrier

\subsubsection*{투자 부문 모형 변수 구성}

\FloatBarrier
\begin{table}[htbp]
\centering
\small
\begin{tabular}{lllll}
\toprule
분류 & 데이터 이름 & 주기 & 변환 & 시차 \\
\midrule
금융 & 주가지수 & 주 & 전월차 & 1 \\
금융 & 국채금리 & 주 & 전월차 & 1 \\
금융 & 회사채금리 & 주 & 전월차 & 1 \\
금융 & 원달러환율 & 주 & 전월차 & 1 \\
금융 & 원자재가격 & 주 & 전월차 & 1 \\
기업경기 & 뉴스심리지수 & 주 & 전월차 & 7 \\
기업경기 & 미국 경제정책 불확실성 지수 & 주 & 전월차 & 5 \\
금융 & 코스피 건설업 지수 & 주 & 전월차 & 1 \\
금융 & 코스피 기계업 지수 & 주 & 전월차 & 1 \\
고용/노동 & 취업자 수(공공업) & 월 & 전월차 & 11 \\
고용/노동 & 취업자 수(건설업) & 월 & 전월차 & 11 \\
수출입 & 수입(자본재, 달러) & 월 & 전월차 & 14 \\
물가 & 생산자물가지수 & 월 & 전월차 & 20 \\
물가 & 생산자물가지수 : 원재료 & 월 & 전월차 & 20 \\
설비투자 & 설비투자지수(계절조정) & 월 & 전월차 & 30 \\
설비투자 & 설비투자 : 기계류 & 월 & 전월차 & 30 \\
설비투자 & 설비투자 : 운송장비 & 월 & 전월차 & 30 \\
건설 & 건설 수주액(총액, 원) & 월 & 전월차 & 30 \\
건설 & 건축 인허가 면적 & 월 & 전월차 & 31 \\
건설 & 건설 착공 면적 & 월 & 전월차 & 31 \\
건설 & 건설 준공액(총액, 원) & 월 & 전월차 & 30 \\
산업생산 & 제조업 출하 : 자본재 & 월 & 전월차 & 30 \\
산업생산 & 제조업 재고 : 자본재 & 월 & 전월차 & 30 \\
산업생산 & 서비스업 : 부동산·임대업 & 월 & 전월차 & 30 \\
산업생산 & 서비스업 : 사업시설·사업지원 & 월 & 전월차 & 30 \\
산업생산 & 광공업생산지수 & 월 & 전월차 & 30 \\
산업생산 & 생산 : 건설업 & 월 & 전월차 & 30 \\
산업생산 & 생산 : 자본재 & 월 & 전월차 & 30 \\
산업생산 & 생산 : 내구재, 계절조정 & 월 & 전월차 & 30 \\
산업생산 & 경기선행지수 & 월 & 전월차 & 30 \\
기업경기 & 기업경기실사지수(BSI) 종합 & 월 & 전월차 & -5 \\
기업경기 & 기업경기실사지수(BSI) 기업경기전망 & 월 & 전월차 & -5 \\
기업경기 & 설비투자(제조업 실적) & 월 & 전월차 & -5 \\
기업경기 & 설비투자(제조업 전망) & 월 & 전월차 & -35 \\
기업경기 & FKI 기업경기지수(전산업, 계절조정) & 월 & 전월차 & -5 \\
기업경기 & 투자 실적(전산업) & 월 & 전월차 & -5 \\
기업경기 & 투자 전망(전산업) & 월 & 전월차 & -35 \\
소비자동향 & 소비자심리지수(종합) & 월 & 전월차 & -5 \\
소비자동향 & 고용상황 전망 & 월 & 전월차 & -5 \\
금융 & 여신금융·상호금융 설비자금대출 & 월 & 전월차 & 45 \\
금융 & 기업대출금리(신규취급분) & 월 & 전월차 & 30 \\
\midrule
합계 & & & & 41 \\
\bottomrule
\end{tabular}
\caption{투자 부문 모형 변수 구성 (41개)}
\label{tab:investment_variables}
\end{table}


\FloatBarrier

\subsubsection*{Nowcasting 시각화 결과}

본 절에서는 생산 모형(전산업생산지수)과 투자 모형(설비투자지수)에 대한 DFM과 MAMBA 모형의 nowcasting 결과를 시각화함. 각 모형별 예측값과 실제값을 비교하여 모형의 성능을 평가함.

\begin{figure}[htbp]
\centering
\begin{subfigure}[b]{0.8\textwidth}
\centering
\includegraphics[width=\textwidth]{images/nowcast/production_nowcast_dfm.png}
\caption{생산 모형 DFM}
\label{fig:appendix_production_dfm}
\end{subfigure}

\vspace{0.3cm}

\begin{subfigure}[b]{0.8\textwidth}
\centering
\includegraphics[width=\textwidth]{images/nowcast/production_nowcast_mamba.png}
\caption{생산 모형 MAMBA}
\label{fig:appendix_production_mamba}
\end{subfigure}

\caption{생산 모형(전산업생산지수)의 DFM과 MAMBA nowcasting 결과 비교. 각 플롯은 모형별 예측값과 실제값을 시간 순서로 비교한 그래프임.}
\label{fig:appendix_production_comparison}
\end{figure}

\begin{figure}[htbp]
\centering
\begin{subfigure}[b]{0.8\textwidth}
\centering
\includegraphics[width=\textwidth]{images/nowcast/investment_nowcast_dfm.png}
\caption{투자 모형 DFM}
\label{fig:appendix_investment_dfm}
\end{subfigure}

\vspace{0.3cm}

\begin{subfigure}[b]{0.8\textwidth}
\centering
\includegraphics[width=\textwidth]{images/nowcast/investment_nowcast_mamba.png}
\caption{투자 모형 MAMBA}
\label{fig:appendix_investment_mamba}
\end{subfigure}

\caption{투자 모형(설비투자지수)의 DFM과 MAMBA nowcasting 결과 비교. 각 플롯은 모형별 예측값과 실제값을 시간 순서로 비교한 그래프임.}
\label{fig:appendix_investment_comparison}
\end{figure}

\textbf{생산 모형(전산업생산지수) 결과}
\begin{itemize}
    \item \textbf{DFM 모형:} 그림~\ref{fig:appendix_production_dfm}에서 DFM 모형의 nowcasting 결과를 확인할 수 있음. 모형은 전반적으로 실제값을 잘 추적하며, 발표 시점에 가까워질수록 예측 정확도가 향상됨.
    \item \textbf{MAMBA 모형:} 그림~\ref{fig:appendix_production_mamba}에서 MAMBA 모형의 nowcasting 결과를 확인할 수 있음. MAMBA 모형은 DFM과 유사한 성능을 보이며, 월별 전망값 변동이 DFM 모형보다 작게 나타남.
\end{itemize}

\textbf{투자 모형(설비투자지수) 결과}
\begin{itemize}
    \item \textbf{DFM 모형:} 그림~\ref{fig:appendix_investment_dfm}에서 DFM 모형의 nowcasting 결과를 확인할 수 있음. 투자 지수는 생산 지수에 비해 변동성이 크며, 모형의 예측이 일부 구간에서 실제값과 차이를 보임.
    \item \textbf{MAMBA 모형:} 그림~\ref{fig:appendix_investment_mamba}에서 MAMBA 모형의 nowcasting 결과를 확인할 수 있음. MAMBA 모형의 성과가 DFM 대비 소폭 부진하였으나, 전반적으로 유사한 패턴을 보임.
\end{itemize}

