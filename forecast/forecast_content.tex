\section{개요}
\label{sec:overview}

\subsection{연구 목적}
\label{sec:macro_forecast_necessity}

정확한 거시경제 변수 예측은 정책 수립, 기업 경영 전략, 금융 시장 분석 등 여러 영역에서 중요함 \cite{stock2002forecasting, banbura2012nowcasting}. 본 파트에서는 한국 거시경제의 대표 변수인 생산, 투자, 소비에 대해 고빈도 정보를 활용한 예측(nowcasting/forecasting) 문제를 설정하고, 상태공간 모형과 딥러닝 모형의 성능을 동일한 데이터와 평가 기준으로 비교함 \cite{banbura2012nowcasting, bok2019frbny}.

\subsection{거시 경제 변수 예측 주요 이슈}
\label{sec:forecast_issues}

거시경제 nowcasting/forecasting에서 핵심적인 도전 과제는 다음과 같음:
\begin{itemize}
    \item \textbf{고차원 공변량:} 많은 거시·금융·서베이 변수를 동시에 사용하면 과적합과 계산 부담이 커짐 \cite{stock2002forecasting}.
    \item \textbf{혼합 주기/비동기 발표:} 주·월·분기 데이터가 섞이고 발표시점이 달라(jagged edges) 결측이 구조적으로 발생함 \cite{banbura2012nowcasting}.
    \item \textbf{비선형/구조변화:} 위기·팬데믹 구간 등에서 선형 가정이 성능을 제한할 수 있음 \cite{huber2020nowcasting, andreini2020deep}.
\end{itemize}

\section{데이터와 전처리}
\label{sec:data_introduction}

\subsubsection{기본 데이터 탐색}
\label{subsec:data_exploration}

\begin{itemize}
  \item 시계열 간 스케일 비율이 크게 달라 수치적 정밀도 문제를 야기할 수 있음.
  \item 일부 시계열이 매우 낮은 분산을 보여 수치적 불안정성을 유발할 수 있음.
  \item 완전한 관측값(complete cases)의 비율이 낮아 대부분의 관측값이 하나 이상의 결측치를 포함함.
\end{itemize}

데이터 품질 및 통계량 대시보드는 그림~\ref{fig:data_quality_dashboard}에 제시됨.
\begin{figure}[htbp]
    \centering
    \includegraphics[width=\textwidth]{forecast/images/data_quality_dashboard.png}
    \caption{데이터 품질 및 통계량 대시보드}
    \label{fig:data_quality_dashboard}
\end{figure}

\subsubsection{타겟 변수 및 설명변수 구성}
\label{subsec:key_variables}

본 파트에서는 세 가지 주요 거시경제 변수를 타겟으로 함:
\begin{itemize}
    \item \textbf{생산:} 전산업생산지수(KOIPALL.G)
    \item \textbf{투자:} 설비투자지수(KOEQUIPTE)
    \item \textbf{소비:} 도소매판매액(KOWRCCNSE)
\end{itemize}

\begin{itemize}
  \item 세 부문 모형 모두 총 41개 변수로 구성됨.
  \item 포함 변수:
  \begin{itemize}
    \item 고용, 산업생산, 서베이(기업경기, 소비자 동향) 등 주요 월간 지수.
    \item 주간 데이터.
    \item 주가지수 등 금융변수, 뉴스심리지수, 미국 경제정책불확실성 지수.
  \end{itemize}
  \item 기업경기동향 조사는 해당월 중 발표되어 속보성이 높음.
\end{itemize}

\subsubsection{전처리}
\label{subsec:preprocessing}

본 연구에서는 모든 모형에 동일한 전처리 파이프라인을 적용하여 공정한 비교를 보장함.

\begin{itemize}
  \item \textbf{변환(Transformation):} 각 시계열의 특성에 맞는 변환을 적용함. 변환 유형: lin(수준값), log(로그), chg(전기대비 차분), ch1(전년동기대비 차분), pch(전기대비 성장률), pc1(전년동기대비 성장률), cha(연율화 차분), pca(연율화 성장률).
  \item \textbf{결측치 처리(Imputation):} 다음 순서로 처리함:
  \begin{enumerate}
    \item forward-fill: 이전 값으로 채움.
    \item backward-fill: 이후 값으로 채움.
    \item naive forecaster: 마지막 관측값으로 채움.
  \end{enumerate}
  \item \textbf{표준화(Scaling):} 모든 모형에 RobustScaler를 적용함. 중앙값(median)을 0으로, 사분위수 범위(IQR)를 1로 조정하여 이상치에 강건한 표준화를 수행함.
  \item \textbf{주파수:} 원본 데이터는 주간 주파수로 제공됨. 모든 모형은 주간 주파수로 학습하고 예측을 생성하며, 일관된 예측을 위해 주간 데이터를 그대로 사용함(리샘플링 없음).
\end{itemize} 

모형별 주간-월간 변환 방식은 다음과 같음:
\begin{itemize}
    \item \textbf{DFM/DDFM:} 주간 데이터를 기본으로 하며, 혼합주기 옵션을 통해 tent kernel이 자동으로 적용되어 주간/월간 데이터를 통합 처리함 \cite{mariano2003new}. 예측 생성 시 horizon은 개월 단위로 지정되며, 모형 내부에서 자동으로 주 단위로 변환됨(1개월 = 4주). 예측 결과는 주간 단위로 생성되며, 평가를 위해 월간으로 평균 집계함.
    
    \item \textbf{딥러닝 모형(TFT, Chronos, LSTM):} 모든 딥러닝 모형은 주간 주파수로 학습하고 주간 단위로 예측을 생성함. 예측 생성 후 평가를 위해 주간 예측을 월간으로 변환하는데, 이는 월별로 주간 예측값을 평균 집계하는 방식으로 수행됨. 구체적으로, 각 월에 해당하는 주간 예측값들을 평균하여 월간 예측값을 생성함.
\end{itemize}

\begin{itemize}
  \item 예측 평가 시 모든 모형의 주간 예측을 월간으로 평균 집계한 후, 원본 타겟 변수(월간 주파수)와 비교함.
  \item 모든 모형을 동일한 기준으로 평가하기 위한 것으로, 주간 예측의 세부 패턴보다는 월간 집계 수준에서의 예측 정확도를 중시함.
\end{itemize}

세 대상 변수에 대한 전처리 결과는 그림~\ref{fig:preprocessed_targets}에 제시됨.
\begin{figure}[htbp]
    \centering
    \includegraphics[width=\textwidth]{forecast/images/preprocessed_targets.png}
    \caption{전처리된 타겟 변수 시계열}
    \label{fig:preprocessed_targets}
\end{figure}

\section{모형과 실험 설계}
\label{sec:methodology}

\subsection{예측 모형}
\label{sec:forecasting_models}

\subsubsection{딥러닝 시계열 모형}
\label{subsec:deep_learning_models}

\begin{itemize}
  \item \textbf{예측 생성 방식:}
  \begin{itemize}
    \item 직접 장기 예측(direct long-horizon forecasting): 전체 예측 시점(88주 = 22개월)을 한 번에 예측.
    \item 재귀적 예측(recursive forecasting): 짧은 구간씩 예측을 반복하여 전체 시점에 도달.
  \end{itemize}
  \item \textbf{Temporal Fusion Transformer (TFT):}
  \begin{itemize}
    \item Attention 기반 아키텍처로, 다중 시점 예측과 해석 가능성을 결합한 모형 \cite{lim2021temporal}.
    \item LSTM을 지역 처리에 사용하고 self-attention을 장기 의존성에 사용함.
    \item Variable Selection Networks를 통해 변수별 중요도를 해석할 수 있어 경제 예측에 유용함 \cite{lim2021temporal}.
    \item 본 연구에서는 재귀적 예측 방식으로 사용하며, 24주 구간을 반복 예측하여 전체 88주(22개월) 예측을 생성함. 모형은 50개의 공변량과 함께 학습되며, 예측 시에도 동일한 공변량을 제공하여 학습 시와 일관된 조건에서 예측을 수행함.
  \end{itemize}
  \item \textbf{Chronos:}
  \begin{itemize}
    \item 사전 훈련된 foundation model로, 대규모 시계열 데이터로 사전 훈련되어 다양한 시계열 패턴을 학습함 \cite{ansari2024chronos}.
    \item Transformer 기반 아키텍처를 사용하여 장기 의존성을 포착함.
    \item 본 연구에서는 직접 장기 예측 방식으로 사용하며, 전체 88주(22개월)를 한 번에 예측함.
  \end{itemize}
  \item \textbf{LSTM:}
  \begin{itemize}
    \item 순환 신경망(RNN)의 변형으로, forget gate, input gate, output gate를 통해 정보의 흐름을 제어하며 장기 의존성을 학습할 수 있음 \cite{hochreiter1997long}.
    \item Gradient vanishing 문제를 완화하여 긴 시계열에서도 효과적으로 학습할 수 있음.
    \item 본 연구에서는 직접 장기 예측 방식으로 사용하며, 전체 88주(22개월)를 한 번에 예측함.
  \end{itemize}
\end{itemize}

\subsubsection{상태공간 모형}
\label{subsec:state_space_models}

\begin{itemize}
  \item \textbf{동적요인모형(DFM):}
  \begin{itemize}
    \item 많은 시계열에서 공통 요인을 추출해 소수의 동태적 요인으로 설명하는 차원축소 기법 \cite{stock2002forecasting}.
    \item 관측식과 상태식을 갖는 state-space 형태로 표현됨.
    \item EM 알고리즘으로 파라미터를 추정하고, Kalman filter를 통해 요인을 추정함 \cite{bok2019frbny}.
    \item 혼합주기 데이터와 비동기적 데이터 발표(jagged edges)를 처리하는 데 강점이 있음 \cite{banbura2012nowcasting, bok2019frbny}.
  \end{itemize}
  \item \textbf{심층 동적요인모형(DDFM):}
  \begin{itemize}
    \item 오토인코더 기반 비선형 인코더를 사용해 요인 구조를 학습함으로써 전통적 DFM의 선형 가정을 완화함 \cite{andreini2020deep}.
    \item 비선형 인코더는 고차원 거시 데이터의 복잡한 상호작용을 더 적은 요인으로 포착함.
    \item 요인층 뒤에는 선형 state-space를 두어 필터링·스무딩 안정성을 유지함.
    \item 학습은 두 단계로 구성됨:
    \begin{enumerate}
      \item 오토인코더를 통해 재구성 오차를 최소화하여 요인 구조를 학습.
      \item 학습된 요인을 사용하여 전이 행렬을 추정하고 Kalman filter를 통해 최종 스무딩을 수행함.
    \end{enumerate}
    \item 전통적 DFM의 요인 식별 제약 문제를 자연스럽게 해결하며, 혼합주기 데이터와 대규모 변수 집합을 효율적으로 처리할 수 있음.
  \end{itemize}
\end{itemize}

\subsection{실험 구성}
\label{sec:experiment_design}

\subsubsection{평가 기준}
\label{subsec:evaluation_criteria}

본 연구에서는 데이터를 세 구간으로 분할하여 모형 학습 및 평가를 수행함:

\begin{itemize}
    \item \textbf{훈련 기간(Train):} 1985년 1월부터 2019년 12월까지 (35년간). 모든 모형은 이 기간의 데이터를 사용하여 학습함. 이 기간은 충분히 긴 시계열을 제공하여 모형이 장기적 패턴과 계절성을 학습할 수 있도록 함.
    
    \item \textbf{최근 기간(Recent):} 2020년 1월부터 2023년 12월까지 (4년간). 이 기간은 훈련 기간과 테스트 기간 사이의 중간 구간으로, 원래는 DFM과 DDFM 모형의 state update를 위해 사용하려고 했으나, 코드 구현상의 문제로 인해 실제로는 사용하지 않음. 대신, 각 모형은 훈련 기간 데이터로만 학습한 후 테스트 기간에 대해 예측을 생성함.
    
    \item \textbf{테스트 기간(Test):} 2024년 1월부터 2025년 10월까지 (22개월). 모든 모형의 예측 성능을 평가하는 기간으로, 실제 예측 상황을 시뮬레이션함. 각 모형은 훈련 기간 데이터로 학습한 후, 테스트 기간에 대해 예측을 생성함. 구체적으로, DFM은 발산 문제를 완화하기 위해 재귀적 예측(6개월 청크 단위)을 사용하며, DDFM은 발산 문제가 발생하지 않아 재귀적 예측 없이 한 번에 전체 22개월을 예측하는 원타임 예측 방식을 사용함.
\end{itemize}

\begin{itemize}
  \item 모든 모형은 주간 주파수로 예측을 생성하며, 원본 타겟 변수가 월간 주파수이므로 주간 예측을 월간으로 평균 집계하여 비교함.
  \item 주간 예측을 월간으로 평균 집계한 후, 각 예측 시점(1--22개월)에 대한 지표를 평균하여 최종 성능 지표로 사용함.
\end{itemize}

\subsubsection{하이퍼 패러미터}
\label{subsec:hyperparameters}

\begin{itemize}
  \item \textbf{CHRONOS:} amazon/chronos-t5-tiny (pre-trained foundation model), prediction length 24 weeks, robust scaler.
  \item \textbf{DDFM:} encoder layers [64, 32], num factors 3, epochs 50, learning rate 0.005, batch size 100, factor order 2, robust scaler.
  \item \textbf{DFM:} max EM iterations 5000, convergence threshold $1.0 \times 10^{-5}$, 3 factors, AR lag 1, mixed frequency enabled, robust scaler.
  \item \textbf{LSTM:} input size 96 weeks, hidden size 64, 2 layers, learning rate 0.001, epochs 50, batch size 32, robust scaler.
  \item \textbf{TFT:} input size 96 weeks, hidden size 64, 4 attention heads, dropout 0.1, learning rate 0.001, max epochs 10, batch size 256, max covariates 50, robust scaler.
\end{itemize}

\subsubsection{성능 지표}
\label{subsec:performance_metrics}

\begin{itemize}
  \item \textbf{표준화된 지표:} sMAE, sMSE, sRMSE. 각 지표는 훈련 데이터의 표준편차로 정규화하여 계산함. 변수 간 스케일 차이를 제거하여 직접 비교 가능하게 하며, 특히 거시경제 변수처럼 단위와 크기가 다른 변수들을 비교할 때 필수적임 \cite{stock2002forecasting}.
  \item \textbf{절대 지표:} MAE, MSE, RMSE. 원본 단위에서의 예측 오차를 나타냄.
  \item 모든 모형은 주간 주파수로 학습하고 예측을 생성함. 주간 예측값을 월간으로 평균 집계한 후, 실제 월간 값과 비교하여 메트릭을 계산함. 모든 지표는 1개월부터 22개월까지의 예측 시점에 대해 계산한 후 평균하여 최종 성능 지표로 사용함.
\end{itemize}

\section{실험 결과}
\label{sec:experiment_result}

\subsection{예측 결과 요약}
\label{sec:forecasting_results_comparison}

\begin{itemize}
  \item 대상 변수: 생산(KOIPALL.G), 투자(KOEQUIPTE), 소비(KOWRCCNSE).
  \item 비교 모형: DFM, DDFM, TFT, Chronos, LSTM.
  \item 모든 모형은 주간 주파수로 예측을 생성하며, 주간 예측을 월간으로 평균 집계한 후 1개월부터 22개월까지의 시점에 대해 평가함.
  \item 예측 결과는 아래 전체 시점 평균 성능 섹션에서 요약되며, 시점별 상세 결과는 부록 표들(표~\ref{tab:koipallg_forecasts}, 표~\ref{tab:koequipte_forecasts}, 표~\ref{tab:kowrccnse_forecasts})에 포함됨.
\end{itemize}

\subsubsection{전체 시점 평균 성능}

변수별로 최우수 모형이 다르게 나타났으며, 특히 예측 시점(단기 vs 장기)에 따라 강점이 달라지는 양상이 관찰됨. DFM은 전이 행렬의 불안정성으로 인해 테스트 구간에서 오차가 크게 나타났음(논의 섹션 참조).

\begin{itemize}
  \item \textbf{KOIPALL.G(생산):} Chronos(sMAE=1.44) > LSTM(sMAE=1.87) > DDFM(sMAE=1.72) > TFT(sMAE=2.07) > DFM(sMAE=3.55). 단기 예측에서는 DDFM이 가장 우수하나, 장기 예측에서는 Chronos와 LSTM이 우수함.
  
  \item \textbf{KOEQUIPTE(투자):} TFT(sMAE=0.53) > DDFM(sMAE=1.50) > Chronos(sMAE=2.60) > DFM(sMAE=3.99) > LSTM(sMAE=5.04). 단기 예측에서는 DDFM이 가장 우수하나, 장기 예측에서는 TFT가 크게 우수함.
  
  \item \textbf{KOWRCCNSE(소비):} DDFM(sMAE=0.32) > TFT(sMAE=1.48) > Chronos(sMAE=2.44) > LSTM(sMAE=2.87) > DFM(sMAE=3.82). 단기 및 장기 예측 모두에서 DDFM이 가장 우수함.
\end{itemize}

테스트 기간 동안의 예측값과 실제값 비교는 그림~\ref{fig:forecast_vs_actual_all}에 제시됨. KOIPALL.G에서는 Chronos와 LSTM이 실제값을 비교적 잘 추적하며, KOEQUIPTE에서는 TFT가 실제값을 가장 정확하게 추적함. KOWRCCNSE에서는 DDFM이 실제값을 가장 정확하게 추적함. DFM은 재귀적 예측으로 인해 예측값이 들쭉날쭉한 패턴을 보임.
\begin{figure}[htbp]
    \centering
    \includegraphics[width=\textwidth]{forecast/images/forecast_vs_actual_all.png}
    \caption{예측값 vs 실제값 비교: 목표 변수 월별 실제값과 모델 예측값}
    \label{fig:forecast_vs_actual_all}
\end{figure}

\subsubsection{시점별 성능 패턴}
\label{subsec:horizon_performance}

\begin{itemize}
  \item \textbf{단기(1--6개월):} DDFM이 상대적으로 강한 경향이 관찰됨.
  \item \textbf{장기(13--22개월):} 생산·투자에서는 TFT/Chronos/LSTM이 상대적으로 안정적인 경우가 있었고, 소비에서는 DDFM 우위가 유지됨.
  \item 시점별 상세 수치는 부록 표(표~\ref{tab:koipallg_forecasts}--\ref{tab:kowrccnse_forecasts})에 제시됨.
\end{itemize}

\subsection{논의}
\label{sec:discussion}

\begin{itemize}
  \item \textbf{변수별 이질성:} 생산/투자/소비는 변동성, 구조변화, 공변량의 정보량이 달라 단일 모형이 항상 우세하지 않을 수 있음.
  \item \textbf{DFM 안정성 이슈:} 본 실험에서는 전이 행렬의 불안정(고유값 $>1$)으로 예측이 발산하거나 과도하게 수렴하는 현상이 관찰되었고, 이를 완화하기 위해 DFM에만 재귀적 예측(청크 단위)을 적용함. 이 설정 차이는 모형 간 비교의 공정성에 영향을 줄 수 있으므로, 추후에는 (i) 안정화(regularization/고유값 제약) 강화, (ii) 재귀 예측/원타임 예측을 모형별로 모두 비교하는 형태로 재설계를 권장함.
  \item \textbf{실험 설계 제약:} Recent 구간(2020--2023)의 상태 업데이트가 계획대로 반영되지 못했으며, 하이퍼파라미터 튜닝도 모형 간 완전한 동등성으로 수행되지는 못했음.
\end{itemize}

\section{결론}
\label{sec:conclusion}

본 파트의 비교 실험에서는 변수별로 우수한 모형이 달랐으며(생산: Chronos, 투자: TFT, 소비: DDFM), 시점(단기/장기)에 따라서도 상대적 우위가 달라질 수 있음을 확인함. 다만 DFM 안정성 이슈 및 실험 설계 제약(Recent 미활용, 예측 방식 차이 등)이 존재하므로, 후속 연구에서는 안정화 강화와 공정한 예측 설정(재귀/원타임의 동등 비교), 그리고 체계적인 튜닝/검증을 포함한 재설계를 권장함.
